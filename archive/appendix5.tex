\chapter{Verzeichnis der Paradigmen}\label{appendix5}

Paradigma 1: Althochdeutsches Substantiv \citep[183–217]{Braune2004}  S. 248

Paradigma 2: Mittelhochdeutsches Substantiv \citep[183–199]{Paul2007}  S. 249

Paradigma 3: Substantivflexion der deutschen Standardsprache \citep[158–169]{Eisenberg2006}  S. 250

Paradigma 4: Substantivflexion von Issime \citep[144–205]{Zürrer1999}  S. 251

Paradigma 5: Substantivflexion von Visperterminen \citep[119–134]{Wipf1911}  S. 252

Paradigma 6: Substantivflexion von Jaun \citep[255–272]{Stucki1917}  S. 253

Paradigma 7: Substantivflexion des Sensebezirks \citealt[179–190]{Henzen1927}  S. 254

Paradigma 8: Substantivflexion von Uri \citep[173–185]{Clauß1929}  S. 254

Paradigma 9: Substantivflexion von Vorarlberg \citep[231–261]{Jutz1925}  S. 255

Paradigma 10: Substantivflexion von Zürich \citep[108–119]{Weber1987}  S. 255

Paradigma 11: Substantivflexion von Bern \citep[82–90]{Marti1985}  S. 255

Paradigma 12: Substantivflexion von Huzenbach \citep[92–98]{Baur1967}  S. 255

Paradigma 13: Substantivflexion von Saulgau \citep[100–109]{Raichle1932}  S. 256

Paradigma 14: Substantivflexion von Stuttgart \citep[149–152]{Frey1975}  S. 256

Paradigma 15: Substantivflexion von Petrifeld \citep[59–62]{Moser1937}  S. 256

Paradigma 16: Substantivflexion von Elisabethtal \citep[50–52]{Žirmunskij1928/29}  S. 257

Paradigma 17: Substantivflexion des Kaiserstuhls \citep[359–373]{Noth1993}  S. 257

Paradigma 18: Substantivflexion des Münstertals \citep[40–44]{Mankel1886}  S. 257

Paradigma 19: Substantivflexion von Colmar \citep[71–76]{Henry1900}  S. 257

Paradigma 20: Substantivflexion des Elsass (Ebene) \citep[25–71]{Beyer1963}  S. 258

Paradigma 21: Starke und schwache Adjektive im Althochdeutschen \citep[217–227]{Braune2004}  S. 259

Paradigma 22: Starke und schwache Adjektive im Mittelhochdeutschen \citep[200–203]{Paul2007}  S. 260

Paradigma 23: Starke und schwache Adjektive in der Standardsprache \citep[177–184]{Eisenberg2006}  S. 260

Paradigma 24: Starke und schwache Adjektive in Issime \citep[267–268]{Zürrer1999}, \citep[90–97]{Perinetto1981}  S. 261

Paradigma 25: Starke und schwache Adjektive in Visperterminen \citep[134–135]{Wipf1911}  S. 261

Paradigma 26: Starke und schwache Adjektive in Jaun \citep[272–275]{Stucki1917}  S. 262

Paradigma 27: Starke und schwache Adjektive im Sensebezirk \citep[190–192]{Henzen1927}  S. 262

Paradigma 28: Starke und schwache Adjektive in Uri \citep[185–187]{Clauß1929}  S. 263

Paradigma 29: Starke und schwache Adjektive in Vorarlberg \citep[261–267]{Jutz1925}  S. 263

Paradigma 30: Starke und schwache Adjektive in Zürich \citep[121–126]{Weber1987}  S. 264

Paradigma 31: Starke und schwache Adjektive in Bern \citep[117–120]{Marti1985}  S. 264

Paradigma 32: Starke und schwache Adjektive in Huzenbach \citep[98–99]{Baur1967}  S. 265

Paradigma 33: Starke und schwache Adjektive in Saulgau \citep[109–113]{Raichle1932}  S. 265

Paradigma 34: Starke und schwache Adjektive in Stuttgart \citep[157–159]{Frey1975}  S. 266

Paradigma 35: Starke und schwache Adjektive in Petrifeld \citep[62–63]{Moser1937}  S. 266

Paradigma 36: Starke und schwache Adjektive in Elisabethtal \citep[52]{Žirmunskij1928/29}  S. 267

Paradigma 37: Starke und schwache Adjektive im Kaiserstuhl \citep[407–410]{Noth1993}  S. 267

Paradigma 38: Starke und schwache Adjektive im Münstertal \citep[44–45]{Mankel1886}  S. 268

Paradigma 39: Starke und schwache Adjektive in Colmar \citep[77–80]{Henry1900}  S. 268

Paradigma 40: Starke und schwache Adjektive im Elsass (Ebene) \citep[114–146]{Beyer1963}  S. 269

Paradigma 41: Personalpronomen des Althochdeutschen \citep[241–245]{Braune2004}  S. 270

Paradigma 42: Personalpronomen des Mittelhochdeutschen \citep[210–214]{Paul2007}  S. 271

Paradigma 43: Personalpronomen der deutschen Standardsprache \citep[169–177]{Eisenberg2006}  S. 272

Paradigma 44: Personalpronomen von Issime \citep[206–312]{Zürrer1999}  S. 273

Paradigma 45: Personalpronomen von Visperterminen \citep[139–141]{Wipf1911}  S. 274

Paradigma 46: Personalpronomen von Jaun \citep[280–282]{Stucki1917}  S. 275

Paradigma 47: Personalpronomen des Sensebezirks \citep[196–198]{Henzen1927}  S. 276

Paradigma 48: Personalpronomen von Uri \citep[190–192]{Clauß1928}  S. 277

Paradigma 49: Personalpronomen von Vorarlberg \citep[271–274]{Jutz1925}  S. 278

Paradigma 50: Personalpronomen von Zürich \citep[153–162]{Weber1987}  S. 278

Paradigma 51: Personalpronomen von Bern \citep[92–97]{Marti1985}  S. 279

Paradigma 52: Personalpronomen von Huzenbach \citep[102–103]{Baur1967}  S. 280

Paradigma 53: Personalpronomen von Saulgau \citep[116–117]{Raichle1932}  S. 280

Paradigma 54: Personalpronomen von Stuttgart \citep[160–161]{Frey1975}  S. 281

Paradigma 55: Personalpronomen von Petrifeld \citep[64–65]{Moser1937}  S. 281

Paradigma 56: Personalpronomen von Elisabethtal \citep[52]{Žirmunskij1928/29}  S. 282

Paradigma 57: Personalpronomen des Kaiserstuhls \citep[393–398]{Noth1993}  S. 282

Paradigma 58: Personalpronomen des Münstertals \citep[46–47]{Mankel1886}  S. 283

Paradigma 59: Personalpronomen von Colmar \citep[81–83]{Henry1900}  S. 283

Paradigma 60: Personalpronomen des Elsass (Ebene) \citep[151–159]{Beyer1963}  S. 284

Paradigma 61: Interrogativpronomen des Althochdeutschen \citep[252–253]{Braune2004}  S. 285

Paradigma 62: Interrogativpronomen des Mittelhochdeutschen \citep[222–223]{Paul2007}  S. 285

Paradigma 63: Interrogativpronomen der deutschen Standardsprache \citep[169–177]{Eisenberg2006}  S. 285

Paradigma 64: Interrogativpronomen von Issime \citep[258–259, 306]{Zürrer1999}  S. 285

Paradigma 65: Interrogativpronomen von Visperterminen \citep[143]{Wipf1911}  S. 285

Paradigma 66: Interrogativpronomen von Jaun \citep[285–286]{Stucki1917}  S. 285

Paradigma 67: Interrogativpronomen des Sensebezirks \citep[201–202]{Henzen1927}  S. 286

Paradigma 68: Interrogativpronomen von Uri \citep[196]{Clauß1929}  S. 286

Paradigma 69: Interrogativpronomen von Vorarlberg \citep[282]{Jutz1925}  S. 286

Paradigma 70: Interrogativpronomen von Zürich \citep[144–145]{Weber1987}  S. 286

Paradigma 71: Interrogativpronomen von Bern \citep[106]{Marti1985}  S. 286

Paradigma 72: Interrogativpronomen von Huzenbach \citep[105]{Baur1967}  S. 286

Paradigma 73: Interrogativpronomen von Saulgau \citep[120]{Raichle1932}  S. 286

Paradigma 74: Interrogativpronomen von Stuttgart \citep[162]{Frey1975}  S. 286

Paradigma 75: Interrogativpronomen von Petrifeld \citep[66–67]{Moser1937}  S. 287

Paradigma 76: Interrogativpronomen von Elisabethtal \citep[53]{Žirmunskij1928/29}  S. 287

Paradigma 77: Interrogativpronomen des Kaiserstuhl \citep[385]{Noth1993}  S. 287

Paradigma 78: Interrogativpronomen des Münstertals \citep[48]{Mankel1886}  S. 287

Paradigma 79: Interrogativpronomen von Colmar \citep[85–86]{Henry1900}  S. 287

Paradigma 80: Interrogativpronomen des Elsass (Ebene) \citep[164–167]{Beyer1963}  S. 287

Paradigma 81: Demonstrativpronomen des Althochdeutschen \citep[247–249]{Braune2004}  S. 288

Paradigma 82: Bestimmter Artikel und Demonstrativpronomen des Mittelhochdeutschen \citep[217–219]{Paul2007}  S. 288

Paradigma 83: Bestimmter Artikel und Demonstrativpronomen der deutschen Standardsprache \citep[169–177]{Eisenberg2006}  S. 288

Paradigma 84: Bestimmter Artikel und Demonstrativpronomen von Issime \citep[4–12, 81]{Perinetto1981}  S. 289

Paradigma 85: Bestimmter Artikel und Demonstrativpronomen von Visperterminen \citep[141]{Wipf1911}  S. 289

Paradigma 86: Bestimmter Artikel und Demonstrativpronomen von Jaun \citep[282–283]{Stucki1917}  S. 289

Paradigma 87: Bestimmter Artikel und Demonstrativpronomen des Sensebezirks \citep[200–201]{Henzen1927}  S. 290

Paradigma 88: Bestimmter Artikel und Demonstrativpronomen von Uri \citep[194–195]{Clauß1929}  S. 290

Paradigma 89: Bestimmter Artikel und Demonstrativpronomen von Vorarlberg \citep[276–279]{Jutz1925}  S. 290

Paradigma 90: Bestimmter Artikel und Demonstrativpronomen von Zürich \citep[101–104, 139–140]{Weber1987}  S. 291

Paradigma 91: Bestimmter Artikel und Demonstrativpronomen von Bern \citep[77–79, 102–103]{Marti1985}  S. 291

Paradigma 92: Bestimmter Artikel und Demonstrativpronomen von Huzenbach \citep[100, 104–105]{Baur1967}  S. 291

Paradigma 93: Bestimmter Artikel und Demonstrativpronomen von Saulgau \citep[115–116, 119–120]{Raichle1932}  S. 292

Paradigma 94: Bestimmter Artikel und Demonstrativpronomen von Stuttgart \citep[154–155]{Frey1975}  S. 292

Paradigma 95: Bestimmter Artikel und Demonstrativpronomen von Petrifeld \citep[63–64, 65]{Moser1937}  S. 292

Paradigma 96: Bestimmter Artikel und Demonstrativpronomen von Elisabethtal \citep[52]{Žirmunskij1928/29}  S. 293

Paradigma 97: Bestimmter Artikel und Demonstrativpronomen des Kaiserstuhls \citep[359–378]{Noth1993}  S. 293

Paradigma 98: Bestimmter Artikel und Demonstrativpronomen des Münstertals \citep[47–48]{Mankel1886}  S. 293

Paradigma 99: Bestimmter Artikel und Demonstrativpronomen von Colmar \citep[68–70, 83]{Henry1900}  S. 294

Paradigma 100: Bestimmter Artikel und Demonstrativpronomen des Elsass (Ebene) \citep[72–78, 84–88]{Beyer1963}  S. 294

Paradigma 101: Possessivpronomen des Althochdeutschen \citep[245–246]{Braune2004}  S. 295

Paradigma 102: Unbestimmter Artikel und Possessivpronomen des Mittelhochdeutschen \citep[216–217, 231]{Paul2007}  S. 295

Paradigma 103: Unbestimmter Artikel und Possessivpronomen der deutschen Standardsprache \citep[169–177]{Eisenberg2006}  S. 296

Paradigma 104: Unbestimmter Artikel und Possessivpronomen von Issime \citep[13–16, 82–84]{Perinetto1981}  S. 297

Paradigma 105: Unbestimmter Artikel und Possessivpronomen von Visperterminen \citep[137, 144]{Wipf1911}  S. 298

Paradigma 106: Unbestimmter Artikel und Possessivpronomen von Jaun \citep[277–278, 284–285]{Stucki1917}  S. 299

Paradigma 107: Unbestimmter Artikel und Possessivpronomen des Sensebezirks \citep[194, 198–199]{Henzen1927}  S. 300

Paradigma 108: Unbestimmter Artikel und Possessivpronomen von Uri \citep[189, 193–194]{Clauß1929}  S. 301

Paradigma 109: Unbestimmter Artikel und Possessivpronomen von Vorarlberg \citep[269–270, 274–276]{Jutz1925}  S. 302

Paradigma 110: Unbestimmter Artikel und Possessivpronomen von Zürich \citep[104–107, 135–139]{Weber1987}  S. 303

Paradigma 111: Unbestimmter Artikel und Possessivpronomen von Bern \citep[79, 98–101]{Marti1985}  S. 304

Paradigma 112: Unbestimmter Artikel und Possessivpronomen von Huzenbach \citep[101, 104]{Baur1967}  S. 305

Paradigma 113: Unbestimmter Artikel und Possessivpronomen von Saulgau \citep[116–119]{Raichle1932}  S. 305

Paradigma 114: Unbestimmter Artikel und Possessivpronomen von Stuttgart \citep[156]{Frey1975}  S. 306

Paradigma 115: Unbestimmter Artikel und Possessivpronomen von Petrifeld \citep[64–66]{Moser1937}  S. 306

Paradigma 116: Unbestimmter Artikel und Possessivpronomen von Elisabethtal \citep[52]{Žirmunskij1928/29}  S. 307

Paradigma 117: Unbestimmter Artikel und Possessivpronomen des Kaiserstuhls \citep[376, 380–384]{Noth1993}  S. 307

Paradigma 118: Unbestimmter Artikel und Possessivpronomen des Münstertals \citep[45–47]{Mankel1886}  S. 308

Paradigma 119: Unbestimmter Artikel und Possessivpronomen von Colmar \citep[70–71, 84–85]{Henry1900}  S. 308

Paradigma 120: Unbestimmter Artikel und Possessivpronomen des Elsass (Ebene) \citep[78–83, 98–109]{Beyer1963}  S. 309


\begin{verbatim}%%move bib entries to  localbibliography.bib
\end{verbatim}