\chapter{Verzeichnis der Tabellen}\label{appendix3}

Tabelle 2.1: Resultate syntaktischer Komplexität, Move/Merge \citep[16]{Garzonio2016}  S. 23

Tabelle 2.2: Arten der Komplexität (\citealt[9]{Rescher1998}; \citealt[23]{Sinnemäki2011})  S. 25

Tabelle 2.3: Einordnung der diskutierten Arbeiten in die Komplexitätsarten nach \citet{Rescher1998} und \citet{Sinnemäki2011}  S. 27

Tabelle 3.1: Die untersuchten alemannischen Dialekte  S. 36

Tabelle 3.2: Die untersuchten alemannischen Dialekte und ihre sprachexternen Eigenschaften  S. 41

Tabelle 4.1: Flexion des Verbs \textit{laudo} \citep[74]{SadlerSpencer2001}   S. 59

Tabelle 4.2: Flexion des Deponens \textit{loquor} \citep[75]{SadlerSpencer2001}   S. 59

Tabelle 4.3: Lexikalische Repräsentation \citep[124]{AckermanStump2004}   S. 64

Tabelle 4.4: Beispiele für Content-Cell, Form-Cell und Realisierung  S. 65

Tabelle 4.5: Flexion des Nomens HṚD in Sanskrit \citep[121]{AckermanStump2004}   S. 67

Tabelle 4.6: Flexion des bestimmten Artikels in Jaun \citep[282]{Stucki1917}  S. 74

Tabelle 4.7: Typen von Synkretismen basierend auf \citet[212–217]{Stump2001}  S. 76

Tabelle 4.8: Präsens Indikativ rumänischer Verben aus (\citealt[214]{Stump2001}, hier gekürzt)  S. 77

Tabelle 4.9: Flexion der lateinischen Verben \textit{fat\=er\=\i} und \textit{mon\=ere}, aus \citep[122]{AckermanStump2004}   S. 79

Tabelle 4.10: C-P, FC und Realisierung des Nominativs und Akkusativs Maskulin und Neutrum  S. 79

Tabelle 4.11: Flexion der Substantive in der deutschen Standardsprache basierend auf \citet[158–167]{Eisenberg2006}  S. 82

Tabelle 4.12: Flexion der starken Adjektive in der deutschen Standardsprache basierend auf \citet[178]{Eisenberg2006}  S. 82

Tabelle 4.13: Nicht-kanonische Phänomene in einem Paradigma (übernommen aus \citealt[95]{Camilleri2012})  S. 92

Tabelle 4.14: Typen an Stammformsystemen \citep[2-4]{FinkelStump2007}   S. 95

Tabelle 5.1: Umlaut im Alemannischen von Zürich (basierend auf \citealt[111–119]{Weber1987}  S. 107

Tabelle 5.2: Wa-/wo-Stämme im Alt- und Mittelhochdeutschen (\citealt[193]{Braune2004}, \citealt[143, 189]{Paul2007})  S. 112

Tabelle 5.3: Dialekte mit \textit{n} (Hiatvermeidung) und Zentralisierung der Mittelsilbe  S. 116

Tabelle 5.4: Dialekte mit \textit{n} (Hiatvermeidung), aber ohne Zentralisierung der Mittelsilbe  S. 116

Tabelle 5.5: Dialekte ohne \textit{n} (Hiatvermeidung)  S. 116

Tabelle 5.6: \textit{N} zur Hiatvermeidung in Jaun (basierend auf \citealt[255–272]{Stucki1917})  S. 117

Tabelle 5.7: \textit{N} (Hiatvermeidung) und Suffix –\textit{ene} in Uri (basierend auf \citealt[173–185]{Clauß1929})  S. 118

Tabelle 5.8: \textit{N} zur Hiatvermeidung in Issime und Visperterminen (basierend auf \citealt[144–205]{Zürrer1999} und \citealt[119–134]{Wipf1911})  S. 122

Tabelle 5.9: \textit{N} zur Hiatvermeidung in Bern und Elisabethtal (basierend auf \citealt[82–90]{Marti1985} und \citealt[50–52]{Žirmunskij1928/29})  S. 122

Tabelle 5.10: \textit{N} zur Hiatvermeidung und Plural des Typs –\textit{ene} in Petrifeld (basierend auf \citealt[59–62]{Moser1937})  S. 123

Tabelle 5.11: \textit{N} zur Hiatvermeidung und Plural des Typs –\textit{ene} in Kaiserstuhl (basierend auf \citealt[359–373]{Noth1993})  S. 123

Tabelle 5.12: Plural des Typs –\textit{ene} in Huzenbach (basierend auf \citealt[92–98]{Baur1967})  S. 124

Tabelle 5.13: Blöcke der Substantivflexion der Varietäten ohne Kasusmarkierung im Plural  S. 125

Tabelle 5.14: Blöcke der Substantivflexion der Varietäten mit Kasusmarkierung im Plural  S. 126

Tabelle 5.15: Starke und schwache Adjektivflexion in Issime anhand des Lexems \textit{naw} ‘neu’ (\citealt[90–97]{Perinetto1981}, \citealt[267–268]{Zürrer1999})  S. 132

Tabelle 5.16: Wa-/wo-Stämme im Althochdeutschen am Beispiel der Lexeme \textit{garo} ‘bereit’ und \textit{blint} ‘blint’ \citep[220, 225]{Braune2004}  S. 133

Tabelle 5.17: Freie Variation in der starken Flexion des Althochdeutschen \citep[220, 223]{Braune2004}  S. 134

Tabelle 5.18: Freie Variation in den untersuchten Varietäten  S. 135

Tabelle 5.19: Betontes Pluralparadigma des Personalpronomens in Issime \citep[206–312]{Zürrer1999}  S. 137

Tabelle 5.20: 3. Person Singular Neutrum belebt und unbelebt in Bern \citep[92–97]{Marti1985}  S. 140

Tabelle 5.21: Herkunft der Akkusativform der 3. Person Singular Neutrum unbelebt  S. 141

Tabelle 5.22: 3. Person Singular Neutrum belebt und unbelebt im Sensebezirk \citep[196–198]{Henzen1927}  S. 141

Tabelle 5.23: Russische Substantivflexion, belebt und unbelebt (gekürztes Paradigma aus \citealt[166]{Corbett1991})  S. 143

Tabelle 5.24: Interrogativpronomen von Jaun \citep[285–286]{Stucki1917}  S. 146

Tabelle 5.25: Bestimmter Artikel und Demonstrativpronomen von Jaun \citep[282–283]{Stucki1917}  S. 147

Tabelle 5.26: Gemeinsame Formen des bestimmten Artikels und des Demonstrativpronomens  S. 150

Tabelle 5.27: Syntaktisch bedingte Variation im bestimmten Artikel  S. 153

Tabelle 5.28: Freie Variation im unbestimmten Artikel und im Possessivpronomen  S. 156

Tabelle 5.29: Zellen des unbestimmten Artikels mit freier Variation  S. 157

Tabelle 5.30: Formen des unbestimmten Artikels bei syntaktisch bedingter Variation im Dativ  S. 159

Tabelle 5.31: Syntaktisch bedingte Variation im bestimmten und unbestimmten Artikel  S. 161

Tabelle 5.32: Präfigierte unbestimmte Artikel  S. 161

Tabelle 5.33: Paradigmen des Possessivpronomens in den Dialekten  S. 163

Tabelle 5.34: Wurzel-/Stammalternationen im Possessivpronomen der Dialekte  S. 164

Tabelle 6.1: Gesamtkomplexität  S. 179

Tabelle 6.2: Komplexität der Interrogativpronomen  S. 180

Tabelle 6.3: Komplexität der Adjektive  S. 180

Tabelle 6.4: Anzahl der Realisierungsregeln und Flexionsklassen der Substantive  S. 181

Tabelle 6.5: Komplexität der Substantive  S. 182

Tabelle 6.6: Komplexität der Personalpronomen  S. 182

Tabelle 6.7: Komplexität des unbestimmten Artikels/Possessivpronomens  S. 183

Tabelle 6.8: Komplexität des bestimmten Artikels/Demonstrativpronomens  S. 183

Tabelle 6.9: Anzahl (nicht) isolierte Dialekte mit diachroner Komplexifizierung und Komplexifizierungswert  S. 185

Tabelle 6.10: Anzahl Dialekte pro Dialektgruppe mit diachroner Komplexifizierung und Komplexifizierungswert  S. 185

Tabelle 6.11: Anzahl diachron komplexifizierter Kategorien  S. 187

Tabelle 6.12: Innovationen und Archaismen in den alemannischen Dialekten  S. 190

Tabelle 6.13: Diachron komplexifizierte Phänomene im Personalpronomen  S. 191

Tabelle 6.14: Diachron komplexifizierte Phänomene im bestimmten Artikel/Demonstrativpronomen  S. 192

Tabelle 6.15: Diachron komplexifizierte Phänomene im unbestimmten Artikel/Possessivpronomen  S. 193

Tabelle 6.16: Diachron komplexifizierte Phänomene im Personalpronomen, bestimmten Artikel/Demonstrativpronomen und unbestimmten Artikel/Possessivpronomen  S. 194

Tabelle 6.17: Gesamtkomplexität (Dialektgruppe)  S. 203

Tabelle 6.18: Durchschnittliche Komplexität pro Dialektgruppe und Kategorie  S. 203

Tabelle 6.19: Stadtdialekt mit geringster Komplexität im Vergleich mit Landdialekten (Schwäbisch und Oberrheinalemannisch)  S. 208

Tabelle 6.20: Stadt- und Landdialekte im Hochalemannischen  S. 208

Tabelle 6.21: Anzahl Pluralmarker pro moderne Varietät  S. 212

Tabelle 6.22: Höhe der Komplexität der Dialekte in allen Kategorien (außer Interrogativpronomen)  S. 219

Tabelle 6.23: Durchschnittliche Komplexität der isolierten und nicht isolierten Dialekte pro Dialektgruppe (Zahl = nicht isoliert komplexer als isoliert)  S. 220

Tabelle 6.24: Durchschnittliche Anzahl Innovationen pro Kategorie, nach Dialektgruppen und isolierten/ nicht isolierten Dialekten  S. 221

Tabelle 6.25 (6.12): Innovationen und Archaismen in den alemannischen Dialekten  S. 232


\begin{verbatim}%%move bib entries to  localbibliography.bib
\end{verbatim}