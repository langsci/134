\chapter{Flexionsparadigmen und Realisierungsregeln}\label{5}

Ziel dieses Kapitels ist, die in \chapref{4} eingeführte Theorie und Messmethode auf die in dieser Arbeit untersuchten Varietäten zu übertragen. Da sich die Phänomene der analysierten Kategorien in den verschiedenen Varietäten wiederholen, wird nicht jedes Paradigma und jedes System an RRs jeder Varietät einzeln besprochen. Vielmehr geht es darum, exemplarisch die Typen an RRs für jede Kategorie und die Typen an Problemen und deren Lösung vorzustellen. Es soll also an Beispielen dargestellt werden, wie die Varietäten dieses Samples mit der in \chapref{4} eingeführten Theorie und Messmethode analysiert wurden. Die vollständigen Paradigmen und RRs der hier untersuchten Varietäten befinden sich im Anhang A bzw. im Anhang B.

Die Typen von Phänomenen, Problemen und Analysen wiederholen sich besonders innerhalb derselben Kategorie. Deshalb ist dieses Kapitel nach Kategorien gegliedert: \isi{Substantive} (\sectref{5.1}), \isi{Adjektive} (\sectref{5.2}), \isi{Personalpronomen} (\sectref{5.3}), \isi{Interrogativpronomen} (\sectref{5.4}), \isi{bestimmter Artikel}/\isi{Demonstrativpronomen} (\sectref{5.5}) und \isi{unbestimmter Artikel}/\isi{Possessivpronomen} (\sectref{5.6}). Es soll hier noch darauf hingewiesen werden, dass am Ende des Abschnitts \sectref{5.1} (\sectref{5.1.6}) anhand eines Beispiels die Systematisierungsarbeit gezeigt wird, d.h., die Analyseschritte von den Angaben aus den Quellen zum Paradigma und vom Paradigma zu den RRs. Dies wird anhand der Substantivflexion von Jaun vorgestellt.

Vorausgeschickt werden hier zwei grundsätzliche Überlegungen, die alle folgenden Kapitel betreffen: Erstens die Definition der Konzepte \isi{Wurzel} und Stamm und zweitens die Trennbarkeit eines Wortes in seine \isi{Wurzel} und \isi{Affixe}.

Erstens soll kurz wiederholt werden, was unter \isi{Wurzel} und Stamm verstanden wird. Dies wurde in \sectref{4.1.3.3} ausführlich erklärt. Dabei sind zwei grundsätzliche Annahmen in Erinnerung zu rufen: a) Nur \isi{Wurzeln} stehen im \isi{Radikon}, welche ausschließlich die Form darstellen und keine Bedeutung tragen; b) Flexion ist ein rein phonologischer Prozess, Flexionsaffixe tragen ebenfalls keine Bedeutung. Durch die Flexion, d.h. durch die RRs, wird ein flektiertes Wort von einer \isi{Wurzel} abgeleitet. Diese Idee wird auf die Stammbildung übertragen, denn auch bei der Stammbildung wird von einer \isi{Wurzel} eine neue Form abgeleitet. Bei der Flexion und Stammbildung handelt es sich also formal um dieselben Prozesse. Beispielsweise kann von der \isi{Wurzel} \textit{kind} der Pluralstamm \textit{kind}-\textit{ər} gebildet werden, an den weiteres Material suffigiert werden kann (z.B. \textit{kind}-\textit{ər}-\textit{n}), aber nicht muss. Des Weiteren ist festzuhalten, dass nicht von jeder \isi{Wurzel} zuerst ein Stamm gebildet werden muss, bevor Flexionsaffixe angehängt werden können. Das Wort \textit{kind}-\textit{əs} entsteht also dadurch, dass die \isi{Realisierungsregel} \mbox{RR \textsubscript{C, \{\textsc{case:gen}, \textsc{num:sg}\}, \textsc{n[}\textsc{ic:}1} \textsubscript{${\veebar}$}\textsubscript{ 2} \textsubscript{${\veebar}$}\textsubscript{ 3} \textsubscript{${\veebar}$}\textsubscript{ 5} \textsubscript{${\veebar}$}\textsubscript{ 9]} ($\langle$X,$\sigma$$\rangle$) = \textsubscript{def} $\langle$X\textit{s}ˊ,$\sigma$ $\rangle$} von der \isi{Wurzel} \textit{kind} das Wort \textit{kind}-\textit{əs} ableitet.

Zweitens können die \isi{Wurzel} und die \isi{Affixe} in den Wörtern der einen Wortarten voneinander getrennt werden, in den Wörtern anderer Wortarten jedoch nicht, wenn man sie synchron analysiert, was in dieser Arbeit gemacht wird. \isi{Substantive}, \isi{Adjektive} und \isi{Possessivpronomen} sind in eine \isi{Wurzel} und \isi{Affixe} dividierbar. Die \isi{Wurzel} steht im \isi{Radikon}, die \isi{Affixe} und \is{Modifikation}Modifikationen des Stammes oder der \isi{Wurzel} werden durch RRs bestimmt. Die Formen der übrigen Wortarten jedoch (\isi{Personalpronomen}, \isi{Interrogativpronomen}, \isi{Demonstrativpronomen}, bestimmter und \isi{unbestimmter Artikel}) können aus synchroner Perspektive nicht weiter in eine \isi{Wurzel} und \isi{Affixe} aufgeteilt werden. Die RRs definieren folglich die gesamte Form. Dies stellt weder ein Problem für das der Messmethode zugrunde liegende Modell noch für die Messmethode selbst dar, da sowohl die \isi{Wurzeln} aus dem \isi{Radikon} als auch die Flexion (d.h. die RRs) nur die Form und nicht die Funktion/Bedeutung eines Wortes definieren. Beim \isi{unbestimmten Artikel} bilden folgende Varietäten eine Ausnahme, in denen eine \isi{Wurzel} von \isi{Suffixen} getrennt werden kann: Mittelhochdeutsch und die deutsche Standardsprache sowie die Dialekte von Visperterminen und Issime. Muss bei einer Wortart einer Varietät die gesamte Form stipuliert werden, wird dies in den einzelnen Kapiteln noch einmal kurz erwähnt.

\section{Substantive}\label{5.1}

\subsection{Flexionsklassen}\label{5.1.1}\label{5.1.1.}

Wie bereits in \sectref{4.3.2} erwähnt wurde, werden die \isi{Flexionsklassen} als eine Art Instruktion gesehen, wie die RRs miteinander kombiniert werden. Das Flexionssystem der \isi{Substantive} kann man sich also als ein Inventar an RRs und als eine Menge an Kombinationen dieser RRs vorstellen. Dabei gilt: Werden z.B. die RRs auf fünf unterschiedliche Weisen kombiniert, weist das System fünf \isi{Flexionsklassen} auf. Nicht nur die RRs tragen folglich zur Komplexität der Flexion bei, sondern auch die \isi{Flexionsklassen}. Denn es kommt vor, wie in \sectref{4.3.2} dargestellt, dass wenn eine Varietät A und eine Varietät B genau dieselbe Anzahl RRs aufweisen, Varietät A diese zehn Mal kombiniert, Varietät B aber nur acht Mal. Die \isi{Flexionsklassen} haben also keine Funktion oder Bedeutung im System, sie stellen lediglich die Anzahl unterschiedlicher Kombinationen der RRs dar.

Es stellt sich nun die Frage, wann eine neue \isi{Flexionsklasse} eröffnet wird. Erstens unterscheiden sich zwei \isi{Flexionsklassen} in mindestens einer RR, egal um welche Flexionsart es sich handelt (\isi{Affix}, Wur\-zel-/Stamm\-al\-ter\-na\-tion etc.). Weist eine \isi{Flexionsklasse} im Gegensatz zu allen anderen \isi{Flexionsklassen} keine overte Markierung auf, wird auch für diesen Fall eine \isi{Flexionsklasse} angenommen. Die \isi{Flexionsklassen} werden also weder nach germanischen Stämmen (z.B. \citealt{Braune2004}) noch nach Deklinationstypen (z.B. \citealt{EisenbergGelhausHenneSittaWellmann1998}) eingeteilt. Ob folglich eine neue \isi{Flexionsklasse} eröffnet wird oder nicht, hängt ausschließlich davon ab, ob diese sich in mindestens einer RR von einer anderen unterscheidet. Beispielsweise ist in der deutschen Standardsprache für die Lexeme, die ihren Plural auf -\textit{ər} bilden (\textit{Kind}{}-\textit{Kindər}, \textit{Wald}{}-\textit{Wäldər}), nur eine \isi{Flexionsklasse} anzusetzen (und nicht zwei). Denn alle Wörter mit einem -\textit{ər} als Pluralsuffix werden umgelautet. Wendet man den \isi{Umlaut} auf \textit{Wald} an, entsteht \textit{Wäld}, wendet man ihn auf \textit{Kind} an, passiert nichts, da \textit{i} nicht umgelautet werden kann. Im Gegensatz dazu sind im Issime Alemannischen zwei \isi{Flexionsklassen} für Wörter anzusetzen, die ihren Plural mit dem \isi{Suffix} -\textit{er} bilden: \textit{l}{\textit{a}}\textit{m}-\textit{l}{\textit{a}}\textit{mmer} ʻLammʼ, \textit{l}{\textit{a}}\textit{n}-\textit{l}{\textit{e}}\textit{nner} ʻLandʼ \citep[164]{Zürrer1999}. Denn nicht alle Wörter mit -\textit{er} als Pluralsuffix lauten den Wurzelvokal um, auch wenn dies möglich wäre. Die beiden \isi{Flexionsklassen} unterscheiden sich also dadurch, dass die eine eine RR für den \isi{Umlaut} hat, die andere nicht. Des Weiteren gibt es auch keine Ober- und Unterklassen, wie dies z.B. bei der Einteilung nach Stämmen üblich ist (z.B. ja- und wa-Stäm\-me als Teil der a-Stäm\-me). Alle \isi{Flexionsklassen} stehen also gleichberechtigt nebeneinander und tragen in gleicher Weise zur Komplexität bei. Deswegen können die \isi{Flexionsklassen} nummeriert werden, ohne dass die Nummer etwas über die \isi{Flexionsklasse} aussagt. Auch andere Symbole wären vorstellbar, wie z.B. die Buchstaben A-Z. Aus praktischen Gründen werden jedoch in dieser Arbeit Zahlen verwendet. Zweitens wird nur dann eine \isi{Flexionsklasse} eröffnet, wenn mindestens zwei Lexeme nach dieser \isi{Flexionsklasse} flektieren. Drittens werden in einer \isi{Flexionsklasse} nur jene Zellen angenommen, die in der Flexion der \isi{Substantive} tatsächlich unterschieden werden. Z.B. macht das Althochdeutsche im Singular und Plural durchaus Kasusunterschiede (Paradigma 1), während im Alemannischen des Sensebezirks sowohl im Singular als auch im Plural alle \isi{Kasus} zusammengefallen sind (Paradigma 7). Da also im Alemannischen des Sensebezirks \isi{Kasus} durch overte Markierung zwar im Pronomen, im \isi{Adjektiv} und in den Determinierern (Paradigma 27, 47, 67, 87, 107), aber nicht im \isi{Substantiv} unterschieden wird, gibt es auch keinen Grund, im Substantivparadigma Zellen für die \isi{Kasus} anzusetzen. Viertens werden Varianten, die phonologisch erklärt werden können, nicht berücksichtigt, da ausschließlich die Komplexität der Flexionsmorphologie gemessen werden soll. Z.B. eröffnen die Varianten -\textit{es}/-\textit{s} des Genitiv Singular in der deutschen Standardsprache keine neuen \isi{Flexionsklassen}, da diese \isi{Variation} phonologisch bedingt ist \citep[224-225]{EisenbergGelhausHenneSittaWellmann1998}.

\subsection{Realisierungsregeln für Affixe}\label{5.1.2}

Dieses Kapitel hat zum Ziel, die RRs zur Suffigierung vorzustellen. Die RRs für die Wur\-zel-/Stamm\-al\-ter\-na\-tio\-nen werden im nachfolgenden Kapitel präsentiert. In den hier untersuchten Varietäten können drei Kategorien durch \isi{Suffixe} kodiert werden: \isi{Numerus}, \isi{Kasus} und Possessiv. Zuerst wird hier also gezeigt, wie die RRs aussehen, die ein \isi{Suffix} für \isi{Numerus} und \isi{Kasus} definieren. Zweitens wird begründet, weshalb in den meisten Varietäten ein Possessiv-S angenommen wird. Schließlich können mehrere \isi{Suffixe} in derselben Zelle des Paradigmas stehen. Drittens wird also gezeigt, welche Art von RRs freie \isi{Variation} adäquat erfasst.

\noindent
{Numerus- und Kasussuffixe}: Bei den \isi{Affixen} der Substantivflexion in den untersuchten Varietäten handelt es sich einzig um \isi{Suffixe}. Diese \isi{Suffixe} drücken sowohl \isi{Kasus} wie auch \isi{Numerus} aus, wie die folgende RR\footnote{Die Notation der RRs wurde in \sectref{4.1.3} eingeführt.} für die deutsche Standardsprache zeigt:

% \ea%20
\begin{exe}
\exr{ex:key:20}
 RR \textsubscript{C}, \textsubscript{\{\textsc{case:dat}, \textsc{num:pl}\}}, \textsubscript{\textsc{n[}\textsc{ic:} 1} \textsubscript{\tiny $\veebar$}\textsubscript{ 2} \textsubscript{\tiny $\veebar$}\textsubscript{ 3} \textsubscript{\tiny $\veebar$}\textsubscript{ 4} \textsubscript{\tiny $\veebar$}\textsubscript{ 5} \textsubscript{\tiny $\veebar$}\textsubscript{ 6} \textsubscript{\tiny $\veebar$}\textsubscript{ 7} \textsubscript{${\veebar}$}\textsubscript{ 8]} ($\langle$X,$\sigma$ $\rangle$) = \textsubscript{def} $\langle$X\textit{ən}ˊ,$\sigma$ $\rangle$\\
\end{exe}

\noindent
{Possessiv-S}: Neben \isi{Numerus} und \isi{Kasus} weisen fast alle Dialekte und die deutsche Standardsprache ein besitzanzeigendes s-\isi{Suffix} auf, das hier Possessiv-S genannt wird und nicht mit dem Genitiv-S verwechselt werden darf. Dass das Possessiv-S angenommen werden muss, hat zwei Gründe. Erstens weisen die Dialekte, in denen es vorkommt, keine Genitivsuffixe auf, d.h., in diesen Dialekten gibt es keinen Genitiv. Zweitens wird das Possessiv-S nur an Eigennamen und Berufsbezeichnungen suffigiert, aber sowohl an Maskulina als auch an Feminina. Deswegen ist auch für die deutsche Standardsprache ein Possessiv-S anzusetzen (z.B. \textit{Annas Schwester}), weil ein Genitiv-S nie an Feminina suffigiert werden kann. Die RR für das Possessiv-S sieht wie folgt aus:

\usecounter{equation}\setcounter{equation}{45}
\ea%46
\label{ex:key:46}
 RR \textsubscript{C,} \textsubscript{\{\textsc{poss:yes}\}}\textsubscript{,} \textsubscript{\textsc{n[}\textsc{proper noun}]} ($\langle$X,$\sigma$ $\rangle$) = \textsubscript{def} $\langle$X\textit{s}ˊ,$\sigma$ $\rangle$\\
\z

Schließlich ist im Alt- und Mittelhochdeutschen \textsc{freie Variation} zu beobachten, und zwar wird eine Zelle durch zwei Formen definiert, wobei die eine aus der \isi{Wurzel} und einem \isi{Suffix} besteht und die andere nur aus der \isi{Wurzel}. Dazu werden zwei RRs (\ref{ex:key:47} + \ref{ex:key:48}) benötigt. Da beide RRs gleich spezifisch sind und in demselben \isi{Block} stehen, definieren beide RRs die Zelle Dativ Singular der \isi{Flexionsklasse} 7 (vgl. Diskussion zur \isi{Variation} und Bedingung \ref{ex:key:29} in \sectref{4.1.3.3}).

\ea%47
\label{ex:key:47}
 RR \textsubscript{C,} \textsubscript{\{\textsc{case:dat}, \textsc{num:sg}\},} \textsubscript{\textsc{n[}\textsc{ic:}7]} ($\langle$X,$\sigma$ $\rangle$) = \textsubscript{def} $\langle$X\textit{e}ˊ,$\sigma$ $\rangle$
\z

\ea%48
\label{ex:key:48}
 RR \textsubscript{C,} \textsubscript{\{\textsc{case:dat,} \textsc{num:sg}\}}, \textsubscript{\textsc{n[}\textsc{ic:}7]} ($\langle$X,$\sigma$ $\rangle$) = \textsubscript{def} $\langle$Xˊ,$\sigma$ $\rangle$
\z

Das Beispiel dazu stammt ebenfalls aus dem Althochdeutschen. Der Dativ Singular der \isi{Flexionsklasse} 7 weist sowohl \isi{Wurzel} + \isi{Suffix} -\textit{e} \REF{ex:key:47} als auch nur \isi{Wurzel} \REF{ex:key:48} auf, z.B. \textit{fater} und \textit{fater-e} \citep[214]{Braune2004}. Auf den ersten Blick sieht die RR \REF{ex:key:48} wie die RR \textit{Identity Function Default} aus (vgl. \sectref{4.1.3.2}, die hier wiederholt wird:

% \ea%26
\begin{exe}
\exr{ex:key:26}
\isi{Identity Function Default} […]\\
RR\textsubscript{n,\{\},U} ($\langle$X,$\sigma $$\rangle$) = \textsubscript{def} ($\langle$X,$\sigma$ $\rangle$). \citep[53]{Stump2001}
\end{exe}      

Die RR \textit{Identity Function Default} definiert, dass, wenn in einem \isi{Block} für ein bestimmtes Set an morphosyntaktischen Eigenschaften keine RR gefunden wird, mit der \isi{Wurzel} nichts passiert. Die RR \REF{ex:key:48} dagegen besagt, dass die Form für den Dativ Singular der \isi{Flexionsklasse} 7 der Form der \isi{Wurzel} entspricht. In \sectref{4.1.3} wurde gezeigt, dass die RRs in einem \isi{Block} miteinander in Konkurrenz stehen und dass immer jene Regel eine bestimmte Zelle definiert, die am spezifischsten für diese Zelle ist. Ist eine Zelle mit einer Form (durch eine RR) gefüllt, ist diese Zelle für weitere potentielle RRs aus demselben \isi{Block} blockiert. Sind jedoch zwei oder mehrere RRs für eine bestimmte Zelle gleich spezifisch, finden beide RRs Anwendung und folglich definieren zwei Formen diese Zelle. Stehen also in einer Zelle zwei Formen, eine bestehend aus \isi{Wurzel}+\isi{Suffix} und eine nur aus der \isi{Wurzel}, muss die RR für die \isi{Wurzel} gleich spezifisch sein wie jene für die Form \isi{Wurzel}+\isi{Suffix}. Da nun die RR \textit{Identity Function Default} \REF{ex:key:26} weniger spezifisch ist als die RR \REF{ex:key:47}, blockiert die RR \REF{ex:key:47} die RR \REF{ex:key:26}. Weil aber sowohl \textit{fatere} wie auch \textit{fater} in der Zelle Dativ Singular stehen, muss eine RR für \textit{fater} angenommen werden, die genauso spezifisch ist wie die RR für \textit{fatere}, was durch die RRs \REF{ex:key:47} und \REF{ex:key:48} gewährleistet ist.

Wichtig ist hier also festzuhalten, dass die Formen, die in freier \isi{Variation} stehen, durch gleich komplexe RRs repräsentiert sind, und zwar unabhängig davon, ob z.B. zwei \isi{Suffixe} in derselben Zelle stehen oder ein \isi{Suffix} und eine \isi{Wurzel}. Würden wir eine freie \isi{Variation} (z.B. \isi{Suffix}/\isi{Suffix}) als komplexer ansehen als eine andere freie \isi{Variation} (z.B. \isi{Suffix}/\isi{Wurzel}), wäre zu definieren, um wie viel die eine mehr oder weniger komplex ist. Das heißt, man würde von einer theoriegeleiteten Komplexitätsmessung, die nicht absolut, aber maximal möglich objektiv ist, zu einer eher von der Intuition geleiteten Komplexitätsmessung übergehen. Des Weiteren wird hier jede untersuchte Varietät ausschließlich streng synchron analysiert, was für die freie \isi{Variation} folgende Konsequenz hat. Zu einem bestimmten Zeitpunkt kann es freie \isi{Variation} geben und ob diese freie \isi{Variation} sich stabilisiert oder ob die eine oder andere Form sich durchsetzt, kann synchron nicht eruiert werden, auch wenn wir aus diachronen Daten das Wissen dazu haben. Synchron sind zwei (oder mehr) Formen in derselben Zelle des Paradigmas, die beide durch RRs definiert werden müssen. Auf das Beispiel aus dem Althochdeutschen bezogen, bedeutet das also Folgendes. Wir wissen, dass die alte Form \textit{fater} lautet, die neue Form \textit{fatere} \citep[214]{Braune2004}. Es handelt sich diachron also um einen Aufbau an Komplexität: Im ältesten Althochdeutsch gibt es keine RR für die Zelle Dativ Singular, da die Defaultform \textit{fater} verwendet wird; im jüngeren Althochdeutsch gibt es hingegen eine RR für die Zelle Dativ Singular, da die suffigierte Form \textit{fater}-\textit{e} verwendet wird. Für den Übergang zwischen diesen beiden Stadien sind, wie oben dargestellt, zwei RRs anzusetzen. Genau dasselbe gilt für den umgekehrten Fall, wenn ein Wandel von einer suffigieren Form (eine RR) zu einer Form ohne Markierung (keine RR) vorliegt: Es gibt einen Zeitpunkt, in dem beide Formen grammatisch sind, weshalb auch zwei RRs angenommen werden müssen, um die Formen in der Zelle zu definieren. Freie \isi{Variation} führt folglich immer zu einer höheren Komplexität, unabhängig davon, ob das Endprodukt des Wandels von einer Form A zu einer Form B höher oder niedriger in seiner Komplexität ist. Denn synchron stehen zwei (oder mehr) Formen in einer Zelle des Paradigmas, welche durch RRs definiert werden müssen.

\subsection{Realisierungsregeln für Wurzel-/Stammalternationen}\label{5.1.3}

In diesem Kapitel werden die RRs zu den Wur\-zel-/Stamm\-al\-ter\-na\-tio\-nen vorgestellt. Die in diesem Sample vorkommenden Wur\-zel-/Stamm\-al\-ter\-na\-tio\-nen können drei Typen zugeordnet werden: \is{Modifikation}Modifikation eines Vokals oder eines Konsonanten in der \isi{Wurzel}, Wurzelerweiterung (woraus Stämme entstehen) und \is{Subtraktion}Subtraktion. Den Defaultstamm bildet die \isi{Wurzel}, wobei es sich um die Form des Nominativs Singular handelt.

Zu den \is{Modifikation}\textsc{Modifikationen} der \isi{Wurzel} gehören drei Phänomene: \isi{Umlaut}, Diphthongierung und Velarisierung. In allen hier untersuchten Varietäten wird der \isi{Umlaut} zur Pluralmarkierung verwendet. In \sectref{4.1.3.2} wurde die RR für den \isi{Umlaut} eingeführt, die in \REF{ex:key:12} wiederholt ist:

% \ea%12
\begin{exe}
\exr{ex:key:12}
 RR \textsubscript{A,} \textsubscript{\{\textsc{num:pl}\},} \textsubscript{\textsc{n[}\textsc{ic:} 1} \textsubscript{\tiny $\veebar$}\textsubscript{ 3} \textsubscript{\tiny $\veebar$}\textsubscript{ 7} \textsubscript{\tiny $\veebar$}\textsubscript{ 8]} ($\langle$X,$\sigma$ $\rangle$) = \textsubscript{def} $\langle$Ẍˊ,$\sigma$ $\rangle$
\end{exe}

Dies ist die Regel für die deutsche Standardsprache, in der jeder Wurzelvokal genau einen \isi{Umlaut} hat. In vielen untersuchten Dialekten\footnote{Dies gilt für das Alemannische des Sensebezirks, von Uri, Zürich, Bern, Saulgau, Petrifeld, Elisabethtal, des Münstertals und des Elsass (Ebene).} wird jedoch ein Primär- und ein Sekundärumlaut zur Pluralmarkierung unterschieden, wobei der Primärumlaut nur vom Wurzelvokal \textit{a} gebildet wird. Das Alemannische von Zürich weist zum Primär- und Sekundärumlaut noch einen zweiten Sekundärumlaut auf (vgl. \tabref{table5.1}). Zum Wurzelvokal \textit{a} lautet der Primärumlaut \textit{e}, der erste Sekundärumlaut \textit{æ}, der zweite Sekundärumlaut \textit{ö}. Der zweite Sekundärumlaut kommt im Gegensatz zu den anderen nur in Langvokalen vor. Trotzdem muss auch dieser definiert werden, da lange Wurzelvokale auch den Primär- oder ersten Sekundärumlaut zeigen.

%{\tabref{table5.1}: \isi{Umlaut} im Alemannischen von Zürich (basierend auf \citealt[111-119]{Weber1987})}

\begin{table}
\caption{Umlaut im Alemannischen von Zürich (basierend auf \citealt[111-119]{Weber1987})}\label{table5.1}
\begin{tabular}{lll}
\lsptoprule
{Singular} & {Plural} & {Umlaut}\\
\midrule
gascht ‘Gast’ & gescht & Primärumlaut\\
schl\=ag ‘Schlag’ & schl\=eg & Primärumlaut\\
bank ‘Bank’ & bænk & Sekundärumlaut 1\\
rom\=an ‘Roman’ & romǣn & Sekundärumlaut 1\\
sal\=at ‘Salat’ & salȫt & Sekundärumlaut 2\\
\lspbottomrule
\end{tabular}
\end{table}

Da Primär- und Sekundärumlaut zur Pluralmarkierung synchron nicht mehr phonologisch erklärt werden können, ist die \isi{Variation} in der Morphologie zu verorten und folglich durch RRs auszudrücken. Für das Alemannische von Zürich werden also drei RRs für den \isi{Umlaut} benötigt:

\ea%49
\label{ex:key:49}
 RR \textsubscript{A,} \textsubscript{\{\textsc{num:pl}\},} \textsubscript{\textsc{n[}\textsc{ic:} 2} \textsubscript{\tiny $\veebar$}\textsubscript{ 5} \textsubscript{\tiny $\veebar$}\textsubscript{ 6]} ($\langle$X,$\sigma$ $\rangle$) = \textsubscript{def} $\langle$Ẍˊ,$\sigma$ $\rangle$
\z

\ea%50
\label{ex:key:50}
 RR \textsubscript{A,} \textsubscript{\{\textsc{num:pl}\},} \textsubscript{\textsc{n[}\textsc{ic:} 1} \textsubscript{\tiny $\veebar$}\textsubscript{ 4]} ($\langle$X,$\sigma$ $\rangle$) = \textsubscript{def} $\langle$Ẍ[\textit{a} $\rightarrow$ \textit{e}]ˊ,$\sigma$ $\rangle$
\z

\ea%51
\label{ex:key:51}
 RR \textsubscript{A,} \textsubscript{\{\textsc{num:pl}\},} \textsubscript{\textsc{n[}\textsc{ic:} 3]} ($\langle$X,$\sigma$ $\rangle$) = \textsubscript{def} $\langle$Ẍ[\textit{\=a} $\rightarrow$ \textit{ȫ}]ˊ,$\sigma$ $\rangle$\\
\z

Die RR \REF{ex:key:49} bildet den ersten Sekundärumlaut, den Default-\isi{Umlaut}, nach dem auch alle anderen Wurzelvokale umgelautet werden. Lautet der Wurzelvokal \textit{a}, werden noch zwei spezifischere Regeln gebraucht, nämlich eine für den Primärumlaut \REF{ex:key:50} und eine für den zweiten Sekundärumlaut \REF{ex:key:51}.

Ist einer bestimmten \isi{Flexionsklasse} eine RR zugeordnet, die einen \isi{Umlaut} bildet, werden alle Wörter dieser \isi{Flexionsklasse} umgelautet, wenn dies möglich ist. In der deutschen Standardsprache z.B. stehen alle Lexeme, die einen Plural auf -\textit{ər} bilden und umgelautet werden, in derselben \isi{Flexionsklasse} (\textit{Wald}-\textit{Wäldər}, \textit{Kind}-\textit{Kindər}). Wörter wie \textit{Kind} haben keine eigene \isi{Flexionsklasse}, denn Wörter, die den Plural auf -\textit{ər} bilden, werden immer umgelautet. Wird der \isi{Umlaut} auf \textit{Kind} angewendet, passiert mit dem Wurzelvokal nichts, denn \textit{i} kann nicht umgelautet werden. Im Gegensatz dazu sind z.B. im Alemannischen von Issime zwei \isi{Flexionsklassen} für den Plural auf -\textit{er} nötig, denn nicht alle Wörter, deren Wurzelvokal umlautbar wäre, werden auch umgelautet: \textit{lam}-\textit{lammer} ‘Lamm’, \textit{lan}-\textit{lenner} ‘Land, Dorf’ \citep[164]{Zürrer1999}. Deswegen sind für die Substantivflexion von Issime zwei \isi{Flexionsklassen} für den Plural auf -\textit{er} anzusetzen (vgl. \isi{Flexionsklassen} 10 und 11 in Paradigma 4).  

Schließlich ist noch zu entscheiden, ob es sich im Althochdeutschen um einen phonologisch oder morphologisch bedingten \isi{Umlaut} handelt. Dies betrifft die \isi{Flexionsklassen} 3 und 14 (\textit{gast}/\textit{gesti} und \textit{anst}/\textit{ensti}; i-Stäm\-me) sowie die \isi{Flexionsklasse} 9 (\textit{lamb}/\textit{lembir}; iz-/az-Stämme) (vgl. Paradigma 1 im Anhang A). Es wird hier davon ausgegangen, dass der \isi{Umlaut} sowohl phonologisch als auch bereits morphologisch bedingt ist. In den \isi{Flexionsklassen} 3 und 14 zeigen alle Formen im Plural einen \isi{Umlaut}, also nicht nur jene, die in der auf die \isi{Wurzel} folgenden Silbe ein \textit{i} aufweisen (z.B. \textit{enst}-\textit{i} Nominativ/Akkusativ Plural, \textit{enst}-\textit{in} Dativ Plural), sondern auch jene Formen, die kein nachfolgendes \textit{i} haben (z.B. \textit{enst}-\textit{o} Genitiv Plural). Ursprünglich stand auch im Genitiv Plural ein \textit{i} (\textit{enst}-\textit{io}), welches aber nur noch in der frühesten Phase des Althochdeutschen belegt ist \citep[201]{Braune2004}. Es empfiehlt sich hier jedoch nicht, Belege aus der frühesten Phase zu berücksichtigen, da es nur äußerst wenige Belege gibt und diese geringe Menge sich folglich für eine Gesamtanalyse nicht anbietet. Der \isi{Umlaut} im Genitiv Plural (\textit{enst}-\textit{o}) muss synchron also morphologisch bedingt sein. Zwei Analysen sind folglich möglich: a) Der \isi{Umlaut} markiert den Genitiv Plural, b) der \isi{Umlaut} markiert den gesamten Plural. Da Letzteres klar wahrscheinlicher ist als Ersteres (der \isi{Umlaut} setzt sich mehr und mehr als Pluralmarker durch), wird für den gesamten Plural ein morphologischer \isi{Umlaut} angenommen. Anders sieht dies im Singular der \isi{Flexionsklasse} 14 aus, für die von einem phonologisch bedingten \isi{Umlaut} ausgegangen wird: \textit{anst} (Nominativ/Akkusativ Singular), \textit{enst}-\textit{i} (Dativ/Genitiv Singular). Da im Singular nur dann ein \isi{Umlaut} auftritt, wenn in der nachfolgenden Silbe ein \textit{i} steht, und dieser auch später nicht morphologisiert wird, kann angenommen werden, dass es sich dabei auch im Althochdeutschen ausschließlich um einen phonologisch bedingten \isi{Umlaut} handelt. Folglich ist für das Althochdeutsche eine RR für den Pluralumlaut anzusetzen (morphologisch bedingt), nicht jedoch für den \isi{Umlaut} im Singular (phonologisch bedingt).

Damit sind die Ausführungen zum \isi{Umlaut} abgeschlossen. Zum Thema \is{Modifikation}Modifikation gehören noch die Diphthongierung und die Velarisierung, welche im Folgenden beschrieben werden.

Das Alemannische des Münstertals markiert den Plural u.a. durch Diphthongierung (vgl. Paradigma 18). Dies trifft nur auf nasalierte, lange Wurzelvokale zu: \textit{p\~{\=a}t} ‘Band’(Singular), \textit{pain} (Plural) \citep[43]{Mankel1886}. Der Diphthong wird durch folgende RR gebildet (die Subtraktion von \textit{t} wird weiter unten diskutiert, RR \REF{ex:key:59}), wobei < { ̑}> für Diphthongierung steht:

\ea%52
\label{ex:key:52}
 RR \textsubscript{A,} \textsubscript{\{\textsc{num:pl}\},} \textsubscript{\textsc{n[}\textsc{ic:} 7]} ($\langle$X,$\sigma$ $\rangle$) = \textsubscript{def} $\langle$X̑ˊ,$\sigma$ $\rangle$\\
\z

Schließlich gehört zu den \is{Modifikation}Modifikationen noch die Velarisierung. Im Alemannischen des Elsass (Ebene) wird in den \isi{Flexionsklassen} 6 (ohne \isi{Umlaut}) und 7 (mit \isi{Umlaut}) der auslautende Konsonant velarisiert\footnote{\citet{Beyer1963} spricht von Palatalisierung \citep[63]{Beyer1963}. Es ist jedoch anzunehmen, dass es sich dabei eher um velare Nasale handelt.} (vgl. Paradigma 20): \textit{hund}-\textit{hung} ‘Hund’, \textit{hand}-\textit{hæng} ‘Hand’ \citep[63]{Beyer1963}. Dies wird durch folgende RR ausgedrückt:

\ea%53
\label{ex:key:53}
 RR \textsubscript{B,} \textsubscript{\{\textsc{num:pl}\},} \textsubscript{\textsc{n[}\textsc{ic:} 6} \textsubscript{\tiny $\veebar$}\textsubscript{ 7]} ($\langle$X,$\sigma$ $\rangle$) = \textsubscript{def} $\langle$X *\textit{nd} $\rightarrow$ \textit{ŋ}ˊ,$\sigma$ $\rangle$ \\
\z

Der zweite Fall von Wur\-zel-/Stamm\-al\-ter\-na\-tion ist die \textsc{Wurzelerweiterung}. Es handelt sich hier zwar um \isi{Suffixe}. Da diese \isi{Suffixe} jedoch die \isi{Wurzel} erweitern (z.B. Pluralstamm), an die weitere \isi{Suffixe} angehängt werden können (z.B. Kasussuffixe), wird dieser Prozess Wurzelweiterung genannt und in diesem Kapitel zu den Wur\-zel-/Stamm\-mo\-di\-fi\-ka\-tio\-nen behandelt.

Die \isi{Flexionsklasse} 3 von Visperterminen weist einen Singular- und einen Pluralstamm auf, wobei der Pluralstamm durch die Suffigierung mit -\textit{m} entsteht: \textit{ar-o} ‘Arm’ (Nominativ Singular), \textit{arm-a} (Nominativ Plural) \citep[122]{Wipf1911}. Wir können hier von einer Wurzelerweiterung und einem Pluralstamm sprechen, da die \isi{Flexionsklasse} 3 im Plural dieselben Kasussuffixe aufweist wie die \isi{Flexionsklasse} 1: \textit{tag}-\textit{a}, \textit{ar}-\textit{m}-\textit{a} (Nominativ/Akkusativ Plural), \textit{tag}-\textit{u}, \textit{ar}-\textit{m}-\textit{u} (Dativ Plural), \textit{tag}-\textit{o}, \textit{ar}-\textit{m}-\textit{o} (Genitiv Plural). In der \isi{Flexionsklasse} 9 wird im Plural ein \textit{n} suffigiert, jedoch nur im Dativ Plural: \textit{hor-u} ‘Horn’ (Nominativ Plural), \textit{horn-u} (Dativ Plural) \citep[130]{Wipf1911}. \textit{Arm}- wird durch die RR \REF{ex:key:54} definiert, \textit{horn}- durch die RR \REF{ex:key:55}.

\ea%54
\label{ex:key:54}
 RR \textsubscript{A,} \textsubscript{\{\textsc{num:pl}\},} \textsubscript{\textsc{n[}\textsc{ic:} 3]} ($\langle$X,$\sigma$ $\rangle$) = \textsubscript{def} $\langle$X\textit{m}ˊ,$\sigma$ $\rangle$
\z

\ea%55
\label{ex:key:55}
 RR \textsubscript{A,} \textsubscript{\{\textsc{case:dat}, \textsc{num:pl}\},} \textsubscript{\textsc{n[}\textsc{ic:} 9]} ($\langle$X,$\sigma$ $\rangle$) = \textsubscript{def} $\langle$X\textit{n}ˊ,$\sigma$ $\rangle$
\z

Die RR \REF{ex:key:54} stellt also den Pluralstamm zur Verfügung (\textit{arm}-), an den weitere Kasusmarker suffigiert werden können (z.B. Nominativ Plural \textit{arm}-\textit{a}, Dativ Plural \textit{arm}-\textit{u}, vgl. Paradigma 5, \citealt[120]{Wipf1911}). Identisch sehen alle RRs für das Pluralsuffix aus, egal ob danach noch weitere Kasussuffixe angehängt werden oder nicht. Ob nach dem Pluralsuffix noch weitere Kasussuffixe folgen, ergibt sich automatisch daraus, ob nach dem \isi{Block} für das Pluralsuffix noch weitere \isi{Blöcke} für Kasussuffixe kommen. Einen etwas komplexeren Fall stellt das \textit{n} dar: Es kann sowohl ein Pluralsuffix sein (z.B. in Issime \textit{uav}-\textit{n}-\textit{a} ‘Ofen’, \citealt[164]{Zürrer1999}) als auch eingefügt werden, um einen Hiatus zu vermeiden (z.B. \textit{chötti}-\textit{n}-\textit{i} ‘Kette’, \citealt[164]{Zürrer1999}). In welchen Varietäten und \isi{Flexionsklassen} was zutrifft, wird in \sectref{5.1.4} dargestellt.

Im Althochdeutschen variiert der Stamm der Diminutiva, die drei Stämme aufweisen: a) Der Default-Stamm (=\isi{Wurzel}) endet auf ein \textit{\=i} (Nominativ und Akkusativ Singular, \textit{chindil\=i} ‘Kind’); b) folgt ein Monophthong, wird ein \textit{n} eingefügt (\textit{chindil\=in}-); c) folgt ein Diphthong, wird \textit{\=i} getilgt (\textit{chindil}-) (vgl. Paradigma 1; \citealt[187]{Braune2004}). Die Tilgung wird weiter unten diskutiert. Der n-Einschub tritt in den \isi{Kasus} Dativ und Genitiv Singular und Plural auf, in denen ein \isi{Suffix} an die \isi{Wurzel} angehängt wird, das mit einem Monophthong anlautet. Dadurch entsteht die Abfolge \textit{\=i}+Monophthong. Es könnte also angenommen werden, dass der n-Einschub der Hiattilgung dient. Erstens werden jedoch üblicherweise im Althochdeutschen \textit{h}, \textit{j} oder \textit{w} zur Hiattilgung verwendet \citep[49]{Armborst1979}. Zweitens ist diachron im Alemannischen \textit{n} im Auslaut weggefallen (\textit{chindil\=i}), im Fränkischen jedoch erhalten (\textit{chindil\=in}) \citep[187]{Braune2004}. Im Alemannischen des 9. Jh. ist diese \isi{Variation} also in der Morphologie zu verorten:

\ea%56
\label{ex:key:56}
 RR \textsubscript{D,} \textsubscript{\{\textsc{case:dat}} \textsubscript{\tiny $\veebar$}\textsubscript{ \GEN\},} \textsubscript{\textsc{n[}\textsc{ic:} 19]} ($\langle$X,$\sigma$ $\rangle$) = \textsubscript{def} $\langle$X\textit{n/}V\_Vˊ,$\sigma$ $\rangle$\\
\z

Die RR \REF{ex:key:56} definiert: Füge im Dativ und Genitiv Singular und Plural der \isi{Flexionsklasse} 19 intervokalisch ein \textit{n} ein. Dies zeigt, dass die Kontextbedingungen für den n-Einschub erst durch die Suffigierung entstehen. Diese RR ist folglich erst in \isi{Block} D anzusetzen, also nach den \isi{Blöcken} B und C, die RRs für Numerus- und Kasussuffixe enthalten (Diskussion zur Abfolge von RRs und \isi{Blöcken} vgl. \sectref{4.1.3.3}).

Drittens sind noch die \is{Subtraktion}\textsc{Subtraktionen} zu behandeln. Die \is{Subtraktion}Subtraktion betrifft sowohl Vokale als auch Konsonanten: \is{Subtraktion}Subtraktion der auslautenden Vokale im Plural des Althochdeutschen und in etlichen Dialekten, \is{Subtraktion}\textit{t}{}-Subtraktion im Alemannischen von Münstertal und die wa-/w\=o-Stäm\-me im Alt- und Mittelhochdeutschen.

Es wurde bereits dargestellt, dass die Diminutiva des Althochdeutschen einen Default-Stamm (=\isi{Wurzel}) auf \textit{\=i} aufweisen (z.B. \textit{chindil\=i}). Dieses \textit{\=i} wird getilgt, wenn ein Diphthong folgt, was im Nominativ und Akkusativ Plural geschieht (z.B. \textit{chindil\=i}-\textit{iu} $\rightarrow$ \textit{chindil}-\textit{iu}). Da auch hier der Kontext zur Tilgung erst gegeben ist, nachdem suffigiert wurde, steht diese RR in \isi{Block} D:

\ea%57
\label{ex:key:57}
 RR \textsubscript{D,} \textsubscript{\{\textsc{case:nom}} \textsubscript{\tiny $\veebar$}\textsubscript{ \textsc{acc}, \textsc{num:pl}\},} \textsubscript{\textsc{n[}\textsc{ic:} 19]} ($\langle$X,$\sigma$ $\rangle$) = \textsubscript{def} $\langle$X \textit{*\=\i} $\rightarrow$ ø\textit{/}\_VVˊ,$\sigma$ $\rangle$\\
\z

Ein ähnlicher Fall der Vokalsubtraktion kommt in vielen alemannischen Dialekten vor. In diesen Dialekten ist im Plural ein wiederkehrendes Muster zu beobachten: An eine \isi{Wurzel}, die auf einen Vokal auslautet, wird ein vokalisches Pluralsuffix angehängt. In solchen Fällen ist in den alemannischen Dialekten zu erwarten, dass ein \textit{n} zur Hiattilgung eingefügt wird, was nicht nur innerhalb eines Wortes, sondern auch über die Wortgrenze hinweg üblich ist, wie dieses Beispiel zeigt:

\ea%58
    \label{ex:key:58}
\gll \textit{W\=are} \hspace{2em} \textit{n} \hspace{2em} \textit{er}\\
wart {} Hiattustilgung {} ihr\\

\citep[321]{Noth1993}
\z

Die Default-Strategie der Phonologie zur Hiat-Tilgung ist also der n-Einschub. Dasselbe wäre also zu erwarten, wenn einer vokalisch auslautenden \isi{Wurzel} ein vokalisches \isi{Suffix} angehängt wird. Im Alemannischen von Huzenbach z.B. enden die Diminutiva auf -\textit{le}, denen ein -\textit{ə} im Plural suffigiert wird. Der Plural lautet aber nicht *\textit{heislenə}, sondern \textit{heislə} ‘Häuschen’ \citep[98]{Baur1967}. Der auslautende Vokal der \isi{Wurzel} muss also im Plural getilgt werden, wenn ein -\textit{ə} folgt. Dies wird durch folgende RR ausgedrückt, die bereits in \sectref{4.1.3.3} eingeführt wurde:

% \ea%44
\begin{exe}
\exr{ex:key:44}
 RR \textsubscript{C, \{\textsc{num:pl}\}, \textsc{n[}\textsc{ic:} 4} \textsubscript{\tiny $\veebar$} \textsubscript{5]} ($\langle$X,$\sigma$ $\rangle$) = \textsubscript{def} $\langle$X *\textit{e} $\rightarrow$ ø/\_əˊ,$\sigma$ $\rangle$
\end{exe}

Auch hier entstehen die Kontextbedingungen erst durch die Suffigierung, weshalb die RR für die Wur\-zel-/Stamm\-al\-ter\-na\-tion in \isi{Block} C steht und die RR der Suffigierung in \isi{Block} B. RRs von diesem Typ gibt es in den folgenden alemannischen Dialekten: Elisabethtal, Kaiserstuhl, Saulgau, Huzenbach, Sensebezirk, Stuttgart, Uri und Petrifeld. Viele alemannische Dialekte weisen diese Wurzelalternation auch im Zusammenhang mit dem Pluralsuffixen des Typs -\textit{ənə} auf, welche im nächsten Kapitel erörtert werden.

Im Gegensatz zu vielen alemannischen Dialekten hat die deutsche Standardsprache keine hiatvermeidende n-Epenthese, noch braucht sie eine RR wie in \REF{ex:key:44}. Auf den ersten Blick suggerieren aber Wörter, deren Singular auf \textit{o} oder \textit{a} auslautet, etwas anderes: \textit{Konto}, \textit{Pizza} (Nominativ Singular), \textit{Kont}-\textit{ən}, \textit{Pizz}-\textit{ən} (Nominativ Plural). Die Tilgung von \textit{o} im Plural wird jedoch von der Phonologie verursacht. Wird an \textit{Konto} die Endung -\textit{ən} suffigiert, können prinzipiell folgende Möglichkeiten entstehen: a) [kon.'to.ən], b) ['kon.to.ən]. In der deutschen Standardsprache wird in einem Simplex die Pänultima akzentuiert, sofern sie „einen akzentuierfähigen Vokal - also nicht /ə/ - enthält, der auch nicht im \isi{Hiat} mit einem folgenden Vokal steht. In diesen letzteren Fällen trifft der Akzent auf die Antepänultima“ \citep[186]{Kohler1995}. Es ist also zuerst von b) mit dem Akzent auf der Antepänultima auszugehen. Des Weiteren wirkt die Regel, dass \isi{Substantive} im Plural (außer -\textit{s}) auf einen zweisilbigen Fuß enden \citep[ 61-62, 106-109]{Wiese1996}: „[…] [W]ith the exception of nouns taking +\textit{s} as the plural marker, nouns in the plural are such that the last syllable must be a schwa syllable, while the preceding syllable is stressed“ \citep[61]{Wiese1996}. Wie beispielsweise *\textit{Schwestərən} kein zulässiger Plural ist, sondern \textit{Schwestərn}, muss auch in *\textit{Kontoən} eine Silbe gekürzt werden. Diese Analyse wirft viele Fragen auf, z.B. weshalb \textit{o} und nicht \textit{ə} getilgt wird, wozu keine detaillierte Analyse gefunden werden konnte. Relevant für diese Arbeit ist jedoch nur, ob es sich dabei um einen morphologischen oder phonologischen Mechanismus handelt. Wie ausgeführt wurde, gibt es klare Indizien dafür, dass die Tilgung phonologisch bzw. phonotaktisch bedingt ist, weshalb dafür keine RR angenommen werden muss.

In einem alemannischen Dialekt ist auch Konsonantensubtraktion zu beobachten. Im Alemannischen des Münstertals bilden Wörter mit einem langen, nasalierten Wurzelvokal den Plural durch Diphthongierung des Wurzelvokals (RR (\ref{ex:key:52})): \textit{p\~{\=a}t} ‘Band’ (Singular), \textit{pain} (Plural) \citep[43]{Mankel1886}. Zusätzlich wird der Plural durch die \is{Subtraktion}Subtraktion des auslautenden \textit{t} markiert. Folgende RR definiert, dass im Plural der \isi{Flexionsklasse} 7 das auslautende \textit{t} getilgt wird:

\ea%59
\label{ex:key:59}
 RR \textsubscript{C,} \textsubscript{\{\textsc{num:pl}\},} \textsubscript{\textsc{n[}\textsc{ic:} 7]} ($\langle$X,$\sigma$ $\rangle$) = \textsubscript{def} $\langle$X *\textit{t} $\rightarrow$ ø/\_\#ˊ,$\sigma$ $\rangle$\\
\z

Schließlich sind in diesem Kontext noch die alt- und mittelhochdeutschen wa-/w\=o-Stäm\-me zu diskutieren (vgl. \tabref{table5.2}). In beiden Varietäten weisen die wa-/w\=o-Stäm\-me dieselben \isi{Suffixe} wie die a-Stäm\-me auf, folglich ist für beide Stämme nur eine \isi{Flexionsklasse} anzunehmen. Der einzige Unterschied in beiden Varietäten besteht darin, dass die \isi{Wurzel} auf \textit{w} endet, wenn ein \isi{Suffix} folgt. Im Auslaut wird das \textit{w} im Althochdeutschen vokalisiert, im Mittelhochdeutschen getilgt. Die \isi{Variation} kann im Althochdeutschen also phonologisch erklärt werden, im Mittelhochdeutschen jedoch nicht. Deshalb braucht es für das Mittelhochdeutsche eine RR, die diese Tilgung durchführt. Im Althochdeutschen steuert die Phonologie diese \isi{Variation}, folglich ist keine RR nötig. In der Folge soll nun genauer darauf eingegangen werden, weshalb die \isi{Variation} im Althochdeutschen phonologisch und im Mittelhochdeutschen morphologisch bedingt ist.

%{\tabref{table5.2}: Wa-/w\=o-Stäm\-me im Alt- und Mittelhochdeutschen (\citealt[193]{Braune2004}, \citealt[143, 189]{Paul2007}}\\

\begin{table}
\caption{Wa-/w\=o-Stämme im Alt- und Mittelhochdeutschen (\citealt[193]{Braune2004}, \citealt[143, 189]{Paul2007})}\label{table5.2}
\begin{tabular}{llll}
\lsptoprule
\multicolumn{2}{c}{{Althochdeutsch}} & \multicolumn{2}{c}{{Mittelhochdeutsch}}\\\cmidrule(lr){1-2}\cmidrule(lr){3-4}
{\textsc{nom}.\textsc{sg}.} & {\textsc{gen.sg.}} & \textsc{nom.sg.} & \textsc{gen.sg.}\\
\midrule
sn\=eo & sn\=ew-es & sn\=e & sn\=ew-es\\
horo & horaw-es & hor & horw-es\\
tou & touw-es & tou & touw-es\\
\lspbottomrule
\end{tabular}
\end{table}

Im Althochdeutschen ist von einer \isi{Wurzel} auf -\textit{w} auszugehen. Steht kein \isi{Suffix} (wie z.B. im Nominativ Singular), tritt das \textit{w} in den Auslaut und wird zu \textit{o} vokalisiert. Diese phonologische Regel dauert im Althochdeutschen fort \citep[68]{VoylesBarrack2014}. Auf der einen Seite weisen zwar auch andere germanische Sprachen ähnliche Prozesse auf, was dafür sprechen würde, dass es sich dabei um ein gemeingermanisches Phänomen handelt. Die Varianten sind aber nicht in allen germanischen Varietäten gleich verteilt, was bedeutet, dass diese Regel in unterschiedlichen Varianten in den einzelnen Sprachen fortdauert. Beispielsweise weist das Nordgermanische dieselbe Verteilung wie das Althochdeutsche auf, im Gotischen und Altenglischen jedoch ist das \textit{w} im Auslaut nach Langvokal erhalten, z.B. Altenglisch \textit{sn\=aw} ‘Schnee’ (Nominativ Singular), \textit{sn\=awes} (Genitiv Singular) \citep[18--19]{KraheMeid1967}.

Weiter stellt sich für das Althochdeutsche die Frage, weshalb aus \textit{w} ein \textit{o} entstanden ist. Dass auslautende Konsonanten zu \textit{o} vokalisiert werden, ist ein durchaus bekanntes Phänomen. Beispielsweise wird im Kroatischen auslautendes \textit{l} zu \textit{o} vokalisiert \citep[84]{Holzer2007}. Für das Althochdeutsche kann folgender Wandel postuliert werden: \textit{w} > \textit{u} > \textit{o}. Dafür spricht, dass alle anderen germanischen Sprachen in diesen Positionen ein \textit{u} aufweisen, das Altsächsische \textit{u} und \textit{o} \citep[18--19]{KraheMeid1967}. Zusätzlich sind hier zwei weitere Wandelerscheinungen wichtig, die wohl parallel abgelaufen sind. Germanisch *\textit{au} vor \textit{h} und besonders vor Dentalen ist zu Althochdeutsch \textit{\=o} geworden, und zwar über die (wenn auch spärlich) belegte Zwischenstufe \textit{ao} \citep[47]{Braune2004}. Parallel dazu läuft der Wandel von auslautendem \textit{ao} zu \textit{\=o}, wobei \textit{ao} aus \textit{aw} entstanden ist \citep[48]{Braune2004}: \textit{fraw\=er}/\textit{frao} > \textit{fr\=o} \citep[112]{Braune2004}. Folglich verhält sich *\textit{au}C wie *\textit{aw}. Schließlich können die Fälle, in denen durch die Vokalisierung \textit{eu} entsteht (*\textit{sn\=ew} > *\textit{sn\=eu} > \textit{sn\=eo}), dadurch erklärt werden, dass im Westgermanischen \textit{eu} im Auslaut zu \textit{eo} wird \citep[68, 167]{VoylesBarrack2014}.

Geklärt werden muss noch, woher das \textit{a} in \textit{horawes} ‘Schmutz’ und der Diphthong in \textit{tou} ‘Tau’ stammen. Beim \textit{a} in \textit{horawes} handelt es sich um einen Sprossvokal, der zwischen \textit{r}+\textit{h}, \textit{r}+\textit{l}, \textit{r}+\textit{w}, \textit{l}+\textit{w} und \textit{s}+\textit{w} eingefügt wird \citep[71]{Braune2004}. Anstelle von \textit{a} können auch \textit{o} oder \textit{e} stehen, wobei die Wahl des Sprossvokals ebenfalls phonologisch bedingt ist: „Der entstehende Vokal erscheint als \textit{a} oder (bes. vor \textit{w}) als \textit{o}, nimmt aber häufig auch die Form eines nebenstehenden Vokals an, wobei in der Regel die Endsilbenvokale, seltener die Stammsilbenvokale, maßgebend sind […]“ \citep[71]{Braune2004}. Auch für den Einschub und die \isi{Variation} des Sprossvokals ist also die Phonologie verantwortlich. Der diphthongische Auslaut von \textit{tou} geht hingegen auf ein geminiertes \textit{w} zurück \citep[109]{Braune2004}. Dieses geminierte \textit{w} wird im Auslaut vereinfacht und „es bleibt also nur der erste Teil des \textit{ww}, der am Silbenschluss einen Diphthong bildet […]“ \citep[111-112]{Braune2004}. Auch dafür gibt es folglich eine phonologische Erklärung, weshalb es in der Morphologie nicht berücksichtigt werden muss.

Im Mittelhochdeutschen ist die \isi{Variation} nicht phonologisch bedingt, denn im Gegensatz zum Althochdeutschen wird im Mittelhochdeutschen ein Segment (d.h. \textit{w}) getilgt und nicht vokalisiert. Aus synchroner Sicht ist diese Tilgung phonologisch nicht voraussagbar, weil auslautendes \textit{w} im Mittelhochdeutschen meistens, aber nicht immer getilgt wird \citep[123]{VoylesBarrack2014}. Geht man vom umgekehrten Fall, d.h., von einer \isi{Wurzel} ohne \textit{w} aus, gibt es ebenfalls keine phonologische Erklärung. Denn \textit{w} steht in den unterschiedlichsten phonologischen Kontexten: intervokalisch wie auch nach verschiedenen Konsonanten. Folglich kann nicht vorausgesagt werden, in welchen phonologischen Kontexten ein \textit{w} eingesetzt werden muss. Diese Gründe sprechen dafür, dass es sich dabei um eine Wurzelalternation handelt, die durch eine RR ausgedrückt wird, welche an die entsprechenden \isi{Flexionsklassen} gebunden ist:

\ea%60
\label{ex:key:60}
 RR \textsubscript{D,} \textsubscript{\{\},} \textsubscript{\textsc{n[}\textsc{ic:} 1} \textsubscript{\tiny $\veebar$}\textsubscript{ 3]} ($\langle$X,$\sigma$ $\rangle$) = \textsubscript{def} $\langle$X *\textit{w} $\rightarrow$ ø/\_\#ˊ,$\sigma$ $\rangle$\\
\z

Diese RR definiert, dass in den \isi{Flexionsklassen} 1 und 3 \textit{w} getilgt wird, sobald es in den Auslaut tritt. Es wird also von einer \isi{Wurzel} auf \textit{w} ausgegangen. Obwohl in der \isi{Flexionsklasse} 1 \textit{w} im Dativ und Genitiv Singular und im gesamten Plural getilgt wird, in der \isi{Flexionsklasse} 3 jedoch nur im Dativ und Genitiv Singular und Plural, kann diese RR für beide \isi{Flexionsklassen} verwendet werden. Denn die Präsenz oder Absenz von \textit{w} ist davon abhängig, ob \textit{w} am Wortende oder im Wortinneren steht, aber nicht abhängig von den morphosyntaktischen Eigenschaften.

In den hier untersuchten alemannischen Dialekten sind wa-/w\=o-Stäm\-me nur in Issime erhalten (vgl. Paradigma 4, \isi{Flexionsklasse} 8; \citealt[163]{Zürrer1999}). Obwohl das \textit{w} in Issime nur intervokalisch auftritt, handelt es sich dabei nicht um ein Element zur Hiatvermeidung. Wie bereits oben dargestellt wurde, wird in den alemannischen Dialekten ein \textit{n} eingefügt, um einen \isi{Hiat} zu vermeiden. Da das \textit{w} in Issime ausschließlich im Plural erscheint, kann es synchron folglich als Pluralmarker analysiert werden. Es wird also durch eine RR suffigiert, wie diese in \REF{ex:key:54} vorgestellt wurde.

\subsection{{Suffixe des Typs -\textit{ənə} }{und \textit{n}} {zur Hiatvermeidung}}\label{5.1.4}

In diesem Kapitel werden drei mit \textit{n} in Zusammenhang stehende Phänomene diskutiert. Erstens ist \textit{n} nur in Issime ein Pluralmarker. Zweitens wird \textit{n} in der Substantivflexion vieler Dialekte zur Vermeidung eines \isi{Hiats} verwendet, was in einigen Dialekten mit einer Schwächung der Mittelsilbe verbunden ist. Beides ist phonologisch bedingt, weshalb es keine RRs dafür braucht. Drittens weisen fünf Dialekte eine Pluralendung des Typs -\textit{ənə} auf (-\textit{ənə}, -\textit{əni}, -\textit{inə} etc.) und vier davon verwenden \textit{n} zur Hiatvermeidung. Bei diesen vier Dialekten ist also zusätzlich zu zeigen, weshalb sowohl ein \textit{n} zur Hiatvermeidung als auch eine Pluralendung des Typs -\textit{ənə} anzunehmen ist. Wie für alle anderen Pluralsuffixe ist auch für die \isi{Suffixe} des Typs -\textit{ənə} eine RR anzusetzen. In drei der fünf Dialekte, die ein Pluralsuffix des Typs -\textit{ənə} aufweisen, wird eine Silbe reduziert, was jedoch phonotaktisch bedingt ist (also keine RR). Punkt zwei und drei werden zusammen erörtert, indem jedes betroffene Paradigma einzeln analysiert wird.

\begin{description}
 \item[\textit{N} als Pluralmarker:]
Wie in vielen anderen Dialekten dient auch in Issime \textit{n} zur Hiatvermeidung. Gleichzeitig weist Issime aber auch ein -\textit{n} als Pluralsuffix auf: \textit{uav}-\textit{e} ‘Ofen’ (Nominativ Singular), \textit{uav}-\textit{n}-\textit{a} (Nominativ Plural), \textit{uav}-\textit{n}-\textit{e} (Dativ Plural) (vgl. \isi{Flexionsklasse} 2 in Paradigma 4, \citealt[164]{Zürrer1999}). Hier wird mit dem \textit{n} klar kein \isi{Hiat} vermieden (es steht nicht intervokalisch), sondern ein Plural markiert, was besonders der Vergleich des Nominativ Singular mit dem Dativ Plural zeigt. Ähnliches kann für den Dialekt von Jaun beobachtet werden: \textit{h\=ar} ʻHaarʼ (Nom/Akk.Sg.), \textit{h\=ar}{}-\textit{ən}{}-\textit{i} (Nom/Akk.Pl.) (vgl. Paradigma 6).

\item[N-Einschub zur Hiatvermeidung und Pluralendung des Typs \textit{-ənə}:] Nun werden folgende Phänomene zusammen betrachtet: n-Einschub zur Hiatvermeidung, die daraus resultierende Zentralisierung des Mittelsilbenvokals (\textit{blatti} > *\textit{blatti}-\textit{n}-\textit{i} > \textit{blattə}-\textit{n}-\textit{i}, ‘Blatt’, Jaun, \citealt[267]{Stucki1917}); Plural des Typs -\textit{ənə}, dadurch verursachte Tilgung des auslautenden Wurzelvokals aufgrund\linebreak phonotaktischer Restriktionen (\textit{wiərde} > * \textit{wiərde}-\textit{ənə} > \textit{wiərd}-\textit{ənə}, ‘Wirtin’, Huzenbach, \citealt[97]{Baur1967}). Die folgenden Tabellen \tabref{table5.3}, \ref{table5.4} und \ref{table5.5} geben eine Übersicht darüber, in welchen Dialekten welche Phänomene zusammen auftreten. Es werden zuerst die Dialekte der \tabref{table5.3} besprochen (von links nach rechts), welche ein \textit{n} zur Hiatvermeidung und eine damit einhergehende Zentralisierung der Mittelsilbe aufweisen, und dann jene der \tabref{table5.4}, deren Mittelsilbe nicht zentralisiert wird. Schließlich wird auf den Dialekt von Huzenbach eingegangen (\tabref{table5.5}), der als einziger in der Substantivflexion kein \textit{n} zur Hiatvermeidung aufweist, aber eine Pluralendung des Typs -\textit{ənə}.
\end{description}

%{\tabref{table5.3}: Dialekte mit \textit{n} (Hiatvermeidung) und Zentralisierung der Mittelsilbe}\\

\begin{table}[p]
\caption{Dialekte mit \textit{n} (Hiatvermeidung) und Zentralisierung der Mittelsilbe}\label{table5.3}
\resizebox{\textwidth}{!}{\begin{tabular}{p{3cm}*{6}{l}}
\lsptoprule
 & Jaun & Sensebezirk & Uri & Vorarlberg & Saulgau & Stuttgart\\\midrule
n (Hiatus) & + & + & + & + & + & +\\
Mittelsilbenzen\-tralisierung & + & + & + & + & + & +\\
Plural Typ -\textit{ənə} & - & + & + & - & - & -\\
Phonotaktisch \mbox{bedingte Tilgung} einer Silbe & - & - & - & - & - & -\\
\lspbottomrule
\end{tabular}}
\end{table}

%{\tabref{table5.4}: Dialekte mit \textit{n}} {(Hiatvermeidung), aber ohne Zentralisierung der Mittelsilbe}\\

\begin{table}[p]
\caption{Dialekte mit \textit{n} (Hiatvermeidung), aber ohne Zentralisierung der Mittelsilbe}\label{table5.4}
\resizebox{\textwidth}{!}{\begin{tabular}{p{3cm}*{6}{l}}
\lsptoprule
 & Issime & Visperterminen & Bern & Petrifeld & Elisabethtal & Kaiserstuhl\\\midrule
n (Hiatus) & + & + & + & + & + & +\\
Mittelsilbenzen\-tralisierung & - & - & - & - & - & -\\
Plural Typ -\textit{ənə} & - & - & - & + & - & +\\
Phonotaktisch \mbox{bedingte Tilgung} einer Silbe & - & - & - & + & - & +\\
\lspbottomrule
\end{tabular}}
\end{table}

%{\tabref{table5.5}: Dialekte ohne \textit{n}} {(Hiatvermeidung)}\\

\begin{table}[p]
\caption{Dialekt ohne \textit{n} (Hiatvermeidung)}\label{table5.5}
\begin{tabularx}{\textwidth}{Xl}
\lsptoprule
 & Huzenbach\\\midrule
n (Hiatus) & -\\
Mittelsilbenzen\-tralisierung & -\\
Plural Typ -\textit{ənə} & +\\
Phonotaktisch bedingte Tilgung einer Silbe & +\\
\lspbottomrule
\end{tabularx}
\end{table}

% Diese drei Tabellen sollten alle zusammen auf einer Querformat-Seite stehen.

In der Substantivflexion von \textbf{Jaun} (Paradigma 6) gehören Maskulina, Neutra und Feminina, deren \isi{Wurzel} auf \textit{i} endet, zu den \isi{Flexionsklassen} 11 (Maskulina und Neutra) bzw. 12 (Feminina). Ihre Flexionsendungen entsprechen denen der \isi{Flexionsklassen} 11 und 12 (vgl. \tabref{table5.6}). Da ihre \isi{Wurzel} jedoch vokalisch auslautet, muss ein \textit{n} eingeschoben werden, wenn das \isi{Suffix} vokalisch anlautet. Außerdem wird schwachtoniges \textit{i} zu \textit{ə}, wenn es im Wortinlaut steht \citep[159-164]{Stucki1917}.

%{\tabref{table5.6}: \textit{N}} {zur Hiatvermeidung in Jaun (basierend auf \citealt[255-272]{Stucki1917}}\\

\begin{table}
\caption{\textit{N} zur Hiatvermeidung in Jaun (basierend auf \citealt[255-272]{Stucki1917})}\label{table5.6}
\resizebox{\textwidth}{!}{\begin{tabular}{lllllll}
\lsptoprule
{\textsc{fk}} & \multicolumn{2}{c}{{\textsc{singular}}}  & \multicolumn{4}{c}{{\textsc{plural}}} \\\cmidrule(lr){2-3}\cmidrule(lr){4-7}
& {\NOM/\AKK/\DAT} & {\GEN} & {\NOM} & {\AKK} & {\DAT} & {\GEN}\\
\midrule
11 & bet ‘Bett’  & bet-s & bet-i & bet-i & bet-ə & bet-ə\\
& blatti ‘Teller’  & blatti-s & blattə-n-i & blattə-n-i & blattə-n-ə & blattə-n-ə\\
\midrule
12 & fr\=ag ‘Frage’  & fr\=ag & fr\=ag-i & fr\=ag-i & fr\=ag-ə & fr\=ag-ə\\
& schieri ‘Schere’ & schieri & schierə-n-i & schierə-n-i & schierə-n-ə & schierə-n-ə\\
\lspbottomrule
\end{tabular}}
\end{table}

Im Gegensatz dazu bilden \textit{h\=ar}/\textit{h\=ar}-\textit{ən}-\textit{i} ‘Haar’ (\isi{Flexionsklasse} 4), \textit{tǖr}/\textit{tǖr}-\textit{ən}-\textit{i} ‘Tür’ (\isi{Flexionsklasse} 14),\largerpage[2]  \textit{matt}-\textit{a}/\textit{matt}-\textit{ən}-\textit{i} ‘Wiese’ (\isi{Flexionsklasse} 16) eigene \isi{Flexionsklassen}. Zwar weisen auch sie dieselben Kasusendungen wie die \isi{Flexionsklassen} 11 und 12 auf: -\textit{s} (Genitiv Singular, nur Maskulina und Neutra), -\textit{i} (Nominativ und Akkusativ Plural), -\textit{ə} (Dativ und Genitiv Plural). Zusätzlich wird aber der Plural durch das \isi{Suffix} -\textit{ən} markiert, was auf die \isi{Flexionsklassen} 11 und 12 nicht zutrifft. Für die \isi{Flexionsklassen} 4, 14 und 16 wäre auch ein Plural auf -\textit{əni} (Nominativ und Akkusativ) und -\textit{ənə} (Dativ und Genitiv) denkbar. Bei dieser Analyse müssten aber zwei neue \isi{Suffixe} angenommen werden (-\textit{əni}, -\textit{ənə}), da sie in keiner anderen \isi{Flexionsklasse} vorkommen. Bei der Analyse mit -\textit{ən} als Pluralsuffix braucht es zusätzlich nur dieses \isi{Suffix}, denn die \isi{Suffixe} -\textit{i} und -\textit{ə} existieren bereits in anderen \isi{Flexionsklassen}. Die Analyse mit -\textit{ən} als Pluralsuffix benötigt also eine RR weniger als die Analyse mit -\textit{əni}/-\textit{ənə}, weshalb sie zu bevorzugen ist.

Im Alemannischen des \textbf{Sensebezirks} verhalten sich die \isi{Substantive}, deren\linebreak \isi{Wurzel} auf \textit{i} auslautet, ähnlich wie jene in Jaun. Die Feminina der auf \textit{i} auslautenden \isi{Substantive} bilden den Plural auf -\textit{ə} (\isi{Flexionsklasse} 3), die Maskulina und Neutra auf -\textit{i} (\isi{Flexionsklasse} 7) (vgl. Paradigma 7): \textit{schwechi}/\textit{schwechə}-\textit{n}-\textit{ə} ‘Schwäche’, \textit{blatti}/\textit{blattə}-\textit{n}-\textit{i} ‘Teller’ \citet[188, 186]{Henzen1927}. \isi{Kasus} wird in diesem Dialekt am \isi{Substantiv} nicht unterschieden. Auch hier wird also an eine \isi{Wurzel}, die auf einen Vokal endet, ein vokalisches Pluralsuffix angehängt, folglich muss ein \textit{n} eingeschoben werden. Wie in Jaun wird auch hier das \textit{i} zu \textit{ə} gesenkt, wenn es im Wortinlaut steht \citep[124]{Henzen1927}. Des Weiteren existiert aber auch das Pluralsuffix -\textit{əni}, das also eine eigene \isi{Flexionsklasse} bildet: \textit{nɛts}/\textit{nɛts}-\textit{əni} ‘Netz’. Da die \isi{Wurzeln} der \isi{Substantive} dieser \isi{Flexionsklasse} konsonantisch auslauten, ist \textit{ə} kein gesenktes \textit{i} und \textit{n} kein Einschub zur Hiatvermeidung.

Im Gegensatz zu den Dialekten von Jaun und des Sensebezirks unterscheidet sich in \textbf{Uri} die Flexion der \isi{Substantiven}, die auf \textit{i} auslauten, in den drei \isi{Genera}. Die Maskulina (\textit{briali} ‘schreiende Person’) gehören zur \isi{Flexionsklasse} 4, die Neutra (\textit{bekchi} ‘Becken’) zur \isi{Flexionsklasse} 11 und die Feminina (\textit{schn\=idəri} ‘Schneiderin’) bilden eine eigene \isi{Flexionsklasse} \REF{ex:key:7} (vgl. Paradigma 8 zusammengefasst in \tabref{table5.7}). Die Maskulina haben dieselben Flexionsendungen wie in \isi{Flexionsklasse} 4. Per Default wird bei einem Hiatus ein \textit{n} eingeschoben, was eine Senkung des \textit{i} zu \textit{ə} als Konsequenz hat \citep[106]{Clauß1929}. Die Feminina bilden eine eigene \isi{Flexionsklasse}. Zwar weisen sie dieselben Flexionsendungen auf wie die \isi{Flexionsklasse} 4; würden aber die Feminina in die \isi{Flexionsklasse} 4 eingeteilt, dann entstünde im Plural *\textit{schn\=idərə}-\textit{n}-\textit{a}, da von der Phonologie per Default ein \textit{n} eingeschoben wird. In den Feminina wird das auslautende \textit{i} der \isi{Wurzel} aber nicht reduziert, sondern getilgt. Da es sich dabei um keinen Prozess handelt, der im gesamten System stattfindet (im Gegensatz zum n-Einschub), ist die Tilgung in der Morphologie zu verorten und folglich durch eine RR zu definieren, wie diese in \sectref{5.1.3} (Subtraktion) vorgestellt wurden:

\ea%61
\label{ex:key:61}
 RR \textsubscript{D,} \textsubscript{\{\textsc{num:pl}\},} \textsubscript{\textsc{n[}\textsc{ic:} 7} \textsubscript{\tiny $\veebar$}\textsubscript{ 11]} ($\langle$X,$\sigma$ $\rangle$) = \textsubscript{def} $\langle$X *\textit{i} $\rightarrow$ ø/\_Vˊ,$\sigma$ $\rangle$\\
\z

Diese RR gilt nur für die Feminina, nicht für die Maskulina. Wären die Feminina und Maskulina in der \isi{Flexionsklasse} 4 und die RR \REF{ex:key:61} der \isi{Flexionsklasse} 4 zugeordnet, würde diese RR auch auf die Maskulina angewendet, wodurch *\textit{brial}-\textit{ɐ} entstehen würde. Die kürzeste Beschreibung dieses Systems, die alle Formen korrekt definiert, erreicht man also, wenn die Maskulina zur \isi{Flexionsklasse} 4 gehören und die Feminina eine eigene \isi{Flexionsklasse} bilden.

%{\tabref{table5.7}: \textit{N}} {(Hiatvermeidung) und \isi{Suffix} -\textit{ənə}} {in Uri (basierend auf Clauß 1929: 173-185})\\

\begin{table}
\caption{\textit{N} (Hiatvermeidung) und Suffix -\textit{ənə} in Uri (basierend auf \citealt[173-185]{Clauß1929})}\label{table5.7}
\begin{tabular}{llll}
\lsptoprule
{\textsc{fk}} & {\textsc{singular}} & \multicolumn{2}{c}{{\textsc{plural}}} \\\cmidrule(lr){3-4}
&  & {\NOM/\AKK} & {\DAT}\\
\midrule
4 & chnacht ‘Knecht’ & chnacht-ɐ & chnacht-ɐ\\
& \mbox{briali ‘schreiende Person’} & brialə-n-ɐ & brialə-n-ɐ\\
\midrule
7 & \mbox{schn\=idəri ‘Schneiderin’} & schn\=idər-ɐ & schn\=idər-ɐ\\
\midrule
11 & bet ‘Bett’ & bet-i & bet-ənɐ\\
& bekchi ‘Becken’ & bekch-i & bekch-ənɐ\\
\midrule
12 & nets ‘Netz’ & nets-i & nets-ɐ\\
\lspbottomrule
\end{tabular}
\end{table}

Die Zuordnung der Neutra zur \isi{Flexionsklasse} 11 hängt auch damit zusammen, dass die \isi{Flexionsklassen} 11 und 12 unterschieden werden. Um den Fall besser zu verstehen, wird nun zuerst die Analyse eingeführt, die hier verwendet wird. Da eine alternative Analyse denkbar ist, aber nur auf den ersten Blick ökonomischer erscheint, wird auch diese vorgestellt und es wird gezeigt, weshalb sie zu verwerfen ist. Für die hier eingeführte Analyse der \isi{Flexionsklassen} 11 und 12 werden folgende RR gebraucht (RR \REF{ex:key:61} bereits oben, hier wiederholt):

\ea%62
\label{ex:key:62}
 RR \textsubscript{C,} \textsubscript{\{\textsc{case:nom}} \textsubscript{\tiny $\veebar$}\textsubscript{ \AKK, \textsc{num:pl}\},} \textsubscript{\textsc{n[}\textsc{ic:} 11} \textsubscript{\tiny $\veebar$}\textsubscript{ 12]} ($\langle$X,$\sigma$ $\rangle$) = \textsubscript{def} $\langle$X\textit{i}ˊ,$\sigma$ $\rangle$
\z

\ea%63
\label{ex:key:63}
 RR \textsubscript{C,} \textsubscript{\{\textsc{case:dat}, \textsc{num:pl}\},} \textsubscript{\textsc{n[}\textsc{ic:} 11]} ($\langle$X,$\sigma$ $\rangle$) = \textsubscript{def} $\langle$X\textit{ənɐ}ˊ,$\sigma$ $\rangle$
\z

\ea%64
\label{ex:key:64}
 RR \textsubscript{C,} \textsubscript{\{\textsc{case:dat}, \textsc{num:pl}\},} \textsubscript{\textsc{n[}\textsc{ic:} 1} \textsubscript{\tiny $\veebar$} \textsubscript{2} \textsubscript{\tiny $\veebar$}\textsubscript{ 3} \textsubscript{\tiny $\veebar$} \textsubscript{4} \textsubscript{\tiny $\veebar$} \textsubscript{5} \textsubscript{\tiny $\veebar$} \textsubscript{6} \textsubscript{\tiny $\veebar$} \textsubscript{7} \textsubscript{${\veebar}$} \textsubscript{8} \textsubscript{${\veebar}$} \textsubscript{9} \textsubscript{${\veebar}$} \textsubscript{10} \textsubscript{${\veebar}$}\textsubscript{ 12]} ($\langle$X,$\sigma$ $\rangle$) = \textsubscript{def} $\langle$X\textit{ɐ}ˊ,$\sigma$ $\rangle$
\z

% \ea%61
\begin{exe}
\exr{ex:key:61}
 RR \textsubscript{D,} \textsubscript{\{\textsc{num:pl}\},} \textsubscript{\textsc{n[}\textsc{ic:} 7} \textsubscript{\tiny $\veebar$}\textsubscript{ 11]} ($\langle$X,$\sigma$ $\rangle$) = \textsubscript{def} $\langle$X *\textit{i} $\rightarrow$ ø/\_Vˊ,$\sigma$ $\rangle$
\end{exe}

Die RR \REF{ex:key:62} definiert, dass der Plural der \isi{Flexionsklassen} 11 und 12 auf -\textit{i} lautet, die RR \REF{ex:key:63}, dass der Dativ Plural der \isi{Flexionsklasse} 11 auf -\textit{ənɐ} lautet, die RR \REF{ex:key:64}, dass der Dativ Plural aller \isi{Flexionsklassen} außer 11 auf -\textit{ɐ} endet. Die Neutra mit einer auf \textit{i} auslautenden \isi{Wurzel} gehören zur \isi{Flexionsklasse} 11. Wie in \isi{Flexionsklasse} 7 wird auch hier das \textit{i} der \isi{Wurzel} getilgt. Die RR \REF{ex:key:61} kann also für die \isi{Flexionsklassen} 7 und 11 verwendet werden.

Die zu verwerfende Analyse geht von der Beobachtung aus, dass der Dativ Plural aller \isi{Flexionsklassen} auf -\textit{ɐ} endet (vgl. Paradigma 8). Im Gegensatz zur \isi{Flexionsklasse} 12 wird in \isi{Flexionsklasse} 11 agglutiniert: \textit{bet}/\textit{bet}-\textit{i}/\textit{bet}-\textit{i}-\textit{n}-\textit{ɐ}>\textit{bet}-\textit{ə}-\textit{n}-\textit{ɐ} ‘Bett’. Ein \textit{n} wird eingeschoben, um den \isi{Hiat} zu vermeiden und das \textit{i} wird inlautend gesenkt. Auf den ersten Blick wäre also die RR \REF{ex:key:63} nicht nötig. Um aber die korrekten Formen der beiden \isi{Flexionsklassen} 11 und 12 zu definieren, müsste aus der RR \REF{ex:key:62} zwei RRs gemacht werden, nämlich eine für die agglutinierende \isi{Flexionsklasse} 11 \REF{ex:key:65} und eine für die \isi{Flexionsklasse} 12 \REF{ex:key:66}:

\ea%65
\label{ex:key:65}
 *RR \textsubscript{B,} \textsubscript{\{\textsc{num:pl}\},} \textsubscript{\textsc{n[}\textsc{ic:} 11]} ($\langle$X,$\sigma$ $\rangle$) = \textsubscript{def} $\langle$X\textit{i}ˊ,$\sigma$ $\rangle$
\z

\ea%66
\label{ex:key:66}
 *RR \textsubscript{C,} \textsubscript{\{\textsc{case:nom}} \textsubscript{\tiny $\veebar$}\textsubscript{ \AKK, \textsc{num:pl}\},} \textsubscript{\textsc{n[}\textsc{ic:} 12]} ($\langle$X,$\sigma$ $\rangle$) = \textsubscript{def} $\langle$X\textit{i}ˊ,$\sigma$ $\rangle$
\z

Die RR \REF{ex:key:65} besagt, dass -\textit{i} im gesamten Plural suffigiert wird, was nur für die \isi{Flexionsklasse} 11, aber nicht für 12 zutrifft. Mit dieser Analyse würde also die RR \REF{ex:key:63} wegfallen, dafür wird eine zusätzliche RR für das Pluralsuffix -\textit{i} gebraucht. Problematisch wird diese Analyse aber erst, wenn die Neutra mit einer auf \textit{i} auslautenden \isi{Wurzel} betrachtet werden (\textit{bekchi} ‘Becken’). Mit den RRs \REF{ex:key:65}, \REF{ex:key:66} und (\REF{ex:key:64}, für alle \isi{Flexionsklassen}) entstehen folgende Formen, unabhängig davon, zu welcher \isi{Flexionsklasse} diese Neutra gezählt werden: *\textit{bekchə}-\textit{n}-\textit{i}, \textit{bekchə}-\textit{n}-\textit{ɐ} (n-Einschub und Senkung des \textit{i} zu \textit{ə}). Nimmt man die RR \REF{ex:key:61} dazu, entstehen diese Formen: \textit{bekch}-\textit{i}, *\textit{bekch}-\textit{ɐ}. Damit mit den RRs \REF{ex:key:65}, \REF{ex:key:66} und (\REF{ex:key:64}, für alle \isi{Flexionsklassen}) die Formen der Neutra korrekt definiert werden, müsste eine zusätzliche RR für die \is{Subtraktion}Subtraktion angenommen werden:

\ea%67
\label{ex:key:67}
 *RR \textsubscript{D,} \textsubscript{\{\textsc{num:pl}\},} \textsubscript{\textsc{n[}\textsc{ic:} 11]} ($\langle$X,$\sigma$ $\rangle$) = \textsubscript{def} $\langle$X *\textit{i} $\rightarrow$ ø/\_iˊ,$\sigma$ $\rangle$
\z

Mit dieser RR wird das \textit{i} der \isi{Wurzel} getilgt, wenn das Pluralsuffix -\textit{i} folgt, womit der n-Einschub verhindert wird (\textit{bekch}-\textit{i}). Da der Kontext dieser RR auf ein nachfolgendes \textit{i} beschränkt ist, entsteht der korrekte Dativ Plural \textit{bekchə}-\textit{n}-\textit{ɐ} (insofern die RR \REF{ex:key:61} auf die \isi{Flexionsklasse} 7 beschränkt wird). Da die erste Analyse mit vier RRs auskommt (\ref{ex:key:61}--\ref{ex:key:64}), für die zweite jedoch fünf RRs nötig sind (\REF{ex:key:61}, \ref{ex:key:64}--\ref{ex:key:67}), ist die erste Analyse zu bevorzugen.

Die Dialekte von \textbf{Vorarlberg, Saulgau und Stuttgart} können zusammen betrachtet werden. Für alle drei Dialekte ist eine \isi{Flexionsklasse} mit einem Pluralsuffix auf -\textit{ə} anzusetzen: für Vorarlberg \isi{Flexionsklasse} 6 (Paradigma 9), für Saulgau \isi{Flexionsklasse} 4 (Paradigma 13), für Stuttgart \isi{Flexionsklasse} 5 (Paradigma 14). In diese \isi{Flexionsklassen} gehören im Dialekt von Vorarlberg jene \isi{Substantive}, deren \isi{Wurzel} auf ein \textit{i} auslautet (\textit{kchöchi}/\textit{kchöchə}-\textit{n}-\textit{ə} ‘Köchin’ \citealt[252-253]{Jutz1925}), im Dialekt von Saulgau und Stuttgart \isi{Substantive} mit einer \isi{Wurzel} auf \textit{e} (Saulgau: \textit{deke}/\textit{dekə}-\textit{n}-\textit{ə} ‘Decke’ \citep[105]{Raichle1932}; Stuttgart: \textit{wəide}/\textit{wəidə}-\textit{n}-\textit{ə} ‘Wirtin’ \citep[151]{Frey1975}). Da in keinem dieser Dialekte konsonantisch auslautende \isi{Wurzeln} einen Plural mit -\textit{ənə} haben, müsste also eine zusätzliche \isi{Flexionsklasse} angesetzt werden, würde man bei den genannten Wörtern von einem Plural auf -\textit{ənə} ausgehen. Dies würde die Beschreibung unnötigerweise komplexer machen, zumal diese Fälle auch problemlos anderen \isi{Flexionsklassen} zugeordnet werden können. In allen drei Dialekten wird ein \textit{n} zur Hiatvermeidung verwendet, wodurch der Auslaut der \isi{Wurzel} ins Wortinnere tritt und dort gesenkt wird (\citealt[175]{Jutz1925}, \citealt[64]{Raichle1932}). In Saulgau und Stuttgart wird eine weitere \isi{Flexionsklasse} für \isi{Substantive} angenommen, deren \isi{Wurzel} auf \textit{e} endet (z.B. Stuttgart: \textit{dischle}/\textit{dischl}-\textit{ə} ‘Tischlein’). Die Begründung entspricht jener für die \isi{Flexionsklassen} 4 und 7 in Uri. Im Gegensatz zu den oben geschilderten Fällen wird hier der auslautende Vokal der \isi{Wurzel} nicht als Konsequenz des n-Einschubs gesenkt, sondern der Vokal getilgt. Aus den bereits für den Dialekt von Uri genannten Gründen (\textit{schn\=idəri}/\textit{schn\=idər}-\textit{a} ‘Schneiderin’) muss eine separate \isi{Flexionsklasse} angenommen werden, wenn der auslautende Wurzelvokal getilgt wird.

Nun werden die Dialekte aus der \tabref{table5.4} besprochen. Die Dialekte von \textbf{Issime, Visperterminen, Bern und Elisabethtal} weisen nicht nur ein \textit{n} zur Hiatvermeidung, aber keine Schwächung der Mittelsilbe auf, sondern haben auch kein Pluralsuffix des Typs -\textit{ənə}. Deswegen können sie hier auch zusammen behandelt werden. Aus den Tabellen \ref{table5.8} und \ref{table5.9} kann entnommen werden, dass bei Wörtern, deren \isi{Wurzeln} auf einen Vokal auslauten, ein \textit{n} eingeschoben wird, wenn ein vokalisches \isi{Suffix} folgt. Da bereits \isi{Flexionsklassen} mit denselben Pluralsuffixen für auf Konsonant auslautende \isi{Wurzeln} angesetzt werden müssen, können Wörter mit vokalisch auslautenden \isi{Wurzeln} denselben \isi{Flexionsklassen} zugeordnet werden. In Visperterminen beispielsweise hat die \isi{Flexionsklasse} 12 folgenden Satz an \isi{Suffixen}: -\textit{sch} (Genitiv Singular), -\textit{i} (Nominativ und Akkusativ Plural), -\textit{u} (Dativ Plural), -\textit{o} (Genitiv Plural). Dies entspricht dem Suffixsatz der Maskulina und Neutra mit einer auf \textit{i} auslautenden \isi{Wurzel} haben, nur dass ein \textit{n} aus phonologischen Gründen eingeschoben wird. Des Weiteren kann festgestellt werden, dass der auslautende Vokal der \isi{Wurzel} nicht verändert wird, wenn \textit{n} + \isi{Suffix} folgen. Außerdem muss im Dialekt von Elisabethtal zusätzlich eine \isi{Flexionsklasse} für jene \isi{Substantive} angenommen werden, deren \isi{Wurzel} auf -\textit{ə} auslautet. Wie in den Dialekten von Uri (\textit{schn\=idəri}/\textit{schn\=idər}-\textit{ɐ} ‘Schneiderin’), Vorarlberg, Saulgau und Stuttgart handelt es sich auch hierbei um eine Tilgung des auslautenden Vokals der \isi{Wurzel}, was durch eine RR abgebildet wird. Schließlich sind in Issime zwei \isi{Flexionsklassen} (16 und 17) anzusetzen, da sie einen Pluralsuffix -\textit{in} haben. Derselbe Fall, der oben erörtert wurde, tritt in Jaun auf (Pluralsuffix -\textit{ən}). Wie für Jaun ist auch für Issime die Analyse mit -\textit{in} als Pluralsuffix kürzer. Andere \isi{Flexionsklassen} weisen denselben Satz an Pluralsuffixen auf: -\textit{i} (Nominativ und Akkusativ Plural), -\textit{u} (Dativ und Genitiv Plural). Zusätzlich muss also nur ein Pluralsuffix -\textit{in} angenommen werden, während in einer Analyse -\textit{ini}/-\textit{inu} zwei RRs gebraucht würden.

%{\tabref{table5.8}: \textit{N}} {zur Hiatvermeidung in Issime und Visperterminen (basierend auf \citealt[144-205]{Zürrer1999}} {und \citealt[119-134]{Wipf1911} }\\

\begin{table}
\caption{\textit{N} zur Hiatvermeidung in Issime und Visperterminen (basierend auf \citealt[144-205]{Zürrer1999} und \citealt[119-134]{Wipf1911}}\label{table5.8}
\resizebox{\textwidth}{!}{\begin{tabular}{l@{ }l*{5}{l}} 
\lsptoprule
\textsc{fk} & \multicolumn{2}{c}{\textsc{singular}} & \multicolumn{4}{c}{\textsc{plural}} \\\cmidrule(lr){2-3}\cmidrule(lr){4-7}
& \textsc{nom/akk/dat} & \textsc{gen} & \textsc{nom} & \textsc{akk} & \textsc{dat} & \textsc{gen}\\
\midrule
\multicolumn{7}{l}{Issime}\\\midrule
9 (\textsc{m+n}) & bet ‘Bett’ & bet-sch & bet-i & bet-i & bet-u & bet-u\\
& berri ‘Beere’ & berri-sch & berri-n-i & berri-n-i & berri-n-u & berri-n-u\\
14 (\textsc{f}) & aksch ‘Axt’ & aksch & aksch-i & aksch-i & aksch-u & aksch-u\\
& chötti ‘Kette’ & chötti & chötti-n-i & chötti-n-i & chötti-n-u & chötti-n-u\\
16 & schuld ‘Schuld’ & schuld & schuld-in-i & schuld-in-i & schuld-in-u & schuld-in-u\\
17 & nacht ‘Nacht’ & nacht & necht-in-i & necht-in-i & necht-in-u & necht-in-u\\
\midrule
\multicolumn{7}{l}{Visperterminen} \\\midrule
12 (\textsc{m+n}) & ber ‘Beere’ & ber-sch & ber-i & ber-i & ber-u & ber-o\\
& redli ‘Rad’ & redli-sch & redli-n-i & redli-n-i & redli-n-u & redli-n-o\\
14 (\textsc{f}) & farb ‘Farbe’ & farb & farb-e & farb-e & farb-u & farb-o\\
& welbi ‘Wölbung’ & welbi & welbi-n-e & welbi-n-e & welbi-n-u & welbi-n-o\\
\lspbottomrule
\end{tabular}}
\end{table}

%{\tabref{table5.9}: \textit{N}} {zur Hiatvermeidung in Bern und Elisabethtal (basierend auf \citealt[82-90]{Marti1985} und \citealt[50-52]{Žirmunskij1928/29}}\\

\begin{table}
\caption{\textit{N} zur Hiatvermeidung in Bern und Elisabethtal (basierend auf \citealt[82-90]{Marti1985} und \citealt[50-52]{Žirmunskij1928/29})}\label{table5.9}
\begin{tabular}{lll} 
\lsptoprule
\textsc{fk} & \textsc{singular} & \textsc{plural}\\
\midrule
Bern &  & \\\midrule
6 & h\=as ‘Haase’ & h\=as-ə\\
& tantə ‘Tante’ & tantə-n-ə\\
& bürdi ‘Bürde, viel’ & bürdi-n-ə\\
\midrule
Elisabethtal &  & \\\midrule
8 & schuld ‘Schuld’ & schuld-ɐ\\
& khuchə ‘Küche’ & khuchə-n-ɐ\\
11 & biəblə ‘Bübchen’ & biəbl-ɐ\\
\lspbottomrule
\end{tabular}
\end{table}

Die Dialekte von \textbf{Petrifeld und des Kaiserstuhls} weisen ebenfalls ein \textit{n} zur Hiatvermeidung und keine Senkung des auslautenden Wurzelvokals auf, wenn er im Wortinneren steht. Dies wird aus den Tabellen \ref{table5.10} und \ref{table5.11} bezüglich der \isi{Flexionsklassen} 4 ersichtlich. Im Gegensatz zu Issime, Visperterminen, Bern und Elisabethtal haben Petrifeld und Kaiserstuhl jedoch Pluralsuffixe des Typs -\textit{ənə}. Es ist phonologisch nicht erklärbar, wie in Petrifeld der auslautende Wurzelvokal \textit{e} zu \textit{i} werden soll bzw. \textit{i} zu \textit{ɐ} im Kaiserstuhl, wenn man davon ausgeht, dass ein \textit{n} zur Hiatvermeidung eingefügt wird. Deswegen ist ein Pluralsuffix -\textit{inə} bzw. -\textit{ɐnɐ} anzunehmen. Des Weiteren kann bezüglich der \isi{Flexionsklasse} 9 von Petrifeld nicht argumentiert werden, dass \textit{e} automatisch von der Phonologie zu \textit{ə} gesenkt wird, da die Abfolge -\textit{enə} möglich ist \citep[42]{Moser1937}. Folglich handelt es sich bei -\textit{ənə} um ein Pluralsuffix. In den \isi{Flexionsklassen} 7 und 9 von Petrifeld wird aber der auslautende Vokal der \isi{Wurzel} getilgt, wenn ein Pluralsuffix folgt, was durch eine RR definiert wird, wie bereits oben für Uri (\textit{schn\=idəri}/\textit{schn\=idər}-\textit{a} ‘Schneiderin’) und weitere Dialekte gezeigt wurde.

%{\tabref{table5.10}: \textit{N}} {zur Hiatvermeidung und Plural des Typs -\textit{ene}} {in Petrifeld (basierend auf \citealt[59-62]{Moser1937}:}\\

\begin{table}
\caption{\textit{N} zur Hiatvermeidung und Plural des Typs -\textit{ene} in Petrifeld (basierend auf \citealt[59-62]{Moser1937})}\label{table5.10}
\begin{tabular}{lll}
\lsptoprule
{\textsc{fk}} & {\textsc{singular}} & {\textsc{plural}}\\
\midrule
4 & bek ‘Bäcker’ & bek-ə\\
& glokə ‘Glocke’ & glokə-n-ə\\
\midrule
8 & kh\=inege ‘Königin’ & kh\=ineg-inə\\
\midrule
9 & kheche ‘Köchin’ & khech-ənə\\
\midrule
7 & -le & -lə\\
\lspbottomrule
\end{tabular}
\end{table}

%{\tabref{table5.11}: \textit{N}} {zur Hiatvermeidung und Plural des Typs -\textit{ene}} {in Kaiserstuhl (basierend auf \citealt[359-373]{Noth1993}:}\\

\begin{table}
\caption{\textit{N} zur Hiatvermeidung und Plural des Typs -\textit{ene} in Kaiserstuhl (basierend auf \citealt[359-373]{Noth1993})}\label{table5.11}
\begin{tabular}{lll}
\lsptoprule
{\textsc{fk}} & {\textsc{singular}} & {\textsc{plural}}\\
\midrule
4 & grab ‘Grab’ & grab-ɐ\\
& dandɐ ‘Tante’ & dandɐ-n-ɐ\\
& bhatzianti ‘Patientin’ & bhatzianti-n-ɐ\\
\midrule
5 & ghuchi ‘Küche’ & ghuch-ɐnɐ\\
\lspbottomrule
\end{tabular}
\end{table}

\textbf{Huzenbach} stellt im Vergleich zu allen anderen Dialekten einen Sonderfall dar (Tabellen \ref{table5.5} und \ref{table5.12}). Würde man die Substantivflexion gleich wie in den übrigen Dialekten analysieren, dann würde daraus Folgendes resultieren. Wörter wie \textit{wiərde} ‘Wirtin’ \citep[97]{Baur1967}, deren \isi{Wurzeln} auf \textit{e} auslauten und die den Plural auf -\textit{ə} bilden, gehören in die \isi{Flexionsklasse} 4. Ein \textit{n} wird eingefügt, um den \isi{Hiat} zu vermeiden, und \textit{e} wird im Wortinneren zu \textit{ə} gesenkt \citep[75-78]{Baur1967}. Für Wörter wie \textit{heisle} ‘Häuschen’ \citep[98]{Baur1967} muss eine neue \isi{Flexionsklasse} angenommen werden, da der auslautende Vokal der \isi{Wurzel} im Plural getilgt wird. Mit dieser Analyse müsste jedoch für die movierten Feminina wie \textit{naiəre}/\textit{naiərnə} ‘Näherin’ \citep[97]{Baur1967} eine eigene \isi{Flexionsklasse} angesetzt werden (mit einem Pluralsuffix auf -\textit{nə}), was mit der hier verwendeten Analyse nicht nötig ist. Wörter, deren auslautender Wurzelvokal im Plural getilgt wird (\textit{heisle}), gehören zur \isi{Flexionsklasse} 4, die den Plural ebenfalls auf -\textit{ə} bildet. Die Tilgung wird durch eine RR \is{Subtraktion}(Subtraktion) gewährleistet, die dieser \isi{Flexionsklasse} zugeordnet ist. Dieselbe RR wird in der \isi{Flexionsklasse} 6 verwendet. Auch hier fällt der Wurzelauslaut \textit{e} weg, wenn ein \isi{Suffix} (-\textit{ənə}) folgt.

%{\tabref{table5.12}: Plural des Typs -\textit{ene}} {in Huzenbach (basierend auf \citealt[92-98]{Baur1967}}\\

\begin{table}
\caption{Plural des Typs -\textit{ene} in Huzenbach (basierend auf \citealt[92-98]{Baur1967})}\label{table5.12}
\begin{tabular}{lll}
\lsptoprule
{\textsc{fk}} & {\textsc{singular}} & {\textsc{plural}}\\
\midrule
4 & dan ‘Tanne’ & dan-ə\\
& heisle ‘Häuschen’ & heisl-ə\\
\midrule
6 & wiərde ‘Wirtin’ & wiərd-ənə\\
& naiəre ‘Näherin’ & naiər-nə\\
\lspbottomrule
\end{tabular}
\end{table}

Mit dieser Analyse benötigt man keine zusätzliche \isi{Flexionsklasse} für die movierten Feminina, sondern sie können der \isi{Flexionsklasse} 6 zugeordnet werden. Wie bereits dargestellt wurde, wird in dieser \isi{Flexionsklasse} im Plural -\textit{ənə} suffigiert und der auslautende Vokal der \isi{Wurzel} getilgt: \textit{naiəre}→\textit{naiəre}-\textit{ənə}→\textit{naiər}-\textit{ənə} ‘Näherin’. Daraus entsteht ein Wort, auf dessen betonte Wurzelsilbe drei unbetonte Silben folgen. Da dies im gesamten System nicht möglich ist, wird von der Phonologie eine Silbe gekürzt (\textit{naiər}-\textit{nə} ‘Näherin’). Dazu braucht es folglich keine RR und die movierten Feminina können in die \isi{Flexionsklasse} 6 eingruppiert werden. Dass diese drei Silben unbetont sind, ist gesichert, da ihre Nuklei aus einem Schwa bestehen. Nicht gesichert ist jedoch, dass in einem solchen Fall eine Silbe automatisch gekürzt wird, da dieses Phänomen in diesem Dialekt noch nicht untersucht worden ist. Solche Restriktionen sind aber aus anderen Varietäten des Deutschen bekannt. Beispielsweise lauten in der deutschen Standardsprache Wörter auf einen Trochäus (minimaler Fuß = Zweisilber) bzw. Daktylus (maximaler Fuß = Dreisilber) aus \citep[130, 135]{Eisenberg2006}. Dies reguliert u.a., ob im Dativ Plural der \isi{Substantive} ein -\textit{n} oder -\textit{en} suffigiert wird \citep[167-169]{Eisenberg2006}. Dasselbe Phänomen wie in Huzenbach ist auch in den Dialekten von Petrifeld und Kaiserstuhl zu beobachten. In Petrifeld gehören movierte Feminina zur \isi{Flexionsklasse} 9 (vgl. \tabref{table5.10}), in der im Plural -\textit{ənə} suffigiert und der auslautende Vokal der \isi{Wurzel} getilgt wird. Bei den movierten Feminina wird zusätzlich eine Silbe gekürzt, was phonologisch bedingt ist: \textit{schaidəre}→\textit{schaidəre}-\textit{ənə}→\textit{schaidər}-\textit{ənə}→\textit{schaidərnə} ‘Schneiderin’. Genau dasselbe passiert im Dialekt des Kaiserstuhls, in dem movierte Feminina zur \isi{Flexionsklasse} 5 gehören (vgl. \tabref{table5.11}): \textit{lährəri}→\textit{lährəri}-\textit{ana}→\textit{lährər}-\textit{ana}→\textit{lährərna} ‘Lehrerin’.

\subsection{Blöcke}\label{5.1.5}

Wie in \sectref{4.1.3.2} dargestellt wurde, sind die RRs in \isi{Blöcken} organisiert. Dadurch wird gewährleistet, dass die RR in der richtigen Reihenfolge angewendet werden. Beispielsweise ist beim standarddeutschen Wort \textit{Bild-ər-n} (Dativ Plural) wichtig, dass zuerst aus der \isi{Wurzel} \textit{bild} der Plural \textit{bild-ər} entsteht und dann, auf \textit{bild-ər} basierend, der Dativ Plural \textit{bild-ər-n}. Wäre die Reihenfolge der RRs nicht definiert, könnte daraus auch *\textit{bild-n-ər} resultieren. Innerhalb eines Blockes konkurrieren die RRs miteinander, RRs unterschiedlicher \isi{Blöcke} hingegen nicht. Dadurch kann beispielsweise definiert werden, dass zur Pluralmarkierung maximal ein \isi{Suffix} verwendet wird, indem alle Pluralsuffixe im selben \isi{Block} stehen.

Alle hier untersuchten Varietäten entsprechen einem der vier Systeme an \isi{Blöcken} in den Tabellen \ref{table5.13} und \ref{table5.14}. \tabref{table5.13} zeigt jene Varietäten, die mindestens einen \isi{Umlaut} (\isi{Block} A) und Pluralsuffixe (\isi{Block} B) haben; in einigen Dialekten wird zudem subtrahiert (\isi{Block} C). Dass die \isi{Umlaute} und Pluralsuffixe in zwei verschiedenen \isi{Blöcken} stehen, hat zwei Gründe. Erstens stehen die \isi{Umlaute} einerseits und die Pluralsuffixe andererseits miteinander in Konkurrenz, wenn eine Flexion mehrere \isi{Umlaute} und Pluralsuffixe aufweist. Dies beschränkt die Anzahl der \isi{Umlaute} und Pluralsuffixe auf je eine Markierung pro Wort. Zwei \isi{Umlaute} oder zwei Pluralsuffixe kommen also nicht vor. Zweitens definieren diese \isi{Blöcke}, dass ein Wort sowohl einen \isi{Umlaut} als auch ein Pluralsuffix haben kann. Neben den RRs für die \isi{Umlaute} und die Pluralsuffixe braucht es in vielen Dialekten RRs für die \is{Subtraktion}Subtraktion (\isi{Block} C). Außer im Dialekt von Münstertal handelt es sich dabei in allen Dialekten um die Tilgung des auslautenden Vokals der \isi{Wurzel}, wenn ein Vokal folgt. In der Substantivflexion von Münstertal wird im Plural ein auslautendes \textit{t} getilgt.

Des Weiteren wird in einer \isi{Flexionsklasse} im Münstertal im Plural der Wurzelvokal diphthongiert. Da jedoch nicht in derselben \isi{Flexionsklasse} diphthongiert und umgelautet wird, kann die RR für die Diphthongierung ebenfalls in \isi{Block} A stehen (zusammen mit den \isi{Umlauten}). Schließlich wird in den meisten Dialekten in possessiven Kontexten bei Eigennamen ein -\textit{s} suffigiert. Da dieses \isi{Suffix} nur im Singular verwendet wird, kann die RR dazu in \isi{Block} B stehen. Denn die Pluralsuffixe sind auf den Plural und die Possessivsuffixe auf den Singular beschränkt, wodurch diese RRs nicht miteinander in Konflikt treten. Gleiches gilt für die Akkusativ/Dativ Singular Markierung im Dialekt von Elisabethtal. Im Dialekt von Vorarlberg positionieren sich die Pluralsuffixe sowie das \isi{Suffix} für den Dativ Plural in \isi{Block} B, denn ein sie werden nie an dasselbe Wort angehängt (vgl. Paradigma 9).

%{\tabref{table5.13}: \isi{Blöcke} der Substantivflexion der Varietäten ohne Kasusmarkierung im Plural}\\

\begin{table}
\caption{Blöcke der Substantivflexion der Varietäten ohne Kasusmarkierung im Plural}\label{table5.13}
\resizebox{\textwidth}{!}{\begin{tabular}{llll}
\lsptoprule
\isi{Block} A & \isi{Umlaut} & \isi{Block} A & \isi{Umlaut}\\
\isi{Block} B & \isi{Suffixe} Plural & \isi{Block} B & \isi{Suffixe} Plural\\
\isi{Block} C & RR Subtraktion &  & \\
\midrule
\multicolumn{2}{p{.5\linewidth}}{Gilt für folgende Varietäten:} & \multicolumn{2}{p{.5\linewidth}}{Gilt für folgende Varietäten:}\\
\multicolumn{2}{p{.5\linewidth}}{Sensebezirk (Poss-S), Huzenbach (Poss-S), Saulgau (Poss-S), Stuttgart, Petrifeld (Poss-S) (Dat.Sg.), Elisabethtal (Akk/Dat.Sg.), Kaiserstuhl (Poss-S), Münstertal (Poss-S) (Diphthong)}
& \multicolumn{2}{p{.5\linewidth}}{Bern (Poss-S), Vorarlberg (Poss-S) (Dat.Pl.), Colmar, Elsass (Ebene)}\\
\lspbottomrule
\end{tabular}}
\end{table}

%{\tabref{table5.14}: \isi{Blöcke} der Substantivflexion der Varietäten mit Kasusmarkierung im Plural}\\

\begin{table}
\caption{Blöcke der Substantivflexion der Varietäten mit Kasusmarkierung im Plural}\label{table5.14}
\resizebox{\textwidth}{!}{\begin{tabular}{llll}
\lsptoprule
\isi{Block} A & \isi{Umlaut} & \isi{Block} A & \isi{Umlaut}\\
\isi{Block} B & \isi{Suffixe} Plural & \isi{Block} B & \isi{Suffixe} Plural\\
\isi{Block} C & \isi{Suffixe} \isi{Kasus} & \isi{Block} C & \isi{Suffixe} \isi{Kasus}\\
\isi{Block} D & RR Subtraktion &  & \\
\midrule
\multicolumn{2}{p{.5\linewidth}}{Gilt für folgende Varietäten:} & \multicolumn{2}{p{.5\linewidth}}{Gilt für folgende Varietäten:}\\
\multicolumn{2}{p{.5\linewidth}}{Althochdeutsch, Mittelhochdeutsch, Uri (Poss-S)} & \multicolumn{2}{p{.5\linewidth}}{Issime, Visperterminen, Jaun, Zürich, Standard}\\
\lspbottomrule
\end{tabular}}
\end{table}

Auf die \isi{Blöcke} in \tabref{table5.14} treffen genau dieselben Beobachtungen zu. Die RR für die \is{Subtraktion}Subtraktion definieren im Althochdeutschen und im Dialekt von Uri, dass der auslautende Vokal der \isi{Wurzel} bei Suffigierung getilgt wird, im Mittelhochdeutschen, dass auslautendes \textit{w} getilgt wird. Die \isi{Blöcke} in \tabref{table5.14} unterscheiden sich von jenen in \tabref{table5.13} nur darin, dass ein zusätzlicher \isi{Block} für die Kasussuffixe angenommen werden muss. Der \isi{Block} mit den Pluralsuffixen steht vor dem \isi{Block} mit den Kasussuffixen. Dies definiert, dass, wenn im Plural \isi{Numerus} und \isi{Kasus} separat markiert werden, zuerst \isi{Numerus} und dann \isi{Kasus} ausgedrückt wird (z.B. \textit{Bild}-\textit{ər}-\textit{n}). Für die Kasussuffixe wird nur ein \isi{Block} und nicht je einer pro \isi{Numerus} benötigt. Denn entweder treten diese RR nicht direkt miteinander in Konkurrenz, da sie für den Singular oder für den Plural definiert sind, oder es handelt sich um einen \isi{Synkretismus}. In diesem Fall bleibt das Feature \isi{Numerus} in der RR unterspezifiziert. Dies kann an der \isi{Flexionsklasse} 4 des Althochdeutsch gut dargestellt werden (vgl. Paradigma 1). Betrachtet man nur diese \isi{Flexionsklasse}, sind folgende RR anzusetzen:

\ea%68
\label{ex:key:68}
 RR \textsubscript{C,} \textsubscript{\{\textsc{case:nom}} \textsubscript{\tiny $\veebar$}\textsubscript{ \AKK\},} \textsubscript{\textsc{n[}\textsc{ic:} 4]} ($\langle$X,$\sigma$ $\rangle$) = \textsubscript{def} $\langle$X\textit{i}ˊ,$\sigma$ $\rangle$
\z

\ea%69
\label{ex:key:69}
 RR \textsubscript{C,} \textsubscript{\{\textsc{case:dat}, \textsc{num:sg}\},} \textsubscript{\textsc{n[}\textsc{ic:} 4]} ($\langle$X,$\sigma$ $\rangle$) = \textsubscript{def} $\langle$X\textit{e}ˊ,$\sigma$ $\rangle$
\z

\ea%70
\label{ex:key:70}
 RR \textsubscript{C,} \textsubscript{\{\textsc{case:gen}, \textsc{num:sg}\},} \textsubscript{\textsc{n[}\textsc{ic:} 4]} ($\langle$X,$\sigma$ $\rangle$) = \textsubscript{def} $\langle$X\textit{es}ˊ,$\sigma$ $\rangle$
\z

\ea%71
\label{ex:key:71}
 RR \textsubscript{C,} \textsubscript{\{\textsc{case:dat}, \textsc{num:pl}\},} \textsubscript{\textsc{n[}\textsc{ic:} 4]} ($\langle$X,$\sigma$ $\rangle$) = \textsubscript{def} $\langle$X\textit{im}ˊ,$\sigma$ $\rangle$
\z

\ea%72
\label{ex:key:72}
 RR \textsubscript{C,} \textsubscript{\{\textsc{case:gen}, \textsc{num:pl}\},} \textsubscript{\textsc{n[}\textsc{ic:} 4]} ($\langle$X,$\sigma$ $\rangle$) = \textsubscript{def} $\langle$X\textit{o}ˊ,$\sigma$ $\rangle$
\z

RR \REF{ex:key:68} definiert, dass im Nominativ und Akkusativ Singular und Plural (\isi{Numerus} unterspezifiziert) -\textit{i} suffigiert wird. Die morphosyntaktischen Eigenschaften der RRs (\ref{ex:key:69}-\ref{ex:key:72} sind so definiert, dass die RR nicht gegeneinander in Konkurrenz treten.

\subsection{Von der Ortsgrammatik zu den Paradigmen und Realisierungsregeln}\label{5.1.6}

In diesem Kapitel soll die Systematisierungsarbeit anhand der Substantivflexion im Dialekt von Jaun vorgestellt werden. Es wird gezeigt, wie in dieser Arbeit aufgrund der Angaben in einer \isi{Ortsgrammatik} ein Paradigma erstellt wird und wie die RRs dieses Paradigmas formuliert werden. Eine Beschreibung dieser Systematisierungsarbeit mit allen Details (d.h. jede einzelne Entscheidung in allen 20 untersuchten Varietäten) kann hier nicht geleistet werden. Vielmehr sollen hier die wichtigsten Herausforderungen und Fragen anhand von repräsentativen Beispielen thematisiert werden.

{Grundsätzlich gilt:} \isi{Ortsgrammatiken} liefern die Datengrundlage der Dialekte in dieser Arbeit. Jede \isi{Ortsgrammatik} systematisiert die sprachlichen Daten auf unterschiedliche Weise. Damit aber die verschiedenen Varietäten miteinander verglichen werden können, müssen auf der Basis der Informationen in den \isi{Ortsgrammatiken} nach einheitlichen Regeln Paradigmen erstellt werden. Auf der Grundlage dieser Paradigmen wiederum werden die RRs hergeleitet.

Beide Schritte, also von den Quellen zum Paradigma und vom Paradigma zu den RRs, sind mit viel Analysen der Sprachdaten verknüpft. Als Hauptfragen ergeben sich: a) Ist die Modifikation einer Wortform phonologisch oder morphologisch bedingt? b) Wie wird segmentiert, d.h., wo fängt ein \isi{Suffix} an und wo hört es auf? c) Was ist das Minimum an \isi{Flexionsklassen}, die angenommen werden müssen? d) Welche \isi{Affixe} können in einer RRs zusammengefasst werden, sind also \isi{Synkretismen}, und welche nicht? Exemplarisch sollen nun die beiden Schritte (\isi{Ortsgrammatik} $\rightarrow$ Paradigma, Paradigma $\rightarrow$ RRs) anhand der Substantivflexion von Jaun dargestellt werden.

{Von der \isi{Ortsgrammatik} zum Paradigma:} Die Seiten zu den \isi{Substantiven} in der Grammatik des Dialekts von Jaun \citep{Stucki1917} sind in \citet{Stucki2011} abgebildet. Vergleicht man diesen Auszug in \citet{Stucki2011} mit dem Paradigma 6, das auf diesem Auszug basiert, wird offensichtlich, dass nicht einfach abgeschrieben werden kann, sondern dass die Daten analysiert und neu systematisiert werden müssen. Im Auszug der \isi{Ortsgrammatik} fällt erstens auf, dass kaum Paradigmen vorhanden sind. Ebenso gibt es keine expliziten Angaben und Erklärungen zu Fragen der Segmentierung der \isi{Affixe} sowie, ob ein \isi{Affix} morphosyntaktische Funktionen kodiert oder ob es sich dabei um eine phonologisch bedingte Modifikation der Wortform handelt. Zweitens werden die \isi{Substantive} in drei \isi{Genera} und in starke oder schwache Flexion eingeteilt, was sechs \isi{Flexionsklassen} ergeben würde. Wie aus dem Paradigma 6 jedoch ersichtlich wird, müssen für den Dialekt von Jaun 16 \isi{Flexionsklassen} angenommen werden. Drittens ist oft die Rede von germanischen Stämmen, die jedoch zum Teil schon für das Althochdeutsche problematisch sind (z.B. u-Stämme) und für die heutigen Dialekte nicht mehr zur Kategorisierung herangezogen werden können. Aus diesen Gründen und auch, weil jede \isi{Ortsgrammatik} eine eigene Definition von \isi{Flexionsklassen} besitzt, welche in den \isi{Ortsgrammatiken} jedoch nicht explizit thematisiert wird, werden in dieser Arbeit die \isi{Flexionsklassen} nach einheitlichen Regeln bestimmt, wie diese in \sectref{5.1.1} vorgestellt wurden. Folglich besteht das Erstellen der Paradigmen nicht in einem einfachen Abschreiben, sondern die Daten aus der \isi{Ortsgrammatik} bedürfen einer detaillierten und umfassenden Analyse, um die Paradigmen zu erstellen. Dies soll nun genauer dargestellt werden, indem u.a. auch auf Teilanalysen aus den vorangehenden Kapiteln Bezug genommen wird.

In \citeauthor{Stucki1917}s \citeyearpar{Stucki1917} Grammatik werden die Wurzelvokale mit ihren \textsc{Umlauten} gelistet und mit Beispielen belegt (s. \citealt[§200c]{Stucki2011}). Weiter diskutiert wird dies jedoch nicht, wie z.B., dass \textit{a} zwei umgelautete Entsprechungen hat ([ɛ] und [æ]), diese dem Primär- und Sekundärumlaut entsprechen, was in den heutigen Dialekten aber nicht mehr phonologisch bedingt ist und folglich Teil der Morphologie ist. Wie in \sectref{5.1.3} \is{Modifikation}(Modifikationen) gezeigt wurde, werden dafür also zwei verschiedene \isi{Flexionsklassen} benötigt. Aus demselben Grund müssen für die \isi{Substantive}, die den Plural auf -\textit{ər} bilden, zwei \isi{Flexionsklassen} angenommen werden: Es gibt \isi{Substantive} mit dem Plural auf -\textit{ər}, die den Wurzelvokal umlauten, und andere, die den Wurzelvokal nicht umlauten. Dass dies nicht mit der Phonologie begründet werden kann, steht nicht in der \isi{Ortsgrammatik} (vgl. \citealt[§205.2]{Stucki2011}), sondern beruht auf der hier vorgenommenen Analyse der Beispiele in der \isi{Ortsgrammatik}. Wann eine neue \isi{Flexionsklasse} angenommen wird, basiert auf der in dieser Arbeit verwendeten Definition von \isi{Flexionsklassen} (vgl. \sectref{5.1.1}), die nicht jener der \isi{Ortsgrammatik} entspricht.

Eine weitere Frage ist jene der \textsc{Segmentierung}, welche sich erstens gut anhand des Plurals der \isi{Flexionsklassen} 4, 14 und 16 illustrieren lässt (vgl. Paradigma 6). Die \isi{Ortsgrammatik} von Jaun gibt als Pluralmarker -\textit{əni} und -\textit{ənə} an (vgl. \citealt[§206, §211 und §212.2]{Stucki2011}). Eine weitere mögliche Analyse wäre, -\textit{ən} als Pluralmarker, -\textit{i} als Marker für Nominativ/Akkusativ Plural und -\textit{ə} für Dativ/Genitiv Plural anzunehmen, zumal -\textit{i} und -\textit{ə} auch in den \isi{Flexionsklassen} 11 und 15 für dieselben \isi{Kasus} verwendet werden (vgl. Paradigma 6). Wie in \sectref{5.1.4} dargestellt wurde, ist die zweite Analyse (-\textit{ən}-\textit{i} und -\textit{ən}-\textit{ə}) ökonomischer, da eine RR weniger angenommen werden muss, weswegen die zweite mögliche Analyse und nicht jene der \isi{Ortsgrammatik} ausgewählt wird. Betroffen vom Problem der Segmentierung ist zweitens auch der Singular der \isi{Flexionsklassen} 6, 15 und 16 (vgl. Paradigma 6; \citealt[§203.2.b.β und §212]{Stucki2011}). Für die \isi{Flexionsklasse} 6 kann von einer \isi{Wurzel} \textit{chaschtə} ‘Kasten’ und nicht \textit{chascht}-\textit{ə} ausgegangen werden, da alle Zellen des Paradigmas über diese Form verfügen. Für den Nominativ/Akkusativ Plural darf angenommen werden, dass, wenn zwei zentralisierte Vokale mit derselben Quantität und Qualität aufeinandertreffen, diese verschmelzen (*\textit{chaschtə}-\textit{ə} > \textit{chascht}-\textit{ə}). Bezüglich der \isi{Flexionsklassen} 15 und 16 sind auch für den Singular \isi{Suffixe} anzunehmen: \textit{tsung}-\textit{a} (Nominativ/Akkusativ Singular), \textit{tsung}-\textit{ə} (Dativ/Genitiv Singular + Plural), \textit{tsung}-\textit{i} (Nominativ/Akkusativ Plural). Würde man \textit{tsunga} als \isi{Wurzel} annehmen, müsste man erklären, weshalb das auslautende \textit{a} getilgt wird, wenn ein \textit{ə} oder \textit{i} suffigiert wird. Außerdem wäre dafür eine RR nötig, da diese Tilgung kein phonologischer Automatismus sein kann, denn der hiatvermeidende Automatismus ist die n-Epenthese. Aufgrund dieses Paradigmas scheint es also plausibler, eine RR für den Nominativ/Akkusativ Singular anzusetzen, die ein -\textit{a} suffigiert, als eine RR, die ein wurzelauslautendes \textit{a} tilgt. Dazu, wie diese \isi{Substantive} zu segmentieren sind, finden sich keine Angaben in der \isi{Ortsgrammatik}. Die Segmentierung muss selbst vorgenommen werden. Besonders aufwändig in der Analyse sind drittens die \isi{Substantive} auf -\textit{i}, z.B.: \textit{schieri} (Sg.), \textit{schierəni} (Nominativ/Akkusativ Plural), \textit{schierənə} (Dativ/Genitiv Plural) ʻSchereʼ (\citealt[§206 und §211]{Stucki2011}). Die \isi{Ortsgrammatik} von Jaun beschreibt diese knapp als auf \textit{i} auslautende \isi{Substantive}, die den Plural auf -\textit{əni} bilden (\citealt[§206 und §211]{Stucki2011}). Es stellen sich hier folgende Fragen: a) Ist das \textit{i} im Singular Teil der \isi{Wurzel} oder ein \isi{Suffix}? b) Lautet der Plural -\textit{əni}/-\textit{ənə} (neue \isi{Flexionsklasse}) oder -\textit{ən}-\textit{i}/-\textit{ən}-\textit{ə} (wie \isi{Flexionsklassen} 4, 14, 16)? c) Wenn \textit{i} Teil der \isi{Wurzel} ist, wäre auch eine Zentralisierung von \textit{i} vorstellbar, wenn es im Inlaut steht, d.h., wenn ein \isi{Suffix} folgt (\textit{schieri} > \textit{schierə}-\textit{ni}). Dann hätte man es mit einem phonologischen Prozess zu tun, für den folglich keine RRs angesetzt werden müssen. Angenommen es handelt sich bei i>ə um einen phonologischen Prozess, dann ist weiter der Frage nachzugehen, ob der Plural -\textit{ni}/-\textit{nə}  oder -\textit{i}/-\textit{ə} (mit n-Epenthese zur Tilgung des \isi{Hiats}) lautet. Wie in \sectref{5.1.4} dargestellt wurde (vgl. auch \tabref{table5.6}), ist es am plausibelsten und ökonomischsten, von einer \isi{Wurzel} auf \textit{i} auszugehen und von einem Plural auf -\textit{i}/-\textit{ə} mit n-Epenthese zur Tilgung des \isi{Hiats} und mit Zentralisierung von \textit{i} zu \textit{ə}. Die n-Epenthese und die Zentralisierung von \textit{i} zu \textit{ə} sind phonologische Mechanismen (herauszufinden anhand des Teils zur Phonologie in der \isi{Ortsgrammatik}), d.h., sie finden immer und automatisch statt, folglich müssen dafür keine RRs angenommen werden. Daraus ergibt sich weiter, dass der Plural -\textit{i}/-\textit{ə} lautet und dass das \textit{i} zur \isi{Wurzel} gehört. Des Weiteren muss für auf \textit{i} auslautende \isi{Substantive} keine eigene \isi{Flexionsklasse} angesetzt werden, denn diese \isi{Substantive} funktionieren aufgrund dieser Analyse exakt wie \isi{Substantive}, für die auf jeden Fall eigene \isi{Flexionsklassen} angenommen werden müssen (vgl. \tabref{table5.6} in \sectref{5.1.4}).

Anhand dieser Beispiele wird klar, wie vielschrittig der Weg von der \isi{Ortsgrammatik} bis zum Paradigma erfordert. Das Ziel ist also, mithilfe der Menge an Daten in der \isi{Ortsgrammatik} eine linguistisch adäquate Analyse durchzuführen und somit eine adäquate Beschreibung zu erhalten. Gleichzeitig soll die Beschreibung, d.h. das Paradigma, so ökonomisch, also so kurz wie möglich ausfallen, um Redundanzen zu vermeiden, die aus der Beschreibung entstehen und nicht aus dem Sprachsystem resultieren.

{Vom Paradigma zu den RRs:} Nun wird noch gezeigt, wie vorgegangen wird, um auf der Basis des Paradigmas 6 die RRs zu bestimmen. Als erstes muss die Anzahl der \isi{Blöcke} eruiert werden. Der Dialekt von Jaun unterscheidet \isi{Umlaute} sowie \isi{Suffixe}, die nur \isi{Numerus} markieren, und \isi{Suffixe}, die nur \isi{Kasus} markieren. Wie in \sectref{5.1.5} gezeigt wurde, sind also drei \isi{Blöcke} anzunehmen, um die Wortform korrekt aufzubauen: \isi{Block} A für \isi{Umlaute}, \isi{Block} B für Numerussuffixe, \isi{Block} C für Kasussuffixe.

Schritt für Schritt werden nun die RR eingeführt. Da ein Primär- und Sekundärumlaut unterschieden wird, braucht es dafür zwei RRs in \isi{Block} A:

\ea%73
\label{ex:key:73}
 RR \textsubscript{A, \{\textsc{num:pl}\}, \textsc{n[}\textsc{ic:}1} \textsubscript{\tiny $\veebar$}\textsubscript{ 6} \textsubscript{\tiny $\veebar$}\textsubscript{ 9} \textsubscript{\tiny $\veebar$}\textsubscript{ 13]} ($\langle$X,$\sigma$ $\rangle$) = \textsubscript{def} $\langle$Ẍˊ,$\sigma$ $\rangle$
\z

\ea%74
\label{ex:key:74}
 RR \textsubscript{A, \{\textsc{num:pl}\}, \textsc{n[}\textsc{ic:}3]} ($\langle$X,$\sigma$ $\rangle$) = \textsubscript{def} $\langle$Ẍ[\textit{a} $\rightarrow$ \textit{e}]ˊ,$\sigma$ $\rangle$
\z

Es folgen in \isi{Block} B die RRs, die nur Plural markieren. Es handelt sich dabei um drei \isi{Suffixe}, nämlich -\textit{ən}, -\textit{ə} und -\textit{ər}:

\ea%75
\label{ex:key:75}
 RR \textsubscript{B, \{\textsc{num:pl}\}, \textsc{n[}\textsc{ic:}4} \textsubscript{\tiny $\veebar$}\textsubscript{ 14} \textsubscript{\tiny $\veebar$}\textsubscript{ 16]} ($\langle$X,$\sigma$ $\rangle$) = \textsubscript{def} $\langle$X\textit{ən}ˊ,$\sigma$ $\rangle$
\z

\ea%76
\label{ex:key:76}
 RR \textsubscript{B, \{\textsc{num:pl}\}, \textsc{n[}\textsc{ic:}7]} ($\langle$X,$\sigma$ $\rangle$) = \textsubscript{def} $\langle$X\textit{ə}ˊ,$\sigma$ $\rangle$
\z

\ea%77
\label{ex:key:77}
 RR \textsubscript{B, \{\textsc{num:pl}\}, \textsc{n[}\textsc{ic:}9} \textsubscript{\tiny $\veebar$}\textsubscript{ 10]} ($\langle$X,$\sigma$ $\rangle$) = \textsubscript{def} $\langle$X\textit{ər}ˊ,$\sigma$ $\rangle$
\z

In \isi{Block} C stehen die RRs für \isi{Kasus}. Aus dem Paradigma 6 ist ersichtlich, dass im Plural Nominativ und Akkusativ immer zusammenfallen. Für beide \isi{Kasus} ist also nur eine RR nötig. Die drei Allomorphe sind -\textit{a}, -\textit{ə} und -\textit{i}:

\ea%78
\label{ex:key:78}
 RR \textsubscript{C, \{\textsc{case:nom}} \textsubscript{\tiny $\veebar$}\textsubscript{ \textsc{acc},\textsc{num:pl}\}, \textsc{n[}\textsc{ic:}1]} ($\langle$X,$\sigma$ $\rangle$) = \textsubscript{def} $\langle$X\textit{a}ˊ,$\sigma$ $\rangle$
\z

\ea%79
\label{ex:key:79}
 RR \textsubscript{C, \{\textsc{case:nom}} \textsubscript{\tiny $\veebar$}\textsubscript{ \textsc{acc},\textsc{num:pl}\}, \textsc{n[}\textsc{ic:}5} \textsubscript{\tiny $\veebar$}\textsubscript{ 6]} ($\langle$X,$\sigma$ $\rangle$) = \textsubscript{def} $\langle$X\textit{ə}ˊ,$\sigma$ $\rangle$
\z

\ea%80
\label{ex:key:80}
 RR \textsubscript{C, \{\textsc{case:nom}} \textsubscript{\tiny $\veebar$}\textsubscript{ \textsc{acc},\textsc{num:pl}\}, \textsc{n[}\textsc{ic:}4} \textsubscript{\tiny $\veebar$}\textsubscript{ 11} \textsubscript{\tiny $\veebar$}\textsubscript{ 12} \textsubscript{\tiny $\veebar$}\textsubscript{ 14} \textsubscript{\tiny $\veebar$}\textsubscript{ 15} \textsubscript{\tiny $\veebar$}\textsubscript{ 16]} ($\langle$X,$\sigma$ $\rangle$) = \textsubscript{def} $\langle$X\textit{i}ˊ,$\sigma$ $\rangle$
\z

Auch der Dativ und der Genitiv Plural werden nicht unterschieden. Es gibt zwei Allomorphe, nämlich -\textit{ə} und -\textit{nə}:

\ea%81
\label{ex:key:81}
 RR \textsubscript{C, \{\textsc{case:dat}} \textsubscript{\tiny $\veebar$}\textsubscript{ \GEN,\textsc{num:pl}\}, \textsc{n[}\textsc{ic:}1} \textsubscript{\tiny $\veebar$}\textsubscript{ 2} \textsubscript{\tiny $\veebar$}\textsubscript{ 3} \textsubscript{\tiny $\veebar$}\textsubscript{ 4} \textsubscript{\tiny $\veebar$}\textsubscript{ 8} \textsubscript{\tiny $\veebar$}\textsubscript{ 9} \textsubscript{\tiny $\veebar$}\textsubscript{ 10} \textsubscript{${\veebar}$}\textsubscript{ 11} \textsubscript{${\veebar}$}\textsubscript{ 12} \textsubscript{${\veebar}$}\textsubscript{ 13} \textsubscript{${\veebar}$}\textsubscript{ 14} \textsubscript{${\veebar}$}\textsubscript{ 15} \textsubscript{${\veebar}$}\textsubscript{ 16]} ($\langle$X,$\sigma$ $\rangle$) = \textsubscript{def} $\langle$X\textit{ə}ˊ,$\sigma$ $\rangle$
\z

\ea%82
\label{ex:key:82}
 RR \textsubscript{C, \{\textsc{case:dat}} \textsubscript{\tiny $\veebar$}\textsubscript{ \GEN,\textsc{num:pl}\}, \textsc{n[}\textsc{ic:}5} \textsubscript{\tiny $\veebar$}\textsubscript{ 6]} ($\langle$X,$\sigma$ $\rangle$) = \textsubscript{def} $\langle$X\textit{nə}ˊ,$\sigma$ $\rangle$
\z

Im Singular wird der Genitiv in den \isi{Flexionsklassen} 1 bis 10 mit dem \isi{Suffix} -\textit{s} markiert:

\ea%83
\label{ex:key:83}
 RR \textsubscript{C, \{\textsc{case:gen},\textsc{num:sg}\}, \textsc{n[}\textsc{ic:}1} \textsubscript{\tiny $\veebar$}\textsubscript{ 2} \textsubscript{\tiny $\veebar$}\textsubscript{ 3} \textsubscript{\tiny $\veebar$}\textsubscript{ 4} \textsubscript{\tiny $\veebar$}\textsubscript{ 5} \textsubscript{\tiny $\veebar$}\textsubscript{ 6} \textsubscript{\tiny $\veebar$}\textsubscript{ 7} \textsubscript{\tiny $\veebar$}\textsubscript{ 8} \textsubscript{${\veebar}$}\textsubscript{ 9} \textsubscript{${\veebar}$}\textsubscript{ 10} \textsubscript{${\veebar}$}\textsubscript{ 11]} ($\langle$X,$\sigma$ $\rangle$) = \textsubscript{def} $\langle$X\textit{s}ˊ,$\sigma$ $\rangle$\\
\z

Schließlich müssen noch die RRs für den Singular der \isi{Flexionsklassen} 15 und 16 bestimmt werden. Es stellen sich hier einige Fragen bezüglich der \isi{Synkretismen}. Das \isi{Suffix} -\textit{a} kommt sowohl im Nominativ/Akkusativ Singular der \isi{Flexionsklassen} 15 und 16 als auch im Nominativ/Akkusativ Plural vor. der \isi{Flexionsklasse} 1. Da sich jedoch diese \isi{Suffixe} in mehr als einer Eigenschaft unterscheiden (detailliert beschrieben in \sectref{4.1.3.3}), nämlich im \isi{Numerus} und \isi{Flexionsklasse}, braucht es dafür zwei RRs: eine RRs für den ominativ/Akkusativ Plural der \isi{Flexionsklasse} 1 (\ref{ex:key:78}) und eine für den Nominativ/Akkusativ Singular der \isi{Flexionsklassen} 15 und 16:

\ea%84
\label{ex:key:84}
 RR \textsubscript{C, \{\textsc{case:nom}} \textsubscript{\tiny $\veebar$}\textsubscript{ \textsc{acc},\textsc{num:sg}\}, \textsc{n[}\textsc{ic:}15} \textsubscript{\tiny $\veebar$}\textsubscript{ 16]} ($\langle$X,$\sigma$ $\rangle$) = \textsubscript{def} $\langle$X\textit{a}ˊ,$\sigma$ $\rangle$\\
\z

Der Dativ/Genitiv Singular der \isi{Flexionsklassen} 15 und 16 fällt mit dem Dat/Gen.Pl. derselben \isi{Flexionsklassen} zusammen (-\textit{ə}). Sie unterscheiden sich also nur in einer Eigenschaft, nämlich \isi{Numerus}, folglich bräuchte es nur eine RR. Es gibt also zwei Möglichkeiten, denn alle \isi{Flexionsklassen} außer den \isi{Flexionsklassen} 5 und 6 bilden den Dativ/Genitiv Plural ebenfalls auf -\textit{ə}: a) eine RR für Dativ/Genitiv Singular und Dativ/Genitiv Plural der \isi{Flexionsklassen} 15 und 16 sowie eine RR für den Dativ/Genitiv Plural aller \isi{Flexionsklassen} außer 15 und 16 sowie 5 und 6 (anderes \isi{Suffix}), b) eine RR für Dativ/Genitiv Singular der \isi{Flexionsklassen} 15 und 16 sowie eine für den Dativ/Genitiv Plural aller \isi{Flexionsklassen} außer der \isi{Flexionsklassen} 5 und 6. Welche Analyse gewählt wird, hat also keinen Einfluss auf die Anzahl RRs. Es stellt sich folglich die Frage, was morphologisch gesehen adäquater erscheint. Mit zwei Ausnahmen markieren alle \isi{Flexionsklassen} den Dativ/Genitiv Plural mit dem \isi{Suffix} -ə, aber nur die \isi{Flexionsklassen} 15 und 16 zeigen einen \isi{Synkretismus} zwischen dem Dativ/Genitiv Singular und dem Dativ/Genitiv Plural. Deswegen wird hier die Analyse b) gewählt:

\ea%85
\label{ex:key:85}
 RR \textsubscript{C, \{\textsc{case:dat}} \textsubscript{\tiny $\veebar$}\textsubscript{ \GEN,\textsc{num:sg}\}, \textsc{n[}\textsc{ic:}15} \textsubscript{\tiny $\veebar$}\textsubscript{ 16]} ($\langle$X,$\sigma$ $\rangle$) = \textsubscript{def} $\langle$X\textit{ə}ˊ,$\sigma$ $\rangle$\\
\z

\section{Adjektive}\label{5.2}

\subsection{Allgemeines und Realisierungsregeln, starke und schwache Flexion}\label{5.2.1}

\subsubsection{Starke und schwache Flexion} Alle der hier untersuchten Varietäten unterscheiden in den Adjektiven eine starke und schwache Flexion. In welchem syntaktischen Kontext eine stark oder schwach flektierte Form verwendet wird, regelt die Syntax. Da die Distribution also syntaktisch bedingt ist, muss die Morphologie lediglich die Formen zur Verfügung stellen. Dies ist auch der Grund, weshalb keine sogenannte gemischte Flexion angenommen wird, wie dies manchmal in Bezug auf die Standardsprache gemacht wird: Mit den Paradigmen der starken und schwachen Flexion sind bereits alle Formen vorhanden.

\subsubsection{Definition der Form} In allen Varietäten dieses Samples wird in der Adjektivflexion ausschließlich suffigiert. Eine Ausnahme bilden nur die wa-/w\=o-Stäm\-me im Mittelhochdeutschen, auf die im folgenden Kapitel eingegangen wird. Die morphosyntaktischen Eigenschaften, die in den RRs definiert werden müssen, sind die folgenden: Numerus, Kasus, Genus, starke oder schwache Flexion. Mit Ausnahmen des Mittelhochdeutschen werden keine Blöcke benötigt, da nie mehr als ein Suffix auftritt. Speziell hervorzuheben ist hier nur, dass alle höchstalemannischen Dialekte (außer Visperterminen) im Plural der starken Flexion Genus unterscheiden. Dies kann sicher als Archaismus gewertet werden, da auch das Alt- und Mittelhochdeutsche eine Genusunterscheidung im Plural der starken Flexion (Althochdeutsch teils auch in der schwachen Flexion) aufweisen. In allen anderen der hier untersuchten alemannischen Dialekten sind die Genera im Plural zusammengefallen.

\subsubsection{Beispiel} Anhand der Adjektivflexion von Issime soll nun illustriert werden, wie ein System an RRs für die Adjektivflexion aussieht. \tabref{table5.15} zeigt das Paradigma der starken und schwachen Adjektivflexion in Issime, \REF{ex:key:86}-\REF{ex:key:96} die dazugehörigen RRs. 

%{\tabref{table5.15}: Starke und schwache Adjektivflexion in Issime anhand des Lexems \textit{naw}} {‘neu’ (\citealt[90-97]{Perinetto1981}, \citealt[267-268]{Zürrer1999}}\\

\begin{table}
\caption{Starke und schwache Adjektivflexion in Issime anhand des Lexems \textit{naw} ‘neu’ (\citealt[90-97]{Perinetto1981}, \citealt[267-268]{Zürrer1999})}\label{table5.15}
\resizebox{\textwidth}{!}{\begin{tabular}{*{9}{l}}
\lsptoprule
\multicolumn{9}{l}{{stark}}\\
& \multicolumn{4}{c}{\textsc{singular}} & \multicolumn{4}{c}{\textsc{plural}}\\\cmidrule(lr){2-5}\cmidrule(lr){6-9}
& \NOM & \AKK & \DAT & \GEN & \NOM & \AKK & \DAT & \GEN\\\midrule
\textsc{m} & naw-e & naw-e & naw-e & naw-s & naw-ø & naw-ø & naw-ø & naw-er\\
\textsc{n} & naw-s & naw-s & naw-s & naw-s & naw-i & naw-i & naw-i & naw-er\\
\textsc{f} & naw-ø & naw-ø & naw-ø & naw-er & naw-ø & naw-ø & naw-ø & naw-er\\
\midrule
\multicolumn{9}{l}{{schwach}} \\
& \multicolumn{4}{c}{\textsc{singular}} & \multicolumn{4}{c}{\textsc{plural}}\\\cmidrule(lr){2-5}\cmidrule(lr){6-9}
& \NOM & \AKK & \DAT & \GEN & \NOM & \AKK & \DAT & \GEN\\\midrule
\textsc{m} & naw-e & naw-e & naw-e & naw-e & \multirow{3}{*}{naw-u} & \multirow{3}{*}{naw-u} & \multirow{3}{*}{naw-e} & \multirow{3}{*}{naw-u}\\
\textsc{n} & naw-ø & naw-ø & naw-e & naw-e &  &  &  & \\
\textsc{f} & naw-u & naw-u & naw-u & naw-u &  &  &  & \\
\lspbottomrule
\end{tabular}}
\end{table}

\ea%86
\label{ex:key:86}
 RR \textsubscript{A,} \textsubscript{\{\textsc{case:nom}} \textsubscript{\tiny $\veebar$}\textsubscript{ \AKK} \textsubscript{\tiny $\veebar$}\textsubscript{ \DAT, \textsc{num:sg}, \textsc{gend:m}\},} \textsubscript{\textsc{adj[]}} ($\langle$X,$\sigma$ $\rangle$) = \textsubscript{def} $\langle$X\textit{e}ˊ,$\sigma$ $\rangle$
\z

\ea%87
\label{ex:key:87}
 RR \textsubscript{A,} \textsubscript{\{\textsc{case:nom}} \textsubscript{\tiny $\veebar$}\textsubscript{ \AKK} \textsubscript{\tiny $\veebar$}\textsubscript{ \DAT, \textsc{num:sg}, \textsc{gend:n}\},} \textsubscript{\textsc{adj[strong]}} ($\langle$X,$\sigma$ $\rangle$) = \textsubscript{def} $\langle$X\textit{s}ˊ,$\sigma$ $\rangle$
\z

\ea%88
\label{ex:key:88}
 RR \textsubscript{A,} \textsubscript{\{\textsc{case:gen}, \textsc{num:sg}, \textsc{gend:m}} \textsubscript{\tiny $\veebar$}\textsubscript{ \textsc{n}\},} \textsubscript{\textsc{adj[strong]}} ($\langle$X,$\sigma$ $\rangle$) = \textsubscript{def} $\langle$X\textit{s}ˊ,$\sigma$ $\rangle$
\z

\ea%89
\label{ex:key:89}
 RR \textsubscript{A,} \textsubscript{\{\textsc{case:gen}, \textsc{num:sg}, \textsc{gend:m}} \textsubscript{\tiny $\veebar$}\textsubscript{ \textsc{n}\},} \textsubscript{\textsc{adj[weak]}} ($\langle$X,$\sigma$ $\rangle$) = \textsubscript{def} $\langle$X\textit{e}ˊ,$\sigma$ $\rangle$ 
\z

\ea%90
\label{ex:key:90}
 RR \textsubscript{A,} \textsubscript{\{\textsc{case:dat}, \textsc{num:sg}, \textsc{gend:n}\},} \textsubscript{\textsc{adj[weak]}} ($\langle$X,$\sigma$$\rangle$) = \textsubscript{def} $\langle$X\textit{e}ˊ,$\sigma$$\rangle$
\z

\ea%91
\label{ex:key:91}
 RR \textsubscript{A,} \textsubscript{\{\textsc{case:gen}, \textsc{num:sg}, \textsc{gend:f}\},} \textsubscript{\textsc{adj[strong]}} ($\langle$X,$\sigma$$\rangle$) = \textsubscript{def} $\langle$X\textit{er}ˊ,$\sigma$$\rangle$
\z

\ea%92
\label{ex:key:92}
 RR \textsubscript{A,} \textsubscript{\{\textsc{num:sg}, \textsc{gend:f}\},} \textsubscript{\textsc{adj[weak]}} ($\langle$X,$\sigma$$\rangle$) = \textsubscript{def} $\langle$X\textit{u}ˊ,$\sigma$$\rangle$
\z

\ea%93
\label{ex:key:93}
 RR \textsubscript{A,} \textsubscript{\{\textsc{case:nom}} \textsubscript{\tiny $\veebar$}\textsubscript{ \AKK} \textsubscript{\tiny $\veebar$}\textsubscript{ \GEN, \textsc{num:pl}\},} \textsubscript{\textsc{adj[weak]}} ($\langle$X,$\sigma$$\rangle$) = \textsubscript{def} $\langle$X\textit{u}ˊ,$\sigma$$\rangle$
\z

\ea%94
\label{ex:key:94}
 RR \textsubscript{A,} \textsubscript{\{\textsc{case:dat}, \textsc{num:pl}\},} \textsubscript{\textsc{adj[weak]}} ($\langle$X,$\sigma$$\rangle$) = \textsubscript{def} $\langle$X\textit{e}ˊ,$\sigma$$\rangle$
\z

\ea%95
\label{ex:key:95}
 RR \textsubscript{A,} \textsubscript{\{\textsc{case:gen}, \textsc{num:pl}\},} \textsubscript{\textsc{adj[strong]}} ($\langle$X,$\sigma$$\rangle$) = \textsubscript{def} $\langle$X\textit{er}ˊ,$\sigma$$\rangle$
\z

\ea%96
\label{ex:key:96}
 RR \textsubscript{A,} \textsubscript{\{\textsc{case:nom}} \textsubscript{\tiny $\veebar$}\textsubscript{ \AKK} \textsubscript{\tiny  $\veebar$}\textsubscript{ \DAT, \textsc{num:pl}, \textsc{gend:n}\},} \textsubscript{\textsc{adj[strong]}} ($\langle$X,$\sigma$$\rangle$) = \textsubscript{def} $\langle$X\textit{i}ˊ,$\sigma$$\rangle$
\z

Aus den RRs wird ersichtlich, dass keine RR anzusetzen ist, wenn dem \isi{Adjektiv} keine Endung suffigiert wird (im Paradigma -ø). Hier wirkt die RR \textit{Identitiy Function Default} (vgl. RR \REF{ex:key:26} in \sectref{4.1.3.2}, die definiert, dass an der \isi{Wurzel} keine Veränderungen vorgenommen werden, wenn keine RR vorhanden ist). Diese RR muss für jede Varietät dieses Samples angenommen werden. Des Weiteren zeigen die RRs (\ref{ex:key:86}--\ref{ex:key:96}), dass nicht nur Kasus- und Genussynkretismen abgebildet werden können (z.B. Kasussynkretismus RR \REF{ex:key:86} mittels Disjunktion, \REF{ex:key:92} mittels Unterspezifikation; Genussynkretismus RR \REF{ex:key:88} mittels Disjunktion, \REF{ex:key:95} mittels Unterspezifikation), sondern auch jener \isi{Synkretismus}, wenn starke und schwache Flexion nicht unterschieden werden (\REF{ex:key:86}, mittels Unterspezifikation). Weisen die starke und schwache Flexion für ein bestimmtes Bündel an morphosyntaktischen Einheiten identische Formen auf, kann die Art der Flexion (stark/schwach) unterspezifiziert bleiben. Dasselbe gilt, wenn \isi{Numerus} nicht unterschieden würde, was in der Adjektivflexion von Issime nicht vorkommt.

\subsection{Wa-/w\=o-Stämme}\label{5.2.2}

Wa-/w\=o-Stäm\-me sind nur im Alt- und Mittelhochdeutschen vorhanden, die wie die wa-/w\=o-Stäm\-me der \isi{Substantive} behandelt werden (vgl. \sectref{5.1.3}). Im Althochdeutschen ist die Vokalisierung von \textit{w} zu \textit{o} phonologisch bedingt (also keine RR), während im Mittelhochdeutschen die Tilgung des \textit{w} aus Mangel an einer möglichen synchronen phonologischen Erklärung in der Morphologie zu verorten ist.

Aus \tabref{table5.16} wird ersichtlich, dass im Althochdeutschen die wa-/w\=o-Stäm\-me dieselben Flexionsendungen aufweisen wie die a-Stäm\-me. Dafür, dass es sich beim auslautenden \textit{o} im Nominativ Singular der wa-/w\=o-Stäm\-me nicht um eine Flexionsendung handelt, sprechen zwei Gründe. Erstens weisen auch a-Stäm\-me keine Endung auf (\textit{blint} ‘blind’) und da das Set an \isi{Suffixen} der wa-/w\=o-Stäm\-me jedem der a-Stäm\-me entspricht, würde es nur wenig Sinn machen, im Nominativ von einer Ausnahme auszugehen. Vielmehr wissen wir aus der Diskussion zu den \isi{Substantiven}, dass \textit{w} zu \textit{o} vokalisiert wird, wenn es im Auslaut steht. Es passt also ins Gesamtsystem, wenn man annimmt, dass aus *\textit{garw} > \textit{garo} ‘bereit’ entsteht. Zweitens wird die \isi{Wurzel} (\textit{garw}-) verwendet, wenn ein \isi{Suffix} folgt, d.h. auch ein \isi{Suffix} auf -\textit{o}. Dafür spricht die Form \textit{garawu} des Instrumentals. Die w- und a-Stäm\-me fallen folglich zusammen und die Vokalisierung ist phonologisch bedingt, weshalb dafür keine RR nötig ist. Der Sprossvokal \textit{a} (\textit{gara}\textit{w}-) und weshalb \textit{w} zu \textit{o} (und nicht zu \textit{u}) vokalisiert wird, wurde genauer in \sectref{5.1.3} besprochen.

%{\tabref{table5.16}: Wa-/w\=o-Stämme im Althochdeutschen am Beispiel der Lexeme \textit{garo}} {‘bereit’ und \textit{blint}} {‘blind’ \citep[220, 225]{Braune2004}}\\

\begin{table}
\caption{ Wa-/w\=o-Stämme im Althochdeutschen am Beispiel der Lexeme \textit{garo} ‘bereit’ und \textit{blint} ‘blind’ \citep[220, 225]{Braune2004}}\label{table5.16}
\resizebox{\textwidth}{!}{\begin{tabular}{llllll}
\lsptoprule
\multicolumn{6}{l}{starke Flexion, Singular, Maskulin, wa-/w\=o-Stämme}\\
& {\NOM} & {\AKK} & {\DAT} & {\GEN} & {\INSTR}\\\midrule
{wa-/w\=o-Stämme} & garaw-\=er/garo & garaw-an & garaw-emo & garaw-es & garaw-u\\
{a-Stämme} & blint-\=er/blint & blint -an & blint -emo & blint -es & blint -u\\
\lspbottomrule
\end{tabular}}
\end{table}

Im Mittelhochdeutschen ist die \isi{Variation} nicht phonologisch bedingt, was bereits in \sectref{5.1.3} erörtert wurde. Es wurde gezeigt, dass weder eine Tilgung von \textit{w} noch eine Einfügung von \textit{w} phonologisch voraussagbar ist. Folglich ist die \isi{Variation} durch eine RR zu definieren. Wie bei den \isi{Substantiven} wird auch bei den \isi{Adjektiven} davon ausgegangen, dass aus dem \isi{Radikon} auf \textit{w} auslautende \isi{Wurzeln} kommen. Nachdem durch RRs Endungen suffigiert worden sind (\isi{Block} A), wird in einem zweiten \isi{Block} (B) eine RR benötigt, die \textit{w} tilgt, wenn es im Auslaut steht:

\ea%97
\label{ex:key:97}
 RR \textsubscript{B,} \textsubscript{\{\},} \textsubscript{\textsc{adj[]}} ($\langle$X,$\sigma$ $\rangle$) = \textsubscript{def} $\langle$X *\textit{w} $\rightarrow$ ø/\_\#ˊ,$\sigma$ $\rangle$\\
\z

\subsection{Freie Variation}\label{5.2.3}

In etlichen der hier untersuchten Varietäten kommen zwei verschiedene Formen in derselben Zelle des Paradigmas vor. Da die Distribution nicht weiter erklärt werden kann, ist von freier \isi{Variation} auszugehen. Es können prinzipiell zwei Arten von freier \isi{Variation} unterschieden werden, deren RRs aber gleich aussehen: Erstens weist eine Zelle des Paradigmas zwei \isi{Suffixe} auf, zweitens weist eine Zelle ein \isi{Suffix} und zusätzlich nur die \isi{Wurzel} auf. Dies soll anhand der starken Flexion im Althochdeutschen dargestellt werden.

\tabref{table5.17} zeigt, dass an die a-Stäm\-me im Nominativ Singular -\textit{er} (pronominale Endung) oder auch nichts (nominal) suffigiert werden kann. An die i-Stäm\-me können -\textit{er} (pronominale Endung) oder -\textit{i} (nominale Endung) suffigiert werden.

%{\tabref{table5.17}: Freie \isi{Variation} in der starken Flexion des Althochdeutschen \citep[220, 223]{Braune2004}}\\

\begin{table}
\caption{Freie Variation in der starken Flexion des Althochdeutschen \citep[220, 223]{Braune2004}}\label{table5.17}
\begin{tabular}{ll}
\lsptoprule
& \textsc{nom.sg.}\\\midrule
{a-Stämme} & blint-\=er / blint ‘blint’\\
{i-Stämme} & m\=ar-\=er / m\=ar-i ‘berühmt’\\
\lspbottomrule
\end{tabular}
\end{table}

Wie bereits für die \isi{Substantive} dargestellt (vgl. \sectref{5.1.2}), wird auch eine RR für Fälle wie \textit{blint} benötigt, welche gleich spezifisch sein muss wie die RR für das \isi{Suffix} -\textit{er} (\textit{blint-er}). Nur so können beide RRs angewendet werden und definieren zwei Formen für dieselbe Zelle des Paradigmas (vgl. \sectref{4.1.3.3}).

Schließlich ist noch interessant zu erwähnen, dass freie \isi{Variation} mit zwei \isi{Suffixen} nur im Althochdeutschen und im Alemannischen von Huzenbach vorkommt (vgl. \tabref{table5.18}). Die \isi{Variation} mit \isi{Wurzel} und \isi{Suffix} ist häufiger: Sie tritt Alt-, Mittelhochdeutsch, Jaun, Sensebezirk, Stuttgart, Kaiserstuhl, Colmar auf. Außerdem sind von der freien \isi{Variation} ausschließlich Nominativ und/oder Akkusativ Singular und/oder Plural betroffen. Tritt freie \isi{Variation} im Nominativ und Akkusativ Plural auf, ist sie auf das Feminin beschränkt. Im Singular können keine Verallgemeinerung bezüglich des \isi{Genus} gemacht werden.

%{\tabref{table5.18}: Freie \isi{Variation} in den untersuchten Varietäten}\\
\begin{table}
\caption{ Freie Variation in den untersuchten Varietäten}\label{table5.18}
\begin{tabularx}{\textwidth}{lQX}
\lsptoprule
Varietät & \isi{Suffix}\slash Stamm & \isi{Suffix}\slash \isi{Suffix}\\\midrule
Althochdeutsch & stark: \textsc{nom.sg.m.+n.+f.}; \textsc{akk.sg.n.} & stark: \textsc{nom.sg.m.+n.+f.}; \textsc{akk.sg.n.}\\
Mittelhochdeutsch & stark: \textsc{nom.sg.m.+n.+f.}; \textsc{akk.sg.n.} & \\
Jaun & stark: \textsc{nom./akk.pl.f.} schwach: \textsc{nom./akk.sg.m.+n.} &  \\
Sensebezirk & stark: \textsc{nom./akk.pl.f.} & \\
Huzenbach & & stark: \textsc{nom.sg.m.} \\
Stuttgart & schwach: \textsc{nom./akk.sg.f.}  & \\
Kaiserstuhl & stark: \textsc{nom./akk.sg.n.} & \\
Colmar & schwach:    \textsc{nom./akk.sg.n.+f.} & \\
\lspbottomrule
\end{tabularx}
\end{table}



\section{Personalpronomen}\label{5.3}

\subsection{Allgemeines und Realisierungsregeln}\label{5.3.1}

{Morphosyntaktische Eigenschaften}: In den \isi{Personalpronomen} werden folgende morphosyntaktischen Eigenschaften unterschieden: \isi{Numerus}, \isi{Kasus}, Person, \isi{Genus}, be\-tont/un\-be\-tont, be\-lebt/un\-be\-lebt. Die \isi{Personalpronomen} aller Varietäten in diesem Sample differenzieren \isi{Numerus}, \isi{Kasus}, Person und \isi{Genus} in der 3. Person Singular. Eine Genusunterscheidung in der 3. Person Plural ist nur im Alt- und Mittelhochdeutschen sowie im Alemannischen von Issime und Jaun (beide Höchstalemannisch) zu beobachten. Alle Varietäten außer der Standardsprache weisen jeweils ein Paradigma für das betonte und unbetonte \isi{Personalpronomen} auf, wobei diese unterschiedlich vollständig sind. Darauf wird in \sectref{5.3.2} genauer eingegangen. \isi{Belebtheit} wird nur in der 3. Person Singular Neutrum und nur in einigen Dialekten unterschieden. Dies wird in \sectref{5.3.3} erörtert.

{Definition der Form}: Die Formen der \isi{Personalpronomen} aller Varietäten müssen durch RRs definiert werden, weil sie nicht weiter unterteilbar sind, d.h., nicht weiter unterteilbar in eine \isi{Wurzel} und \isi{Affixe}. Somit sind die \isi{Personalpronomen} vergleichbar mit Kasus- oder Numerussuffixen der \isi{Substantive} oder \isi{Adjektive}. Dies stellt weder für das der Messmethode zugrunde liegende Modell noch für die Messmethode selbst ein Problem dar, wie dies am Anfang dieses Kapitels beschrieben wurde. Jede Form jeder Zelle wird also durch eine RR definiert. Folglich befinden sich alle RR in demselben \isi{Block}. Eine Ausnahme hiervon bilden die Doppelformen des Plurals im Alemannischen von Issime. Diese sollen im folgenden Abschnitt beschrieben werden. Außerdem soll anhand des Plurals der \isi{Personalpronomen} von Issime ein System an RRs für das \isi{Personalpronomen} gezeigt werden.

{Zusammengesetzte Formen in Issime}: \tabref{table5.19} stellt das Paradigma des betonten \isi{Personalpronomens} im Plural von Issime dar. Jede Person weist sowohl eine einfache Form (z.B. \textit{wir}) als auch eine Doppelform auf (z.B. \textit{wirendri}), welche in unterschiedlichen Kontexten verwendet werden \citep[216-221]{Zürrer1999}. Die Doppelform setzt sich aus dem einfachen \isi{Personalpronomen} und dem Indefinitpronomen \textit{andere} zusammen. Die einfache Form ist aus dem Althochdeutschen ererbt, die Doppelform aus dem Piemontesischen, Frankoprovenzalischen, gesprochenen Französischen und/oder Italienischen nachgebildet \citep[215]{Zürrer1999}, z.B. Piemontesisch \textit{noj-autri} \citep[72]{BreroBertodatti1988}. Wie in \sectref{3.3.3} beschrieben wurde, werden alle vier Sprachen im Aostatal gesprochen. Im Italienischen und Frankoprovenzalischen gibt es solche Doppelformen in der 1. und 2. Person Plural, im Französischen und Piemontesischen auch in der 3. Person Plural \citep[215]{Zürrer1999}. Im alemannischen Dialekt von Issime wurde nicht nur das Muster zur Bildung solcher Doppelformen übernommen, sondern auch in das bereits vorhandene Kasussystem integriert.

%{\tabref{table5.19}: Betontes Pluralparadigma des \isi{Personalpronomens} in Issime \citep[206-312]{Zürrer1999}}\\

\begin{table}
\caption{Betontes Pluralparadigma des Personalpronomens in Issime \citep[206--312]{Zürrer1999}}\label{table5.19}
\begin{tabular}{lllll}
\lsptoprule
{\textsc{person}} & {\textsc{nom}} & {\textsc{akk}} & {\textsc{dat}} & {\textsc{gen}}\\
\midrule
& \multicolumn{4}{c}{einfaches Personalpronomen}\\\midrule
1. & wir & ündsch & ündsch & ündsch-uru\\
2. & ir & auw & auw & auw-uru\\
3. & dschi & dschi & ürj-u & ürj-u, ürj-uru\\
\midrule
& \multicolumn{4}{c}{zusammengesetztes Personalpronomen}\\\midrule
1. & wir-endri & ündsch-endri & ündsch-enandre & ündsch-erandru\\
2. & ir-endri & auw-endri & auw-enandre & auw-erandru\\
3. & dschi-endri & dschi-endri & ürj-enandre & ürj-erandru\\
\lspbottomrule
\end{tabular}
\end{table}

Da die Doppelformen aus zwei Teilen bestehen, werden auch zwei \isi{Blöcke} benötigt, damit z.B. \textit{wir-endri} definiert ist und nicht \textit{endri-wir}. Des Weiteren sind alle Paradigmen auf ein Minimum zu reduzieren, was auch für die Doppelformen gilt. Beispielsweise wäre der Dativ in vier \isi{Affixe} einteilbar: einfache Form + \textit{en} + \textit{andr} + \textit{e} (vier RRs). Kürzer fällt jedoch die Beschreibung aus, wenn man von der einfachen Form + \textit{enandre} ausgeht (zwei RRs). Da \textit{enandre} im gesamten Dativ unabhängig von Person vorkommt, kann \textit{enandre} als Dativmarker analysiert werden. Die Analyse einfache Form + \textit{enandre} bildet also den Dativ adäquat ab und ist die kürzeste Beschreibung.

Es sollen nun die RRs für den Plural des betonten \isi{Personalpronomens} von Issime gezeigt werden. Aus Gründen der Anschaulichkeit entspricht das System an RRs nicht ganz jenem, das zur Komplexitätsmessung verwendet wird (vgl. RRs im Anhang B), da die unbetonten \isi{Personalpronomen} hier ausgelassen werden. Zuerst sind die einfachen Formen zu definieren, die auch in den Doppelformen vorkommen:

\ea%98
\label{ex:key:98}
 RR \textsubscript{A,} \textsubscript{\{\textsc{case:nom}, \textsc{num:pl}, \textsc{pers:1}\},} \textsubscript{\textsc{pron.pers[stress:+]}} ($\langle$X,$\sigma$$\rangle$) = \textsubscript{def} $\langle$\textit{wir}ˊ,$\sigma$$\rangle$
\z

\ea%99
\label{ex:key:99}
 RR \textsubscript{A,} \textsubscript{\{\textsc{case:nom}, \textsc{num:pl}, \textsc{pers:2}\},} \textsubscript{\textsc{pron.pers[stress:+]}} ($\langle$X,$\sigma$$\rangle$) = \textsubscript{def} $\langle$\textit{ir}ˊ,$\sigma$$\rangle$
\z

\ea%100
\label{ex:key:100}
 RR \textsubscript{A,} \textsubscript{\{\textsc{case:nom}} \textsubscript{\tiny $\veebar$}\textsubscript{ \AKK, \textsc{num:pl}, \textsc{pers:3}\},} \textsubscript{\textsc{pron.pers[stress:+]}} ($\langle$X,$\sigma$$\rangle$) = \textsubscript{def} $\langle$\textit{dschi}ˊ,$\sigma$ $\rangle$
\z

\ea%101
\label{ex:key:101}
 RR \textsubscript{A,} \textsubscript{\{\textsc{case: akk}}\textsubscript{ ${\veebar}$}\textsubscript{ \DAT} \textsubscript{\tiny $\veebar$}\textsubscript{ \GEN, \textsc{num:pl}, \textsc{pers:1}\},} \textsubscript{\textsc{pron.pers[stress:+]}} ($\langle$X,$\sigma$$\rangle$) = \textsubscript{def} $\langle$\textit{ündsch}ˊ,$\sigma$$\rangle$
\z

\ea%102
\label{ex:key:102}
 RR \textsubscript{A,} \textsubscript{\{\textsc{case: akk}}\textsubscript{ ${\veebar}$}\textsubscript{ \DAT} \textsubscript{\tiny $\veebar$}\textsubscript{ \GEN, \textsc{num:pl}, \textsc{pers:2}\},} \textsubscript{\textsc{pron.pers[stress:+]}} ($\langle$X,$\sigma$$\rangle$) = \textsubscript{def} $\langle$\textit{auw}ˊ,$\sigma$$\rangle$
\z

\ea%103
\label{ex:key:103}
 RR \textsubscript{A,} \textsubscript{\{\textsc{case: dat}} \textsubscript{\tiny $\veebar$}\textsubscript{ \GEN, \textsc{num:pl}, \textsc{pers:3}\},} \textsubscript{\textsc{pron.pers[stress:+]}} ($\langle$X,$\sigma$$\rangle$) = \textsubscript{def} $\langle$\textit{ürj}ˊ,$\sigma$$\rangle$
\z

\noindent
Es folgen die Dativ- und Genitivsuffixe für die einfache Form (\isi{Block} B):

\ea%104
\label{ex:key:104}
 RR \textsubscript{B,} \textsubscript{\{\textsc{case: dat}} \textsubscript{\tiny $\veebar$}\textsubscript{ \GEN, \textsc{num:pl}, \textsc{pers:3}\},} \textsubscript{\textsc{pron.pers[stress:+}, \textsc{form:simple]}} ($\langle$X,$\sigma$ $\rangle$) = \textsubscript{def} $\langle$X\textit{u}ˊ,$\sigma$ $\rangle$
\z

\ea%105
\label{ex:key:105}
 RR \textsubscript{B,} \textsubscript{\{\textsc{case: gen}, \textsc{num:pl}\},} \textsubscript{\textsc{pron.pers[stress:+}, \textsc{form:simple]}} ($\langle$X,$\sigma$ $\rangle$) = \textsubscript{def} $\langle$X\textit{uru}ˊ,$\sigma$ $\rangle$
\z

Auch die RRs für den zweiten Teil der Doppelformen stehen in \isi{Block} B. Dies ist möglich, da in den RRs die Form als \textit{simple} oder \textit{composed} klar definiert ist sowie der zweite Teil der Doppelformen und die Dativ-/Genitivsuffixe nie zusammen an dasselbe Wort suffigiert werden:

\ea%106
\label{ex:key:106}
 RR \textsubscript{B,} \textsubscript{\{\textsc{case:} \textsc{nom}} \textsubscript{\tiny $\veebar$}\textsubscript{ \AKK, \textsc{num:pl}\},} \textsubscript{\textsc{pron.pers[stress:+}, \textsc{form:composed]}} ($\langle$X,$\sigma$$\rangle$) = \textsubscript{def} $\langle$X\textit{endri}ˊ,$\sigma$$\rangle$
\z

\ea%107
\label{ex:key:107}
 RR \textsubscript{B,} \textsubscript{\{\textsc{case: dat}, \textsc{num:pl}\},} \textsubscript{\textsc{pron.pers[stress:+}, \textsc{form:composed]}} ($\langle$X,$\sigma$$\rangle$) = \textsubscript{def} $\langle$X\textit{enandre}ˊ,$\sigma$$\rangle$
\z

\ea%108
\label{ex:key:108}
 RR \textsubscript{B,} \textsubscript{\{\textsc{case: gen}, \textsc{num:pl}\},} \textsubscript{\textsc{pron.pers[stress:+}, \textsc{form:composed]}} ($\langle$X,$\sigma$$\rangle$) = \textsubscript{def} $\langle$X\textit{erandru}ˊ,$\sigma$$\rangle$
\z

In den RRs (\ref{ex:key:105}--\ref{ex:key:108}) bleibt Person unterspezifiziert, da diese RRs die 1.-3. Person definieren. Dasselbe gilt für die RRs (\ref{ex:key:98}--\ref{ex:key:103}), die die Form (\textit{simple}/\textit{composed}) nicht spezifizieren. Weil auch Betonung ein binärer Parameter ist, verhält er sich wie z.B. \isi{Numerus} und Form: Weist eine Zelle für das betonte und unbetonte \isi{Personalpronomen} dieselbe Form auf, ist der Parameter Betonung unterspezifiziert.

Schließlich muss noch hervorgehoben werden, dass die einfachen und zusammengesetzten Formen nicht in freier \isi{Variation} (d.h. beide Formen in einer Zelle) sind, sondern jeweils ein eigenes Paradigma bilden (vgl. \tabref{table5.19}). Würden beide Formen in derselben Zelle stehen, müsste für alle einfachen Formen, an die in \isi{Block} B kein weiteres Material suffigiert wird, eine RR in \isi{Block} B angesetzt werden, die definiert, dass nichts suffigiert wird (vgl. Diskussion zur freien \isi{Variation} in den Abschnitten \sectref{5.1.2} und \sectref{5.2.3}). Das würde die Beschreibung des Systems deutlich verlängern, also Komplexität hinzufügen. Um dies verständlicher zu machen, sollen anhand der 1. Person Plural Nominativ beide Möglichkeiten (freie \isi{Variation}, zwei Paradigmen) aufgezeigt werden.

Angenommen, die einfache (\textit{wir}) und zusammengesetzte (\textit{wirendri}) Form stünden in derselben Zelle, dann sind folgende RRs nötig:\largerpage[2]

% \ea%98
\begin{exe}[(98)]
\exr{ex:key:98}
 RR \textsubscript{A,} \textsubscript{\{\textsc{case:nom}, \textsc{num:pl}, \textsc{pers:1}\},} \textsubscript{\textsc{pron.pers[stress:+]}} ($\langle$X,$\sigma$$\rangle$) = \textsubscript{def} $\langle$\textit{wir}ˊ,$\sigma$$\rangle$
\end{exe}

\begin{exe}[(106.1)]%106.1
\exspecial{106} \label{ex:key:106.1}
RR \textsubscript{B,} \textsubscript{\{\textsc{case:} \textsc{nom}} \textsubscript{\tiny $\veebar$}\textsubscript{ \AKK, \textsc{num:pl}\},} \textsubscript{\textsc{pron.pers[stress:+}, \textsc{form:composed]}} ($\langle$X,$\sigma$$\rangle$) = \\\textsubscript{def} $\langle$X\textit{endri}ˊ,$\sigma$$\rangle$
\end{exe}

\begin{exe}[(106.2)]%106.2
\exspecial{106} \label{ex:key:106.2}
RR \textsubscript{B,} \textsubscript{\{\textsc{case:} \textsc{nom}} \textsubscript{\tiny $\veebar$}\textsubscript{ \AKK, \textsc{num:pl}\},} \textsubscript{\textsc{pron.pers[stress:+}, \textsc{form:simple]}} ($\langle$X,$\sigma$$\rangle$) = \textsubscript{def} $\langle$Xˊ,$\sigma$$\rangle$
\end{exe}

In \isi{Block} A wird definiert, dass in der Zelle Nominativ \textit{wir} steht \REF{ex:key:98}, in \isi{Block} B, dass der zusammengesetzten Form -\textit{endri} suffigiert wird \REF{ex:key:106.1} und dass der einfachen Form nichts suffigiert wird \REF{ex:key:106.2}. Würde die RR \REF{ex:key:106.2} nicht angenommen, wäre in der Zelle Nominativ keine einfache Form definiert. Es braucht also drei RRs, damit gewährleistet ist, dass in derselben Zelle eine einfache und eine zusammengesetzten Form stehen.

Angenommen, die einfachen und zusammengesetzten \isi{Personalpronomen} haben jeweils ein eigenes Paradigma, so sind nur die RRs \REF{ex:key:98} und \REF{ex:key:106} nötig (vgl. \tabref{table5.18}). Die RR \REF{ex:key:98} definiert \textit{wir} für beide Paradigmen (der Parameter \textit{Form} ist unterspezifiziert) und die RR \REF{ex:key:106} das \isi{Suffix} -\textit{endri} für das zusammengesetzte \isi{Personalpronomen}. Benötigt werden also nur zwei RRs, weshalb diese Analyse zu bevorzugen ist.

\subsection{Betont und unbetont}\label{5.3.2}

Alle der hier untersuchten Varietäten außer der Standardsprache haben betonte und unbetonte \isi{Personalpronomen}, d.h., in einem betonten Kontext wird eine andere Form des \isi{Personalpronomens} verwendet als in einem unbetonten Kontext. Bei der unbetonten Variante handelt es sich meistens um eine phonetisch reduzierte Form im Vergleich zur betonten Form. Die unbetonten Formen können also als \textit{special clitics} bezichnet werden \citep[510--511]{ZwickyPullum1983}. Da diese phonetische Reduktion jedoch nicht in der gesamten Sprache gilt, es sich also nicht um Regeln handelt, die automatisch auf das gesamte System angewendet werden, bilden die unbetonten \isi{Personalpronomen} ein zusätzliches Paradigma neben jenem der betonten \isi{Personalpronomen}. Sie sind folglich ebenfalls durch RRs zu definieren. Formalisiert wird die Betonung durch den Parameter \textit{Stress} (Betonung), welcher die Ausprägung + oder - haben kann.

Bereits das Alt- und Mittelhochdeutsche weisen unbetonten \isi{Personalpronomen} auf, jedoch nur in der 3. Person Singular und Plural (vgl. Paradigmen 41 und 42). Alle alemannischen Dialekte mit Ausnahme von Elisabethtal haben aber ein vollständiges Paradigma mit unbetonten \isi{Personalpronomen}, d.h. auch in der 1. und 2. Person Singular und Plural. Dies kann als Ausbau und Grammatikalisierung der Kategorie unbetonter \isi{Personalpronomen} interpretiert werden, während die Standardsprache diese Kategorie abgebaut hat. Erstaunlicherweise existieren im Dialekt von Elisabethtal unbetonte Formen nur in der 3. Person Singular Maskulin und Neutrum.\largerpage[2] Da alle anderen schwäbischen Dialekte dieses Samples ein vollständiges Paradigma für die unbetonten \isi{Personalpronomen} haben, kann dies als Abbau gewertet werden.\footnote{Denkbar wäre natürlich auch, dass die Beschreibung unvollständig ist. Da jedoch \citeauthor{Žirmunskij1928/29}s \citeyearpar{Žirmunskij1928/29} Ausführungen sonst eine sehr hohe Genauigkeit aufweisen, kann davon ausgegangen werden, dass auch diese adäquat ist.}

\subsection{Belebt und unbelebt}\label{5.3.3}

{Dialekte}: Einige alemannische Dialekte unterscheiden in der 3. Person Singular Neutrum zwischen belebt und unbelebt. Es handelt sich um folgende Dialekte: Jaun, Sensebezirk, Uri (also alle höchstalemannischen Dialekte außer der Walser Dialekte), Bern und Zürich (also alle hochalemannischen Dialekte außer Vorarlberg), Kaiserstuhl und Elsass (Ebene) (Oberrheinalemannisch). In keinem der schwäbischen Dialekte kommt dieses Phänomen vor. Die Dialekte von Jaun, Bern und des Sensebezirks differenzieren \isi{Belebtheit} im betonten und unbetonten Paradigma, die Dialekte von Uri, Zürich, des Kaiserstuhls und des Elsass (Ebene) nur im betonten Paradigma.

{Das System}: In den Dialekten, die belebte und unbelebte Formen unterscheiden, wird das Neutrum verwendet, um sich auf weibliche Menschen zu beziehen (für männliche Menschen wird das Maskulin verwendet). In den Walser Dialekten, die belebt und unbelebt nicht differenzieren, bezieht man sich mit dem Neutrum auf weibliche und männliche Menschen. \tabref{table5.20} zeigt das Paradigma der 3. Person Singular Neutrum belebt und unbelebt des Dialektes von Bern.

%{\tabref{table5.20}: 3. Person Singular Neutrum belebt und unbelebt in Bern \citep[92-97]{Marti1985}}\\

\begin{table}
\caption{3. Person Singular Neutrum belebt und unbelebt in Bern \citep[92-97]{Marti1985}}\label{table5.20}
\begin{tabular}{lllll} 
\lsptoprule
&  & {\NOM} & {\AKK} & {\DAT}\\
\midrule
{Singular} & {3.\textsc{n}.unbelebt} & ǣs & ǣs & \=im\\
& {3.\textsc{n}.belebt} & ǣs & \=ins & \=im\\
\lspbottomrule
\end{tabular}
\end{table}

Die unbelebten Formen sind jene, die für eine 3. Person Singular Neutrum zu erwarten sind: Nominativ und Akkusativ fallen zusammen, während der Dativ eine eigene Form hat. In der belebten Form werden auch Nominativ und Akkusativ unterschieden. Man kann sich nun fragen, ob die belebten Formen zum Neutrum gehören oder ob sie neben Maskulin, Feminin und Neutrum ein eigenes \isi{Genus} bilden. Die grundsätzlichere Frage lautet also, mit welchem Kriterium \isi{Genus} ermittelt werden kann. \citet{Corbett1991} schlägt Kongruenz vor: „[…] the determining criterion of gender is agreement […]. Saying that a language has three genders implies that there are three classes of nouns which can be distinguished syntactically by the agreements they take“ \citep[4]{Corbett1991}. Da die Wörter, die durch die belebten Pronomen ersetzt werden können, Kongruenz im Neutrum aufweisen, gehören die belebten Pronomen zum Neutrum. Bevor versucht wird, dieses Phänomen einzuordnen und zu erklären, wird nun zuerst gezeigt, woher die Akkusativform des belebten Neutrums stammen könnte und anschließend, weshalb der Dialekt des Sensebezirks einen Sonderfall darstellt.

{Ursprung des belebten Akkusativs}: Es stellt sich also die Frage, woher die Form des Akkusativs Neutrum belebt stammt. Eine mögliche Erklärung ist, dass sie vom Akkusativ Maskulin der 3. Person Singular abgeleitet ist, da ihre Formen die größten Ähnlichkeiten aufweisen, was in \tabref{table5.21} zusammengefasst ist. Der einzige Unterschied zwischen den beiden Formen ist das auslautende \textit{s} im Neutrum. Es könnte davon ausgegangen werden, dass \textit{s} als eine Art Default-Marker für das Neutrum fungiert (z.B. \textit{das}, \textit{meinəs}, \textit{schönəs} etc.) und somit an die Maskulinform angehängt wird, um eine Neutrumform zu bilden. Dies entspricht der Erklärung, die \citet{Stucki1917} in Bezug auf den Dialekt von Jaun gibt \citep[281]{Stucki1917}.

%{\tabref{table5.21}: Herkunft der Akkusativform der 3. Person Singular Neutrum unbelebt}\\

\begin{table}
\caption{Herkunft der Akkusativform der 3. Person Singular Neutrum unbelebt}\label{table5.21}
\begin{tabular}{lll} 
\lsptoprule
& \textsc{3.sg.m.akk} & \mbox{\textsc{3.sg.n.}unbelebt.\AKK}\\
\midrule
Jaun & ẽ & ẽs\\
Uri & inɐ & inəss\\
Zürich & in & ins\\
Bern & \=in & \=ins\\
Kaiserstuhl & \=inɐ & \=inəs\\
Elsass (Ebene) & inə & inəs\\
\lspbottomrule
\end{tabular}
\end{table}

Alle Dialekte außer das Alemannische des Sensebezirks, die \isi{Belebtheit} unterscheiden, weisen parallele Formen zum Paradigma von Bern auf, d.h. unbelebt Nominativ=Akkusativ${\neq}$Dativ und belebt Nominativ${\neq}$Akkusativ${\neq}$Dativ. Dabei handelt es sich bei den unbelebten Formen um jene für ein Neutrum zu erwartende Formen und die belebten Formen weisen die in \tabref{table5.21} gelisteten Akkusativformen auf. 

{Das System im Dialekt des Sensebezirks}: Das Paradigma des Sensebezirks\linebreak weicht davon ab (vgl. \tabref{table5.22}). Zwar wird auch hier zwischen belebten und unbelebten Formen im Neutrum unterschieden, aber ein anderer Wandel ist hier eingetreten: In den meisten nominalen Wortarten wurde die Akkusativ- durch die Dativform ersetzt \citep{Bucheli2010}. Dieser Wandel ist in die Formen des unbelebten Paradigmas eingedrungen, jedoch nicht in jene des belebten Paradigmas. Im belebten Paradigma ist also der alte Akkusativ erhalten, während er im unbelebten Paradigma durch den Dativ ersetzt wurde. Dieser Dialekt hat folglich keine spezielle Form für den belebten Akkusativ grammatikalisiert. Zusammenfassend kann man also festhalten, dass im Dialekt des Sensebezirks der Akkusativ der 3. Person Singular Neutrum unbelebt wie der Dativ lautet und der Akkusativ der 3. Person Singular Neutrum belebt wie der Nominativ.

%{\tabref{table5.22}: 3. Person Singular Neutrum belebt und unbelebt im Sensebezirk \citep[196-198]{Henzen1927}}\\

\begin{table}
\caption{3. Person Singular Neutrum belebt und unbelebt im Sensebezirk \citep[196-198]{Henzen1927}}\label{table5.22}
\begin{tabular}{lllll} 
\lsptoprule
&  & {\NOM} & {\AKK} & {\DAT}\\
\midrule
{Singular} & \textsc{3.n.}unbelebt & æs & \=im & \=im\\
& \textsc{3.n.}belebt & æs & æs & \=im\\
\lspbottomrule
\end{tabular}
\end{table}

{RRs und Parameter \textit{Animacy}}: Wenn also \isi{Belebtheit} unterschieden wird, muss dies durch die RRs definiert werden. Neben den Parametern zur Betonung und zu den morphosyntaktischen Eigenschaften muss folglich noch ein Parameter \isi{Belebtheit} spezifiziert werden (ANIM für \textit{Animacy}). Folgende RRs bilden das Paradigma der \tabref{table5.22} ab:

\ea%109
\label{ex:key:109}
 RR \textsubscript{A,} \textsubscript{\{\textsc{case: nom}, \textsc{num:sg}, \textsc{pers:3}, \textsc{gend:n}\},} \textsubscript{\textsc{pron.pers[stress:+]}} ($\langle$X,$\sigma$ $\rangle$) = \textsubscript{def} $\langle$X\textit{æs}ˊ,$\sigma$ $\rangle$
\z

\ea%110
\label{ex:key:110}
 RR \textsubscript{A,} \textsubscript{\{\textsc{case: dat}, \textsc{num:sg}, \textsc{pers:3}, \textsc{gend:n}\},} \textsubscript{\textsc{pron.pers[stress:+]}} ($\langle$X,$\sigma$$\rangle$) = \textsubscript{def} $\langle$X\textit{\=im}ˊ,$\sigma$$\rangle$
\z

\ea%111
\label{ex:key:111}
 RR \textsubscript{A,} \textsubscript{\{\textsc{case: akk}, \textsc{num:sg}, \textsc{pers:3}, \textsc{gend:n}, \textsc{anim:-}\},} \textsubscript{\textsc{pron.pers[stress:+]}} ($\langle$X,$\sigma$$\rangle$) = \textsubscript{def} $\langle$X\textit{\=im}ˊ,$\sigma$$\rangle$
\z

\ea%112
\label{ex:key:112}
 RR \textsubscript{A,} \textsubscript{\{\textsc{case: akk}, \textsc{num:sg}, \textsc{pers:3}, \textsc{gend:n}, \textsc{anim:+}\},} \textsubscript{\textsc{pron.pers[stress:+]}} ($\langle$X,$\sigma$$\rangle$) = \textsubscript{def} $\langle$X\textit{æs}ˊ,$\sigma$$\rangle$
\z

Es soll nun ein Versuch vorgenommen werden, dieses Phänomen zu erklären, da es meines Wissens in keinem anderen deutschen Dialekt vorkommt. Dazu soll zuerst ein Vergleich mit älteren und jüngeren Sprachstufen angestellt werden, die mit dem Deutschen eng verwandt sind. Anschließend werden die Ergebnisse erklärt.

{Sprachvergleich}: In der niederländischen und englischen Standardsprache fallen in der 3. Person Singular Neutrum des \isi{Personalpronomens} der Nominativ und der Akkusativ zusammen, engl. \textit{it}, niederl. \textit{het} \citep[36]{GabrielRoodzant2010}. Mit dem Neutrum werden unbelebte Entitäten pronominalisiert, mit dem Maskulin und Feminin wird auf belebte Entitäten verwiesen, engl. \textit{he}\slash\textit{she}, niederl. \textit{hij}\slash \textit{zij} \citep[36]{GabrielRoodzant2010}. Interessanterweise wird in den belebten \isi{Personalpronomen} zwischen Subjekt- und Objektkasus unterschieden, in den unbelebten Pronomen jedoch nicht: engl. \textit{he}/\textit{him}, \textit{she}/\textit{her}, \textit{it}/\textit{it}; niederl. \textit{hij}/\textit{hem}, \textit{zij}/\textit{haar}, \textit{het}/\textit{het} \citep[36]{GabrielRoodzant2010}. 

Die skandinavischen Standardsprachen können in zwei Gruppen geteilt werden. Zur ersten gehören Isländisch und Färöisch, deren \isi{Personalpronomen} der 3. Person Singular mit drei \isi{Genera} und ohne Unterscheidung von \isi{Belebtheit} wie die deutsche Standardsprache funktionieren (\citealt[76]{Pétursson1981}; \citealt[200]{BarnesWeyhe2002}). Zur zweiten Gruppe zählen Norwegisch (Bokmål), Schwedisch und Dänisch (\citealt[317--332]{FaarlundLieVannebo1997}; \citealt[128--136]{HolmesHinchliffe1994}; \citealt[141-155]{AllanHomlmesLundskær-Nielsen1995}). In diesen Sprachen wird in der 3. Person Singular des \isi{Personalpronomens} zwischen belebt und unbelebt unterschieden. Bei den belebten \isi{Personalpronomen} wird weiter zwischen Maskulin und Feminin (Sexus) differenziert (z.B. norw. \textit{han}/\textit{hun}), bei den unbelebten \isi{Personalpronomen} zwischen Utrum und Neutrum (\isi{Genus}) (z.B. norw. \textit{den}/\textit{det}) \citep[317]{FaarlundLieVannebo1997}. Des Weiteren weist in den belebten \isi{Personalpronomen} der Objektkasus eine andere Form auf als der Subjektkasus, während in den unbelebten \isi{Personalpronomen} Subjekt- und Objektkasus zusammenfallen (wie das Englische und das Niederländische).

Die älteren Stufen Gotisch, Altnordisch, Altenglisch, Altsächsisch und Althochdeutsch haben drei \isi{Genera}, funktionieren wie Deutsch, Isländisch und Färöisch und weisen im Neutrum eine einheitliche Form für den Nominativ und Akkusativ auf \citep[54]{KraheMeid1967}. Im Altnordischen stammt die Form für die 3. Person Singular Neutrum aus dem \isi{Demonstrativpronomen} ( \citealt[54]{KraheMeid1967}, \citealt[108]{Gutenbrunner1951} ).

Die hier untersuchten germanischen Sprachen können also in zwei Gruppen eingeteilt werden. Erstens verfügen die alten germanischen Stufen wie auch\linebreak Deutsch, Isländisch und Färöisch in der 3. Person Singular des \isi{Personalpronomens} über drei \isi{Genera} und machen keine zusätzliche Unterscheidung zwischen belebt und unbelebt. Zweitens differenzieren Norwegisch, Schwedisch, Dänisch, Englisch und Niederländisch zuerst zwischen belebt und unbelebt. Anschließend wird innerhalb der belebten \isi{Personalpronomen} Sexus (Maskulin und Feminin) kodiert und innerhalb der unbelebten \isi{Personalpronomen} \isi{Genus} (Utrum und Neutrum). Die Genusunterscheidung trifft nur auf die skandinavischen Sprachen zu. In dieser zweiten Gruppe unterscheidet außerdem das belebte \isi{Personalpronomen} einen Subjekt- von einem Objektkasus, das unbelebte \isi{Personalpronomen} jedoch nicht.

Die modernen romanischen Sprachen verfügen über ein Zwei-Genus-System (Maskulin und Feminin), Latein jedoch über ein Drei-Genus-System wie u.a. die deutsche Standardsprache. Im Neutrum wird die Form \textit{id} für Nominativ und Akkusativ verwendet \citep[90]{Touratier2013}. Dasselbe gilt für das Neugriechische \citep[95]{HoltonMackridgePhilippaki-Warburton2002} wie auch für das Altkirchenslawische \citep[50]{Trunte2005}. Das Altgriechische verwendet für den Nominativ und die obliquen \isi{Kasus} unterschiedliche Pronomen \citep[92]{Smyth1984}, weshalb es hier für den Vergleich nicht berücksichtigt werden kann. Bereits für das Urindogermanische mit einem Drei-Genus-System wird von einem \isi{Synkretismus} zwischen Nominativ und Akkusativ im Neutrum ausgegangen \citep[67]{Tichy2004}. Man nimmt aber des Weiteren an, dass dem Drei-Genus-System ein Zwei-Genus-System vorausgegangen ist:

\begin{quote}
Diese [Zweiheit] bestand vermutlich auf der einen Seite aus einer Klasse A, wo eine Nom./Akk.-Dif\-fe\-ren\-zie\-rung möglich war ( […] vom Sprecher als Träger einer Verbalhandlung vorstellbar, agensfähig), auf der anderen Seite aus einer Klasse B, wo dies gerade ausgeschlossen war ( […] vom Sprecher als Träger einer Verbalhandlung nicht vorstellbar, nicht agensfähig). \citep[323]{Meier-Brügger2010}
\end{quote}

Im Gegensatz zum Altkirchenslawischen weist die Substantivflexion der modernen slawischen Sprachen einen parallelen Fall zu den \isi{Belebtheit} unterscheidenden alemannischen Dialekten auf, was kurz anhand des Russischen gezeigt werden soll. Das Russische vefügt über drei \isi{Genera} (Maskulin, Feminin, Neutrum) und unterscheidet be\-lebt/un\-be\-lebt im Maskulin Singular und im Plural aller \isi{Genera} (\tabref{table5.23}). Dabei variiert die Akkusativform: Handelt es sich um belebte Entitäten, lautet der Akkusativ wie der Genitiv, handelt es sich um unbelebte Entitäten, lautet der Akkusativ wie der Nominativ.

%{\tabref{table5.23}: Russische Substantivflexion, belebt und unbelebt (gekürztes Paradigma aus \citet[166]{Corbett1991})}\\

\begin{table}
\caption{Russische Substantivflexion, belebt und unbelebt (gekürztes Paradigma aus \citealt[166]{Corbett1991})}\label{table5.23}
\resizebox{\textwidth}{!}{\begin{tabular}{lllllll} 
\lsptoprule
& {\textit{student}} {(m)} & {\textit{dub}} {(m)} & {\textit{sestra}} {(f)} & {\textit{škola}} {(f)} & {\textit{čudovišče} }{(n)} & {\textit{vino} }{(n)}\\
& {ʻStudentʼ} & {ʻEicheʼ} &  {ʻSchwesterʼ} &  {ʻSchuleʼ} &  {ʻMonsterʼ} & {ʻWeinʼ}\\
\midrule
\multicolumn{7}{c}{{\textsc{singular}}}\\\midrule
{\NOM} & student & dub & sestr-a & škol-a & čudovišč-e & vin-o\\
{\AKK} & student-a & dub & sestr-u & škol-u & čudovišč-e & vin-o\\
{\GEN} & student-a & dub-a & sestr-y & škol-y & čudovišč-a & vin-a\\
\midrule
\multicolumn{7}{c}{{\textsc{Plural}}}\\\midrule
{\NOM} & student-y & dub-y & sestr-y & škol-y & čudovišč-a & vin-a\\
{\AKK} & student-ov & dub-y & sester & škol-y & čudovišč & vin-a\\
{\GEN} & student-ov & dub-ov & sester & škol & čudovišč & vin\\
\lspbottomrule
\end{tabular}}
\end{table}

In den alemannischen Dialekten und in den slawischen Sprachen stellt die Kategorie \isi{Belebtheit} eine Neuerung dar. In den alemannischen Dialekten taucht dies im \isi{Personalpronomen} auf, in den slawischen Sprachen in den \isi{Substantiven}. Übrigens weist im Russischen der Akkusativ in der 3. Person Singular des \isi{Personalpronomens} dieselbe Form wie der Genitiv auf und beide \isi{Kasus} unterschieden sich vom Nominativ. Darauf kann aber an dieser Stelle nicht weiter eingegangen werden. Die alemannischen Dialekte und das Russische besitzen drei \isi{Genera} und ein Subgenus \isi{Belebtheit}, wobei das Subgenus im Russischen im Maskulin Singular und im Plural aller \isi{Genera} vorkommt, in den alemannischen Dialekten nur im Neutrum Singular. Des Weiteren beinhaltet \isi{Belebtheit} im Russischen Menschen wie Tiere, während \isi{Belebtheit} in den alemannischen Dialekten nur Menschen einschließt. Bezüglich der Form verwendet das Russische eine bereits vorhandene Form, nämlich die Genitivform (für Akkusativ belebt). Die alemannischen Dialekte haben für den belebten Akkusativ eine neue Form hervorgebracht. Eine Ausnahme bildet diesbezüglich der Dialekt des Sensebezirks, der wie das Russische eine bereits vorhandene Form benutzt: Der Akkusativ der unbelebten Form lautet wie der Dativ, der Akkusativ der belebten Form wie der Nominativ. Besonders auffällig ist, dass in beiden Sprachen jeweils nur der Akkusativ in Abhängigkeit von \isi{Belebtheit} variiert. Nimmt man noch das Urindogermanische und jene germanischen Sprachen hinzu, die im \isi{Personalpronomen} be\-lebt/un\-be\-lebt unterscheiden, kann dies allgemeiner zusammengefasst werden: Sprachen, die in bestimmten Wortarten eine Kategorie \isi{Belebtheit} aufweisen, unterscheiden in diesen Wortarten Nominativ und Akkusativ (bzw. Subjekt- und Objektkasus), wenn der Referent belebt ist, während diese \isi{Kasus} zusammenfallen, wenn der Referent unbelebt ist. Dies bedarf einer Erklärung.

{Erklärungsversuch}: Einen interessanten Ansatz bietet \citet{Comrie1996}. Er geht davon aus, dass in einer Transitivkonstruktion das Agens einen hohen Grad an \isi{Belebtheit} und das Patiens einen niedrigen Grad an \isi{Belebtheit} aufweist. Verfügt nun das Patiens einen hohen Grad an \isi{Belebtheit}, liegt ein markierter Fall vor, der auch formal markiert werden muss \citep[128]{Comrie1996}. Da in Transitivkonstruktionen (der Nominativ-Akkusativ-Sprachen) der prototypische \isi{Kasus} für Patiens der Akkusativ ist, muss in diesen Sprachen also der Akkusativ belebt besonders markiert werden. In dieselbe Richtung geht \citet{Bossong1998}: Das prototypische Subjekt ist belebt, das prototypische Objekt unbelebt, da das Subjekt die Handlung ausführt, während das Objekt die Handlung erfährt \citep[201]{Bossong1998}. In diesen Fällen muss das Objekt nicht speziell markiert werden. Wenn ein Objekt jedoch durch seine inhärente Semantik, indem es belebt ist, ein potentielles Subjekt darstellt, „[…] il s’avère nécessaire de lui conférer une marque spécifique permettant de le distinguer du sujet sans ambiguïté [et] parce qu’il[…] correspond[…] moins bien à la sémantique de l’objet prototypique […]“ \citep[202]{Bossong1998}. Dies nennt \citet{Bossong1998} differentielle Objektmarkierung. Das erklärt also, weshalb in den alemannischen Dialekten im Neutrum belebt der Akkusativ vom Nominativ unterschieden wird, jedoch nicht im Neutrum unbelebt. Interessanterweise widerspricht das System vom Dialekt des Sensebezirks Bossongs differentiellen Objektmarkierung, denn dieser Dialekt zeigt einen Nominativ-Akkusativ-\isi{Synkretismus} im belebten Paradigma und eine Nominativ-Akkusativ-Unterscheidung im unbelebten Paradigma. Aufgrund dieser Beobachtungen wird klar, dass dies weiterer Beschäftigung und Analysen bedarf.

Des Weiteren übernimmt \citet{Bossong1998} \citeauthor{Silverstein1976}s \citeyearpar{Silverstein1976} Belebtheitshierarchie, in der Deiktika am belebtesten sind \citep[203]{Bossong1998}. Diese Belebtheitshierarchie kann also erklären, weshalb in den alemannischen Dialekten \isi{Belebtheit} am \isi{Personalpronomen}, nicht aber an anderen nominalen Kategorien markiert wird.\footnote{Es sei hier nur darauf hingewiesen, dass in den Dialekten des Sensebezirks und von Jaun der Akkusativ und der Dativ des \isi{Personalpronomens} mit einem vorausgehenden \textit{i}{}- markiert werden (also eine Distinktion Objekt vs. Subjekt), während in den übrigen Determinierern nur der Dativ mit einem vorausgehenden \textit{i}{}- markiert wird \citep[85]{Seiler2003}.}\largerpage

Schließlich ist noch zu klären, weshalb in den alemannischen Dialekten das \isi{Personalpronomen} der 3. Person Singular im Maskulin und Neutrum belebt eine Akkusativform hat, die sich von der Nominativform unterscheidet, das Feminin aber nicht. Oder andersherum gefragt: Weshalb kommt diese Neuerung (neue Akkusativform, die sich vom Nominativ unterscheidet) nur im Neutrum, aber nicht im Feminin vor? Die prototypischen \isi{Genera} für belebte Entitäten sind Maskulin und Feminin, für unbelebte Entitäten das Neutrum. Stehen nun auch belebte Entitäten im Neutrum, müssen diese Fälle und besonders der Akkusativ dieser Fälle speziell markiert werden. Dies widerspricht jedoch den Ergebnissen aus dem Singular Neutrum in der russischen Substantivflexion. Diese Diskussion ließe sich noch vertiefen, zumal vieles ungeklärt bleibt oder widersprüchlich ist. Beispielsweise wird nicht klar, weshalb gerade die Walser Dialekte, in denen mit dem Neutrum auf männliche wie weibliche Menschen referiert wird, im Neutrum keine Kategorie \isi{Belebtheit} grammatikalisiert haben. Diese Diskussion würde hier jedoch zu weit führen und ist Aufgabe zukünftiger Forschung.

\subsection{Freie Variation}\label{5.3.4}

Freie \isi{Variation} kommt im Alt- und Mittelhochdeutschen sowie in der Standardsprache bezüglich der \isi{Personalpronomen} nicht vor. Im Gegensatz dazu ist freie \isi{Variation} ein durchaus übliches Phänomen in den alemannischen Dialekten. Nur die Dialekte von Saulgau, Colmar, des Münstertals und des Kaiserstuhls weisen keine freie \isi{Variation} in den \isi{Personalpronomen} auf. Mit Abstand am häufigsten kommt freie \isi{Variation} im Nominativ vor, gefolgt vom Akkusativ und Genitiv. Am seltensten ist freie \isi{Variation} in Dativ zu beobachten. Beispielsweise lautet im Dialekt des Sensebezirks die 1. Person Plural Nominativ \textit{wiər} und \textit{miər} \citep[196]{Henzen1927}.

Wie bereits mehrmals dargestellt wurde (\sectref{4.1.3.3}, \sectref{5.1.2}, \sectref{5.2.3}), müssen bei freier \isi{Variation} die RRs gleich spezifisch sein. Weist eine Zelle des Paradigmas mehr als eine Form auf, müssen die RRs gleich spezifisch gestaltet und demselben \isi{Block} zugewiesen werden. Nur so wird gewährleistet, dass zwei RRs zwei Formen für dieselbe Zelle definieren.

\section{Interrogativpronomen}\label{5.4}

{Morphosyntaktische Eigenschaften}: Im \isi{Interrogativpronomen} \textit{wer}/\textit{was} werden \isi{Kasus} und \isi{Belebtheit} unterschieden. Zwar entsprechen die belebten Formen formal einem Maskulin und die unbelebten einem Neutrum. Da aber mit \textit{wer} auf belebte Entitäten verwiesen wird und mit \textit{was} auf unbelebte, wird in den RRs nicht \isi{Genus} sondern ±belebt definiert. Wie beim \isi{Personalpronomen} lautet der Parameter \textit{Animacy} und wird in den RR mit ANIM abgekürzt.\largerpage

{Definition der Form}: Wie die \isi{Adjektive} (Ausnahme: Mittelhochdeutsch) weist auch das \isi{Interrogativpronomen} maximal ein \isi{Suffix} auf. Es muss also keine Abfolge von Affigierungen durch verschiedene \isi{Blöcke} definiert werden. Des Weiteren macht es keinen Unterschied bezüglich der Anzahl RRs, ob durch die RR die ganze Form des \isi{Interrogativpronomens} definiert wird (z.B. \textit{wer}, \textit{wen} etc.) oder ob dieses in \isi{Wurzel} und \isi{Suffix} geteilt wird (z.B. \textit{w}-\textit{er}, \textit{w}-\textit{en} etc.). Daraus resultieren zwei Konsequenzen für die RRs: Durch die RRs wird die gesamte Form definiert und alle RRs stehen im selben \isi{Block}.

{Freie Variation}: Auch im \isi{Interrogativpronomen} kommt \isi{Variation} vor, d.h., eine Zelle des Paradigmas hat zwei Formen. Wie bereits referiert und analog zu den oben besprochenen Kategorien (\sectref{4.1.3.3}, \sectref{5.1.2}, \sectref{5.2.3}, \sectref{5.3.4}) sind für beide Formen gleich spezifische RRs anzunehmen. Beispielsweise verfügt der Akkusativ Singular belebt im Dialekt von Jaun über zwei Formen, nämlich \textit{wær} und \textit{wɛm} (vgl. \tabref{table5.24}). Zwar ist die Distribution syntaktisch bedingt: \textit{Wær} in freier Verwendung, \textit{wɛm} nach Präposition \citep[285]{Stucki1917}, trotzdem muss die Morphologie für eine Zelle des Paradigmas beide Formen bilden. Folglich braucht es dafür zwei gleich spezifische RRs.

{Beispiel}: Als Beispiel für ein System an RRs bezüglich des \isi{Interrogativpronomens} werden in der Folge die RRs für das \isi{Interrogativpronomen} von Jaun gelistet. \tabref{table5.24} zeigt das Paradigma des \isi{Interrogativpronomens} von Jaun, (\ref{ex:key:113}--\ref{ex:key:116}) die dazugehörigen RRs.

%{\tabref{table5.24}: \isi{Interrogativpronomen} von Jaun \citep[285-286]{Stucki1917}}\\

\begin{table}
\caption{Interrogativpronomen von Jaun \citep[285-286]{Stucki1917}}\label{table5.24}
\begin{tabular}{llll} 
\lsptoprule
& {\NOM} & {\AKK} & {\DAT}\\\midrule
{belebt} & wær & wær/wemm & wɛm\\
{unbelebt} & was & was & wɛm\\
\lspbottomrule
\end{tabular}
\end{table}

RRs \REF{ex:key:113} und \REF{ex:key:114} definieren den \isi{Synkretismus} zwischen Nominativ und Akkusativ, RR \REF{ex:key:115} jenen zwischen belebt und unbelebt im Dativ. Schließlich wird in RR \REF{ex:key:116} die zweite Form des Akkusativs belebt bestimmt.

\ea%113
\label{ex:key:113}
 RR \textsubscript{A,} \textsubscript{\{\textsc{case:} \textsc{nom}} \textsubscript{\tiny $\veebar$}\textsubscript{ \AKK, \textsc{anim:+}\},} \textsubscript{PRON.INTER} ($\langle$X,$\sigma$ $\rangle$) = \textsubscript{def} $\langle$\textit{wær}ˊ,$\sigma$ $\rangle$
\z

\ea%114
\label{ex:key:114}
 RR \textsubscript{A,} \textsubscript{\{\textsc{case:} \textsc{nom}} \textsubscript{\tiny $\veebar$}\textsubscript{ \AKK, ANIM:-\},} \textsubscript{PRON.INTER} ($\langle$X,$\sigma$ $\rangle$) = \textsubscript{def} $\langle$\textit{was}ˊ,$\sigma$ $\rangle$
\z

\ea%115
\label{ex:key:115}
 RR \textsubscript{A,} \textsubscript{\{\textsc{case: dat}\},} \textsubscript{PRON.INTER} ($\langle$X,$\sigma$ $\rangle$) = \textsubscript{def} $\langle$\textit{wɛm}ˊ,$\sigma$ $\rangle$
\z

\ea%116
\label{ex:key:116}
 RR \textsubscript{A,} \textsubscript{\{\textsc{case: akk}, \textsc{anim:+}\},} \textsubscript{PRON.INTER} ($\langle$X,$\sigma$ $\rangle$) = \textsubscript{def} $\langle$\textit{wɛm}ˊ,$\sigma$ $\rangle$
\z

\section{Bestimmter Artikel / Demonstrativpronomen}\label{5.5}

\subsection{Allgemeines und Realisierungsregeln}\label{5.5.1}

{Zwei Wortarten, eine Kategorie}: Der \isi{bestimmte Artikel} und das einfache \isi{Demonstrativpronomen} bilden zusammen eine Kategorie. Wie in \sectref{4.3.2} erörtert wurde, hat dies zwei Gründe. Erstens sollen die Paradigmen auf ein Minimum reduziert werden, da nur so unterschiedliche Paradigmen aus unterschiedlichen Grammatiken verglichen werden können. Dies ist am besten zu erreichen, indem ähnliche oder sogar gleiche Paradigmen Teil derselben Kategorie sind. Da unter allen untersuchten Wortarten der \isi{bestimmte Artikel} und das \isi{Demonstrativpronomen} die größten Ähnlichkeiten aufweisen, formen sie eine Kategorie. Zweitens ist der \isi{bestimmte Artikel} aus dem \isi{Demonstrativpronomen} entstanden.

{Definition der Form}: Wie in den \isi{Personalpronomen} sind die Formen des \isi{bestimmten Artikels} und des \isi{Demonstrativpronomens} synchron nicht weiter unterteilbar, weswegen die gesamte Form durch die RR definiert wird (vgl. \sectref{5.3.1}). Da unterschiedliche \isi{Affixe} nicht aneinandergereiht sind und ihre Abfolge also nicht definiert werden muss, stehen alle RRs in demselben \isi{Block}.

Als Beispiel für ein System an RRs soll hier der Dialekt von Jaun herangezogen werden (\tabref{table5.25}, RRs (\ref{ex:key:118}--\ref{ex:key:136})), an dem in diesem Unterkapitel unterschiedliche Eigenschaften der Kategorie \isi{bestimmter Artikel}/\isi{Demonstrativpronomen} illustriert werden. Die zu unterscheidenden morphosyntaktischen Eigenschaften sind \isi{Kasus}, \isi{Numerus}, \isi{Genus}.

%{\tabref{table5.25}: Bestimmter Artikel und \isi{Demonstrativpronomen} von Jaun \citep[282-283]{Stucki1917}}\\

\begin{table}
\caption{Bestimmter Artikel und Demonstrativpronomen von Jaun \citep[282-283]{Stucki1917}}\label{table5.25}
\begin{tabular}{lllllllll}
\lsptoprule
 \multicolumn{5}{c}{{bestimmter Artikel}}  & \multicolumn{4}{c}{{Demonstrativpronomen}}\\\cmidrule(lr){1-5}\cmidrule(lr){6-9}
& {\NOM} & {\AKK} & {\DAT} & {\GEN} &  & {\NOM} & {\AKK} & {\DAT}\\\midrule
\textsc{m.sg} & dər & dər/ə & dəm/əm & ts & \textsc{m.sg} & dær & dær & dɛm\\
\textsc{n.sg} & ts & ts & dəm/əm & ts &     \textsc{n.sg} & das & das & dɛm\\
\textsc{f.sg} & di/t & di/t & dər & dər &   \textsc{f.sg} & di & di & dɛr\\
\textsc{pl} & di/t & di/t & də & dər &      \textsc{m./n.pl} & di & di & dɛnə\\
&  &  &  &  & \textsc{f.pl} & diu & diu & dɛnə\\
\lspbottomrule
\end{tabular}
\end{table}

Die Kategorie \isi{bestimmter Artikel}/\isi{Demonstrativpronomen} ist mit DET1 kodiert. Sowohl bei dieser Kategorie als auch bei der Kategorie \isi{unbestimmter Artikel}/\isi{Possessivpronomen} handelt es sich um Determinierer, die jeweils große Ähnlichkeiten in ihren Formen, aber nicht in ihrer Semantik aufweisen. Da es hier\linebreak jedoch nur um die Form geht, kann die Kategorie \isi{bestimmter Artikel}\slash\is{Demonstrativpronomen}Demonstra-\linebreak tivpronomen mit DET1 und die Kategorie \isi{unbestimmter Artikel}\slash Possessivprono-\linebreak men\is{Possessivpronomen} mit DET2 (\sectref{5.6}) abgekürzt werden. Fallen die Formen des \isi{bestimmten Artikels} und des \isi{Demonstrativpronomens} zusammen, muss der Parameter DET1 nicht weiter definiert werden. Weisen die beiden Wortarten zwei unterschiedliche Formen auf, ist die Wortart zu spezifizieren.

{Beispiel}: Für beide Fälle finden sich Beispiele im Dialekt von Jaun. Der Nominativ Singular des \isi{bestimmten Artikels} lautet \textit{dər} und des \isi{Demonstrativpronomens} \textit{dær} \citep[282]{Stucki1917}. Es sind also zwei unterschiedliche Formen, die durch zwei RRs bestimmt werden müssen (vgl. RRs (\ref{ex:key:118}) und (\ref{ex:key:130})). Im Gegensatz dazu weisen der \isi{bestimmte Artikel} und das \isi{Demonstrativpronomen} im Nominativ/Akkusativ Singular Feminin dieselbe Form auf, nämlich \textit{di} \citep[282]{Stucki1917}. Für diese Form ist folglich nur eine RR nötig:

\ea%117
\label{ex:key:117}
 RR \textsubscript{A,} \textsubscript{\{\textsc{case:nom}} \textsubscript{\tiny $\veebar$}\textsubscript{ \textsc{acc}}\textsubscript{, \textsc{num:sg}, \textsc{gend:f}\},} \textsubscript{DET1} ($\langle$X,$\sigma$ $\rangle$) = \textsubscript{def} $\langle$\textit{di}ˊ,$\sigma$ $\rangle$ \\
\z

In allen anderen Zellen weisen die beiden Wortarten unterschiedliche Formen auf (vgl. \tabref{table5.25}), worauf in \sectref{5.5.3} noch eingegangen wird. Die \isi{Variation} im Akkusativ Singular Maskulin und im Nominativ/Akkusativ Singular Feminin und Plural wird in \sectref{5.5.5} referiert. Es kann aber schon vorweggenommen werden, dass ihre Distribution syntaktisch bedingt ist und die Formen durch RRs definiert werden müssen. Ein immer wiederkehrendes Phänomen in dieser Kategorie sind Kasus- und Genussynkrestismen (z.B. RR \ref{ex:key:118}, \ref{ex:key:124}). Des Weiteren wird \isi{Genus} nicht spezifiziert, wenn die Formen aller drei \isi{Genera} zusammenfallen (RR \ref{ex:key:136}). Es folgen hier die RRs des \isi{bestimmten Artikels} (dazu gehört auch die oben eingeführte RR \REF{ex:key:117}, die für die Wortart unterspezifiziert ist):

\ea%118
\label{ex:key:118}
 RR \textsubscript{A,} \textsubscript{\{\textsc{case:nom}} \textsubscript{\tiny $\veebar$}\textsubscript{ \textsc{acc}}\textsubscript{, \textsc{num:sg}, \textsc{gend:m}\},} \textsubscript{\textsc{det1[art.def]}} ($\langle$X,$\sigma$ $\rangle$) = \textsubscript{def} $\langle$\textit{də}\textit{r}ˊ,$\sigma$ $\rangle$
\z

\ea%119
\label{ex:key:119}
 RR \textsubscript{A,} \textsubscript{\{\textsc{case:}}\textsubscript{\textsc{acc}}\textsubscript{, \textsc{num:sg}, \textsc{gend:m}\},} \textsubscript{\textsc{det1[art.def]}} ($\langle$X,$\sigma$ $\rangle$) = \textsubscript{def} $\langle$\textit{ə}ˊ,$\sigma$ $\rangle$
\z

\ea%120
\label{ex:key:120}
 RR \textsubscript{A,} \textsubscript{\{\textsc{case:nom}} \textsubscript{\tiny $\veebar$}\textsubscript{ \textsc{acc}}\textsubscript{, \textsc{num:sg}, \textsc{gend:n}\},} \textsubscript{\textsc{det1[art.def]}} ($\langle$X,$\sigma$ $\rangle$) = \textsubscript{def} $\langle$\textit{ts}ˊ,$\sigma$ $\rangle$
\z

\ea%121
\label{ex:key:121}
 RR \textsubscript{A,} \textsubscript{\{\textsc{case:nom}} \textsubscript{\tiny $\veebar$}\textsubscript{ \textsc{acc}}\textsubscript{, \textsc{num:sg}, \textsc{gend:f}\},} \textsubscript{\textsc{det1[art.def]}} ($\langle$X,$\sigma$ $\rangle$) = \textsubscript{def} $\langle$\textit{t}ˊ,$\sigma$ $\rangle$
\z

\ea%122
\label{ex:key:122}
 RR \textsubscript{A,} \textsubscript{\{\textsc{case:nom}} \textsubscript{\tiny $\veebar$}\textsubscript{ \textsc{acc}}\textsubscript{, \textsc{num:pl}\},} \textsubscript{\textsc{det1[art.def]}} ($\langle$X,$\sigma$ $\rangle$) = \textsubscript{def} $\langle$\textit{t}ˊ,$\sigma$ $\rangle$
\z

\ea%123
\label{ex:key:123}
 RR \textsubscript{A,} \textsubscript{\{\textsc{case:nom}} \textsubscript{\tiny $\veebar$}\textsubscript{ \textsc{acc}}\textsubscript{, \textsc{num:pl}\},} \textsubscript{\textsc{det1[art.def]}} ($\langle$X,$\sigma$ $\rangle$) = \textsubscript{def} $\langle$\textit{di}ˊ,$\sigma$ $\rangle$
\z

\ea%124
\label{ex:key:124}
 RR \textsubscript{A,} \textsubscript{\{\textsc{case:dat}, \textsc{num:sg}, \textsc{gend:m}} \textsubscript{\tiny $\veebar$} \textsubscript{\scshape n}\textsubscript{\},} \textsubscript{\textsc{det1[art.def]}} ($\langle$X,$\sigma$ $\rangle$) = \textsubscript{def} $\langle$\textit{də}\textit{m}ˊ,$\sigma$ $\rangle$
\z

\ea%125
\label{ex:key:125}
 RR \textsubscript{A,} \textsubscript{\{\textsc{case:dat}, \textsc{num:sg}, \textsc{gend:m}} \textsubscript{\tiny $\veebar$} \textsubscript{\scshape n}\textsubscript{\},} \textsubscript{\textsc{det1[art.def]}} ($\langle$X,$\sigma$ $\rangle$) = \textsubscript{def} $\langle$\textit{əm}ˊ,$\sigma$ $\rangle$
\z

\ea%126
\label{ex:key:126}
 RR \textsubscript{A,} \textsubscript{\{\textsc{case:gen}, \textsc{num:sg}, \textsc{gend:m}} \textsubscript{\tiny $\veebar$} \textsubscript{\scshape n}\textsubscript{\},} \textsubscript{\textsc{det1[art.def]}} ($\langle$X,$\sigma$ $\rangle$) = \textsubscript{def} $\langle$\textit{ts}ˊ,$\sigma$ $\rangle$
\z

\ea%127
\label{ex:key:127}
 RR \textsubscript{A,} \textsubscript{\{\textsc{case:dat}} \textsubscript{\tiny ${\veebar}$} \textsubscript{\GEN, \textsc{num:sg}, \textsc{gend:f}\},} \textsubscript{\textsc{det1[art.def]}} ($\langle$X,$\sigma$ $\rangle$) = \textsubscript{def} $\langle$\textit{dər}ˊ,$\sigma$ $\rangle$
\z

\ea%128
\label{ex:key:128}
 RR \textsubscript{A,} \textsubscript{\{\textsc{case:dat}, \textsc{num:pl}\},} \textsubscript{\textsc{det1[art.def]}} ($\langle$X,$\sigma$ $\rangle$) = \textsubscript{def} $\langle$\textit{də}ˊ,$\sigma$ $\rangle$
\z

\ea%129
\label{ex:key:129}
 RR \textsubscript{A,} \textsubscript{\{\textsc{case:gen}, \textsc{num:pl}\},} \textsubscript{\textsc{det1[art.def]}} ($\langle$X,$\sigma$ $\rangle$) = \textsubscript{def} $\langle$\textit{dər}ˊ,$\sigma$ $\rangle$
\z
\noindent
Die RRs des \isi{Demonstrativpronomens} sind die folgenden (plus die für die Wortart unterspezifizierte RR \ref{ex:key:117}):

\ea%130
\label{ex:key:130}
 RR \textsubscript{A,} \textsubscript{\{\textsc{case:nom}} \textsubscript{\tiny $\veebar$}\textsubscript{ \textsc{acc}}\textsubscript{, \textsc{num:sg}, \textsc{gend:m}\},} \textsubscript{\textsc{det1[pron.dem]}} ($\langle$X,$\sigma$ $\rangle$) = \textsubscript{def} $\langle$\textit{dæ}\textit{r}ˊ,$\sigma$ $\rangle$
\z

\ea%131
\label{ex:key:131}
 RR \textsubscript{A,} \textsubscript{\{\textsc{case:nom}} \textsubscript{\tiny $\veebar$}\textsubscript{ \textsc{acc}}\textsubscript{, \textsc{num:sg}, \textsc{gend:n}\},} \textsubscript{\textsc{det1[pron.dem]}} ($\langle$X,$\sigma$ $\rangle$) = \textsubscript{def} $\langle$\textit{das}ˊ,$\sigma$ $\rangle$
\z

\ea%132
\label{ex:key:132}
 RR \textsubscript{A,} \textsubscript{\{\textsc{case:nom}} \textsubscript{\tiny $\veebar$}\textsubscript{ \textsc{acc}}\textsubscript{, \textsc{num:pl}, \textsc{gend:m}} \textsubscript{\tiny $\veebar$}\textsubscript{ \textsc{n}\},} \textsubscript{\textsc{det1[pron.dem]}} ($\langle$X,$\sigma$ $\rangle$) = \textsubscript{def} $\langle$\textit{di}ˊ,$\sigma$ $\rangle$
\z

\ea%133
\label{ex:key:133}
 RR \textsubscript{A,} \textsubscript{\{\textsc{case:nom}} \textsubscript{\tiny $\veebar$}\textsubscript{ \textsc{acc}}\textsubscript{, \textsc{num:pl}, \textsc{gend:f}\},} \textsubscript{\textsc{det1[pron.dem]}} ($\langle$X,$\sigma$ $\rangle$) = \textsubscript{def} $\langle$\textit{diu}ˊ,$\sigma$ $\rangle$
\z

\ea%134
\label{ex:key:134}
 RR \textsubscript{A,} \textsubscript{\{\textsc{case:dat}, \textsc{num:sg}, \textsc{gend:m}} \textsubscript{\tiny $\veebar$}\textsubscript{ \textsc{n}\},} \textsubscript{\textsc{det1[pron.dem]}} ($\langle$X,$\sigma$ $\rangle$) = \textsubscript{def} $\langle$\textit{dɛm}ˊ,$\sigma$ $\rangle$
\z

\ea%135
\label{ex:key:135}
 RR \textsubscript{A,} \textsubscript{\{\textsc{case:dat}, \textsc{num:sg}, \textsc{gend:f}\},} \textsubscript{\textsc{det1[pron.dem]}} ($\langle$X,$\sigma$ $\rangle$) = \textsubscript{def} $\langle$\textit{dɛr}ˊ,$\sigma$ $\rangle$
\z

\ea%136
\label{ex:key:136}
 RR \textsubscript{A,} \textsubscript{\{\textsc{case:dat}, \textsc{num:pl}\},} \textsubscript{\textsc{det1[pron.dem]}} ($\langle$X,$\sigma$ $\rangle$) = \textsubscript{def} $\langle$\textit{dɛnə}ˊ,$\sigma$ $\rangle$
\z

\subsection{Possessiv-Artikel \textit{s}}\label{5.5.2}

Wie in \sectref{5.1.2} gezeigt wurde, ist in der Substantivflexion einiger Varietäten ein Possessiv-s anzunehmen, das an Eigennamen und Berufsbezeichnungen sowohl im Maskulin als auch im Feminin suffigiert wird. Einige dieser Dialekte haben einen speziellen Artikel, nämlich \textit{ts} oder \textit{s}, der ausschließlich in diesen possessiven Kontexten verwendet wird. Es handelt sich dabei um folgende Dialekte: Uri (Paradigma 88), Vorarlberg (Paradigma 89), Huzenbach (Paradigma 92), Saulgau (Paradigma 93), Petrifeld (Paradigma 95) und Kaiserstuhl (Paradigma 97).

Analog zur RR \REF{ex:key:46} zum Possessiv-S in der Substantivflexion (vgl. \sectref{5.1.2}) ist auch für den \isi{Possessivartikel} eine RR anzunehmen. Sie bestimmt, dass dieser Artikel nur in Possessivkontexten vorkommt. Folgende RR zeigt diese für den Dialekt von Uri:

\ea%137
\label{ex:key:137}
 RR \textsubscript{A,} \textsubscript{\{POSS:+, \textsc{num:sg}\},} \textsubscript{\textsc{det1[art.def]}} ($\langle$X,$\sigma$ $\rangle$) = \textsubscript{def} $\langle$\textit{ts}ˊ,$\sigma$ $\rangle$ \\
\z

Die Kategorie Possessiv trifft weder eine Genus- noch Kasusunterscheidung und tritt nur im Singular auf. Sie steht also nicht in Konkurrenz zu den genus- und kasusspezifizierten Formen, sondern definiert eigene Zellen im Paradigma (vgl. z.B. Paradigma 88 für den Dialekt von Uri).

\subsection{Diachrone Differenzierung des bestimmten Artikels und des Demonstrativpronomens}\label{5.5.3}

Wie bereits in \sectref{4.3.2} dargestellt, ist der \isi{bestimmte Artikel} aus dem einfachen \isi{Demonstrativpronomen} entstanden. Für das Althochdeutsche ist noch von keinem grammatikalisierten \isi{bestimmten Artikel} auszugehen \citep[24]{Schrodt2004}, sondern noch von einem einfachen \isi{Demonstrativpronomen}. Das Mittelhochdeutsche verfügt über einen \isi{bestimmten Artikel}, dessen Gebrauch aber teils von jenem der modernen deutschen Standardsprache abweicht \citep[380-381]{Paul2007}.

Die Formen des \isi{bestimmten Artikels} und des einfachen \isi{Demonstrativpronomens} im Mittelhochdeutschen und in der deutschen Standardsprache fallen vollständig zusammen (vgl. Paradigmen 82 und 83). Eine Ausnahme bildet hier nur der Dativ Singular Maskulin und Neutrum in der deutschen Standardsprache, der nach Präpositionen eine andere Form zeigt (vgl. \sectref{5.5.5}). Im Gegensatz dazu weisen alle alemannischen Dialekte für die beiden Wortarten unterschiedliche Paradigmen auf (z.B. \tabref{table5.25} für Jaun). Dass synchron der \isi{bestimmte Artikel} der alemannischen Dialekte eine andere Form als das \isi{Demonstrativpronomen} zeigt und es sich bei dieser Form diachron um eine reduzierte Form des \isi{Demonstrativpronomens} handelt, ist ein Symptom dafür, dass die Grammatikalisierung des Artikels in diesen Dialekten weiter fortgeschritten ist als im Mittelhochdeutschen und in der Standardsprache. Dabei können jene Dialekte, in denen keine Formen identisch sind, von den Dialekten unterschieden werden, in denen einige wenige Zellen für die beiden Wortarten gleiche Formen haben. Die Dialekte, in denen die beiden Wortarten zu hundert Prozent verschiedene Formen aufweisen, sind: Visperterminen, Uri, Zürich, Huzenbach, Saulgau, Stuttgart, Petrifeld, Elisabethtal, Kaiserstuhl, Münstertal und Colmar. Die Dialekte, in denen sich die beiden Wortarten einzelne Formen teilen, sind in \tabref{table5.26} gelistet, wie auch die Zellen der beiden Wortarten, die zusammenfallen. Betroffen vom Zusammenfall sind unterschiedliche Zellen. Am häufigsten jedoch weist der Dativ Singular Feminin des \isi{bestimmten Artikels} und des \isi{Demonstrativpronomens} dieselben Formen auf.

% {\tabref{table5.26}: Gemeinsame Formen des \isi{bestimmten Artikels} und des Demonstrativpronomens}

\begin{table}
\caption{Gemeinsame Formen des bestimmten Artikels und des Demonstrativpronomens}\label{table5.26}
\begin{tabular}{l>{\scshape}l}
\lsptoprule
{Dialekt} & {\upshape zusammengefallene Zellen}\\
\midrule
Issime & dat./gen.sg.f, gen.pl\\
Jaun & nom./akk.sg.f\\
Sensebezirk & nom./akk.sg.f, nom./akk.pl\\
Vorarlberg & dat.sg.m./n./f\\
Bern & dat.sg.f\\
Elsass (Ebene) & nom./akk.sg.m., dat.sg.f\\
\lspbottomrule
\end{tabular}
\end{table}

\subsection{Freie Variation im bestimmten Artikel und Demonstrativpronomen}\label{5.5.4}

In der Hälfte der untersuchten Varietäten kommt freie \isi{Variation} im \isi{bestimmten Artikel} und/oder im \isi{Demonstrativpronomen} vor. In beiden Wortarten sind davon viele unterschiedliche Zellen des Paradigmas betroffen, im Dativ Singular Maskulin und Neutrum des \isi{bestimmten Artikels} jedoch kommt sie häufiger vor als in den übrigen Zellen.

Für die RRs der freien \isi{Variation} gilt dasselbe wie für die RRs der bis hier diskutierten Kategorien (vgl. \sectref{5.1.2}, \sectref{5.2.3}, \sectref{5.3.4}, \sectref{5.4}): Die RRs der beiden Varianten müssen gleich spezifisch sein, damit sie einander bei der Definition der Form für eine bestimmte Zelle nicht blockieren. Theoretisch begründet wurde dies in \sectref{4.1.3.3} Als Bespiel fungiert hier das \isi{Demonstrativpronomen} von Colmar, in dem Nominativ und Akkusativ Singular Maskulin jeweils zwei Formen aufweisen, nämlich \textit{tar} und \textit{ta} \citep[83]{Henry1900}. Damit für eine Zelle zwei Formen definiert sind, werden zwei gleich spezifische RRs benötigt wie in \REF{ex:key:138} und \REF{ex:key:139} bezüglich des Paradigmas von Colmar.

\ea%138
\label{ex:key:138}
 RR \textsubscript{A,} \textsubscript{\{\textsc{case:nom}} \textsubscript{\tiny $\veebar$}\textsubscript{ \textsc{acc}}\textsubscript{, \textsc{num:sg}, \textsc{gend:m}\},} \textsubscript{\textsc{det1[pron.dem]}} ($\langle$X,$\sigma$ $\rangle$) = \textsubscript{def} $\langle$\textit{tar}ˊ,$\sigma$ $\rangle$
\z

\ea%139
\label{ex:key:139}
 RR \textsubscript{A,} \textsubscript{\{\textsc{case:nom}} \textsubscript{\tiny $\veebar$}\textsubscript{ \textsc{acc}}\textsubscript{, \textsc{num:sg}, \textsc{gend:m}\},} \textsubscript{\textsc{det1[pron.dem]}} ($\langle$X,$\sigma$ $\rangle$) = \textsubscript{def} $\langle$\textit{ta}ˊ,$\sigma$ $\rangle$
\z

\subsection{Syntaktisch bedingte Variation im bestimmten Artikel}\label{5.5.5}

Von der freien \isi{Variation} ist die syntaktisch bedingte \isi{Variation} zu unterscheiden und innerhalb dieser zwei Faktoren: a) Der \isi{bestimmte Artikel} ist nur Teil einer NP oder diese NP gehört zu einer PP; b) der Artikel steht vor einem \isi{Adjektiv} oder einem \isi{Substantiv}. Im Gegensatz zur freien \isi{Variation} ist von der syntaktisch bedingten \isi{Variation} also nur der \isi{bestimmte Artikel} betroffen. Ob es sich um freie oder syntaktisch bedingte \isi{Variation} handelt, hat keinen Einfluss auf das Aussehen der RR, denn die Morphologie stellt nur die Formen zur Verfügung. Wie die Formen distribuiert sind, ist Aufgabe der Syntax, d.h., die Regeln der Distribution dürfen nicht in den RRs definiert werden. Trotzdem soll die syntaktische bedingte \isi{Variation} kurz besprochen werden, denn wenn in Abhängigkeit von der syntaktischen Umgebung unterschiedliche Artikel verwendet werden, impliziert dies, dass die Morphologie unterschiedliche Formen dieses Artikels definieren muss. In der Folge wird zuerst auf die Abfolge Artikel + \isi{Adjektiv} und dann auf die Abfolge Präposition + Artikel + \isi{Substantiv} vs. Artikel + \isi{Substantiv} eingegangen. Eine Übersicht bietet \tabref{table5.27}.

{Artikel + Adjektiv}: Einige alemannische Dialekte weisen im Nominativ und Akkusativ Singular Feminin sowie im Nominativ und Akkusativ Plural zwei Formen des \isi{bestimmten Artikels} auf, wobei es sich immer um die Varianten \textit{di} und \textit{d} handelt (vgl. Paradigmen 86, 87, 88, 89, 91). \textit{Di} wird vor einem \isi{Adjektiv} verwendet, \textit{d} vor einem \isi{Substantiv} (die Qualität von \textit{d} kann je nach Dialekt variieren), z.B. \textit{t Han} ʻdie Handʼ, \textit{di grosi Han} ʻdie große Handʼ \citep[187, 191, 200]{Henzen1927}. Diese \isi{Variation} ist nur in den hoch- und höchstalemannischen Dialekten zu finden, mit Ausnahme der Walser Dialekte und des Dialektes von Zürich. Zwar zeigen auch die schwäbischen Dialekte von Huzenbach (Paradigma 92) und Petrifeld (Paradigma 95) \isi{Variation}, aber nur im Nominativ und Akkusativ Plural. Außerdem finden sich in den Grammatiken dieser schwäbischen Dialekte keine Angaben zur Distribution dieser Varianten, weshalb von freier \isi{Variation} ausgegangen werden kann.

{(Präposition +) Artikel + Substantiv}: Die Form des Artikels variiert auch in Abhängigkeit davon, ob die NP Teil einer PP ist oder nicht, d.h., ob dem Artikel eine Präposition vorangeht oder nicht. Von dieser \isi{Variation} betroffen sind der Akkusativ Singular Maskulin und der Dativ Singular Maskulin/Neutrum, die nun nacheinander besprochen werden.

Im Akkusativ Singular Maskulin steht eine Form des Typs \textit{der} oder \textit{de}, wenn dem Artikel keine Präposition vorangeht, eine Form des Typs \textit{e}, wenn dem Artikel eine Präposition vorangeht. Beispiel: \textit{dər} \textit{w\=i} ʻden Weinʼ, \textit{für} \textit{ə} \textit{w\=i} ʻfür den Weinʼ \citep[283]{Stucki1917}. Dies kommt in den hoch- und höchstalemannischen Dialekten (außer den Walser Dialekten und dem Dialekt von Zürich) vor, also in genau denselben Dialekten, die bereits die \isi{Variation} vor \isi{Adjektiv} aufgewiesen haben (vgl. Paradigmen 86, 87, 88, 89. 91). Auch im Dialekt von Saulgau ist dieselbe \isi{Variation} zu beobachten, wobei jedoch nach der Präposition die Form \textit{n} (und nicht \textit{ə}) steht (Paradigma 93). Schließlich zeigt der Dialekt des Elsass (Ebene) eine ähnliche \isi{Variation}: \textit{Dər} steht, wenn keine Präposition vorangeht, \textit{dər} oder \textit{də}, wenn eine Präposition vorangeht (Paradigma 100).

Dieselbe syntaktisch bedingte \isi{Variation} (±Präposition + Artikel), aber im Dativ Singular Maskulin/Neutrum weisen folgende Varietäten auf: die drei elsässischen Dialekte (Paradigmen 98, 99, 100), die höchstalemannischen Dialekte außer Issime (Paradigmen 85, 86, 87, 88), der Dialekt von Saulgau (Paradigma 93) und die deutsche Standardsprache (Paradigma 83). Beim Artikel, der nach einer Präposition steht, handelt es sich um eine reduzierte Form jenes Artikels, dem keine Präposition vorangeht, z.B. \textit{dum}/\textit{um} (Sensebezirk, \citealt[200]{Henzen1927}). Eine phonologische Erklärung kann jedoch nicht gefunden werden, da keine Vokalcluster o.ä. auftreten. Anders sieht dies aus, wenn man nur die Abfolge Präposition + Artikel betrachtet: Lautet die Präposition vokalisch aus und der Artikel vokalisch an, wird der anlautende Vokal des Artikels getilgt. Im Dialekt des Sensebezirks z.B. lautet der Artikel \textit{um}, wenn die Präposition konsonantisch auslautet, aber \textit{m}, wenn die Präposition vokalisch auslautet \citep[200]{Henzen1927}. Ebenfalls phonologisch bedingt ist die \isi{Variation} im Dialekt von Zürich: Geht dem Artikel keine oder eine konsonantisch auslautende Präposition voraus, lautet der Artikel \textit{əm}, geht dem Artikel eine vokalisch auslautende Präposition voraus, lautet der Artikel \textit{m} \citep[103]{Weber1987}. Da die Varianten phonologisch bedingt sind, müssen für diese Formen keine RRs angenommen werden. Im Gegensatz zum Dialekt von Zürich ist im Dialekt von Visperterminen die \isi{Variation} nicht phonologisch bedingt. Die beiden Varianten \textit{dum} (keine Präposition oder konsonantisch auslautende Präposition) und \textit{m} (vokalisch auslautende Präposition) lauten beide konsonantisch an \citet[141-142]{Wipf1911}. Beide Formen sind also durch RRs zu definieren. Schließlich weisen folgende Dialekte im Dativ Singular Maskulin/Neutrum freie \isi{Variation} auf: Elisabethtal (Paradigma 96), Vorarlberg (Paradigma 89) und Jaun (Paradigma 86, zusätzlich zur syntaktisch bedingten \isi{Variation}).

Zusammenfassend kann festgehalten werden, dass in den Dialekten von Uri, Jaun und des Sensebezirks (alle höchstalemannisch) syntaktisch bedingte \isi{Variation} besonders häufig vorkommt. In etwas geringerem Maße sind die hochalemannischen Dialekte (außer Zürich), die elsässischen Dialekte sowie der schwäbische Dialekt von Saulgau betroffen.

% % {\tabref{table5.27}: Syntaktisch bedingte \isi{Variation} im bestimmten Artikel}\\

\begin{table}
\caption{Syntaktisch bedingte Variation im bestimmten Artikel}\label{table5.27}
\begin{tabular}{p{2.25cm}p{4cm}p{4cm}}
\lsptoprule
{± Präposition\newline + best. Artikel:} & \textsc{akk.sg.m} & \textsc{dat.sg.m/n}\\
\midrule
{Dialekte:} & 
• Höchstalemannisch\newline\hspaceThis{•} (außer Walser)\newline • Hochalemannisch\newline\hspaceThis{•} (außer Zürich)\newline • Saulgau\newline • Elsass (Ebene) & 
• Höchstalemannisch\newline\hspaceThis{•} (außer Issime)\newline • Saulgau\newline • Elsässischen Dialekte\newline • Standardsprache\\
\midrule
{best. Artikel\newline ±\isi{Adjektiv}:} & \multicolumn{2}{l}{\textsc{nom/akk.sg.f + nom/akk.pl}}\\
\midrule
{Dialekte:} & \multicolumn{2}{l}{• Höchstalemannisch (außer Walser)}\\ & \multicolumn{2}{l}{• Hochalemannisch (außer Zürich)}\\
\lspbottomrule
\end{tabular}
\end{table}
%AUFZÄHLUNG SIEHT SELTSAM AUS

\section{Unbestimmter Artikel / Possessivpronomen}\label{5.6}

\subsection{Allgemeines und Realisierungsregeln}\label{5.6.1}

{Zwei Wortarten, eine Kategorie}: Der \isi{unbestimmte Artikel} und das \isi{Possessivpronomen} bilden zusammen eine Kategorie, was in \sectref{4.3.2} bereits begründet wurde und hier nur kurz wiederholt wird. Erstens weist die Flexion dieser Wortarten im Vergleich zu den anderen Wortarten die größten Ähnlichkeiten auf. Wenn die Paradigmen auf ein Minimum reduziert werden sollen, ist dies am besten zu erreichen, wenn Wortarten mit ähnlicher Flexion dieselbe Kategorie bilden. Zweitens hat sich die Flexion dieser Wortarten aus diachroner Sicht parallel entwickelt.

{Parameter DET}: In den RRs wird die Kategorie des \isi{unbestimmten Artikels} und des \isi{Possessivpronomens} mit DET2 kodiert. Wie in \sectref{5.5.1} zum \isi{bestimmten Artikel} und zum \isi{Demonstrativpronomen} gezeigt wurde, handelt es sich bei den Wortarten der Kategorien DET1 und DET2 um Determinierer, die jeweils in ihrer Form, aber nicht in ihrer Semantik große Ähnlichkeiten zeigen. Deshalb können diese Kategorien mit DET1 und DET2 kodiert werden.

Eine weitere allgemeine Beobachtung betrifft das Inventar an Wortarten. Alle der hier untersuchten Varietäten verfügen über einen \isi{unbestimmten Artikel} und ein \isi{Possessivpronomen}. Davon weicht nur das Althochdeutsche ab, das über keinen grammatikalisierten \isi{unbestimmten Artikel} verfügt \citep[26]{Schrodt2004}. Im weiteren Verlauf dieses Unterkapitels wird für beide Wortarten beschrieben, ob und welche Arten von \isi{Affixen} und Wur\-zel-/Stamm\-al\-ter\-na\-tio\-nen sie aufweisen, welche \isi{Blöcke} anzunehmen und welche morphosyntaktischen Eigenschaften zu unterscheiden sind. Diese Fragen werden zuerst für den \isi{unbestimmten Artikel} und dann für das \isi{Possessivpronomen} beantwortet. Abgeschlossen wird dieses Unterkapitel mit einigen Bemerkungen zu \isi{Synkretismen}.

Der \textsc{unbestimmte Artikel} weist in vielen untersuchten Varietäten keine \isi{Affixe} auf, was in der Diskussion der vorangehenden Wortarten schon  dargestellt wurde. In diesen Varietäten definieren die RRs also die gesamte Form. Auch sind keine \isi{Blöcke} nötig. Folgende RR definiert die Form des \isi{unbestimmten Artikels} im Dativ Singular Maskulin/Neutrum im Dialekt des Kaiserstuhls, der \textit{imɐ} lautet \citep[376]{Noth1993}:

\ea%140
\label{ex:key:140}
 RR \textsubscript{A,} \textsubscript{\{\textsc{case:dat}}\textsubscript{, \textsc{num:sg}, \textsc{gend:m}} \textsubscript{\tiny $\veebar$} \textsubscript{\scshape n}\textsubscript{\},} \textsubscript{\textsc{det2[art.indef]}} ($\langle$X,$\sigma$ $\rangle$) = \textsubscript{def} $\langle$\textit{imɐ}ˊ,$\sigma$ $\rangle$ \\
\z

Ausnahmen bilden folgende Phänomene: \isi{Suffixe} und \isi{Präfixe}. Präfigierung sowie ihre RRs und \isi{Blöcke} werden in \sectref{5.6.5} vorgestellt. Präfigierung gibt es in den folgenden Dialekten: Jaun, Sensebezirk, Uri, Zürich, Bern. Suffigierung kommt im Mittelhochdeutschen, in der deutschen Standardsprache und in den Walser Dialekten (Issime und Visperterminen) vor. Die RRs entsprechen dem Typ, der bereits für die \isi{Substantive} und die \isi{Adjektive} eingeführt wurde: Die \isi{Wurzel}\linebreak stammt aus dem \isi{Radikon}, die RRs definieren die \isi{Suffixe}. Dies ist am \isi{unbestimmten Artikel} \textit{ein-ən} (Akkusativ Singular Neutrum) in der deutschen Standardsprache exemplifiziert:

\ea%141
\label{ex:key:141}
 RR \textsubscript{A,} \textsubscript{\{\textsc{case:}}\textsubscript{\textsc{acc}}\textsubscript{, \textsc{num:sg}, \textsc{gend:m}\},} \textsubscript{\textsc{det2[]}} ($\langle$X,$\sigma$ $\rangle$) = \textsubscript{def} $\langle$X\textit{ən}ˊ,$\sigma$ $\rangle$\\
\z

Die RR von Varietäten, deren \isi{unbestimmter Artikel} suffigiert wird, stehen im selben \isi{Block}. Mehrere \isi{Blöcke} sind nicht nötig, da jede Zelle maximal ein \isi{Suffix} aufweist, die Abfolge von \isi{Suffixen} muss also nicht definiert werden. Für Varietäten mit präfigiertem unbestimmtem Artikel müssen zwei \isi{Blöcke} angenommen werden, was in \sectref{5.6.5} erörtert wird. Die morphosyntaktischen Eigenschaften, die im \isi{unbestimmten Artikel} unterschieden werden müssen, sind \isi{Kasus} und \isi{Genus}. \isi{Numerus} ist nur im Mittelhochdeutschen zu bestimmen, weil nur das Mittelhochdeutsche im Plural einen \isi{unbestimmten Artikel} hat.

Beim \textsc{Possessivpronomen} kommt die \isi{Wurzel} aus dem \isi{Radikon}, die \isi{Affixe} werden durch RRs definiert. Das \isi{Possessivpronomen} verfügt ausschließlich über \isi{Suffixe}. Zusätzlich alterniert die \isi{Wurzel} in den folgenden Varietäten: Issime, Vorarlberg, Zürich, Huzenbach, Saulgau und Petrifeld. Wie und in welchen Kontexten die \isi{Wurzeln}/Stämme variieren und wie die RRs aussehen, wird in \sectref{5.6.8} besprochen. Verfügt eine Varietät im \isi{Possessivpronomen} nur über \isi{Suffixe}, ist nur ein \isi{Block} nötig, weil nie mehr als ein \isi{Suffix} angehängt wird. Kommen zu den \isi{Suffixen} noch Wur\-zel-/Stamm\-al\-ter\-na\-tio\-nen hinzu, sind zwei \isi{Blöcke} anzunehmen, worauf in \sectref{5.6.8} genauer eingegangen wird.

Die meisten modernen Dialekte in diesem Sample haben im \isi{Possessivpronomen} unterschiedliche Suffixparadigmen, und zwar abhängig davon, an welches \isi{Possessivpronomen} suffigiert wird. Z.B. weist der Nominativ/Akkusativ Singular Feminin im Dialekt des Kaiserstuhls ein \isi{Suffix} -\textit{i} auf, aber nur im \isi{Possessivpronomen} der 3. Person Singular Feminin und der 3. Person Plural (\textit{ir}-\textit{i}, \citealt[382]{Noth1993}). In allen anderen \isi{Possessivpronomen} steht im Nominativ/Akkusativ Singular Feminin kein \isi{Suffix} (\textit{mi}, \citealt{Noth1993}: 380). In den RRs für die \isi{Possessivpronomen} sind also nicht nur die morphosynaktischen Eigenschaften des \isi{Suffixes} zu bestimmen, sondern auch jene des \isi{Possessivpronomens}. Darauf wird in \sectref{5.6.6} noch genauer eingegangen. Die RRs für das genannte Beispiel sehen also wie folgt aus:

\ea%142
\label{ex:key:142}
 RR \textsubscript{A,} \textsubscript{\{\textsc{case:} \textsc{nom}} \textsubscript{\tiny $\veebar$}\textsubscript{ \textsc{acc}}\textsubscript{, \textsc{num:sg}, \textsc{gend:f}\},} \textsubscript{\textsc{det2[pron.poss}, \textsc{pers:3}, \textsc{num:sg}, \textsc{gend:f}]} ($\langle$X,$\sigma$$\rangle$) = \textsubscript{def} $\langle$X\textit{i}ˊ,$\sigma$$\rangle$
\z

\ea%143
\label{ex:key:143}
 RR \textsubscript{A,} \textsubscript{\{\textsc{case:} \textsc{nom}} \textsubscript{\tiny $\veebar$}\textsubscript{ \textsc{acc}}\textsubscript{, \textsc{num:sg}, \textsc{gend:f}\},} \textsubscript{\textsc{det2[pron.poss}, \textsc{pers:3}, \textsc{num:pl}]} ($\langle$X,$\sigma$$\rangle$) = \textsubscript{def} $\langle$X\textit{i}ˊ,$\sigma$$\rangle$
\z

In den geschwungenen Klammern werden die morphosyntaktischen Eigenschaften des \isi{Suffixes} definiert. In den eckigen Klammern stehen weitere Angaben zur Kategorie DET2, wie Wortart und weitere Angaben zu dieser Wortart, d.h., um welches \isi{Possessivpronomen} es sich handelt. Welche Dialekte unterschiedliche Suffixparadigmen für welche \isi{Possessivpronomen} haben, wird in \sectref{5.6.6} gezeigt.

Schließlich werden die \textsc{Synkretismen} in der Kategorie DET2 wie bei den übrigen Kategorien erfasst. Die RR \REF{ex:key:140} zeigt einen Genussynkretismus, die RR \REF{ex:key:142} einen Kasussynkretismus. Wird \isi{Numerus} nicht unterschieden, bleibt \isi{Numerus} unterspezifiziert. Dasselbe gilt auch für DET2: Weisen der \isi{unbestimmte Artikel} und das \isi{Possessivpronomen} dieselbe Form auf, so muss DET2 nicht weiter spezifiziert werden (z.B. RR \ref{ex:key:141}).

\subsection{Diachrone Differenzierung der Paradigmen des unbestimmten Artikels und des Possessivpronomens}\label{5.6.2}

Die deutsche Standardsprache hat ein Set an \isi{Suffixen} für den \isi{unbestimmten Artikel} und das \isi{Possessivpronomen} (vgl. Paradigma 103). Bereits im Mittelhochdeutschen weist die Flexion des \isi{unbestimmten Artikels} und des \isi{Possessivpronomens} keine Unterschiede auf, denn beide werden stark flektiert (\citealt[216-217]{Paul2007}, Paradigma 102). Wenn dem \isi{Possessivpronomen} aber ein Artikel vorangeht, kann es stark oder schwach flektiert werden \citep[369]{Paul2007}.

Im Gegensatz zum Mittelhochdeutschen und zur deutschen Standardsprache verfügen die hier untersuchten alemannischen Dialekte über zwei separate Paradigmen für den \isi{unbestimmten Artikel} und das \isi{Possessivpronomen} (vgl. Paradigmen 104-120), wobei in einigen Dialekten die Formen einzelner Zellen der beiden Wortarten identisch sind. Keine \isi{Synkretismen} zwischen den beiden Wortarten sind in folgenden Dialekten zu finden: Issime, Jaun, Sensebezirk, Uri, Bern, Elisabethtal, Kaiserstuhl, Münstertal und Elsass. Sind Zellen der beiden Wortarten identisch, so betrifft das (mit einer Ausnahme) stets den Dativ Singular Feminin, und zwar in folgenden Dialekten: Vorarlberg, Zürich, Huzenbach, Saulgau, Stuttgart, Petrifeld und Colmar. Dies tangiert also vor allem östliche hochalemannische Dialekte, die schwäbischen Dialekte (mit Ausnahme der Sprachinsel Elisabethtal) und einen elsässischen Dialekt. Auch im Dialekt von Visperterminen sind einige Zellen der beiden Wortarten gleich, aber im Nominativ und Akkusativ Singular Neutrum sowie im Dativ und Genitiv Singular Maskulin und Neutrum.

Es kann also festgehalten werden, dass die alemannischen Dialekte (mit Ausnahme einiger wenigen Zellen) - im Gegensatz zum Mittelhochdeutschen, aber vor allem im Gegensatz zur deutschen Standardsprache - für den \isi{unbestimmten Artikel} und für das \isi{Possessivpronomen} zwei unterschiedliche Paradigmen grammatikalisiert haben. Zur Verdeutlichung soll dies am Beispiel des Nominativs Singular Feminin illustriert werden: Im Dialekt von Visperterminen lautet der \isi{unbestimmte Artikel} \textit{a}, das \isi{Possessivpronomen} \textit{m\=in}-\textit{i}, in der deutschen Standardsprache \textit{ein}-\textit{ə} bzw. \textit{mein}-\textit{ə}. Für die deutsche Standardsprache ist also nur ein Satz an \isi{Suffixen} für beide Wortarten nötig, für den Dialekt von Visperterminen jedoch zwei. Die Grammatikalisierung dieser beiden Wortarten ist folglich in den alemannischen Dialekten weiter fortgeschritten als in der deutschen Standardsprache. Die gleichen Beobachtungen wurden in \sectref{5.5.3} bezüglich des \isi{bestimmten Artikels} und des \isi{Demonstrativpronomens} gemacht.

\subsection{Unbestimmter Artikel und Possessivpronomen: Freie Variation}\label{5.6.3}

Wie für alle bis hierhin besprochenen Wortarten ist freie \isi{Variation} auch im \isi{unbestimmten Artikel} und im \isi{Possessivpronomen} zu finden. Dabei kann unterschieden werden, ob \isi{Wurzel} + \isi{Suffix} und nur die \isi{Wurzel} zueinander in freier \isi{Variation} stehen oder ob zwei \isi{Suffixe} in derselben Zelle vorkommen. Diese Typen an freier \isi{Variation} kommen in der Adjektivflexion ebenfalls vor (vgl. \sectref{5.2.3}). \tabref{table5.28} gibt einen Überblick darüber, welcher Typ freier \isi{Variation} in welcher Wortart und in welchen Dialekten zu beobachten ist. Dabei kann festgehalten werden, dass es sich um ein weit verbreitetes Phänomen handelt.

% {\tabref{table5.28}: Freie \isi{Variation} im \isi{unbestimmten Artikel} und im Possessivpronomen}\\

\begin{table}
\caption{Freie Variation im unbestimmten Artikel und im Possessivpronomen}\label{table5.28}
\begin{tabularx}{\textwidth}{XXX} 
\lsptoprule
& {unbestimmter Artikel} & {Possessivpronomen}\\
\midrule
{freie \isi{Variation} \isi{Suffix}/-ø} & Mhd & Ahd, Mhd, Jaun, Sensebezirk, Uri, Saulgau, Stuttgart\\
\midrule
{freie \isi{Variation} \isi{Suffix}/ Suffix} & Visperterminen, Sensebezirk, Uri, Vorarlberg, Zürich, Bern, Huzenbach, Saulgau, Petrifeld & Sensebezirk, Vorarlberg, Bern, Huzenbach, Stuttgart, Kaiserstuhl\\
\lspbottomrule
\end{tabularx}
\end{table}

Im \isi{Possessivpronomen} sind alle \isi{Kasus}, \isi{Genera} und beide \isi{Numeri} von freier \isi{Variation} betroffen. Es kann keine Tendenz festgestellt werden. Dies ist jedoch bezüglich des \isi{unbestimmten Artikels} möglich, wie \tabref{table5.29} zeigt. Freie \isi{Variation} kommt im \isi{unbestimmten Artikel} nur im Singular und besonders häufig im Dativ vor.

% {\tabref{table5.29}: Zellen des \isi{unbestimmten Artikels} mit freier Variation}\\

\begin{table}
\caption{Zellen des unbestimmten Artikels mit freier Variation}\label{table5.29}
\begin{tabularx}{\textwidth}{>{\scshape}XX}
\lsptoprule
{\upshape Zellen mit freier \isi{Variation} im\newline \isi{unbestimmten Artikel}:} & {Dialekte:}\\
\midrule
dat.sg.m/n/f & Sensebezirk, Vorarlberg, Zürich, Bern, Saulgau\\
dat.sg.m/n & Visperterminen, Uri\\
nom.sg.m & Huzenbach, Petrifeld\\
akk.sg.m & Petrifeld\\
akk.sg.n & Mittelhochdeutsch\\
\lspbottomrule
\end{tabularx}
\end{table}

Wie schon mehrmals gezeigt, müssen die RRs gleich spezifisch sein, damit sie dieselbe Zelle des Paradigmas definieren können. Dies wurde in \sectref{4.1.3.3} theoretisch begründet. Folglich muss auch die \isi{Wurzel} stipuliert werden, wenn diese mit \isi{Wurzel} + \isi{Suffix} in freier \isi{Variation} steht. Nur so wird gewährleistet, dass beide Formen dieselbe Zelle definieren. Dies wird an den folgenden RRs für das \isi{Possessivpronomen} der 1. und 2. Person Singular im Dialekt des Sensebezirks illustriert (vgl. Paradigma 107). Die RRs \REF{ex:key:144} und \REF{ex:key:145} definieren -ø/-\textit{ər (}z.B. \textit{mi}, \textit{minər}\footnote{-\textit{N}- dient synchron der Hiatbeseitigung und wird von der Phonologie automatisch eingefügt. Genauer diskutiert wird dies in \sectref{5.6.8}.} im Nominativ und Akkusativ Plural), die RRs \REF{ex:key:146} und \REF{ex:key:147} -\textit{um}/-\textit{m} (z.B. \textit{mim}, \textit{minum}\footnote{Vgl. vorangehende Fußnote.} im Dativ Singular Maskulin und Neutrum) \citep[198]{Henzen1927}.

\ea%144
\label{ex:key:144}
 RR \textsubscript{A,} \textsubscript{\{\textsc{case:} \textsc{nom}} \textsubscript{\tiny $\veebar$}\textsubscript{ \textsc{acc}}\textsubscript{, \textsc{num:pl}\}}\textsubscript{,} \textsubscript{\textsc{det2[pron.poss}, \textsc{pers:1}} \textsubscript{\tiny $\veebar$} \textsubscript{\textsc{2}}\textsubscript{, \textsc{num:sg}]} ($\langle$X,$\sigma$ $\rangle$) = \textsubscript{def} $\langle$Xˊ,$\sigma$ $\rangle$
\z

\ea%145
\label{ex:key:145}
 RR \textsubscript{A,} \textsubscript{\{\textsc{case:} \textsc{nom}} \textsubscript{\tiny $\veebar$}\textsubscript{ \textsc{acc}}\textsubscript{, \textsc{num:pl}\}}\textsubscript{,} \textsubscript{\textsc{det2[pron.poss}, \textsc{pers:1}} \textsubscript{\tiny $\veebar$} \textsubscript{\textsc{2}}\textsubscript{, \textsc{num:sg}]} ($\langle$X,$\sigma$ $\rangle$) = \textsubscript{def} $\langle$X\textit{ər}ˊ,$\sigma$ $\rangle$
\z

\ea%146
\label{ex:key:146}
 RR \textsubscript{A,} \textsubscript{\{\textsc{case: dat}, \textsc{num:sg}, \textsc{gend: m}} \textsubscript{\tiny $\veebar$} \textsubscript{\textsc{n}\}}\textsubscript{,} \textsubscript{\textsc{det2[pron.poss}, \textsc{pers:1}} \textsubscript{\tiny $\veebar$} \textsubscript{\textsc{2}}\textsubscript{, \textsc{num:sg}]} ($\langle$X,$\sigma$ $\rangle$) = \textsubscript{def}   $\langle$X\textit{um}ˊ,$\sigma$ $\rangle$
\z

\ea%147
\label{ex:key:147}
 RR \textsubscript{A,} \textsubscript{\{\textsc{case: dat}, \textsc{num:sg}, \textsc{gend: m}} \textsubscript{\tiny $\veebar$} \textsubscript{\textsc{n}\}}\textsubscript{,} \textsubscript{\textsc{det2[pron.poss}, \textsc{pers:1}} \textsubscript{\tiny $\veebar$} \textsubscript{\textsc{2}}\textsubscript{, \textsc{num:sg}]} ($\langle$X,$\sigma$ $\rangle$) = \textsubscript{def}   $\langle$X\textit{m}ˊ,$\sigma$ $\rangle$
\z

\subsection{Unbestimmter Artikel: Syntaktisch bedingte Variation}\label{5.6.4}

{(Präposition +) Artikel + Substantiv}: Wie im \isi{bestimmten Artikel} (vgl. \sectref{5.5.5}) gibt es auch im \isi{unbestimmten Artikel} syntaktisch bedingte \isi{Variation}. Im Gegensatz zum \isi{bestimmten Artikel} kommt diese jedoch nur im Kontext ± vorangehende Präposition vor. Im \isi{bestimmten Artikel} betrifft dies die Zellen Akkusativ Singular Maskulin und Dativ Singular Maskulin/Neutrum, im \isi{unbestimmten Artikel} ebenfalls Akkusativ und Dativ, aber jeweils alle drei \isi{Genera}. In der Folge wird zuerst die \isi{Variation} im Dativ und anschließend jene im Akkusativ vorgestellt.

{Dativ}: In allen Dialekten fallen im Dativ Singular die Formen des Maskulins und des Neutrums zusammen, das Feminin hat eine eigene Form, z.B. \textit{æmə}/\textit{əmə} (mask. und neut.), \textit{ænərə}/\textit{ənərə} (fem.) (Münstertal, \citealt[45]{Mankel1886}). Alle Dialekte, deren \isi{unbestimmter Artikel} syntaktisch bedingte \isi{Variation} zeigt, haben zwei Formen: eine für den Kontext, wenn keine Präposition vorangeht, eine für den Kontext, wenn eine konsonantisch auslautende Präposition vorangeht. Im Dialekt des Münstertals wird \textit{æmə}/\textit{ænərə} verwendet, wenn dem \isi{unbestimmten Artikel} keine Präposition vorangeht, \textit{əmə}/\textit{ənərə}, wenn dem \isi{unbestimmten Artikel} eine konsonantisch auslautende Präposition vorangeht \citep[45-46]{Mankel1886}. Diese \isi{Variation} kann phonologisch nicht erklärt werden. In allen betroffenen Dialekten sind folglich im Dativ für Maskulin/Neutrum und Feminin jeweils zwei \isi{unbestimmte Artikel} anzunehmen, d.h. zwei RRs. Des Weiteren kann die Form des \isi{unbestimmten Artikels}, der einer vokalisch auslautenden Präposition folgt, stets von einer der beiden beschriebenen Formen abgeleitet werden. Im Dialekt von Münstertal lautet der \isi{unbestimmte Artikel} nach vokalisch auslautender Präposition \textit{mə}/\textit{nərə} \citep[45]{Mankel1886}. Um einen \isi{Hiat} zu vermeiden, wird also dem \isi{unbestimmten Artikel} nach vokalisch auslautender Präposition die erste Silbe getilgt: \textit{zu}-\textit{əmə} $\rightarrow$ \textit{zu}-\textit{mə}. Interessanterweise wird hier jedoch nicht die Defaultstrategie verwendet, um einen \isi{Hiat} zu vermeiden, nämlich das Einfügen von \textit{n}. Man würde also eigentlich erwarten: *\textit{zu}-\textit{n}-\textit{əmə}. Da hier nicht die Defaultstratiegie eingesetzt wird, die für die gesamte Phonologie gilt (n-Epenthese), sondern eine Silbe getilgt wird, muss dies durch eine RR definiert werden. Man benötigt aber nicht für jede Form des \isi{unbestimmten Artikels} nach auf Vokal auslautender Präposition eine RR. Weil die erste Silbe einer bereits vorhandenen Form getilgt wird, ist nur eine RR nötig, die diese Tilgung definiert (vgl. Anhang B: Elsass (Ebene) RR \REF{ex:key:81}, Münstertal RR \REF{ex:key:81}, Zürich RR \REF{ex:key:95}, Bern RR \REF{ex:key:111}, Uri RR \REF{ex:key:124}, Sensebezirk RR \REF{ex:key:124}, Jaun RR \REF{ex:key:136}). Folglich stehen diese Tilgungsregeln auch im letzten \isi{Block}.

\tabref{table5.30} gibt eine Übersicht, in welchen Dialekten im Dativ des \isi{unbestimmten Artikels} syntaktisch bedingte \isi{Variation} auftritt. Die Dialekte von Petrifeld und Colmar weisen nach konsonantisch und vokalisch auslautender Präposition denselben Artikel auf. In allen anderen Dialekten wird der \isi{unbestimmte Artikel} nach vokalisch auslautender Präposition (PP V) entweder von jener Form abgeleitet, die einer konsonantisch auslautenden Präposition folgt (PP K), oder von jener Form, der keine Präposition vorausgeht (NP).

% {\tabref{table5.30}: Formen des \isi{unbestimmten Artikels} bei syntaktisch bedingter \isi{Variation} im Dativ}\\

\begin{table}
\caption{Formen des unbestimmten Artikels bei syntaktisch bedingter Variation im Dativ}\label{table5.30}
\begin{tabularx}{\textwidth}{l*{5}{X}} 
\lsptoprule
& \multicolumn{3}{c}{\textsc{dat.m+n}} & \multicolumn{2}{c}{\textsc{dat.f}}\\\cmidrule(lr){2-4}\cmidrule(lr){5-6}
& \textsc{np} & \textsc{pp (k)} & \textsc{pp (v)} & \textsc{np} & \textsc{pp (k)}\\
\midrule
Elsass (Ebene) & imə & əmə & mə & inərə & ərə\\
% % \midrule
Münstertal & æmə & əmə & mə & ænərə & ənərə\\
% % \midrule
Colmar & eme & me & me & enre & re\\
% % \midrule
Petrifeld & imə & əmə & əmə & inrə & rə\\
% % \midrule
Zürich & əmənə & əmə & mənə, mə & ənərə & ərə\\
% % \midrule
Bern & əmənə, əmnə, əmə & əmənə, əmnə, əmə & mənə & ənərə, ərə & ənərə, ərə\\
% % \midrule
Uri & ɑmənɐ, ɑmɐ & əmɐ & mənɐ, mɐ & ɑnərɐ & ərɐ\\
% % \midrule
Sensebezirk & (a,i)məna, (a,i)ma & əməna & məna, ma & (a,i)nəra, əra & ənəra\\
% % \midrule
Jaun & əmənə & əmənə & mənə & andərə & ərə\\
\lspbottomrule
\end{tabularx}
\end{table}

Der \isi{unbestimmte Artikel} im Dativ zeigt also syntaktisch bedingte \isi{Variation} in den höchstalemannischen Dialekten (außer den Walser Dialekte), in den elsässischen Dialekten sowie im Dialekt von Bern, Zürich und Petrifeld (vgl. \tabref{table5.30}). Der \isi{bestimmte Artikel} weist syntaktisch bedingte \isi{Variation} in den höchstalemannischen Dialekten (außer Issime), in den elsässischen Dialekten sowie im Dialekt von Saulgau (vgl. \tabref{table5.27}). Es kann also festgehalten werden, dass syntaktisch bedingte \isi{Variation} im Dativ vor allem ein Phänomen der höchstalemannischen Dialekte (außer den Walser Dialekte) und der elsässischen Dialekte ist (vgl. \tabref{table5.31}).

Schließlich zählen etliche Grammatiken unterschiedliche Varianten für den Dativ auf, ohne jedoch auf ihre Distribution einzugehen. Wie beim \isi{bestimmten Artikel} wird in diesen Fällen auch beim \isi{unbestimmten Artikel} davon ausgegangen, dass es sich um freie \isi{Variation} handelt. Zum Beispiel hat der Dialekt von Bern drei Varianten: \textit{əmənə}, \textit{əmnə}, \textit{əmə} (\citealt[79]{Marti1985}; vgl. auch \tabref{table5.30}). Dies betrifft folgende Dialekte: Visperterminen (nur Maskulin/Neutrum), Sensebezirk, Bern, Vorarlberg, Saulgau.

{Akkusativ}: Wie der \isi{bestimmte Artikel} variiert auch der \isi{unbestimmte Artikel} im Akkusativ Singular, jedoch in allen \isi{Genera}, während im \isi{bestimmten Artikel} nur das Maskulin davon betroffen ist. Im Akkusativ Singular fallen Maskulin und Feminin zusammen, das Neutrum hat eine eigene Form, z.B. \textit{a}/\textit{əna} (Maskulin/Feminin), \textit{as}/\textit{ənas} (Neutrum) (Sensebezirk, \citealt[194]{Henzen1927}). Die kürzeren Formen \textit{a}/\textit{as} werden verwendet, wenn dem \isi{unbestimmten Artikel} keine Präposition vorangeht, die längeren Formen \textit{əna}/\textit{ənas}, wenn eine Präposition vorangeht. Bei den längeren Formen handelt es sich um präfigierte Formen, auf deren Bildung im nachfolgenden Abschnitt \sectref{5.6.5} noch eingegangen wird. Es kann aber schon festgehalten werden, dass für die kürzeren und längeren Formen jeweils RRs benötigt werden, um ihre Form zu definieren. Lautet die Präposition konsonantisch aus, so folgen die genannten Formen \textit{əna}/\textit{ənas}. Lautet die Präposition vokalisch aus, wird der anlautende Vokal des \isi{unbestimmten Artikels} getilgt (\textit{na}/\textit{nas}), um einen \isi{Hiat} zu vermeiden. Im Gegensatz zum Dativ ist im Akkusativ der \isi{unbestimmte Artikel} nach vokalisch auslautender Präposition immer von jener Form abzuleiten, welche einer konsonantisch auslautenden Präposition folgt. Wie beim Dativ des \isi{unbestimmten Artikels} ist für die Tilgung der ersten Silbe auch im Akkusativ eine RR im letzten \isi{Block} anzunehmen, da nicht die Defaultstrategie zur Hiatvermeidung verwendet wird (n-Epenthese), sondern die erste Silbe getilgt wird. Genauer gesagt, reicht eine RR für den Akkusativ und den Dativ, da eine RR für beide \isi{Kasus} definiert werden kann, indem beide \isi{Kasus} bei den morphosyntaktischen Eigenschaften in der RR angegeben werden (vgl. Liste an RRs oben bezüglich des Dativs).

Wie beim \isi{unbestimmten Artikel} im Dativ ist die \isi{Variation} im Akkusativ ebenfalls in den Dialekten Jaun, Sensebezirk und Uri vorhanden (höchstalemannisch, außer den Walser Dialekte). Dazu kommen die Dialekte von Bern und Zürich. \tabref{table5.31} fasst die \isi{Variation} im bestimmten und \isi{unbestimmten Artikel} zusammen. Die \isi{Variation} ±Präposition+Artikel ist sowohl im Akkusativ als auch im Dativ im bestimmten und \isi{unbestimmten Artikel} zu beobachten, wobei im unbestimmten alle \isi{Genera} betroffen sind, während dies im \isi{bestimmten Artikel} nur für das Maskulin und Neutrum gilt. Ausschließlich der \isi{bestimmte Artikel} variiert in Abhängigkeit eines nachfolgenden \isi{Adjektivs}, und zwar nur im Nominativ/Akkusativ Singular Feminin und im Nominativ/Akkusativ Plural. Bezüglich der Dialekte kann ein höchstalemannisches Zentrum an Artikelvariation beobachtet werden: Die höchstalemannischen Dialekte mit Ausnahme der Walser Dialekte weisen alle fünf Variationen auf. Als mögliche Hypothese könnte das Fehlen dieser \isi{Variation} in den Walser Dialekten damit erklärt werden, dass diese Dialekte als isoliert und archaisch gelten können und somit resistenter gegenüber Neuerungen sind. Die hochalemannischen Dialekte zeigen \isi{Variation} vor allem im Akkusativ (weniger im Dativ) sowie im Kontext ±\isi{Adjektiv}. Demgegenüber variieren die elsässischen Dialekte vorwiegend im Dativ. Schließlich weist der Dialekt von Saulgau \isi{Variation} nur im \isi{bestimmten Artikel} im Kontext ±Präposition auf (Akkusativ und Dativ), jedoch nicht im \isi{unbestimmten Artikel}. Das Gegenteil gilt für den Dialekt von Zürich: Nur der \isi{unbestimmte Artikel} variiert, aber ebenfalls im Akkusativ und Dativ. Dies sind doch erstaunliche areale Muster, welchen in einer weiteren Arbeit nachgegangen werden sollte.

% {\tabref{table5.31}: Syntaktisch bedingte \isi{Variation} im bestimmten und unbestimmten Artikel}\\

\begin{table}
\caption{Syntaktisch bedingte Variation im bestimmten und unbestimmten Artikel}\label{table5.31}
\begin{tabular}{>{\raggedright}p{2.75cm}p{4cm}p{4cm}}
\lsptoprule
± Präposition\newline + unbest. Artikel & \textsc{akk.sg.m/n/f} & \textsc{dat.sg.m/n/f}\\
\midrule
{Dialekte:} & • Höchstalemannisch\newline\hspaceThis{•} (außer Walser)\newline • Bern\newline • Zürich & 
• Höchstalemannisch\newline\hspaceThis{•} (außer Walser)\newline • Elsässische Dialekte\newline • Bern\newline • Zürich\newline • Petrifeld\\
\midrule
{± Präposition\newline + best. Artikel:} & \textsc{akk.sg.m} & \textsc{dat.sg.m/n}\\
\midrule
{Dialekte:} & • Höchstalemannisch\newline\hspaceThis{•} (außer Walser)\newline • Hochalemannisch\newline\hspaceThis{•} (außer Zürich)\newline • Saulgau\newline • Elsass (Ebene)& • Höchstalemannisch\newline\hspaceThis{•} (außer Issime)\newline • Elsässische Dialekte\newline • Saulgau\newline • Standard\\
\midrule
{best. Artikel\newline ±\isi{Adjektiv}:} & \multicolumn{2}{l}{\textsc{nom/akk.sg.f + nom/akk.pl}}\\
\midrule
{Dialekte:} & \multicolumn{2}{l}{• Höchstalemannisch (außer Walser)} \\& \multicolumn{2}{l}{ • Hochalemannisch (außer Zürich)}\\
\lspbottomrule
\end{tabular}
\end{table}
%AUFZÄLUNG SIEHT SELTSAM AUS

\subsection{Unbestimmter Artikel: Präfixe}\label{5.6.5}

Im vorangehenden Kapitel wurde gezeigt, dass die präfigierten Formen des \isi{unbestimmten Artikels} im Akkusativ in einem bestimmten syntaktischen Kontext verwendet werden, und zwar nach einer Präposition. Außerdem kommen diese Formen nur in den folgenden Dialekten vor: Jaun, Sensebezirk, Uri, Zürich, Bern. Hier geht es darum, die Bildung ihrer Form zu erfassen.

Dass es sich um präfigierte Formen handelt, demonstriert ein Vergleich des Akkusativs mit dem Nominativ. \tabref{table5.32} zeigt die Nominativ- und Akkusativformen jener Dialekte, die einen präfigierten Akkusativ aufweisen. Die Nominativform für das Maskulin/Feminin und das Neutrum ist in der Akkusativform beinhaltet. Alle drei \isi{Genera} und alle fünf Dialekte zeigen im Akkusativ dasselbe \isi{Affix}, das an die Nominativform präfigiert wird: -\textit{ən}/-\textit{n}.

% {\tabref{table5.32}: Präfigierte unbestimmte Artikel}\\

\begin{table}
\caption{Präfigierte unbestimmte Artikel}\label{table5.32}
\resizebox{\textwidth}{!}{\begin{tabular}{*{7}{l}}
\lsptoprule
{Dialekt} & \multicolumn{2}{c}{\textsc{nom.sg}} & \multicolumn{4}{c}{\textsc{akk.sg}}\\\cmidrule(lr){2-3}\cmidrule(lr){4-7}
& \textsc{m/f} & \textsc{n} & \multicolumn{2}{c}{\scshape m/f} & \multicolumn{2}{c}{\scshape n}\\
&  &  & \mbox{-präfigiert} & \mbox{+präfigiert} & \mbox{-präfigiert} & \mbox{+präfigiert}\\
\midrule
Jaun & a & as & a & ən-a & as & ən-as\\
Sensebezirk & a & as & a & ən-a & as & ən-as\\
Uri & ɐ & ɐs & ɐ & ən{}-ɐ & ɐs & ən-ɐs\\
Zürich & ən & əs & ən & ən-ən & əs & ən-əs\\
Bern & ə & əs & ə & n-ə & əs & n-əs\\
\lspbottomrule
\end{tabular}}
\end{table}

In diesen Dialekten ist also im Akkusativ Singular aller \isi{Genera} eine RR für das \isi{Präfix} anzunehmen. Wie in \sectref{5.6.1} besprochen wurde, muss die gesamte Form des \isi{unbestimmten Artikels} durch RR definiert werden, was in \isi{Block} A passiert (RRs \ref{ex:key:148} und \ref{ex:key:149}). Die \isi{Präfixe} stehen dann in einem weiteren \isi{Block} B. Folgende RRs exemplifizieren dies am Dialekt des Sensebezirks, wobei \isi{Genus} in RR \REF{ex:key:150} unterspezifiziert bleibt, da in allen \isi{Genera} präfigiert wird:

\ea%148
\label{ex:key:148}
 RR \textsubscript{A,} \textsubscript{\{\textsc{case:nom}} \textsubscript{\tiny $\veebar$}\textsubscript{ \textsc{acc}}\textsubscript{, \textsc{num:sg}, \textsc{gend:m}} \textsubscript{\tiny $\veebar$}\textsubscript{ \textsc{f}\},} \textsubscript{\textsc{det2[art.indef]}} ($\langle$X,$\sigma$ $\rangle$) = \textsubscript{def} $\langle$X\textit{a}ˊ,$\sigma$ $\rangle$
\z

\ea%149
\label{ex:key:149}
 RR \textsubscript{A,} \textsubscript{\{\textsc{case:nom}} \textsubscript{\tiny $\veebar$}\textsubscript{ \textsc{acc}}\textsubscript{, \textsc{num:sg}, \textsc{gend:n}\},} \textsubscript{\textsc{det2[art.indef]}} ($\langle$X,$\sigma$ $\rangle$) = \textsubscript{def} $\langle$X\textit{as}ˊ,$\sigma$ $\rangle$
\z

\ea%150
\label{ex:key:150}
 RR \textsubscript{B,} \textsubscript{\{\textsc{case:}}\textsubscript{\textsc{acc}}\textsubscript{, \textsc{num:sg}\},} \textsubscript{\textsc{det2[art.indef]}} ($\langle$X,$\sigma$ $\rangle$) = \textsubscript{def} $\langle$\textit{ən}Xˊ,$\sigma$ $\rangle$
\z

\subsection{Possessivpronomen: Unterschiedliche Paradigmen}\label{5.6.6}

Wie schon in \sectref{5.6.1} erwähnt wurde, weisen die \isi{Possessivpronomen} unterschiedliche Suffixparadigmen auf, je nachdem um welches \isi{Possessivpronomen} es sich handelt. In diesen Fällen ist in den RRs also auch das \isi{Possessivpronomen} zu definieren, wie dies anhand der RRs \REF{ex:key:142} und \REF{ex:key:143} illustriert wurde. Dies gilt für fast alle alemannischen Dialekte dieses Samples, jedoch nicht für das Alt- und Mittelhochdeutsche sowie die deutsche Standardsprache. Diese drei Varietäten verfügen über nur ein Paradigma an \isi{Suffixen} für alle \isi{Possessivpronomen}. Im Alt- und Mittelhochdeutschen jedoch kommen noch zwei unveränderliche \isi{Possessivpronomen} hinzu, nämlich die 3. Person Singular Feminin und die 3. Person Plural, welche keine \isi{Suffixe} aufweisen. Darauf wird in \sectref{5.6.7} noch genauer eingegangen. Ebenfalls nur ein Paradigma haben die elsässischen Dialekte wie auch die Dialekte von Huzenbach und Visperterminen, wobei Visperterminen zusätzlich dieselben unveränderlichen \isi{Possessivpronomen} wie das Alt- und Mittelhochdeutsche aufweist. Auch die Dialekte von Issime, Jaun und des Sensebezirks haben diese unveränderlichen \isi{Possessivpronomen}, die in der anschließenden Diskussion dieses Kapitels nicht berücksichtigt werden.

Es verfügen also alle Dialekte (außer Visperterminen, Huzenbach, elsässische Dialekte) über unterschiedliche Paradigmen in Abhängigkeit des \isi{Possessivpronomens} (vgl. \tabref{table5.33}). Zwei Paradigmen haben Vorarlberg, Zürich, Stuttgart, Elisabethtal, drei Paradigmen Jaun, Sensebezirk, Uri, Saulgau, Kaiserstuhl, Petrifeld, vier Paradigmen Issime und Bern. Ein areales Muster ist also nicht zu erkennen.

Dafür gibt es ein Muster bezüglich der Art des \isi{Possessivpronomens}. In all diesen Dialekten bilden die \isi{Possessivpronomen} der 1. und 2. Person Singular sowie der 3. Person Singular Maskulin und Neutrum zusammen ein Paradigma, das sich von jenem der 1. und 2. Person Plural unterscheidet, welche ihrerseits ebenfalls zusammen ein Paradigma formen. Die einzige Ausnahme davon ist der Dialekt von Bern.

Für die \isi{Possessivpronomen} der 3. Person Singular Feminin und der 3. Person Plural sind (mit ein paar Ausnahmen) drei Typen festzustellen: Entweder sind sie unveränderlich, bilden zusammen ein drittes Paradigma oder gehören zum Paradigma der 1./2. Person Plural. Dies soll nun genauer beschrieben werden.

Mehrheitlich formen die Pronomen der 3. Person Singular Feminin und der 3. Person Plural zusammen ein drittes Paradigma (Uri, Saulgau und Kaiserstuhl) oder sie funktionieren gleich wie die Pronomen der 1. und 2. Person Plural (Zürich, Stuttgart und Elisabethtal). Jaun und Sensebezirk haben unveränderliche \isi{Possessivpronomen}. In Issime ist die 3. Person Singular Feminin unveränderlich, die 3. Person Plural bildet ein eigenes Paradigma. Dies gilt auch für Petrifeld, wobei hier die 3. Person Singular Feminin mit der 1. und 2. Person Plural geht. Im Dialekt von Vorarlberg wird die 3. Person Singular Feminin und die 3. Person Plural analytisch gebildet, nämlich mit dem \isi{Demonstrativpronomen} + der 3. Person Singular Maskulin/Neutrum \citep[275]{Jutz1925}. Da diese Formen bereits durch RRs zur Verfügung stehen, müssen sie nicht nochmal definiert werden. Bern entspricht keiner der benannten Regelmäßigkeiten.

% {\tabref{table5.33}: Paradigmen des \isi{Possessivpronomens} in den Dialekten}\\

\begin{table}
\caption{Paradigmen des Possessivpronomens in den Dialekten}\label{table5.33}
\begin{tabularx}{\textwidth}{>{\scshape}X>{\scshape}X>{\scshape}X>{\scshape}XX}
\lsptoprule
1.2.sg, 3.sg.m/n & 1.2.pl & \mbox{3.sg.f, 3.pl} ({\upshape unveränderlich}) &  & Jaun, Sensebezirk\\
\midrule
1.2.sg, 3.sg.m/n & 1.2.pl & 3.sg.f ({\upshape \mbox{unveränder}\-lich}) & 3.pl   & Issime\\
\midrule                                                           
1.2.sg, 3.sg.m/n & 1.2.pl & 3.sg.f, 3.pl &                   & Uri, Saulgau, Kaiserstuhl\\
\midrule                                                           
1.2.sg, 3.sg.m/n & 1.-3.pl, 3.sg.f &  &                       & Zürich, Stuttgart, Elisabethtal\\
\midrule                                                           
1.2.sg, 3.sg.m/n & 1.2.pl,3.sg.f & 3.pl &                    & Petrifeld\\
\midrule                                                           
1.2.sg, 3.sg.m/n & 1.2.pl &  &                                 & Vorarlberg\\
\midrule                                                           
1.sg & 2.sg, 3.sg.m/n & 1.pl & 2.3.pl, 3.sg.f               & Bern\\
\lspbottomrule
\end{tabularx}
\end{table}

\subsection{Possessivpronomen: Unveränderliche Formen}\label{5.6.7}

Alt- und Mittelhochdeutsch sowie alle höchstalemannischen Dialekte außer Uri haben in der 3. Person Singular Feminin und in der 3. Person Plural unveränderliche \isi{Possessivpronomen}, z.B. \textit{ira} (3. Person Singular Feminin), \textit{iro} (3. Person Plural) (Althochdeutsch, \citealt[245]{Braune2004}). Diese \isi{Possessivpronomen} flektieren also nicht nach \isi{Kasus}, \isi{Numerus} und \isi{Genus}. Der Dialekt von Issime weist ein solches \isi{Possessivpronomen} nur in der 3. Person Singular Feminin auf, das \isi{Possessivpronomen} der 3. Person Plural flektiert (vgl. \tabref{table5.33} und Paradigma 104). In all diesen Varietäten wurde das unveränderliche \isi{Possessivpronomen} aus dem Genitiv des \isi{Personalpronomens} der 3. Person Singular Feminin bzw. der 3. Person Plural übernommen.

Für diese unveränderlichen \isi{Possessivpronomen} sind keine RRs nötig. Ihre\linebreak Form stammt aus dem \isi{Radikon} wie die \isi{Wurzeln} aller anderen \isi{Possessivpronomen}. Beispielsweise muss für das \isi{Possessivpronomen} \textit{mein}-\textit{ən} (Akkusativ Singular Maskulin, deutsche Standardsprache) nur das \isi{Suffix} durch RRs definiert werden, die \isi{Wurzel} \textit{mein} kommt aus dem \isi{Radikon}. Wird in einer Zelle nicht suffigiert, z.B. \textit{mein} (Nominativ Singular Maskulin, deutsche Standardsprache), wird keine RR benötigt, da die \isi{Wurzel} per Default eingefügt wird (vgl. \sectref{4.1.3.2}, RR \textit{Identity Function Default}). Ebenfalls per Default füllen die unveränderlichen \isi{Possessivpronomen} ihre Paradigmen.

Eine Ausnahme bildet hier nur der Dialekt des Sensebezirks. In diesem Dialekt hat das unveränderliche \isi{Possessivpronomen} folgende Formen: \textit{\=ira}/\textit{\=iras} (3. Person Singular Feminin), \textit{\=irə}/\textit{\=irəs} (3. Person Plural). Auch diese flektieren nicht nach \isi{Numerus}, \isi{Kasus} und \isi{Genus}, weisen jedoch zwei Varianten auf, nämlich eine mit -\textit{s} suffigierte und eine ohne \isi{Suffix}. Es liegt also freie \isi{Variation} des Typs \isi{Wurzel}/\isi{Wurzel}+\isi{Suffix} vor, für den zwei gleich spezifische RRs nötig sind, damit beide RRs dieselbe Zelle definieren (vgl. Diskussion zur freien \isi{Variation} in den \isi{Substantiven} \sectref{5.1.2}, \isi{Adjektiven} \sectref{5.2.3} und \isi{Possessivpronomen} \sectref{5.6.3} sowie die theoretische Begründung \sectref{4.1.3.3}). Für die 3. Person Singular Feminin und die 3. Person Plural im Dialekt des Sensebezirks müssen also folgende RRs angesetzt werden:\largerpage

\ea%151
\label{ex:key:151}
 RR \textsubscript{A,} \textsubscript{\{\},} \textsubscript{\textsc{det2[pron.poss}, \textsc{pers:3}, \textsc{num:sg}, \textsc{gend:f}]} ($\langle$X,$\sigma$$\rangle$) = \textsubscript{def} $\langle$Xˊ,$\sigma$$\rangle$
\z

\ea%152
\label{ex:key:152}
 RR \textsubscript{A,} \textsubscript{\{\},} \textsubscript{\textsc{det2[pron.poss}, \textsc{pers:3}, \textsc{num:sg}, \textsc{gend:f}]} ($\langle$X,$\sigma$$\rangle$) = \textsubscript{def} $\langle$X\textit{s}ˊ,$\sigma$$\rangle$
\z

\ea%153
\label{ex:key:153}
 RR \textsubscript{A,} \textsubscript{\{\},} \textsubscript{\textsc{det2[pron.poss}, \textsc{pers:3}, \textsc{num:pl}]} ($\langle$X,$\sigma$$\rangle$) = \textsubscript{def} $\langle$Xˊ,$\sigma$$\rangle$
\z

\ea%154
\label{ex:key:154}
 RR \textsubscript{A,} \textsubscript{\{\},} \textsubscript{\textsc{det2[pron.poss}, \textsc{pers:3}, \textsc{num:pl}]} ($\langle$X,$\sigma$$\rangle$) = \textsubscript{def} $\langle$X\textit{s}ˊ,$\sigma$$\rangle$
\z

\subsection{Possessivpronomen (und unbestimmter Artikel): Wur\-zel-/Stamm\-al\-ter\-na\-tio\-nen}\label{5.6.8}

Hier werden zwei Wur\-zel-/Stamm\-al\-ter\-na\-tio\-nen im \isi{Possessivpronomen} vorgestellt. Für den ersten Typ sind keine RRs anzunehmen, die Wur\-zel-/Stamm\-al\-ter\-na\-tion wird von der Phonologie gesteuert (n-Einschub zur Hiatvermeidung). Der zweite Typ zeichnet sich dadurch aus, dass die Alternationen durch RRs definiert werden müssen.\largerpage

{N-Einschub zur Hiatvermeidung}: In vielen alemannischen Dialekten hat das \isi{Possessivpronomen} der 1. und 2. Person Singular sowie der 3. Person Singular Maskulin/Neutrum das auslautende \textit{n} verloren. Es taucht jedoch wieder auf, wenn der \isi{Wurzel} ein vokalisch anlautendes \isi{Suffix} folgt, z.B. \textit{mi} (Nominativ/Akkusativ Singular Maskulin), \textit{min}-\textit{a} (Nominativ/Akkusativ Singular Maskulin), \textit{mi}-\textit{s} (Genitiv Singular Maskulin/Neutrum) (Jaun, \citealt[284]{Stucki1917}). Aus synchroner Sicht kann also festgehalten werden, dass ein \textit{n} eingefügt wird, wenn die \isi{Wurzel} vokalisch auslautet und das \isi{Suffix} vokalisch anlautet. Wie bereits in \sectref{5.1.4} gezeigt wurde, wird in den alemannischen Dialekten ein \textit{n} zur Hiatvermeidung per Default eingefügt. Da also die Phonologie den \isi{Hiat} automatisch beseitigt, müssen dafür keine RRs angenommen werden. Dies trifft auf folgende Dialekte zu: Jaun, Sensebezirk, Uri, Vorarlberg, Bern, Saulgau, Stuttgart, Petrifeld, Elisabethtal, Kaiserstuhl und Colmar.

Dasselbe ist im \isi{unbestimmten Artikel} des Dialekts von Visperterminen zu beobachten. Folgt dem \isi{unbestimmten Artikel} ein vokalisch anlautendes \isi{Suffix}, wird ein \textit{n} eingeschoben: \textit{a} (Nominativ Singular Maskulin), \textit{a}-\textit{s} (Genitiv Singular Maskulin), \textit{an}-\textit{um} (Dativ Singular Maskulin) \citep[137]{Wipf1911}.

{Nicht phonologisch bedingte Wur\-zel-/Stamm\-al\-ter\-na\-tio\-nen}: Bei den in der Folge diskutierten Fällen können die Wur\-zel-/Stamm\-al\-ter\-na\-tio\-nen nicht durch allgemeine phonologische Regeln erklärt werden, die für das gesamte System gelten. Deswegen sind dafür RRs anzusetzen. Bei diesen  Wur\-zel-/Stamm\-al\-ter\-na\-tio\-nen handelt es sich sowohl um Einfügungen als auch um Tilgungen. Des Weiteren sind sie entweder vom phonologischen Kontext oder vom morphosyntaktischen Kontext abhängig. Eine Übersicht gibt \tabref{table5.34}. Solche  Wur\-zel-/Stamm\-al\-ter\-na\-tio\-nen kommen in folgenden Dialekten vor, die anschließend in dieser Reihenfolge erörtert werden: Saulgau, Issime, Zürich, Huzenbach, Vorarlberg und Petrifeld.

%{\tabref{table5.34}: \isi{Wurzel-/Stammalternationen} im \isi{Possessivpronomen} der Dialekte}\\
              
\begin{table}
\caption{Wurzel-/Stammalternationen im Possessivpronomen der Dialekte}\label{table5.34}
\begin{tabularx}{\textwidth}{lXl}
\lsptoprule
\multicolumn{3}{c}{{Abhängig vom phonologischen Kontext}}\\\cmidrule(lr){1-3}
Prozess & Dialekt & Beispiel\\
\midrule
Einfügen von K (\textit{r}) & Saulgau & \textit{əisə}, \textit{əisər}-\textit{e}\\
Tilgen von V (\textit{ə}, \textit{u}) & Saulgau,\newline Issime & \textit{iərə}, \textit{iər}-\textit{e} \textit{üriu}, \textit{üri}-\textit{er}\\
Tilgen von K (\textit{n}) & Issime, Zürich & \textit{mein}, \textit{mei}-\textit{s} \textit{m\=in}, \textit{m\=i}-\textit{s}\\
\raggedright Verlust Nasalierung & Huzenbach & \textit{mãẽ}, \textit{mae}-\textit{ərə}\\

\midrule
 \multicolumn{3}{c}{{Abhängig vom morphosyntaktischen Kontext}}\\\cmidrule(lr){1-3}
Prozess & Dialekt & Beispiel\\
\midrule
Einfügen von K (\textit{n}) & Petrifeld & \textit{m\=ai}, \textit{m\=ain}-\textit{rə}\\
Tilgen von K (\textit{r}) & Petrifeld, Vorarlberg & \textit{\=ais}\textit{r}, \textit{\=ais}-\textit{ǝm} \textit{üsər}, \textit{üsə}\\
\raggedright Tilgen eines Segments (\textit{nə}) & Petrifeld & \textit{\=iərnə}, \textit{\=iər}-\textit{əm}\\
\lspbottomrule
\end{tabularx}
\end{table}

Im Dialekt von \textbf{Saulgau} wird im Pronomen der 1. und 2 Person Plural ein \textit{r} eingefügt, wenn ein vokalisch anlautendes \isi{Suffix} folgt. Die \isi{Wurzel} der 1. Person Plural ist \textit{əisə}, lautet das \isi{Suffix} vokalisch an, wird die \isi{Wurzel} durch ein \textit{r} erweitert, z.B. \textit{əisər}-\textit{e} (Plural) \citep[118]{Raichle1932}. Im \isi{Possessivpronomen} der 3. Person Singular Feminin und der 3. Person Plural hingegen wird der auslautende Vokal getilgt, wenn ein vokalisch anlautendes \isi{Suffix} folgt: z.B. \textit{iərə} (=\isi{Wurzel}), \textit{iər}-\textit{e} (Plural) \citep[119]{Raichle1932}. Für beide Fälle sind RRs nötig:

\ea%155
\label{ex:key:155}
 RR \textsubscript{B,} \textsubscript{\{\}}\textsubscript{,} \textsubscript{\textsc{det2[pron.poss}, \textsc{pers:1}} \textsubscript{\tiny $\veebar$}\textsubscript{\textsc{2}}\textsubscript{, \textsc{num:pl}]} ($\langle$X,$\sigma$ $\rangle$) = \textsubscript{def} $\langle$X\textit{r}/\_Vˊ,$\sigma$ $\rangle$
\z

\ea%156
\label{ex:key:156}
 RR \textsubscript{B,} \textsubscript{\{\}}\textsubscript, \textsubscript{\textsc{det2[pron.poss}, \textsc{pers:3}, \textsc{num:pl}]} ($\langle$X,$\sigma$ $\rangle$) = \textsubscript{def} $\langle$X *\textit{ə} $\rightarrow$ ø/\_Vˊ,$\sigma$ $\rangle$
\z

\ea%157
\label{ex:key:157}
 RR \textsubscript{B,} \textsubscript{\{\}}\textsubscript, \textsubscript{\textsc{det2[pron.poss}, \textsc{pers:3}, \textsc{num:sg}, \textsc{gend:f}]} ($\langle$X,$\sigma$ $\rangle$) = \textsubscript{def} $\langle$X *\textit{ə} $\rightarrow$ ø/\_Vˊ,$\sigma$ $\rangle$
\z

Das \isi{Suffix} löst also die Wurzelalternation aus. Deswegen sind die \isi{Suffixe} in \isi{Block} A, die RR für die Wurzelalternationen in \isi{Block} B. Dies gilt auch für alle folgenden Fälle, in denen der phonologische Kontext der Auslöser der Wurzelalternation ist.

In \textbf{Issime} lautet das \isi{Possessivpronomen} der 3. Person Plural \textit{üriu}. Folgt ein vokalisch anlautendes \isi{Suffix}, wird das auslautende \textit{u} getilgt, z.B. \textit{üri}-\textit{er} \citep[84]{Perinetto1981}:

\ea%158
\label{ex:key:158}
 RR \textsubscript{B,} \textsubscript{\{\}}\textsubscript{,} \textsubscript{\textsc{det2[pron.poss}, \textsc{pers:3}, \textsc{num:pl}]} ($\langle$X,$\sigma$ $\rangle$) = \textsubscript{def} $\langle$X *\textit{u} $\rightarrow$ ø/\_Vˊ,$\sigma$ $\rangle$ \\
\z

Ebenfalls im Dialekt von Issime enden die \isi{Possessivpronomen} der 1. und 2. Person Singular sowie der 3. Person Singular Maskulin und Neutrum auf ein \textit{n}. Folgt ihnen ein konsonantisch anlautendes \isi{Suffix}, wird das \textit{n} der \isi{Wurzel} getilgt, z.B. \textit{mein}, \textit{mei}-\textit{s} \citep[83]{Perinetto1981}:

\ea%159
\label{ex:key:159}
 RR \textsubscript{B,} \textsubscript{\{\}}\textsubscript{,} \textsubscript{\textsc{det2[pron.poss}, \textsc{pers:3}, \textsc{num:sg}, \textsc{gend: m}} \textsubscript{\tiny $\veebar$}\textsubscript{ \textsc{n}}\textsubscript{]} ($\langle$X,$\sigma$ $\rangle$) = \textsubscript{def} $\langle$X *\textit{n} $\rightarrow$ ø/\_Kˊ,\mbox{$\sigma$ $\rangle$}
\z

\ea%160
\label{ex:key:160}
 RR \textsubscript{B,} \textsubscript{\{\}}\textsubscript{,} \textsubscript{\textsc{det2[pron.poss}, \textsc{pers:1}} \textsubscript{\tiny $\veebar$}\textsubscript{\textsc{2}, \textsc{num:sg}]} ($\langle$X,$\sigma$ $\rangle$) = \textsubscript{def} $\langle$X *\textit{n} $\rightarrow$ ø/\_Kˊ,$\sigma$ $\rangle$
\z

Dieselbe \isi{Variation} kommt im Dialekt von \textbf{Zürich} vor. Auch sind die gleichen \isi{Possessivpronomen} betroffen, weshalb die RRs hier nicht gelistet werden.

Im Dialekt von \textbf{Huzenbach} ist im \isi{Possessivpronomen} der 1. und 2. Person Singular sowie der 3. Person Singular Maskulin und Neutrum der Diphthong der \isi{Wurzel} nasaliert. Diese Nasalierung geht verloren, wenn das \isi{Suffix} vokalisch anlautet, z.B. \textit{mãẽ}, \textit{mae}-\textit{ərə} \citep[104]{Baur1967}. Die RRs definieren, dass die nasalierten Vokale ihre Nasalierung verlieren, wenn ein Vokal folgt:

\ea%161
\label{ex:key:161}
 RR \textsubscript{B,} \textsubscript{\{\}}\textsubscript{,} \textsubscript{\textsc{det2[pron.poss}, \textsc{pers:1}} \textsubscript{\tiny $\veebar$}\textsubscript{\textsc{2}, \textsc{num:sg}]} ($\langle$X,$\sigma$ $\rangle$) = \textsubscript{def} $\langle$X [+V, +nasal] $\rightarrow$ \mbox{[+V, -nasal]/\_Vˊ,$\sigma$ $\rangle$}
\z

\ea%162
\label{ex:key:162}
 RR \textsubscript{B,} \textsubscript{\{\}}\textsubscript{,} \textsubscript{\textsc{det2[pron.poss}, \textsc{pers:3}, \textsc{num:sg}, \textsc{gend: m}} \textsubscript{\tiny $\veebar$}\textsubscript{ \textsc{n}}\textsubscript{]} ($\langle$X,$\sigma$ $\rangle$) = \textsubscript{def} $\langle$X [+V, +nasal] $\rightarrow$ [+V, -nasal]/\_Vˊ,$\sigma$ $\rangle$
\z

Nun werden die  Wur\-zel-/Stamm\-al\-ter\-na\-tio\-nen besprochen, die von morphosyntaktischen Eigenschaften abhängig sind. Im Dialekt von Vorarlberg weist das \isi{Possessivpronomen} der 1. und 2. Person Plural im Nominativ/Akkusativ Singular Neutrum zwei Formen auf, nämlich \textit{üsə} und \textit{üsər} \citep[276]{Jutz1925}. Vergleicht man diese beiden Formen mit den Formen der übrigen Zellen, ist festzustellen, dass in allen anderen Zellen die Form \textit{üsər} als \isi{Wurzel} angenommen werden kann. Die Tilgung des \textit{r} ist also durch RRs auszudrücken. Gleichzeitig muss aber für die Zelle Nominativ/Akkusativ Singular Neutrum die Form \textit{üsər} durch eine gleich spezifische RRs definiert werden, weil nur so beide Formen in derselben Zelle stehen. Deswegen stehen auch beide Zellen in \isi{Block} A:

\ea%163
\label{ex:key:163}
 RR \textsubscript{A,} \textsubscript{\{\textsc{case:nom}} \textsubscript{\tiny $\veebar$}\textsubscript{ \textsc{acc}, \textsc{num:sg}, \textsc{gend:n}\},} \textsubscript{\textsc{det2[pron.poss}, \textsc{pers:1}} \textsubscript{\tiny $\veebar$}\textsubscript{ 2, \textsc{num:pl}]} ($\langle$X,$\sigma$ $\rangle$) = \textsubscript{def} $\langle$Xˊ,$\sigma$ $\rangle$
\z

\ea%164
\label{ex:key:164}
 RR \textsubscript{A,} \textsubscript{\{\textsc{case:nom}} \textsubscript{\tiny $\veebar$}\textsubscript{ \textsc{acc}, \textsc{num:sg}, \textsc{gend:n}\},} \textsubscript{\textsc{det2[pron.poss}, \textsc{pers:1}} \textsubscript{\tiny $\veebar$}\textsubscript{ 2, \textsc{num:pl}]} ($\langle$X,$\sigma$ $\rangle$) = \textsubscript{def} $\langle$X *\textit{r} $\rightarrow$ øˊ,$\sigma$ $\rangle$
\z

Auch im Dialekt von \textbf{Petrifeld} wird das auslautende \textit{r} der \isi{Wurzel} in einer bestimmten morphosyntaktischen Umgebung getilgt. Dies betrifft die \isi{Possessivpronomen} der 1. und 2. Person Plural wie auch der 3. Person Singular Feminin. Das \textit{r} wird im Dativ Singular Maskulin/Neutrum getilgt: \textit{\=aisr} (=\isi{Wurzel}), \textit{\=ais}-\textit{ǝm} (Dativ Singular Maskulin/Neutrum) \citep[65]{Moser1937}. Man könnte annehmen, dass das \textit{r} wegfällt, wenn ein Vokal folgt. Dies trifft aber nicht zu, was folgende Formen zeigen: \textit{\=aisr}-\textit{ǝ} (Dativ Singular Feminin), \textit{\=aisr}-\textit{e} (Plural) \citep[65]{Moser1937}. Die \isi{Variation} in der \isi{Wurzel} ist also vom morphosyntaktischen Kontext abhängig. In den genannten \isi{Possessivpronomen} sind folglich für den Dativ Singular Maskulin/Neutrum zwei RRs nötig: eine definiert das \isi{Suffix} -\textit{ǝm}, die andere tilgt \textit{r}. Diese RRs müssen in zwei verschiedenen \isi{Blöcken} stehen. Ständen sie im selben \isi{Block}, würden sie zwei Formen definieren (vgl. \sectref{5.6.3}). Die RR für das \isi{Suffix} ist in \isi{Block} B verortet, die RR für die Wur\-zel-/Stamm\-al\-ter\-na\-tion in \isi{Block} C (\isi{Block} A wird unten diskutiert). Die RRs für die Wur\-zel-/Stamm\-al\-ter\-na\-tion sehen wie folgt aus:

\ea%165
\label{ex:key:165}
 RR \textsubscript{C,} \textsubscript{\{\textsc{case:dat}, \textsc{num:sg}, \textsc{gend:m}} \textsubscript{\tiny $\veebar$}\textsubscript{ \textsc{n}\},} \textsubscript{\textsc{det2[pron.poss}, \textsc{pers:1}} \textsubscript{\tiny $\veebar$}\textsubscript{ 2, \textsc{num:pl}]} ($\langle$X,$\sigma$ $\rangle$) = \textsubscript{def} $\langle$X *\textit{r} $\rightarrow$ øˊ,$\sigma$ $\rangle$
\z

\ea%166
\label{ex:key:166}
 RR \textsubscript{C,} \textsubscript{\{\textsc{case:dat}, \textsc{num:sg}, \textsc{gend:m}} \textsubscript{\tiny $\veebar$}\textsubscript{ \textsc{n}\},} \textsubscript{\textsc{det2[pron.poss}, \textsc{pers:3}, \textsc{num:sg}, \textsc{gend:f}]} ($\langle$X,$\sigma$ $\rangle$) = \textsubscript{def} $\langle$X *\textit{r} $\rightarrow$ øˊ,$\sigma$ $\rangle$
\z

Ebenfalls im Dialekt von Petrifeld wird im \isi{Possessivpronomen} der 3. Person Plural ein ganzes Segment getilgt, wenn ein \isi{Suffix} folgt: \textit{\=iərnə} (=\isi{Wurzel}), \textit{\=iər}-\textit{əm} (Dativ Singular) \citep[65-66]{Moser1937}). Die Definition über die morphosyntaktischen Eigenschaften scheint hier also auf den ersten Blick nicht nötig zu sein. Da aber ausschließlich im Dativ Singular ein \isi{Suffix} auftaucht (alle anderen Zellen weisen keine \isi{Suffixe} auf), kann der Kontext am einfachsten über die morphosyntaktischen Eigenschaften beschrieben werden:

\ea%167
\label{ex:key:167}
 RR \textsubscript{C,} \textsubscript{\{\textsc{case:dat}, \textsc{num:sg}, GEND:\},} \textsubscript{\textsc{det2[pron.poss}, \textsc{pers:3}, \textsc{num:pl}]} ($\langle$X,$\sigma$ $\rangle$) = \textsubscript{def} $\langle$X *\textit{nə} $\rightarrow$ øˊ,$\sigma$ $\rangle$ \\
\z

Schließlich muss für Petrifeld noch ein n-Einschub stipuliert werden. Die \isi{Possessivpronomen} der 1. und 2. Person Singular sowie der 3. Person Singular Maskulin/Neutrum lauten vokalisch aus. Wird ihnen ein vokalisches \isi{Suffix} angehängt, wird per Default ein \textit{n} eingeschoben: \textit{m\=ai} (=\isi{Wurzel}), \textit{m\=ain}-\textit{e} (Plural) \citep[65]{Moser1937}. Im Dativ Singular Feminin wird jedoch ebenfalls ein \textit{n} eingeschoben, obwohl das \isi{Suffix} konsonantisch anlautet: \textit{m\=ain}-\textit{rə} \citep[65]{Moser1937}. Dies muss also durch eine RRs in \isi{Block} A definiert werden, die übrigen \isi{Suffixe} stehen in \isi{Block} B. Nur so kann gewährleistet werden, dass zuerst -\textit{n} und dann - \textit{rə} suffigiert wird (\textit{m\=ai}-\textit{n}-\textit{rə}):

\ea%168
\label{ex:key:168}
 RR \textsubscript{A,} \textsubscript{\{\textsc{case:dat}, \textsc{num:sg}, \textsc{gend:f}\},} \textsubscript{\textsc{det2[pron.poss}, \textsc{pers:3}, \textsc{num:sg}, \textsc{gend:m}} \textsubscript{\tiny $\veebar$}\textsubscript{ \textsc{n}]} ($\langle$X,$\sigma$$\rangle$) = \textsubscript{def} $\langle$X\textit{n}ˊ,$\sigma$$\rangle$
\z

\ea%169
\label{ex:key:169}
 RR \textsubscript{A,} \textsubscript{\{\textsc{case:dat}, \textsc{num:sg}, \textsc{gend:f}\},} \textsubscript{\textsc{det2[pron.poss}, \textsc{pers:1}} \textsubscript{\tiny $\veebar$}\textsubscript{ 2, \textsc{num:sg}]} ($\langle$X,$\sigma$$\rangle$) = \textsubscript{def} $\langle$X\textit{n}ˊ,$\sigma$$\rangle$
\z

\section{Synopse}\label{5.7}

In den vorangehenden Kapiteln wurden exemplarisch jene Phänomene sowie ihre Probleme und Lösungen vorgestellt, die in den Varietäten dieser Arbeit vorkommen. Diese Analyse soll hier kurz zusammengefasst werden, und zwar nach Phänomenen und nicht nach Wortarten. Daraus ergeben sich drei Themenblöcke: Zugehörigkeit eines Phänomens zur Morphologie oder zur Phonologie (\sectref{5.7.1}) bzw. zur Morphologie oder zur Syntax (\sectref{5.7.2}) und Typen morphologischer Probleme (\sectref{5.7.3}).

\subsection{Morphologie vs. Phonologie}\label{5.7.1}

Eine zentrale Frage ist, ob ein bestimmtes Phänomen zur Morphologie oder zur Phonologie gehört. Handelt es sich um einen Prozess, der im gesamten Sprachsystem gilt, also per Default ausgeführt wird, so gehört dieser zur Phonologie. Kommt dieser Prozess jedoch nur in einem morphologisch definierbaren Bereich vor, wird er zur Morphologie gezählt. Ein ausgezeichnetes Beispiel sind die wa-/w\=o{}-Stäm\-me bzw. deren Reste im Alt- und Mittelhochdeutschen sowie im Dialekt von Issime. Im Althochdeutschen ist die \isi{Variation} zwischen \textit{sn\=e}\textit{o} ‘Schnee’ und \textit{sn\=e}\textit{wes} phonologisch bedingt, weil es eine Regel gibt, die auslautendes \textit{w} immer zu \textit{o} vokalisiert (vgl. \sectref{5.1.3} \is{Subtraktion}Subtraktion). Im Gegensatz dazu wird im Mittelhochdeutschen auslautendes \textit{w} erstens getilgt und zweitens ist diese Tilgung synchron nicht vorauszusagen, weil auslautendes \textit{w} nicht immer getilgt wird (vgl. \sectref{5.1.3} \is{Subtraktion}Subtraktion). Diese Tilgung gehört also zur Morphologie und wird durch eine an bestimmte \isi{Flexionsklassen} gebundene RR definiert (vgl. \sectref{5.1.3} \is{Subtraktion}Subtraktion). Schließlich konnte für den Dialekt von Issime gezeigt werden, dass es sich bei \textit{w} um einen Pluralmarker handelt, der ebenfalls durch eine RR bestimmt wird (vgl. \sectref{5.1.3} \is{Subtraktion}Subtraktion).

Ein ähnliches Phänomen ist bezüglich des \textit{n} zu finden. In den alemannischen Dialekten wird \textit{n} eingeschoben, um einen \isi{Hiat} zu vermeiden, und zwar nicht nur innerhalb eines Wortes sondern auch zwischen Wörtern. Die Phonologie infigiert also automatisch ein \textit{n} zur Vermeidung eines \isi{Hiats}. Im Dialekt von Issime hingegen stellt \textit{n} auch eine Pluralmarker da, der folglich zur Morphologie gehört (vgl. \sectref{5.1.4}).

In der Folge sollen nun die wichtigsten Phänomene aufgezählt werden, die erstens aus synchroner Sicht phonologisch nicht erklärt werden können (sie gehören zur Morphologie) und zweitens die synchron phonologisch voraussagbar sind (sie gehören zur Phonologie).

\subsubsection{Synchron phonologisch nicht erklärbar}

\begin{description}
\item [\isi{Substantive}:] Dazu gehört das bereits erwähnte \textit{w} im Mittelhochdeutschen wie auch die althochdeutschen Diminutive, in denen das auslautende \textit{\=i} vor Diphthong getilgt wird (vgl. \sectref{5.1.3} \is{Subtraktion}Subtraktion). Außerdem zählen hierzu auch die Fälle in den alemannischen Dialekten, in denen der wurzelauslautende Vokal getilgt wird, wenn ein vokalisch anlautendes \isi{Suffix} folgt. Weil per Default die Phonologie den entstandenen \isi{Hiat} durch den Einschub von \textit{n} beheben würde, gehört die Tilgung des auslautenden Wurzelvokals zur Morphologie (vgl. \sectref{5.1.4}).
\item [\isi{Possessivpronomen}:] Bei den \isi{Possessivpronomen} werden ebenfalls an der \isi{Wurzel} Elemente suffigiert oder subtrahiert (meistens nachdem ein Flexionssuffix angehängt wurde), ohne dass diese phonologisch erklärbar sind (ausführlich diskutiert in \sectref{5.6.8}).
\item [Bestimmter und \isi{unbestimmter Artikel} (in vielen Dialekten):] Im Akkusativ und Dativ nimmt der Artikel eine andere Form an, je nachdem ob ihm eine Präposition vorausgeht oder nicht. Geht dem Artikel eine Präposition voraus, weist der Artikel eine reduzierte Form auf. Diese Reduktion kann aber nicht allgemein für das ganze System beschrieben werden, weshalb auch die reduzierte Form von der Morphologie definiert werden muss (vgl. \sectref{5.5.5} und \sectref{5.6.4}).
\item [Bestimmter Artikel vs. \isi{Demonstrativpronomen} (in allen Dialekten):] Beim bestim-\linebreak mten Artikel handelt es sich um eine reduzierte Form des \isi{Demonstrativpronomens}. Auch diese kann nicht durch einheitliche phonologische Regeln beschrieben werden. Sie muss also von RRs definiert werden (vgl. \sectref{5.5.3}).
\item [Betontes und unbetontes \isi{Personalpronomen} (in allen Dialekten):] Die Form des unbetonten \isi{Personalpronomens} ist ebenfalls eine reduzierte Form des betonten \isi{Personalpronomens}. Auch hier agieren die phonologischen Regeln synchron nicht mehr (vgl. \sectref{5.3.2}).
\end{description}


\subsubsection{Synchron phonologisch/phonotaktisch erklärbare Phänomene}

\begin{description}
\item[\isi{Substantive} der deutschen Standardsprache:] Wird -\textit{s} oder -\textit{n} suffigiert, können diese \isi{Suffixe} als -\textit{s}/-\textit{n} oder -\textit{əs}/-\textit{ən} auftreten (gilt in allen \isi{Kasus} und beiden \isi{Numeri}). Diese \isi{Variation} ist phonologisch und phonotaktisch bedingt, gilt also für das gesamte System (z.B. auch in der Verbflexion) (vgl. \sectref{5.1.1}).
\item[Mittelsilbensenkung in den \isi{Substantiven} (in vielen Dialekten):] In vielen Dialekten\linebreak wird das auslautende -\textit{i} zu -\textit{e} oder -\textit{ə} gesenkt, wenn es in den Inlaut tritt. Vollvokale in der Mittelsilbe sind in diesen Dialekten nicht möglich (vgl. \sectref{5.1.4}).
\item[Kürzung der Endungen des Typs -\textit{ən}\textit{ə} (\isi{Substantive}) (in drei Dialekten):] Durch die\linebreak Suffigierung von -\textit{ən}\textit{ə} folgen der betonten Wurzelsilben drei unbetonte Silben. Aus phonotaktischen Gründen wird eine unbetonte Silbe getilgt, weil Wörter nur auf einen Trochäus oder einen Daktylus auslauten dürfen (vgl. \sectref{5.1.4}).
\end{description}

\subsection{Morphologie vs. Syntax}\label{5.7.2}

Neben phonologischen sind auch syntaktische Phänomene von morphologisch-\linebreak en zu trennen. Es geht vor allem darum, dass die Morphologie Formen zur Verfügung stellt, deren Distribution jedoch von der Syntax geregelt wird. Beispielsweise definieren RRs stark und schwach flektierte \isi{Adjektive}. Wie diese jedoch im Syntagma verteilt sind, wird durch syntaktische Regeln bestimmt (vgl. \sectref{5.2.1}).

Ein weiteres Beispiel findet sich bei den Artikeln. Der \isi{bestimmte Artikel} weist unterschiedliche Formen auf, je nachdem, ob er vor einem \isi{Adjektiv} steht oder nicht, oder ob ihm eine Präposition vorausgeht oder nicht. Der \isi{unbestimmte Artikel} variiert nur in Abhängigkeit von der Präsenz und Absenz einer vorausgehenden Präposition. Zwar ist die Distribution syntaktisch bedingt; die Formen der Artikel müssen jedoch definiert werden, und zwar durch RRs (vgl. \sectref{5.5.5} und \sectref{5.6.4}).

\subsection{Morphologische Phänomene}\label{5.7.3}

In diesem Kapitel\largerpage[2] sollen noch einige morphologische Phänomene zusammengefasst werden, die vor allem \is{nicht-konkatenative Flexion}nicht-konkatenativ sind (\sectref{5.7.3.1} und \sectref{5.7.3.2}) oder dem 1-zu-1-Verhältnis zwischen Form und Funktion widersprechen (\sectref{5.7.3.3} und \sectref{5.7.3.4}). Des Weiteren soll die Definition der \isi{Flexionsklasse} resümiert werden, da diese von den meisten Definitionen abweicht (\sectref{5.7.3.5}). Schließlich werden die diachron neu entstandenen Kategorien aufgelistet (\sectref{5.7.3.6}).

\subsubsection{Wurzel + Realisierungsregeln vs. nur Realisierungsregeln}\label{5.7.3.1}

In den folgenden Wortarten kann eine \isi{Wurzel} von \isi{Affixen} getrennt werden: \isi{Substantive}, \isi{Adjektive} und \isi{Possessivpronomen}. Die \isi{Wurzel} stammt aus dem \isi{Radikon}, von der durch RRs flektierte Wörter abgeleitet werden können. In den übrigen Wortarten (\isi{Personalpronomen}, \isi{Interrogativpronomen}, \isi{Demonstrativpronomen}, bestimmter und \isi{unbestimmter Artikel}) ist eine Trennung von \isi{Wurzel} und \isi{Affixen} nicht möglich. Eine Ausnahme bildet der \isi{unbestimmte Artikel} im Mittelhochdeutschen und in der deutschen Standardsprache sowie in den Dialekten von Visperterminen und Issime. Kann ein flektiertes Wort nicht in eine \isi{Wurzel} und \isi{Affixen} dividiert werden, wird die gesamte Form durch RRs definiert (vgl.\chapref{5}, \sectref{5.3.1}, \sectref{5.4}, \sectref{5.5.1}, \sectref{5.6.1}).

\subsubsection{Nicht-konkatenative Morphologie}\label{5.7.3.2}

Hierzu gehören \is{Modifikation}Modifikationen und \is{Subtraktion}Subtraktionen an der \isi{Wurzel}, woraus neue Stämme entstehen. Innerhalb der in\-fe\-ren\-tiel\-len-re\-a\-li\-sie\-ren\-den Morphologie und anhand der RRs können \is{nicht-konkatenative Flexion}nicht-konkatenative Phänomene problemlos adäquat erfasst werden, was in den Abschnitten \sectref{4.1.2} und \sectref{4.1.3} erörtert wurde.

Zur \is{Modifikation}\textsc{Modifikation} in den \isi{Substantiven} gehören der \isi{Umlaut}, die Diphthongierung (nur Münstertal) und die Velarisierung (nur Elsass (Ebene)). Durch diese \is{Modifikation}Modifikationen werden Pluralstämme abgeleitet (vgl. \sectref{5.1.3}).

Auch durch \is{Subtraktion}\textsc{Subtraktionen} entstehen neue Stämme. Im mittelhochdeutschen \isi{Substantiv} und \isi{Adjektiv} wird wurzelauslautendes \textit{w} getilgt, wenn es in den Auslaut tritt. Im Dialekt von Münstertal fällt das \textit{t} der \isi{Wurzel} im Plural weg. Zudem wird in vielen Dialekten der auslautende Vokal der \isi{Wurzel} getilgt, wenn ein vokalisch anlautendes \isi{Suffix} folgt (z.B. \textit{heisle} ’Häuschen‘ (Sg.), \textit{heisl}-\textit{ə} (Pl.) \citep[98]{Baur1967}). Auch in den althochdeutschen Diminutiva fällt das auslautende -\textit{\=i} der \isi{Wurzel} weg, wenn das \isi{Suffix} mit einem Diphthong anlautet. Wie in \sectref{5.7.1} sowie genauer in den Abschnitten \sectref{5.1.3} und \sectref{5.2.2} erklärt wurde, handelt es sich dabei nicht um voraussagbare phonologische Prozesse. Weitere \is{Subtraktion}Subtraktionen gibt es in den \isi{Possessivpronomen} (vgl. \sectref{5.6.8}).

Bemerkenswert ist hier nicht nur die \is{Subtraktion}Subtraktion an sich, sondern auch die Tatsache, dass die Bedingung für die \is{Subtraktion}Subtraktion erst durch das \isi{Suffix} gegeben ist. Die RR muss also definieren, in welcher phonologischen Umgebung was subtrahiert wird. Des Weiteren steht diese \is{Subtraktion}Subtraktions-RR in jenem \isi{Block}, der auf den \isi{Block} der RR für das \isi{Suffix} folgt. Nur so ist gewährleistet, dass zuerst suffigiert wird, woraus die Bedingung für die \is{Subtraktion}Subtraktion entsteht (vgl. \sectref{5.1.3} und \sectref{4.1.3.2}). In den \isi{Possessivpronomen} sind außerdem auch Fälle zu beobachten, in denen die \isi{Wurzel} erweitert wird, wenn ein \isi{Suffix} angehängt wird (vgl. \sectref{5.6.8}).

\subsubsection{Synkretismen}\label{5.7.3.3}

Es wurde gezeigt, dass \isi{Synkretismen} im \isi{Numerus}, \isi{Kasus} sowie \isi{Genus} auftreten. Außerdem gibt es auch \isi{Synkretismen} zwischen den Wortarten einer Kategorie. Beispielsweise unterscheidet die deutsche Standardsprache die Form des \isi{bestimmten Artikels} nicht von der Form des einfachen \isi{Demonstrativpronomens}. \isi{Synkretismen} können von den RRs auf zwei verschiedene Arten erfasst werden. Fallen alle Features einer morphosyntaktischen Eigenschaft bzw. einer Kategorie zusammen (z.B. keine Kasusunterscheidung), bleibt diese morphosyntaktische Eigenschaft bzw. Kategorie unterspezifiziert. Fallen einige Features zusammen (z.B. Nominativ und Akkusativ), werden diese durch eine ausschließende Disjunktion erfasst (Zeichen ${\veebar}$) (vgl. z.B. \sectref{5.2.1}).

Dieser Typ von \isi{Synkretismus} macht also ein System einfacher, da weniger RRs benötigt werden. Ein zweiter Typ von \isi{Synkretismus} hingegen macht das System komplexer, da er nicht durch eine einzige RR definiert werden kann. Dies ist der Fall, wenn die Features von mehr als einer morphosyntaktischen Eigenschaft/Kategorie variieren. Beispielsweise muss in der starken Adjektivflexion der deutschen Standardsprache das \isi{Suffix} -\textit{ər} (Dativ und Genitiv Feminin Singular sowie Genitiv Plural) durch zwei RRs bestimmt werden, da \isi{Numerus}, \isi{Kasus} und \isi{Genus} variieren (ausführlich diskutiert in \sectref{4.1.3.3}).

\subsubsection{Freie Variation}\label{5.7.3.4}

Alle Formen einer Zelle des Paradigmas müssen definiert werden, d.h. also auch, wenn mehr als eine Form vorhanden ist. In den hier untersuchten Varietäten kommen zwei Typen an freien Varianten vor. Im ersten Typ sind beide Formen suffigiert, wobei die beiden RRs gleich spezifisch sein und im selben \isi{Block} stehen müssen. Nur so können zwei Formen für dieselbe Zelle bestimmt werden (vgl. \sectref{4.1.3.3}). Im zweiten Typ ist eine Form suffigiert, die andere Form besteht nur aus der \isi{Wurzel}. Hier sind ebenfalls zwei RRs nötig: Eine definiert das \isi{Suffix}, die andere definiert, dass mit der \isi{Wurzel} nichts passiert. Würde nur die suffigierte Form definiert werden, würde diese ein zusätzliches Einfügen der \isi{Wurzel} in dieselbe Zelle verhindern. Dass beide Varianten in derselben Zelle stehen, ist nur dann gewährleistet, wenn zwei gleichspezifische RRs desselben \isi{Blocks} für dieselbe Zelle Formen definieren (vgl. \sectref{4.1.3.3}, Beispiel u.a. \sectref{5.1.2}).

\subsubsection{Zugehörigkeit Flexionsklassen}\label{5.7.3.5}

In den Abschnitten \sectref{5.1.1} und \sectref{5.1.4} wurde gezeigt, was in dieser Arbeit unter \isi{Flexionsklasse} verstanden wird. Die wichtigsten Punkte sollen hier noch einmal zusammengefasst werden:

\begin{itemize}
\item 
\isi{Flexionsklassen} sind eine Art Instruktion, wie die RRs miteinander kombiniert werden, d.h., sie zeigen die Anzahl Kombinationen an RRs.
\item 
Folglich werden nur jene Kategorien unterschieden, die auch durch die RRs unterschieden werden. Z.B. wird in der Substantivflexion im Dialekt des Kaiserstuhls kein \isi{Kasus} markiert.
\item 
Zwei \isi{Flexionsklassen} unterscheiden sich in mindestens einer RR. Sie werden also weder nach Stämmen oder Deklinationstypen eingeteilt, noch gibt es Ober- und Unterklassen.
\item 
Eine \isi{Flexionsklasse} hat mindestens zwei Lexeme.
\end{itemize}

\subsubsection{Neue Kategorien}\label{5.7.3.6}

Vergleicht man das Alt- und Mittelhochdeutsche mit den modernen Varietäten, stellt man fest, dass neue Kategorien entstanden sind. Diese sollen in der Folge kurz zusammengefasst werden. Die Zusammenfassung ist nach Wortarten gegliedert: Artikel, \isi{Possessivpronomen} und \isi{Personalpronomen}.

Einen grammatikalisierten \textsc{Artikel} gibt es im Althochdeutschen nicht. Das Mittelhochdeutsche und die deutsche Standardsprache verfügen über einen bestimmten und einen \isi{unbestimmten Artikel}. Die Flexion der Artikel weist jedoch keine Unterschiede zur Flexion des Demonstrativ- bzw. \isi{Possessivpronomens} auf. Es ist also nur ein Satz an RRs für den \isi{bestimmten Artikel}/\isi{Demonstrativpronomen} und ein Satz für den \isi{unbestimmten Artikel}/\isi{Possessivpronomen} nötig. Für die Dialekte benötigt man jedoch für jede der vier Wortarten einen Satz an RRs. Schließlich wurde gezeigt, dass viele Dialekte im Akkusativ und/oder Dativ zwei Artikelvarianten pro Zelle haben, wobei diese syntaktisch distribuiert sind. Trotzdem sind für beide Varianten RRs anzunehmen, da nur RRs Formen definieren (vgl. \sectref{5.5.3} und \sectref{5.6.2}).

Die Flexion des \textsc{Possessivpronomens} im Mittelhochdeutschen und in der deutschen Standardsprache ist also identisch mit der Flexion des \isi{unbestimmten Artikels}. In den Dialekten sind zwei separate Paradigmen anzunehmen. Zusätzlich weisen die meisten Dialekte im \isi{Possessivpronomen} unterschiedliche Paradigmen auf, und zwar in Abhängigkeit davon, um welches \isi{Possessivpronomen} es sich handelt (vgl. \sectref{5.6.6}).

Es wurde auch gezeigt, dass alle Dialekte im \textsc{Personalpronomen} ein betontes und unbetontes Paradigma aufweisen (vgl. \sectref{5.3.2}). Des Weiteren unterscheiden einige Dialekte im Neutrum der 3. Person Singular im Akkusativ eine belebte und eine unbelebte Form (vgl. \sectref{5.3.3}). Schließlich hat der Dialekte von Issime im Plural einfache und zusammengesetzte Formen (vgl. \sectref{5.3.1}).