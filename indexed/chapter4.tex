\chapter{Theoretische Grundlage und Messmethode}\label{4}

Dieses Kapitel stellt die theoretischen Grundlagen vor und wie diese zur Komplexitätsmessung übernommen und/oder angepasst werden müssen (\sectref{4.1}). Davon wird abgeleitet, was ein System mehr oder weniger komplex macht (\sectref{4.2}) und wie in dieser Arbeit Komplexität gemessen wird (\sectref{4.3}).

Zu einem einfacheren Verständnis wird hier sehr kurz zusammengefasst, welche Komponenten die nominale Flexionsmorphologie mehr oder weniger komplex machen. Diese Zusammenfassung basiert auf \citet[28]{BaechlerSeiler2012} und \citet[23]{Baechler2016}.\\

\noindent
Ein System wird durch Folgendes \textsc{komplexer}:

\begin{itemize}
\item 
Anzahl der morphosyntaktischen Eigenschaften, die in der Flexion markiert werden, und zwar unabhängig davon, wie sie ausgedrückt werden (z.\,B.\ Affigierung, Subtraktion, \is{Modifikation}Modifikation der \isi{Wurzel} etc.).
\item 
Je mehr Allomorphie, desto komplexer, z.\,B.\ Pluralallomorphe im \isi{Substantiv} der deutschen Standardsprache (-\textit{ə}, -\textit{ər}, -\textit{n} etc.).
\item 
Mehrfachausdruck derselben morphosyntaktischen Eigenschaft. Beispielsweise wird in \textit{Wälder} der Plural durch den \isi{Umlaut} und durch das \isi{Suffix} -\textit{ər} ausgedrückt.
\item 
Ein bestimmter \isi{Synkretismus}: Wenn die Werte von mindestens zwei morphosyntaktischen Eigenschaften variieren. Zum Beispiel markiert in der starken Adjektivflexion der deutschen Standardsprache -\textit{ər} u.a. den Genitiv Singular Feminin und den Genitiv Plural. Folglich variieren die Werte von zwei morphosyntaktischen Eigenschaften, nämlich \isi{Genus} (Feminin vs. nicht spezfiziert) und \isi{Numerus} (Singular vs. Plural).
\end{itemize}

\noindent\largerpage[2]
Ein System wird durch Folgendes \textsc{simpler}:

\begin{itemize}
\item 
Ein bestimmter \isi{Synkretismus}: Wenn die Werte von maximal einer morphosyntaktischen Eigenschaften variieren.
\item 
Wenn keine overte Markierung vorhanden ist (Unterspezifikation), z.\,B.\ der Nominativ Singular der \isi{Substantive} in der deutschen Standardsprache.
\item 
Unterscheidungen, die in einer Wortart gemacht werden, aber nicht in einer anderen. Beispielsweise wird im Kaiserstuhl Alemannischen \isi{Kasus} am \isi{Substantiv} nicht markiert, jedoch an den Determinierern und am \isi{Adjektiv}.
\item 
Phonologisch erklärbare Allomorphie. Zum Beispiel hängen die Allomorphe des Genitiv Singular \isi{Suffixes} -\textit{əs}/-\textit{s} in der deutschen Standardsprache von der phonologischen Struktur des Wortes ab (Auslaut, Anzahl Silben, Akzent).
\end{itemize}

\section{Theoretische Grundlage}\label{4.1}

\subsection{LFG und Morphologie}\label{4.1.1}

\subsubsection{Grundsätzliches zu LFG} In der \is{Lexical-Functional Grammar (LFG)}Lexical-Functional Grammar (LFG) werden neben einem Lexikon verschiedene Subsysteme angenommen, die jeweils eigene Einheiten und Regeln aufweisen. Zu diesen Subsystemen, die Strukturen genannt werden, gehören beispielsweise die semantische Struktur, Informationsstruktur, phonologische Struktur, Argumentstruktur, funktionale Struktur und Konstituentenstruktur \citep[22-25]{Falk2001}. Diese Subsysteme bilden keine hierarchische, sondern eine parallele Architektur, d.h., sie existieren gleichzeitig parallel, keine Struktur geht einer anderen voraus \citep[23]{Falk2001}. Die Entsprechungen und Beziehungen zwischen den verschiedenen Strukturen werden durch ein System von Funktionsgleichungen abgebildet \citep[51]{Bresnan2001}. Wie in anderen Frameworks wird auch in \is{Lexical-Functional Grammar (LFG)}LFG diskutiert, welche Subsysteme in einer Sprache existieren und folglich modelliert werden müssen. Jedoch gehen alle mir bekannten Publikationen davon aus, dass zumindest eine funktionale Struktur (f-Struk\-tur) und eine Konstituentenstruktur (c-Struk\-tur) angenommen werden müssen. Da diese im weiteren Verlauf dieser Arbeit wieder vorkommen, sollen sie kurz skizziert werden.

Die Unterscheidung zwischen c-Struk\-tur und f-Struk\-tur beruht auf der Beobachtung, dass die Hierarchie von Phrasen und deren Abfolge nichts mit den Funktionen in einem Satz zu tun haben müssen. Man betrachte die beiden Sätze, deren c-Struk\-tu\-ren in \REF{ex:key:1} und deren f-Struk\-tur in \REF{ex:key:2} abgebildet ist: (1.1) \textit{Ein Buch hat Max gelesen}, (1.2) \textit{Max las ein Buch}.\footnote{Die c-Struk\-tur basiert auf \citet[23-44]{Berman2003}.} Gehen wir davon aus, dass Perfekt und Präteritum in der deutschen Standardsprache die Zeitstufe Vergangenheit ausdrücken, weisen beide Sätze dieselben Funktionen auf: Prädikat \textit{lesen} in der Vergangenheit, Subjekt \textit{Max} (mit morphosyntaktischen Eigenschaften), Objekt \textit{ein} \textit{Buch} (mit morphosyntaktischen Eigenschaften). Die Unterschiede betreffen ausschließlich den Aufbau und die Abfolge der Konstituenten. Dieser Tatsache kann Rechnung getragen werden, indem für beide Sätze zwei c-Struk\-tu\-ren (abgebildet in \ref{ex:key:1}) und eine f-Struk\-tur angenommen wird (abgebildet in \ref{ex:key:2}).

\ea%1
    \label{ex:key:1}\begin{multicols}{2}
      \ea 
      \begin{forest}
    [CP
	[NP [DET [ein]] [N [Buch]]]
    [C'
    [C[hat]]
    [VP
    [NP [N[Max]]]
    [VP [V[gelesen]]]]]
    ]
\end{forest}

\ex 
    \begin{forest}
[CP
	[NP [N [Max]]]
    [C'
    [C[las]]
    [VP
    [NP [Det [ein]] [N[Buch]]]]]
    ]
\end{forest}
\z \end{multicols} \z
   

\ea%2
    \label{ex:key:2}
\begin{avm}
f\textsubscript{1} f\textsubscript{4} f\textsubscript{5}  f\textsubscript{6}  
\[PRED & ʻlesen $\langle$SUBJ, OBJ$\rangle$ʼ \\
TENSE & PAST \\
SUBJ & f\textsubscript{2} f\textsubscript{3} \[PRED & ʻMaxʼ\\
		 										GEND & M\\
        										NUM & SG\\
                            					CASE & NOM
                           						 \]\\
OBJ & f\textsubscript{7} f\textsubscript{8} f\textsubscript{9}
\[PRED & ʻBuchʼ\\
 DEF & -\\
 GEND & N\\
 NUM & SG\\
 CASE & ACC
\]\\
\] 
\end{avm}
    \z


Um die c-Struk\-tur und die f-Struk\-tur miteinander zu verbinden, wird ein System von „correspondence functions, or […] projection […] functions“ \citep[24]{Falk2001} benötigt. Dazu wird jedem Knoten in der c-Struk\-tur eine Variable f\textsubscript{x} zugewiesen, wie in der c-Struk\-tur in \REF{ex:key:3}. Die einem bestimmten Knoten entsprechende f-Struk\-tur wird mit derselben Variablen versehen (vgl. \ref{ex:key:2}).

\ea%3
    \label{ex:key:3}
\begin{forest}
[CP\textsubscript{f1}
	[NP\textsubscript{f2} [N\textsubscript{f3} [Max]]]
    [C'\textsubscript{f4}
    [C\textsubscript{f5}[las]]
    [VP\textsubscript{f6}
    [NP\textsubscript{f7} [Det\textsubscript{f8} [ein]] [N\textsubscript{f9}[Buch]]]]]
    ]
\end{forest}\z

  

 

Damit kann nun das Mapping zwischen der c-Struk\-tur und der f-Struk\-tur definiert werden, und zwar anhand einer Reihe von Funktionalgleichungen, was als funktionale Deskription (f-Des\-krip\-ti\-on) bezeichnet wird \citep[68–69]{Falk2001}. Die f-Des\-krip\-ti\-on für die f-Struk\-tur in \REF{ex:key:2} und die c-Struk\-tur in \REF{ex:key:3} steht in \REF{ex:key:4}. Aus der f-Des\-krip\-tion in \REF{ex:key:4} wird ersichtlich, dass die Gleichungen die Beziehung zwischen Mutter- und Tochterknoten ausdrücken. Die Variablen f\textsubscript{x} können also durch Metavariablen (↑↓) ersetzt werden, wobei die Variable ↑ auf den Mutterknoten verweist, die Variable ↓ auf den Tochterknoten \citep[71]{Falk2001}. Beispielsweise kann die Gleichung (f\textsubscript{1} SUBJ) = f\textsubscript{2} durch die Gleichung (↑ SUBJ) = ↓ ersetzt werden.\largerpage[2]

\ea%4
    \label{ex:key:4}
(f\textsubscript{1} SUBJ) = f\textsubscript{2}\\
f\textsubscript{2} = f\textsubscript{3}\\
(f\textsubscript{3} PRED) = ʻMaxʼ\\
(f\textsubscript{3} DEF) = +\\
(f\textsubscript{3} GEND) = M\\
(f\textsubscript{3} NUM) = SG\\
(f\textsubscript{3} CASE) = NOM\\
f\textsubscript{1} = f\textsubscript{4}\\
f\textsubscript{4} = f\textsubscript{5}\\
(f\textsubscript{5} PRED) = ʻlesen $\langle$(f\textsubscript{5} SUBJ) (f\textsubscript{5} OBJ)$\rangle$ʼ\\
f\textsubscript{4} = f\textsubscript{6}\\
(f\textsubscript{6} OBJ) = f\textsubscript{7}\\
f\textsubscript{7} = f\textsubscript{8}\\
f\textsubscript{7} = f\textsubscript{9}\\
(f\textsubscript{8} DEF) = –\\
(f\textsubscript{8} NUM) = SG\\
(f\textsubscript{8} CASE) = ACC\\
(f\textsubscript{8} GEND) = N\\
(f\textsubscript{9} PRED) = ʻBuchʼ\\
(f\textsubscript{9} NUM) = SG\\
(f\textsubscript{9} CASE) = ACC\\
(f\textsubscript{9} GEND) = N\\
\z

Gehen wir von einem Modell mit paralleler Architektur aus,\largerpage kann auch die Morphologie ein Subsystem bilden. Für die Komplexitätsmessung ist eine parallele Architektur vor allem deswegen von Vorteil, weil nur so in Zukunft eventuelle Ausgleichstendenzen zwischen den Subsystemen (z.\,B.\ zwischen Syntax und Morphologie) gemessen werden können. Natürlich stellt sich ganz grundsätzlich die Frage, ob Morphologie als autonome Repräsentationsebene existiert oder sie Teil der Syntax und Phonologie ist. Die unterschiedlichen Positionen dazu können hier nicht umfassend dargestellt werden. Vielmehr soll hier dafür argumentiert werden, dass die Morphologie vielleicht keine universelle Komponente in der Grammatik bildet, aber dass sie, wenn sie in einer Sprache existiert, nach unabhängigen Prinzipien und Regeln strukturiert ist \citep[156]{BörjarsVincentChapman1997}. Neben vielen anderen konnte auch Anderson (1992, vor allem Kapitel 2) zeigen, dass es ein System an Regeln gibt, das Wörter konstruiert, und ein anderes System an Regeln, dass die syntaktische Struktur von Phrasen und Sätzen organisiert \citep[22]{Anderson1992}. Nehmen wir also die Morphologie als Subsystem an, stellt sich die Frage, wie die Einheiten und Regeln dieses Subsystems innerhalb von \is{Lexical-Functional Grammar (LFG)}LFG ausschauen. Traditionell wird in \is{Lexical-Functional Grammar (LFG)}LFG eine morphembasierte Auffassung von Morphologie vertreten; genauer eine le\-xi\-ka\-lisch-in\-kre\-men\-tel\-le Auffassung \citep[112]{AckermanStump2004}. \is{lexikalische Theorie}Lexikalisch heißt, dass Stämme und \isi{Affixe} im Lexikon gelistet sind. Unter inkrementell versteht man den Vorgang, dass ein flektiertes Wort eine bestimmte grammatische Eigenschaft erhält, indem es jenen Marker zu sich nimmt, der genau diese grammatische Eigenschaft trägt. Folgendes leicht abgeändertes Beispiel aus \citet[55]{Bresnan2001} soll dies kurz illustrieren. Die Verbform \textit{lives} besteht aus einem Stamm und einem \isi{Suffix}, die mit funktionalen Schemata im Lexikon gelistet sind:\\\pagebreak

\noindent
\begin{styleNoSpacing}
live  (↑ PRED) = ʻlive $\langle$…$\rangle$ʼ
\end{styleNoSpacing}

\noindent
\begin{styleNoSpacing}
-s  (↑ TENSE) = PRES
\end{styleNoSpacing}
\noindent
\begin{styleNoSpacing}
  (↑ SUBJ) = ↓
\end{styleNoSpacing}
\noindent
\begin{styleNoSpacing}
    (↓ PERS) = 3
\end{styleNoSpacing}
\noindent
\begin{styleNoSpacing}
    (↓ NUM) = SG
\end{styleNoSpacing} \\

\noindent
Werden beide Lexikoneinträge miteinander kombiniert, entsteht die flektierte Wortform \textit{lives} mit den morphologischen Eigenschaften, die vom \isi{Suffix} -\textit{s} stammen. Dieses flektierte Wort kann in die c-Struk\-tur eingefügt werden, wobei eine funktionale Deskription generiert wird (zum besseren Verständnis wird hier auch ein Subjekt angefügt). Leicht abgeändertes Beispiel aus \citet[58]{Bresnan2001}:\\

\begin{forest} for tree={align=center,base=top}
	[S\textsubscript{f1}
    	[(f\textsubscript{1} SUBJ) {=} f\textsubscript{2}\\NP\textsubscript{f2}
        [f\textsubscript{2} {=} f\textsubscript{4}\\N\textsubscript{f4}
        [Max\\
        (f\textsubscript{4} NUM) {=} SG \\
        (f\textsubscript{4} PRED) {=} ʻMaxʼ]]]
        [f\textsubscript{1} {=} f\textsubscript{3}\\
        VP \textsubscript{f3}
        [f\textsubscript{3} {=} f\textsubscript{5}\\V\textsubscript{f5}
        [live   s \textsubscript{f6}\\
        (f\textsubscript{5} PRED) {=} ʻlive $\langle$…$\rangle$ʼ\\
        (f\textsubscript{5} TENSE) {=} PRES\\
        (f\textsubscript{5} SUBJ) {=} f\textsubscript{6}\\
        (f\textsubscript{6} PERS) {=} 3\\
        (f\textsubscript{6} NUM) {=} SG]]]
        ]
\end{forest}\\


Diese funktionale Deskription definiert das Mapping zwischen c- und f-Struk\-tur. Zur Veranschaulichung wird in der Folge noch die f-Struk\-tur dargestellt (\citealt[59]{Bresnan2001}; leicht abgeändert):\\

%AVM
\begin{avm}
f\textsubscript{1} f\textsubscript{3} f\textsubscript{5} 
\[SUBJ & f\textsubscript{2}, f\textsubscript{4}, f\textsubscript{6}
\[PERS & 3\\
 NUM & SG\\
 PRED & ʻMaxʼ
\]\\
TENSE & PRES\\
PRED & ʻlive $\langle$...$\rangle$ʼ
\]
\end{avm}\\

Zwei Dinge können hier festgehalten werden. Erstens können nur vollständige Wortformen in die c-Struk\-tur eingefügt werden, lexikalische Integrität stellt also einen zentralen Grundsatz von \is{Lexical-Functional Grammar (LFG)}LFG dar.\largerpage[2] Somit bilden Syntax und Morphologie zwei verschiedene Module mit eigenen Regeln. Zweitens wird hier aber noch von keinem autonomen morphologischen Subsystem ausgegangen. Dass es ein solches aber braucht, geht auf die Diskussion zur Schnittstelle zwischen Syntax und Morphologie sowie auf die konsequente Trennung von Form und Bedeutung zurück. Diese Diskussion soll hier kurz an einigen Beispielen skizziert werden.

\subsubsection{M-Struktur} \citet{ButtNiñoSegond2004} schlagen eine Modellierung von Hilfsverben vor, die die spezifische Kombination von Hilfs- und Vollverben zur Bildung von Periphrasen in der m(orphologischen)-Struktur verortet. Traditionell werden in \is{Lexical-Functional Grammar (LFG)}LFG Hilfs- und Modalverben wie Vollverben behandelt, wobei jedes verbale Element ein Komplement zu sich nimmt \citep[13]{ButtNiñoSegond2004}. Nach dieser Analyse haben die Sätze in \REF{ex:key:5} die c-Struk\-tur in \REF{ex:key:6} und die f-Struk\-tur in \REF{ex:key:7} \citep[14–15]{ButtNiñoSegond2004}:\largerpage[2.5]

\ea%5
    \label{ex:key:5}
The driver will have turned the lever.\\
Le conducteur aura tourné le levier\\
Der Fahrer wird den Hebel gedreht haben.\\
    \z


\ea%6
    \label{ex:key:6}

{\small\begin{forest}
[S
	[NP [the driver]]
     [VP 
     [AUX [will]] 
     [VP
     [AUX [have]]
     [VP
     [V [turned]]
     [NP [the lever]]]]]]
\end{forest}

%tree (6)
\begin{forest}
[S
	[NP [le conducteur]]
     [VP 
     [AUX [aura]] 
     [VP
     [V [tourné]]
     [NP [le levier]]]]]
\end{forest}}
\z

% AVM (7)
% \begin{styleNoSpacing}
\ea%7
    \label{ex:key:7}
\begin{avm}
\[PRED & ʻwill $\langle$XCOMP$\rangle$ SUBJʼ\\
 TENSE & PRES\\
 SUBJ & 
 \[PRED & ʻdriverʼ\\
  CASE & NOM\\
  GEND & M\\
  NUM & SG\\
  SPEC & DEF  
 \]\\
XCOMP &
\[PRED & ʻperfective $\langle$XCOMP$\rangle$ SUBJʼ\\
 SUBJ & [...]\\
 XCOMP &
 	\[PRED & ʻturn $\langle$SUBJ, OBJ$\rangle$ʼ\\
     SUBJ & [...]\\
     OBJ &
     	\[PRED & ʻleverʼ\\
         CASE & ACC\\
         GEND & M\\
         NUM & SG\\
         SPEC & DEF
        \]\\
    \]\\
\]\\
\]
\end{avm}\\


\begin{avm}
\[PRED & ʻauxiliary $\langle$XCOMP$\rangle$ SUBJʼ\\
 TENSE & FUT\\
 SUBJ & 
 \[PRED & ʻconducteurʼ\\
  PERS & 3\\
  GEND & M\\
  NUM & SG\\
  SPEC & DEF  
 \]\\
 XCOMP &
 	\[PRED & ʻtourner $\langle$SUBJ, OBJ$\rangle$ʼ\\
     SUBJ & [...]\\
     OBJ &
     	\[PRED & ʻlevierʼ\\
         PERS & 3\\
         GEND & M\\
         NUM & SG\\
         SPEC & DEF
        \]\\
    \]\\
\]
\end{avm}\z

Eine wichtige Grundidee in \is{Lexical-Functional Grammar (LFG)}LFG ist, grammatische Funktionen (f-Struk\-tur), wie z.\,B.\ Subjekt und Tempus, vom Aufbau und der Abfolge der Konstituenten (c-Struk\-tur) zu trennen. Die Analyse in \REF{ex:key:7} suggeriert nun, dass sich die beiden Sätze in ihrer funktionalen bzw. hier in ihrer prädikationellen Struktur unterscheiden. Diese Analyse ist jedoch falsch, da in beiden Sätzen ein Futur Perfekt ausgedrückt wird. Der einzige Unterschied in diesen Sätzen besteht darin, dass das Futur Perfekt verschieden kodiert wird. Es gilt also auf der einen Seite, die Gleichheit auf der funktionalen Ebene (Bedeutung) zu modellieren, und auf der anderen Seite die Unterschiede in der Kodierung (Form) dieser Funktion/Bedeutung. Dazu schlagen \citet{ButtNiñoSegond2004} eine f-Struk\-tur \REF{ex:key:8} für die Sätze in \REF{ex:key:5} vor \citep[16]{ButtNiñoSegond2004}:

\ea%8
    \label{ex:key:8}
%AVM 8 
\begin{avm}
\[PRED & ʻturn/tourner $\langle$SUBJ, OBJ$\rangle$ʼ\\
 TENSE & FUTPERF\\
 SUBJ &
 	\[PRED & ʻdriver/conducteurʼ\\
     CASE & NOM\\
     GEND & M\\
     NUM & SG\\
     SPEC & DEF
    \]\\
 OBJ &
 \[PRED & ʻlever/levierʼ\\
  CASE & ACC\\
  GEND & M\\
  NUM & SG\\
  SPEC & DEF
 \]\\
\]
\end{avm}
   \z
   

Die sprachspezifischen Unterschiede in der Form, bei denen es sich hier um morphologische Unterschiede handelt, werden in der m-Struk\-tur kodiert. \REF{ex:key:9} ist die m-Struk\-tur für den englischen Satz, \REF{ex:key:10} die m-Struk\-tur für den französischen Satz (\citealt[18]{ButtNiñoSegond2004}; aus Gründen der Verständlichkeit fallen die m-Struk\-tu\-ren hier ausführlicher aus).

\ea%9
\label{ex:key:9}
\begin{avm}
\[AUX & +\\
 FIN & +\\
 DEP &
 \[AUX & +\\
  FIN & --\\
  VFORM & BASE\\
  DEP &
  	\[AUX & --\\
     FIN & --\\
     VFORM & PERFP
    \]\\
 \]\\
\]
\end{avm}\z


\ea%10
    \label{ex:key:10}
\begin{avm}
\[AUX & +\\
 FIN & +\\
  DEP &
  	\[AUX & --\\
     FIN & --\\
     VFORM & PERFP
    \]\\
\]
\end{avm}
    \z
 

Für die Sätze in \REF{ex:key:5} können wir also festhalten, dass sie die gleichen grammatischen Funktionen aufweisen, was mit einer identischen f-Struk\-tur modelliert werden kann \REF{ex:key:8}. Der Unterschied besteht ausschließlich in der verschiedenen Kodierung der Funktion Futur Perfekt, was durch unterschiedliche m-Struk\-tu\-ren repräsentiert wird (\REF{ex:key:9} und \REF{ex:key:10}).

Durch diese konsequente Trennung von syntaktischer, funktionaler Bedeutung und morphologischer Form können nicht nur Periphrasen adäquat modelliert und sprachübergreifend verglichen werden. Auch Wortformen und Periphrasen, die eine bestimmte Form haben, welche jedoch ihrer Bedeutung widerspricht, stellen kein Problem mehr da. Der englische Satz in \REF{ex:key:5} beispielsweise weist kein Verb auf, das im Futur steht: \textit{will} = Präsens, \textit{have} = Infinitiv, \textit{turned} = Partizip Perfekt. Es stellt sich also die Frage, wie die Periphrase \textit{will have turned} zu ihrer Bedeutung Futur Perfekt kommt. Nimmt man jedoch eine f-Struk\-tur mit der Funktion Futur Perfekt an und eine m-Struk\-tur mit der spezifischen Kombination dieser drei Verben, können diese beiden Strukturen z.\,B.\ mit Funktionalgleichungen miteinander verbunden werden (Projektionen). Ein weiteres Beispiel sind die lateinischen \isi{Deponentia}. Es handelt sich dabei um eine Gruppe von Verben, die eine passive Form, aber eine aktive Bedeutung haben, z.\,B.\ \textit{loquor} (\tabref{table4.2}):

% % % \textbf{\tabref{table4.1}: Flexion des Verbs \textit{laudo}} \textbf{ \citep[74]{SadlerSpencer2001}}\\

\begin{table}
\caption{Flexion des Verbs \textit{laudo} { \citep[74]{SadlerSpencer2001}}}\label{table4.1}
\begin{tabularx}{\textwidth}{XXX}
\lsptoprule
{Imperfektiv} & {Aktiv} & {Passiv}\\\midrule
{Präsens} & laudat & laudatur\\
{Vergangenheit} & laudabat & laudabatur\\
{Futur} & laudabit & laudabitur\\\midrule
{Perfektiv} & {Aktiv} & {Passiv}\\\midrule
{Präsens} & laudavit & laudatus/a/um est\\
{Vergangenheit} & laudaverat & laudatus/a/um erat\\
{Futur} & laudaverit & laudatus/a/um erit\\
\lspbottomrule
\end{tabularx}
\end{table}

% % % \textbf{\tabref{table4.2}: Flexion des \isi{Deponens} \textit{loquor}} \textbf{\citep[75]{SadlerSpencer2001}}\\

\begin{table}
\caption{Flexion des Deponens \textit{loquor} {\citep[75]{SadlerSpencer2001}}}\label{table4.2}
\begin{tabularx}{\textwidth}{XXX}
\lsptoprule
 & {Imperfektiv} & {Perfektiv}\\\midrule
{Präsens} & loquitur & locutus/a/um est\\
{Vergangenheit} & loquebatur & locutus/a/um erat\\
{Futur} & loquar & locutus/a/um erit\\
\lspbottomrule
\end{tabularx}
\end{table}

Das Verb \textit{laudat} (\tabref{table4.1}) und das Deponens-Verb \textit{loquitur} (\tabref{table4.2}) haben für das gleiche f-Feature [VOICE ACTIVE], aber zwei unterschiedliche \is{m-Feature}m-Features: für \textit{laudat} [voice active], für \textit{loquitur} [voice passiv].

Es konnte hier kurz aufgezeigt werden, dass durch die klare Trennung der Komponenten (Funktion, Aufbau und Abfolge von Konstituenten, morphologische Form) Phänomene innerhalb einer Sprache wie auch sprachübergreifende Phänomene adäquat modelliert werden können. Es stellt sich aber nach wie vor die Frage, wie die Verbindung zwischen den morphosyntaktischen Eigenschaften eines flektierten Wortes (= Bedeutung) und seiner Struktur (= Form) aussehen bzw. wie ein flektiertes Wort zu seiner Bedeutung und Form kommt. In allen hier vorgestellten Studien werden die Wortformen innerhalb des Lexikons gebildet, was impliziert, dass die Stämme und \isi{Affixe} im Lexikon gelistet sind. Es wird also von einer le\-xi\-ka\-lisch-in\-kre\-men\-tel\-len Auffassung ausgegangen. Dass diese Auffassung gerade für flektierende Sprache problematisch ist und eine in\-fe\-ren\-tiel\-le-re\-a\-li\-sie\-ren\-de (engl. inferential-realizational) Auffassung mehr leistet, soll im anschließenden Kapitel referiert werden.

\subsection{Lexikalische-inkrementelle vs. in\-fe\-ren\-tiel\-le-re\-a\-li\-sie\-ren\-de Morphologie}\label{4.1.2}

\subsubsection{Definition} Die grundsätzlichen Unterschiede in der morphologischen Theoriebildung kann nach \citeauthor{Stump2001}s \citeyearpar[1-2]{Stump2001} Taxonomie in vier Kategorien eingeteilt werden (lexikalisch vs. inferentiell, inkrementell vs. realisierend), die in der Folge kurz vorgestellt wird. \is{lexikalische Theorie}\textsc{Lexikalische Theorien} gehen davon aus, dass Stämme und Affixe mit ihrer jeweiligen Bedeutung im Lexikon gelistet sind. Im Lexikon stehen also Stämme mit ihren grammatischen und semantischen Bedeutungen, Affixe mit ihren morphosyntaktischen Bedeutungen. In \textsc{inferentiellen Theorien} werden Form und Bedeutung eines Affixes voneinander getrennt. Von Wurzeln\footnote{Genaueres zu Wurzeln und Stämmen vgl. \sectref{4.1.3.3}.} werden Wortformen durch Regeln oder Formeln abgeleitet. Diese Regeln definieren nicht Form und Bedeutung eines Affixes, sondern sie verknüpfen das Auftreten eines Affixes bzw. einer phonetische Modifikation des Wortes mit bestimmten morphosyntaktischen Eigenschaften. In \textsc{inkrementellen Theorien} erhalten Wörter ihre morphosyntaktische Bedeutung, indem sie Affixe mit denselben morphosyntaktischen Eigenschaften zu sich nehmen. Im Gegensatz dazu sind in \is{realisierende Theorie}\textsc{realisierenden Theorien} Wörter mit morphosyntaktischen Eigenschaften verknüpft und diese Verknüpfung lizenziert das Einfügen eines Markers. Alle vier logisch möglichen Kombinationen sind in Theorien zur Flexionsmorphologie zu finden, viele lassen sich aber grundsätzlich entweder der le\-xi\-ka\-li\-schen-in\-kre\-men\-tel\-len oder der in\-fe\-ren\-tiel\-len-re\-a\-li\-sie\-ren\-den Richtung zuordnen. Diese Unterscheidung wird auch morphem- oder wortbasiert genannt und kann wie folgt zusammengefasst werden:

\begin{quote}
[…] what \citet{Stump2001} calls a \textit{lexical-incremental} conception of morphology: such treatments are \textsc{lexical} by virtue of the assumption that affixes, like stems, possess their own separate representations in the lexicon, and they are \textsc{incremental} in that the grammatical properties of a fully inflected word are associated with it only as an effect of its acquiring the morphological markers bearing those properties. \citep[112]{AckermanStump2004}
\end{quote}

\begin{quote}
[…] morphology of the type designated as inferential-realizational in\linebreak Stump’s taxonomy. Recently there has been a resurgence of interest in the so-called Word \& Paradigm approach to morphology […]; what distinguishes this approach from traditional morpheme-based approaches is its premise that language’s inflectional system is \textsc{inferential} rather than lexical (in the sense that it represents inflectional exponents not as lexically listed elements, but as markings licensed by rules by which complex word forms are deduced from simpler roots and stems) and is \textsc{realizational} rather than incremental (in the sense that it treats a word’s association with a particular set of morphosyntactic properties as a precondition for – not a consequence of – the application of the rules licensing the inflectional exponents of those properties). \citep[116]{AckermanStump2004}
\end{quote}

\subsubsection{Vorteile der inferentiellen-realisierenden Theorie} Eine in\-fe\-ren\-tiel\-le-re\-a\-li\-sie\-ren\-de Theorie kann Phänomene in der Flexionsmorphologie genauer und adäquater beschreiben als eine le\-xi\-ka\-li\-sche-in\-kre\-men\-telle. Die wichtigsten Argumente dafür sollen hier \citet{Stump2001} folgend kurz aufgezeigt werden. Für eine \textsc{realisierende Theorie} sprechen vor allem zwei Beobachtungen. Erstens kann eine morphosyntaktische Eigenschaft durch mehrere Affixe ausgedrückt werden: „The morphosyntactic properties associated with an inflected word may exhibit \textsc{extended exponence} in that word’s morphology. That is, a given property may be expressed by more than one morphological marking in the same word“ \citep[4]{Stump2001}. Ein Beispiel dafür aus der Flexion der deutschen Standardsprache ist die Pluralmarkierung in Wörtern wie \textit{Wälder}: Hier wird der Plural sowohl durch das Suffix -\textit{ər} als auch durch den Umlaut kodiert. In inkrementellen Theorien darf eine morphosyntaktische Bedeutung durch maximal einen Affix ausgedrückt werden und Mehrfachausdruck ist zu verhindern: „[b]ecause operations are informationally additive, multiple additions of identical information are precluded“ (\citealt[280]{Steele1995}, zitiert in \citealt[4]{Stump2001}). In \is{realisierende Theorie}realisierenden Theorien existiert diese Einschränkung nicht, denn eine morphosyntaktische Eigenschaft kann das Einfügen eines oder mehrerer Marker verursachen \citep[4]{Stump2001}.

Zweitens kommt auch der umgekehrte Fall vor, d.h., eine morphosyntaktische Eigenschaft wird ohne \isi{Affix} ausgedrückt: „The morphosyntactic properties associated with an inflected word’s individual inflectional markings may underdetermine the properties associated with the word as a whole“ \citep[7]{Stump2001}. In der deutschen Standardsprache kommt dies häufig vor. Beispielsweise wird in der 1. und 3. Person Singular Präteritum die Person nicht markiert: \textit{ging} (1./3. Person Singular Präteritum) vs. \textit{ging}-\textit{st} (2. Person Singular Präteritum). Stark flektierte \isi{Substantive} tragen keine Akkusativmarkierung (\textit{Tag} = Nominativ und Akkusativ), während Nominativ und Akkusativ der schwach flektierten \isi{Substantive} unterschieden werden (\textit{Mensch} = Nominativ, \textit{Mensch}-\textit{en} = Akkusativ). Da in \isi{inkrementellen Theorien} die morphosyntaktischen Eigenschaften einer Wortform von den morphosyntaktischen Eigenschaften der Marker abgeleitet werden, müssen Nullmorphe angenommen werden (bzw. in \isi{inferentiellen Theorien} Regeln, die keine Formveränderungen verursachen) \citep[7–9]{Stump2001}. Für die starke Substantivflexion des Deutschen bedeutet das, dass nicht nur ein Nullmorph für den Akkusativ angenommen werden muss, sondern auch ein anderes Nullmorph für den Dativ, also zwei verschiedene Nullmorpheme. Realisierende Theorien brauchen „nothing so exotic to account for these facts“ \citep[9]{Stump2001}, denn die Marker müssen nicht alle Eigenschaften realisieren. Die \isi{Wurzel} \textit{Tag} ist verknüpft mit der morphosyntaktischen Eigenschaft Genitiv und erst diese Verknüpfung ermöglicht es Regeln, ein Flexionsaffix zu lizenzieren, nämlich -\textit{es}. Gibt es aber für eine bestimmte morphosyntaktische Eigenschaft (z.\,B.\ Akkusativ) keine Regel, die ein Flexionsaffix lizenziert, passiert mit der \isi{Wurzel} nichts.

Es wurde hier gezeigt, dass die \isi{Eins-Zu-Eins-Beziehung} zwischen Form und Bedeutung eines Morphems nicht zutrifft und dass folglich Wörter ihre Bedeutung nicht durch das Aneignen von Morphemen erhalten können. In den zwei besprochenen Beispielen gab es im ersten eine Bedeutung, aber mehrere Formen, im zweiten ebenfalls eine Bedeutung, aber keine Form. Auch die beiden umgekehrten Fälle treten auf, nämlich eine Form/keine Bedeutung und eine Form/ mehrere Bedeutungen \citep[80–82]{Spencer2004}. Für den ersten Fall sind die Fugenelemente in den deutschen Varietäten ein gutes Beispiel. Bei gewissen Komposita muss obligatorisch ein Element eingefügt werden, das aber keine Bedeutung trägt, z.\,B.\ \textit{Universität}-\textit{s}-\textit{zeitung}. Der zweite Fall (eine Form/mehrere Bedeutungen) tritt typischerweise in flektierenden Sprachen auf. In der standarddeutschen Verbflexion trägt das nicht weiter segmentierbare \isi{Suffix} -\textit{st} die Bedeutung 2. Person Singular, das \isi{Suffix} -\textit{t} 2. Person Plural, die Information zu Person und \isi{Numerus} sind also in einem \isi{Suffix} kodiert.

Eine \textsc{inferentielle Theorie} hat\largerpage im Gegensatz zu einer lexikalischen vor allem in zweierlei Hinsicht Vorteile. Erstens wird in \isi{inferentiellen Theorien} die Unterscheidung zwischen konkatenativer und \is{nicht-konkatenative Flexion}nicht-konkatenativer Flexion verworfen. In \is{lexikalische Theorie}lexikalischen Theorien werden diese zwei Arten von Flexion unterschieden, da nur \isi{Affixe} mit ihren morphosyntaktischen Eigenschaften aus dem Lexikon stammen. Erst, nachdem \isi{Wurzel} und \isi{Affix} sich verbunden haben, werden an der \isi{Wurzel} \is{Modifikation}Modifikationen vorgenommen. Als Beispiel nennt \citet{Stump2001} die Analyse des englischen Simple Past von \citet{HalleMarantz1993}, wobei es um die komplementäre Verteilung des Ablauts (z.\,B.\ \textit{sing} – \textit{sang}) und des Default-\isi{Suffixes} -\textit{ed} geht. \citet{HalleMarantz1993} nehmen an, dass \textit{sang} ein Nullsuffix für die Vergangenheit trägt, das mit dem Default-\isi{Suffix} -\textit{ed} in Konkurrenz steht. Des Weiteren löst dieses Nullsuffix eine \is{Modifikation}Modifikation am Wurzelvokal aus. Da das Nullsuffix nun eine kleinere Klasse an Verben subkategorisiert, setzt sich dieses \isi{Suffix} nach \is{P\=aṇinis Prinzip}P\=aṇinis Prinzip durch, wenn es mit dem \isi{Suffix} -\textit{ed} konkurriert. In einer inferentiellen Theorie werden nur zwei Regeln benötigt, die miteinander in Konkurrenz stehen: Eine Regel für den Ablaut (in diesem Bsp. die Ersetzung von \textit{i} durch \textit{a}) und eine für die Suffigierung von -\textit{ed}. Da die Ablautregel beschränkter Anwendung findet als die -\textit{ed}-Regel, setzt sie sich nach \is{P\=aṇinis Prinzip}P\=aṇinis Prinzip durch \citep[10]{Stump2001}. Dass eine Unterscheidung zwischen konkatenativer und \is{nicht-konkatenative Flexion}nicht-konkatenativer Flexion nicht nötig ist, bringt \citet{Stump2001} wie folgt auf den Punkt:

\begin{quote}
[…] but although concatenative and nonconcatenative inflection differ in their phonological expression, there is no convincing basis for assuming that they perform different functions or occupy different positions in the architecture of a language’s morphology […]. Thus, in inferential theories, the morphological rule associated with a given set of morphosytnactic properties may be either affixational or nonconcatenative; the difference between affixational rules and nonconcatenative rules has no theoretical importance. \citep[9]{Stump2001}
\end{quote}

Dies lässt sich auch auf das Deutsche übertragen, z.\,B.\ auf die Pluralmarkierung. Wird der Plural markiert, gibt es dafür drei Möglichkeiten: a) \isi{Suffix} (\textit{Tisch} – \mbox{\textit{Tisch}-\textit{e}}), b) \isi{Umlaut} (\textit{Apfel} – \textit{Äpfel}), c) eine Kombination von beidem (\textit{Wald} – \textit{Wäld}-\textit{ər}). Wie gezeigt wurde, stellen weder unterschiedliche Mechanismen in der Flexion (konkatenativ, nicht-konkatenativ) noch die Kombination dieser Mechanismen und der Mehrfachausdruck derselben Funktion (hier Plural) für eine in\-fe\-ren\-tiel\-le-re\-a\-li\-sie\-ren\-de Theorie ein Problem dar.\largerpage

Eine zweite strittige Unterscheidung, die in \is{lexikalische Theorie}lexikalischen Theorien gemacht wird, ist die Unterscheidung zwischen Bedeutungs- und Kontexteigenschaften eines \isi{Affixes}. Ein \isi{Affix} kann mit zwei Arten von morphosyntaktischen Eigenschaften im Lexikon gelistet sein: a) mit seiner Bedeutung, b) mit Restriktionen seine Subkategorisierung betreffend, d.h., in welchen Kontexten das \isi{Affix} eingesetzt werden darf \citep[10]{Stump2001}. Die deutsche Standardsprache weist im Dativ Plural ein -\textit{n}-\isi{Suffix} auf (\isi{Variation} von -\textit{en}/-\textit{n} ist phonotaktisch erklärbar). In \is{lexikalische Theorie}lexikalischen Theorien muss nun entschieden werden, ob das \isi{Suffix} -\textit{n} die Bedeutung Dativ trägt und eine Beschränkung, nur an Pluralstämme suffigiert zu werden oder ob das \isi{Suffix} -\textit{n} die Bedeutung Dativ Plural hat. Da in \isi{inferentiellen Theorien} Regeln eine Relation zwischen den morphosyntaktischen Eigenschaften eines Wortes und seiner Morphologie herstellen, ist die Unterscheidung zwischen Bedeutungs- und Kontexteigenschaften nicht nötig. Wie diese Regeln aussehen, wird im nachfolgenden \sectref{4.1.3} gezeigt.

Es konnte hier also gezeigt werden, dass ein in\-fe\-ren\-tiel\-ler-re\-a\-li\-sie\-ren\-der Ansatz die Phänomene in der Flexion adäquater beschreiben kann. Traditionell wird in \is{Lexical-Functional Grammar (LFG)}LFG ein le\-xi\-ka\-li\-scher-in\-kre\-men\-tel\-ler Ansatz vertreten (nicht aber z.\,B.\ \citealt{ButtNiñoSegond2004}, vgl. \sectref{4.1.1}). Im folgenden Kapitel soll dargestellt werden, dass ein in\-fe\-ren\-tiel\-ler-realisierender Ansatz durchaus in \is{Lexical-Functional Grammar (LFG)}LFG implementierbar ist.

\subsection{LFG und inferentielle-realisierende Morphologie}\label{4.1.3}

\subsubsection{Content-Paradigm und Form-Paradigm und ihre Relationen}\label{4.1.3.1}

Es wurde gezeigt, dass nicht nur auf der Satzebene Form und Bedeutung auseinandergehalten werden sollen (c-Struk\-tur und f-Struk\-tur), sondern auch auf der Wortebene. Folglich darf auch nur die Bedeutungsseite (und nicht die Formseite) eines Lexems mit der f-Struk\-tur verbunden werden. Am Beispiel periphrastischer und analytischer Formen konstatieren \citet{AckermanStump2004} Folgendes:

\begin{quote}\largerpage
Our claim here, however, is that the syntactic atoms constituting a periphrase may be nothing more than form-theoretic exponents of a unitary content-theoretic element, and that it is this latter element – not its exponents – that determines the periphrase’s f-structure. In particular, we claim that rules of morphology define the (potentially periphrastic) realization of a lexeme’s pairing with a particular set of morphosyntactic properties, and that the association of such a pairing with an f-structure is insensitive to the manner of its realization. \citep[117]{AckermanStump2004}
\end{quote}

Dazu schlagen \citet{AckermanStump2004} ein in\-fe\-ren\-tiel\-les-re\-a\-li\-sie\-ren\-des Modell für die Morphologie vor, welches in \is{Lexical-Functional Grammar (LFG)}LFG implementiert werden kann. Dieses Modell stellt in dieser Arbeit die Grundlage zur Messung morphologischer Komplexität dar und wird nun vorgestellt. Die folgenden Ausführungen beziehen sich auf \citet[117–124]{AckermanStump2004}, wobei die englischen Ausdrücke nicht ins Deutsche übersetzt werden.

Um Bedeutung von Form zu unterscheiden, ist das Lexikon zweigeteilt und besteht aus einem \is{Lexemikon}\textit{Lexemikon} (Bedeutung), das Lexeme mit einer lexikalischen Bedeutung beinhaltet, und einem \textit{Radikon} (Form), das \isi{Wurzeln}, also ausschließlich Formen beinhaltet (schematische Darstellung in \tabref{table4.3}, Beispiele zu Teilparadigmen in \tabref{table4.4}). Jedes Lexem L des \is{Lexemikon}Lexemikons ist mit einem \is{Content-Paradigm}\textit{Content-Paradigm} C-P (L) verbunden. Eine Zelle des \is{Content-Paradigm}\textit{Content-Paradigms} besteht aus einer\pagebreak[4]\largerpage[2] Kombination von L und einem kompletten\footnote{Ein Set an morphosyntaktischen Eigenschaften ist komplett, wenn es alle für die Kategorie/Wortart relevanten morphosyntaktischen Eigenschaften enthält und diese morphosyntaktischen Eigenschaften sich nicht widersprechen (wohlgeformt). Für ein deutsches \isi{Substantiv} ist das Set \{\textsc{nom}, \textsc{sg}, \textsc{masc}\} komplett, aber nicht \{\textsc{nom}, \textsc{sg}\}. Ein komplettes Set muss auch wohlgeformt sein: Das Set \{\textsc{nom}, \textsc{sg}, \textsc{masc}\} ist wohlgeformt, das Set \{\textsc{nom}, \textsc{sg}, \textsc{pl}, \textsc{masc}\} ist es nicht. Genauer beschrieben wird dies in \sectref{4.1.3.2}.} Set an morphosyntaktischen Eigenschaften $\langle$L,$\sigma$$\rangle$. Diese Kombination wird \textit{Content-Cell} genannt. Nur die \textit{Content-Cell} beinhaltet syntaktisch und semantisch interpretierbare Informationen. Das \is{Content-Paradigm}\textit{Content-Paradigm} bildet also die Schnittstelle mit der Syntax (f-Struk\-tur) und Semantik: „A lexeme L’s content paradigm lists the morphosyntactic property sets with which L may be associated in syntax and which determine L’s semantic interpretation in a particular sentential context“ \citep[104]{Stump2016}.

Parallel dazu hat jedes r des Radikons ein \is{Form-Paradigm}\textit{Form-Paradigm} F-P (r), das aus einer Kombination aus r und einem Set an morphosyntaktischen Eigenschaften zusammengesetzt $\langle$r,$\sigma$$\rangle$ ist. Diese Kombination heißt \textit{Form-Cell}. Die \textit{Form-Cells} beinhalten also nicht nur die \isi{Wurzeln} (r), sondern auch die Informationen ($\sigma $), die nötig sind, um die morphologische Realisierung (d.h. die Wortform) herzuleiten. Sie besteht folglich aus jenen morphosyntaktischen Eigenschaften, die morphologisch, also durch die Flexion des Stammes realisiert werden \citep[104]{Stump2016}. Kurz gesagt, hat das \is{Form-Paradigm}\textit{Form-Paradigm} Informationen zur Form, das \is{Content-Paradigm}\textit{Content-Paradigm} Informationen zur Bedeutung.

% % % \textbf{\tabref{table4.3}: Lexikalische Repräsentation \citep[124]{AckermanStump2004}}\\

\begin{table}
\caption{Lexikalische Repräsentation \citep[124]{AckermanStump2004}}\label{table4.3}
\begin{tabularx}{\textwidth}{p{2cm}QQp{2.5cm}} 
\lsptoprule
 & \multicolumn{2}{c}{Lexical representations:} & Realizations (word forms)\\\cmidrule(lr){2-3}
& content-cell \mbox{($\langle$L,$\sigma$$\rangle$ pairings)} & form-cell \mbox{($\langle$r,$\sigma$$\rangle$ pairings)} & \\\midrule
Associations: & \multicolumn{2}{p{6.5cm}}{content-cell associated with form-cells by \textit{rules of paradigm linkage}} &  \cellcolor[gray]{.8} \\\cmidrule(lr){2-3}
& \cellcolor[gray]{.8}  & \multicolumn{2}{p{6cm}}{form-cell associated with realizations by \textit{realization rules}}\\\cmidrule(lr){3-4}
Information represented: & Contentive (functional-se\-man\-tic content, grammatical functions, morphosyntactic properties) & Formal (morphosyntactic property labels); Diacritic/indexical (inflection class membership, root phonology) & Purely phonological\\
\lspbottomrule
\end{tabularx}
\end{table}

Wichtig ist hier jedoch, dass keines dieser Paradigmen Wortformen beinhaltet. Die Wortformen werden von sogenannten \isi{Realisierungsregeln} generiert, die ausschließlich die phonologische Substanz der Wortform definieren. Die \isi{Realisierungsregeln} werden genauer in \sectref{4.1.3.2} eingeführt. Wie die Zellen des \is{Content-Paradigm}\textit{Content-Paradigm} und \is{Form-Paradigm}\textit{Form-Paradigm} sowie die Wortformen realisiert und miteinander verknüpft werden, wird gleich anschließend erklärt. Zur Illustration sollen hier zuerst einige Beispiele vorgestellt werden (\tabref{table4.4}), die zugleich die Vorteile zeigen, wenn ein \is{Content-Paradigm}\textit{Content-Paradigm}, ein \is{Form-Paradigm}\textit{Form-Paradigm} und die Wortformen voneinander getrennt werden.

Für das Verb \textit{machen} ist grundsätzlich festzuhalten, dass jede \textit{Content}{}-\textit{Cell} aus einem Lexem L (\textsc{machen}) und morphosyntaktischen Eigenschaften besteht (z.\,B.\ \textsc{2 sg ind pres act}). Der \textit{Content}{}-\textit{Cell} entspricht eine \textit{Form}{}-\textit{Cell}, die aus der \isi{Wurzel} (\textit{mach}) und ebenfalls morphosyntaktischen Eigenschaften besteht (\textsc{2 sg ind pres act}). Die Wortform, also die Realisierung, lautet \textit{machst}. Die Teilung in \is{Content-Paradigm}\textit{Content-Paradigm}, \is{Form-Paradigm}\textit{Form-Paradigm} und Realisierung scheint hier auf den ersten Blick überflüssig. Die Vorteile zeigen sich in den folgenden Beispielen.

In der Verbflexion der deutschen Standardsprache werden die Formen sowohl synthetisch als auch analytisch gebildet, z.\,B.\ das Präsens synthetisch (\textit{machst}), das Futur analytisch (\textit{wirst} \textit{machen}). Im Italienischen dagegen wird das Futur\pagebreak[4] synthetisch gebildet (\textit{farai}). Für die Bedeutung eines Lexems ist es jedoch unerheblich, wie eine Wortform realisiert wird (vgl. dazu auch die Diskussion in \sectref{4.1.1}). Im Deutschen und Italienischen gibt es also ein Lexem mit der Bedeutung ʻmachenʼ und mit der Bedeutung ʻ2. Singular Indikativ Futur Aktivʼ (\textit{Content}{}-\textit{Paradigm}). Im Deutschen hat dieses Verb die \isi{Wurzel} \textit{mach}, im Italienischen \textit{far} (\textit{Form}{}-\textit{Paradigm}). Realisiert wird das Verb im Deutschen analytisch (\textit{wirst} \textit{machen}), im Italienischen synthetisch (\textit{farai}). Des Weiteren ist hier wichtig, dass sich die Bedeutung ʻFuturʼ nicht aus der Form \textit{wirst machen} ableiten lässt, da weder \textit{wirst} noch \textit{machen} futurische Bedeutung trägt. Vielmehr wird in diesem Modell eine bestimmte Form (synthetisch oder analytisch, \textit{Form-Cell} und Realisierung) mit einer bestimmten Bedeutung (\textit{Content-Cell}) verbunden.

% % % \textbf{\tabref{table4.4}: Beispiele für \isi{Content-Cell}, \isi{Form-Cell} und Realisierung}\\

\begin{table}
\caption{Beispiele für Content-Cell, Form-Cell und Realisierung}\label{table4.4}\small
\resizebox{\textwidth}{!}{\begin{tabular}{lll}
\lsptoprule
{\isi{Content-Cell} <L,$\sigma $>} &  {\isi{Form-Cell} <r,$\sigma $>} & \multicolumn{1}{p{3cm}}{Realisierung \mbox{(Wortformen, rein} phonologisch)}\\
\midrule
\mbox{<\textsc{machen},  \{\textsc{2 sg ind pres act}\}>} & \mbox{<mach, \{\textsc{2 sg ind pres act} \}>} & \textit{machst}\\
\mbox{<\textsc{machen},  \{\textsc{2 sg ind fut act}\}>} & \mbox{<mach,  \{\textsc{2 sg ind fut act } \}>} & \textit{wirst machen}\\
\mbox{<\textsc{fare},    \{\textsc{2 sg ind fut act}\}>} & \mbox{<far,   \{\textsc{2 sg ind fut act } \}>} & \textit{farai}\\
\mbox{<\textsc{sein},    \{\textsc{1 sg ind pres act}\}>} & \mbox{<b,    \{\textsc{1 sg ind pres act} \}>} & \textit{bin}\\
\mbox{<\textsc{sein},    \{\textsc{1 pl ind pres act}\}>} & \mbox{<s,    \{\textsc{1 pl ind pres act} \}>} & \textit{sind}\\
\mbox{<\textsc{laudare}, \{\textsc{3 sg ind pres act}\}>} & \mbox{<laud, \{\textsc{3 sg ind pres act} \}>} & \textit{laudat}\\
\mbox{<\textsc{laudare}, \{\textsc{3 sg ind pres pass}\}>} & \mbox{<laud,\{\textsc{3 sg ind pres pass}\}>} & \textit{laudatur}\\
\mbox{<\textsc{loqui},   \{\textsc{3 sg ind pres act}\}>} & \mbox{<loq,  \{\textsc{3 sg ind pres pass}\}>} & \textit{loquitur}\\
\lspbottomrule
\end{tabular}}
\end{table}

Im Deutschen haben fast alle Verben nur eine \isi{Wurzel} (z.\,B.\ \textit{machen}). Es gibt jedoch auch Verben mit Suppletivstämmen, die also mehr als eine \isi{Wurzel} haben, z.\,B.\ \textit{sein}, \textit{ich} \textit{bin}, \textit{wir} \textit{sind}. Diese \isi{Wurzeln} sind im \textit{Form}{}-\textit{Paradigm} gelistet.

Schließlich kommen wir hier auf die lateinischen \isi{Deponentia} zurück, z.\,B.\ \textit{loqui} (vgl. \tabref{table4.2}, \sectref{4.1.1}). Dabei handelt es sich um Verben, die von der Form her wie ein Passiv aussehen, jedoch eine aktive Bedeutung haben. Die Verben \textit{laudat} und \textit{loquitur} haben dieselbe Bedeutung (3. Singular Indikativ Präsens Aktiv, vgl. \textit{Content}{}-\textit{Cell}), die Form \textit{loquitur} ist jedoch gleich gebildet wie \textit{laudatur}, das eine passive Bedeutung hat (3. Singular Indikativ Präsens Passiv, vgl. \textit{Form}{}-\textit{Cell}). Für das \isi{Deponens} \textit{loqui} lautet $\sigma $ in der \textit{Content}{}-\textit{Cell} also u.a. Aktiv (weil \textit{loquitur} aktive Bedeutung hat), in der \textit{Form}{}-\textit{Cell} jedoch Passiv (weil \textit{loquitur} wie ein Passiv gebildet wird).

Es stellt sich nun die Frage, wie diese Zellen realisiert werden und miteinander verknüpft sind. Zur Veranschaulichung ist dies ebenfalls in \tabref{table4.3} dargestellt. Die Relation zwischen den Zellen des \is{Form-Paradigm}\textit{Form-Paradigm} und der Realisierung dieser Zellen wird durch sogenannte \isi{Realisierungsregeln} (engl. Realization Rules, RR) ausgedrückt, wie diese in in\-fe\-ren\-tiel\-len-re\-a\-li\-sie\-ren\-den Theorien üblich sind (vgl. z.\,B.\ \citealt{Zwicky1985}, \citealt{Anderson1992}). Wie \isi{Realisierungsregeln} genau aussehen, wird in \sectref{4.1.3.2} erklärt. Zur Verdeutlichung soll hier nur ein Beispiel angefügt werden. Der Genitiv Singular \textit{Tag}-\textit{es} hat folgende \isi{Realisierungsregel} (vereinfachte Darstellung): RR\textsubscript{\{\textsc{case:gen}, \textsc{num:sg}, \textsc{gend:masc}\}}, \textsubscript{N}($\langle$X,$\sigma$$\rangle$) = \textsubscript{def} $\langle$X\textit{es}ˊ,$\sigma$$\rangle$.\footnote{Vereinfacht ausgedrückt, besagt diese \isi{Realisierungsregel}, dass einer \isi{Wurzel} -\textit{es} suffigiert wird.} Die Realisierung einer \textit{Form-Cell} $\langle$r,$\sigma$$\rangle$ ist also exakt jene Form, die dadurch definiert ist, dass alle \isi{Realisierungsregeln} auf r angewendet werden, die das Set morphosyntaktischer Einheiten $\sigma $ realisieren.

Auch die Relationen zwischen den Zellen des \is{Content-Paradigm}\textit{Content-Paradigm} und jenen des \is{Form-Paradigm}\textit{Form-Paradigm} sind durch Regeln definiert, aber durch sogenannte \textit{Rules of Paradigm Linkage}. Jeder Zelle des \is{Content-Paradigm}\textit{Content-Paradigm} entspricht eine Zelle des \is{Form-Paradigm}\textit{Form-Paradigm}. Diese Zelle des \is{Form-Paradigm}\textit{Form-Paradigm} wird \is{Form-Correspondent}\textit{Form-Correspondent} (FC) von $\langle$L,$\sigma$$\rangle$ genannt. Die Realisierung einer \textit{Content-Cell} $\langle$L,$\sigma$$\rangle$ besteht also in der Realisierung des FC von $\langle$L,$\sigma$$\rangle$, wobei FC von $\langle$L,$\sigma$$\rangle$ durch die \textit{Rules of Paradigm Linkage} ermittelt wird.

Es wird eine \is{Rule of Paradigm Linkage}\textit{Universal Default Rule of Paradigm Linkage} angenommen: „If root r is stipulated as the primary root of a given lexeme L, then the FC of the content-cell $\langle$L,$\sigma$$\rangle$ is the form-cell $\langle$r,$\sigma$$\rangle$“ \citep[120]{AckermanStump2004}. Davon weichen insbesondere zwei Fälle ab, für die spezifischere \textit{Rules of Paradigm Linkage} definiert werden müssen. Erstens gibt es Lexeme, die zwei \isi{Wurzeln} aufweisen (vgl. \tabref{table4.5} \is{Form-Correspondent}\textit{Form-Correspondent}). Das \is{Content-Paradigm}\textit{Content-Paradigm} besteht aus zwei Zellen: $\langle$L,$\sigma$$\rangle$ und $\langle$L,$\sigma$ˊ$\rangle$. Die Zelle $\langle$L,$\sigma$$\rangle$ hat den FC $\langle$r,$\sigma$$\rangle$ und die Zelle $\langle$L,$\sigma$ˊ$\rangle$ den FC $\langle$rˊ,$\sigma$ˊ$\rangle$, wobei rˊ ${\neq}$ r (d.h. zwei verschiedene \isi{Wurzeln}). Z.B. hat das Nomen HṚD im Sanskrit zwei heteroklitische Stämme: \textit{hṛdaya} und \textit{hṛd}, wobei \textit{hṛdaya} in den direkten und \textit{hṛd} in den indirekten \isi{Kasus} verwendet wird. Sanskrit braucht also eine \is{Rule of Paradigm Linkage}\textit{Rule of Paradigm Linkage}, die den Zellen des \is{Content-Paradigm}\textit{Content-Paradigm} mit direktem \isi{Kasus} einen FC aus dem \is{Form-Paradigm}\textit{Form-Paradigm} mit der \isi{Wurzel} \textit{hṛdaya} zuweist und den Zellen des \is{Content-Paradigm}\textit{Content-Paradigm} mit indirektem \isi{Kasus} einen FC aus dem \is{Form-Paradigm}\textit{Form-Paradigm} mit der \isi{Wurzel} \textit{hṛd}.

% % % \textbf{\tabref{table4.5}: Flexion des Nomens HṚD in Sanskrit \citep[121]{AckermanStump2004}}\\

\begin{table}
\caption{Flexion des Nomens HṚD in Sanskrit \citep[121]{AckermanStump2004}}\label{table4.5}
\begin{tabular}{lll}
\lsptoprule
{Cells in HṚD’s content-paradigm} & {Form-correspondents} & {Realizations}\\\midrule
$\langle$HṚD, \{\textsc{neut nom sg}\}$\rangle$ & $\langle$\textit{hṛdaya}, \{\textsc{neut nom sg}\}$\rangle$ & \textit{hṛdaya–m}\\
$\langle$HṚD, \{\textsc{neut loc sg}\}$\rangle$ & $\langle$\textit{hṛd},    \{\textsc{neut loc sg}\}$\rangle$ &    \textit{hṛd–i}\\
\lspbottomrule
\end{tabular}
\end{table}


Dies kann auf Suppletivstämme übertragen werden. Die Stämme des Verbs \textit{sein} gehen auf unterschiedliche \isi{Wurzeln} zurück, die im \is{Form-Paradigm}\textit{Form-Paradigm} verortet werden können, wie oben dargestellt wurde (vgl. \tabref{table4.4}).

Die zweite Abweichung von der \is{Rule of Paradigm Linkage}\textit{Universal Default Rule of Paradigm Linkage} betrifft jene Fälle, in denen der FC von $\langle$L,$\sigma$$\rangle$ $\langle$r,$\sigma$ˊ$\rangle$ lautet, wobei $\sigma$ $\neq$ $\sigma$ˊ. Ein Beispiel dafür sind die lateinischen \isi{Deponentia}, deren Aktiv (Bedeutung) mit einem passiven Stamm (Form) gebildet wird (vgl. \tabref{table4.2} und \tabref{table4.4}). Diese Verben brauchen also eine \is{Rule of Paradigm Linkage}\textit{Rule of Paradigm Linkage}, die jeder Aktivzelle des \is{Content-Paradigm}\textit{Content-Paradigm} einen FC aus dem passiven \is{Form-Paradigm}\textit{Form-Paradigm} zuweist.

Durch dieses Modell von \citet{AckermanStump2004} wird gewährleistet, dass Form und Bedeutung auf der Wortebene voneinander getrennt werden. Die Verbindungen zwischen Form und Bedeutung sind nicht 1:1 im Lexikon gelistet, d.h., die Form eines Wortes wird nicht direkt mit seiner Bedeutung assoziiert, die Bedeutung eines Wortes ist nicht von seiner Form ableitbar. Vielmehr wird die Relation zwischen Form und Bedeutung durch Regeln abgebildet.

Nun ist noch die Frage nach den Verbindungen dieser morphologischen Komponente des Systems mit der f-Struk\-tur und der c-Struk\-tur zu klären. Die f-Struk\-tur eines Lexems wird ausschließlich von den Informationen im \textit{Content-Pa\-ra\-digm} dieses Lexems projiziert. Die Realisierungen hingegen, die mit einer Zelle des \is{Content-Paradigm}\textit{Content-Paradigm} verknüpft sind, bestimmen nicht die f-Struk\-tur, sondern bilden nur den c-struk\-tu\-rel\-len Ausdruck dieser Zelle. Bezogen auf die Beispiel in \tabref{table4.4} sind es also die \is{Content-Paradigm}\textit{Content-Paradigms} von \textit{machen}, \textit{fare} (2. Singular Indikativ Futur Aktiv) und \textit{loqui} (3. Singular Indikativ Präsens Aktiv), die die f-Struk\-tur projizieren. Die c-Struk\-tur wird jedoch von den Realisierungen gebildet: \textit{wirst} \textit{machen}, \textit{farai} (2. Singular Indikativ Futur Aktiv), \textit{loquitur} (3. Singular Indikativ Präsens Passiv). Damit ist die Darstellung von \citet[117–124]{AckermanStump2004} abgeschlossen.

\subsubsection{Realisierungsregeln und Komplexität}\label{4.1.3.2}

\subsubsubsection{Form der Realisierungsregeln} Es soll nun noch beschrieben werden, wie die Realisierungsregeln aussehen, und zwar basierend auf \citet[40–46 und 50–53]{Stump2001}, was durch eigene Beispiele aus der deutschen Standardsprache ergänzt wird. In den vorherigen Abschnitten wurde gezeigt, dass die Realisierungsregeln flektierte Wörter ableiten. Formal sind Realisierungsregeln Funktionen, wie in \REF{ex:key:11} dargestellt:

\ea%11
    \label{ex:key:11}
    RR \textsubscript{n,$\tau $,C} ($\langle$X,$\sigma$$\rangle$) = \textsubscript{def} $\langle$Yˊ,$\sigma$$\rangle$\\
    \z


RR steht für \isi{Realisierungsregel}. Gefolgt wird eine RR von drei Indizes: Der \isi{Blockindex} n gibt den \isi{Block} an, zu dem die RR gehört; der \isi{Klassenindex} C (Class Index) bezeichnet die Klasse an Lexemen, deren Paradigma die RR definiert; der \isi{Eigenschaften-Set-Index} $\tau $ (Property-Set Index) bestimmt das Set an morphoyntaktischen Eigenschaften, die diese Regel realisiert. Weiter besteht die RR aus drei Variablen. Die Variable $\sigma $ steht für die realisierten morphosyntaktischen Eigenschaften, X für die \isi{Wurzel} des Lexems und Yˊ ist eine phonologische Kette und das Resultat der RR, wobei Yˊ durch weitere RR erweitert werden kann. Z.B. werden für die Wortform \textit{Wäldər} zwei RR benötigt, um den \isi{Umlaut} (\REF{ex:key:12}, \textit{Wäld}) und das \isi{Suffix} -\textit{ər} (\REF{ex:key:13}, \textit{Wäldər}) zu produzieren. Zur Veranschaulichung werden die beiden RR für \textit{Wälder} aufgeführt:

\ea%12
    \label{ex:key:12}
RR \textsubscript{A,} \textsubscript{\{\textsc{num:pl}\},} \textsubscript{N[IC: 1} \textsubscript{${\veebar}$}\textsubscript{ 3} \textsubscript{${\veebar}$}\textsubscript{ 7} \textsubscript{${\veebar}$}\textsubscript{ 8]} ($\langle$X,$\sigma$$\rangle$) = \textsubscript{def} $\langle$Ẍˊ,$\sigma$$\rangle$
    \z



\ea%13
    \label{ex:key:13}
   RR \textsubscript{B,} \textsubscript{\{\textsc{num:pl}\},} \textsubscript{N[IC: 3]} ($\langle$X,$\sigma$$\rangle$) = \textsubscript{def} $\langle$X\textit{ər}ˊ,$\sigma$$\rangle$
    \z

\subsubsubsection{Bedingungen} Bevor die RR weiter referiert wird, muss noch ausgeführt werden, unter welchen Bedingungen eine RR definiert ist. Diese sind in der Bedingung {\textit{Rule-Argument Coherence}} \REF{ex:key:14} zusammengefasst:

\ea%14
    \label{ex:key:14}
\isi{Rule-argument coherence}:\\
RR\textsubscript{n,}\textsubscript{$\tau $}\textsubscript{,C}($\langle$X,$\sigma$$\rangle$) is defined iff (a) $\sigma $ is an extension of $\tau $; (b) \isi{L-index}(X) ${\in}$ C; and (c) $\sigma $ is   a \isi{well-formed} set of morphosyntactic properties for \isi{L-index}(X). \citep[45]{Stump2001}
    \glt
    \z
      


Diese Bedingung besteht also aus drei Bedingungen, nämlich \isi{Extension} und \isi{Wohlgeformtheit}, die die morphosyntaktischen Eigenschaften betreffen, und \textsc{L-index}, womit die Verbindung zwischen einer \isi{Wurzel} und dem Lexem beschrieben wird. Wenn phonologisch identische \isi{Wurzeln} unterschiedlichen Lexemen entsprechen, müssen \isi{Wurzel} und Lexem mit einem Index miteinander verbunden werden. Z.B. haben die Verben MALEN\textsubscript{1} und MAHLEN\textsubscript{2} identische \isi{Wurzeln}, ihre Paradigmen weisen jedoch einen wesentlichen Unterschied auf: Das Lexem MALEN\textsubscript{1} bildet sein Partizip Perfekt schwach (\textit{gemalt}), das Lexem MAHLEN\textsubscript{2} jedoch stark (\textit{gemahlen}). Findet nun eine RR Anwendung auf eine \isi{Wurzel}, so erbt Yˊ den Index von X \REF{ex:key:15}:

\ea%15
    \label{ex:key:15}
\isi{Persistence of L-indexing}:\\
For any realization rule RR\textsubscript{n,}\textsubscript{$\tau $}\textsubscript{,C}, if RR \textsubscript{n,}\textsubscript{$\tau $}\textsubscript{,C} ($\langle$X,$\sigma$$\rangle$) = \textsubscript{def} $\langle$Yˊ,$\sigma$$\rangle$, then \isi{L-index}(Yˊ) = \isi{L-index}(X). \citep[45]{Stump2001}
    \z

Die Bedingungen der \isi{Wohlgeformtheit} und der \isi{Extension} betreffen die morphosyntaktischen Eigenschaften. Da die \isi{Unifikation} eng mit der \isi{Wohlgeformtheit} und der \isi{Extension} zusammenhängt, wir auch diese hier beschrieben. \textsc{Wohlgeformtheit} ist wie folgt definiert \REF{ex:key:16}, wobei Feature mit F und Value mit v abgekürzt wird:

\ea%16
    \label{ex:key:16}
A set $\tau $ of morphosyntactic properties for a Lexeme of category C is WELL-FORMED in some language [l] only if $\tau $ satisfies the following conditions in [l]:\\
\begin{itemize}
\item[a.] For each property F:v ${\in}$ $\tau $, F:v is available to lexemes of category C and v is a permissible value for F.
\item[b.] For any morphosyntactic feature F having v\textsubscript{1}, v\textsubscript{2} as permissible values, if v\textsubscript{1} ${\neq}$ v\textsubscript{2} and   F:v\textsubscript{1} ${\in}$ $\tau $, then F:v\textsubscript{2} ${\notin}$ $\tau $. \citep[41]{Stump2001}
\end{itemize}
    \glt
    \z

Die Bedingung a. verhindert, dass z.\,B.\ ein Verb die Eigenschaft \{\textsc{case:nom}\} und dass das Feature \{\textsc{case}\} den Wert \{\textsc{sg}\} bekommt. Die Bedingung b. stellt sicher, dass ein Set an morphosyntaktischen Eigenschaften keine sich widersprechende Werte enthält. Z.B. kann ein flektiertes Wort nicht gleichzeitig im Singular und Plural stehen. Ein Set \{\textsc{nom, sg, pl, masc}\} ist also nicht wohlgeformt.

Um die Bedingung der \textsc{Extension} zu verstehen, sind noch zwei Arten morphosyntaktischer Features zu unterscheiden: \textit{atom-valued} und \textit{set-valued}. Ein morphosyntaktisches Feature ist \textit{atom-valued}, wenn seine Werte nicht weiter analysiert werden können, z.\,B.\ \{\textsc{case:nom}\}. Von einem \textit{set-valued} morphosyntaktischen Feature spricht man, wenn der Wert des Features selbst ein Set an morphosyntaktischen Einheiten ist, z.\,B.\ \{\textsc{agr:}\{\textsc{pers:1}, \textsc{num:sg}\}\}. Die \isi{Extension} wird in \REF{ex:key:17} definiert:

\ea%17
    \label{ex:key:17}
Where $\sigma $ and $\tau $ are \isi{well-formed} sets of morphosyntactic properties, $\sigma $ is an EXTENSION of $\tau $ iff (i) for any atom-valued feature F and any permissible value v for F, if F:v ${\in}$ $\tau $, then F:v ${\in}$ $\sigma $; and (ii) for any set-valued feature F and any permissible value $\rho $ for F, if F:$\rho $ ${\in}$ $\tau $, then F:$\rho $ˊ ${\in}$ $\sigma $, where $\rho $ˊ is an extension of $\rho $. \citep[41]{Stump2001}
    \z

Ein Beispiel für ein \textit{set-valued} Feature ist in \citet[41]{Stump2001} zu finden. Da die Nominalphrase des Deutschen nur \textit{atom-valued} Features aufweist, wird das Konzept der \isi{Extension} anhand einer möglichen Nominalphrase veranschaulicht: \{\textsc{case:nom}, \textsc{num:sg}, \textsc{gend:f}\} ist eine \isi{Extension} der folgenden Sets an morphosyntaktischen Eigenschaften:

\ea%18
    \label{ex:key:18}
\ea \label{ex:key:18a} \{\textsc{case:nom, num:sg, gend:fem}\}
\ex \label{ex:key:18b} \{\textsc{case:nom, num:sg}\}          
\ex \label{ex:key:18c} \{\textsc{num:sg, gend: fem }\}        
\ex \label{ex:key:18d} \{\textsc{case:nom, gend:fem}\}        
\ex \label{ex:key:18e} \{\textsc{case:nom}\}                  
\ex \label{ex:key:18f} \{\textsc{num:sg}\}                    
\ex \label{ex:key:18g} \{\textsc{gend:fem}\}                  
    \z
    \z
   


\textsc{Unifikation} ist ein zentrales Konzept in \is{Lexical-Functional Grammar (LFG)}LFG, aber auch in anderen Feature-basierten Modellen \citep[17]{Falk2001}. Bezogen auf die RR wird \isi{Unifikation} wie folgt definiert \REF{ex:key:19}:

\ea%19
    \label{ex:key:19}
Where $\sigma $ and $\tau $ are \isi{well-formed} sets of morphosyntactic properties, the UNIFICATION $\rho $ of $\sigma $ and $\tau $ is the smallest \isi{well-formed} set of morphosyntactic properties such that $\rho $ is an extension of both $\sigma $ and $\tau $. \citep[41]{Stump2001}
    \z

Beispielsweise ist die \isi{Unifikation} von \{\textsc{case:nom}, \textsc{num:sg}\} und \{\textsc{gend:fem}\} definiert als \{\textsc{case:nom}, \textsc{num:sg}, \textsc{gend:fem}\}, wohingegen die \isi{Unifikation} von \{\textsc{case:\linebreak nom}, \textsc{num:sg}\} und \{\textsc{num:pl}\} nicht definiert ist. Die Idee der \isi{Unifikation} ist von großer Bedeutung, wenn mehrere RR angewendet werden. Z.B. das Wort \textit{Wäldərn} (Dativ Plural) wird von drei RR gebildet, nämlich von den Regeln in \REF{ex:key:12} und \REF{ex:key:13}, die hier wiederholt werden, sowie von der Regel in \REF{ex:key:20}.\footnote{-\textit{ən}/-\textit{n}-\isi{Variation} ist phonotaktisch bedingt und muss deswegen in der Morphologie nicht berücksichtigt werden.} Die \isi{Unifikation} der Sets an morphosyntaktischen Eigenschaften der Regeln \REF{ex:key:12}, \REF{ex:key:13} und \REF{ex:key:20} lautet \{\textsc{case:dat}, \textsc{num:pl}\}.

\begin{exe}
\exr{ex:key:12} RR \textsubscript{A,} \textsubscript{\{\textsc{num:pl}\},} \textsubscript{N[IC: 1} \textsubscript{${\veebar}$}\textsubscript{ 3} \textsubscript{${\veebar}$}\textsubscript{ 7} \textsubscript{${\veebar}$}\textsubscript{ 8]} ($\langle$X,$\sigma$$\rangle$) = \textsubscript{def} $\langle$Ẍˊ,$\sigma$$\rangle$
\end{exe}

\begin{exe}
\exr{ex:key:13}
RR \textsubscript{B,} \textsubscript{\{\textsc{num:pl}\},} \textsubscript{N[IC: 3]} ($\langle$X,$\sigma$$\rangle$) = \textsubscript{def} $\langle$X\textit{ər}ˊ,$\sigma$$\rangle$
\end{exe}

\ea%20
    \label{ex:key:20}
  RR \textsubscript{C,} \textsubscript{\{\textsc{case:dat}, \textsc{num:pl}\},} \textsubscript{N[IC: 1} \textsubscript{${\veebar}$}\textsubscript{ 2} \textsubscript{${\veebar}$}\textsubscript{ 3} \textsubscript{${\veebar}$}\textsubscript{ 4} \textsubscript{${\veebar}$}\textsubscript{ 5} \textsubscript{${\veebar}$}\textsubscript{ 6} \textsubscript{${\veebar}$}\textsubscript{ 7} \textsubscript{${\veebar}$}\textsubscript{ 8]} ($\langle$X,$\sigma$$\rangle$) = \textsubscript{def} $\langle$X\textit{ən}ˊ,$\sigma$$\rangle$
\z

\subsubsubsection{Blöcke} Nach diesen Ausführungen zu den Bedingungen der RR, sollen nun die drei Indizes (n, $\tau $, C) der RR näher erörtert werden. Zur Erinnerung ist die Form einer RR in \REF{ex:key:11} wiederholt.

\begin{exe}
 \exr{ex:key:11} RR \textsubscript{n,$\tau $,C} ($\langle$X,$\sigma$$\rangle$) = \textsubscript{def} $\langle$Yˊ,$\sigma$$\rangle$
\end{exe}
 
Basierend auf \citet{Anderson1992} geht auch \citet{Stump2001} davon aus, dass die RR in \isi{Blöcken} organisiert sind. Die \isi{Blöcke} bestimmen, in welcher Reihenfolge die RR angewendet werden, wozu jeder \isi{Block} einen Index A–Z bekommt: Z.B. werden zuerst die Regeln des \isi{Blocks} A angewendet, dann jene des \isi{Blocks} B etc. Innerhalb eines \isi{Blocks} konkurrieren die RRs folglich um die gleiche Position in der Abfolge der Regeln (RRs unterschiedlicher \isi{Blöcke} konkurrieren nicht miteinander). Für die Substantivflexion der deutschen Standardsprache sind folgende \isi{Blöcke} denkbar: \isi{Block} A beinhaltet die Regeln zur \is{Modifikation}Modifikation der \isi{Wurzel}, \isi{Block} B die Regeln zum Plural, \isi{Block} C die Regeln zum \isi{Kasus} (vgl. Regeln \REF{ex:key:12}, \REF{ex:key:13} und \REF{ex:key:20}). Da die deutsche Standardsprache unterschiedliche Plural-\isi{Suffixe} aufweist, befinden sich alle RRs dieser Pluralsuffixe in \isi{Block} C und konkurrieren miteinander. Die Regeln für die Pluralsuffixe konkurrieren jedoch nicht mit den Regeln für den \isi{Umlaut}, da diese nicht im selben \isi{Block} sind. Dadurch wird der Tatsache Rechnung getragen, dass der Plural nur durch den \isi{Umlaut} oder nur durch ein \isi{Suffix} oder durch eine Kombination von beiden ausgedrückt werden kann, aber nie durch zwei \isi{Umlaute} oder zwei \isi{Suffixe}.

Zur Konkurrenz der RR stellen sich nun noch zwei Fragen: a) Was passiert, wenn mehrere Regeln auf die gleichen morphosyntaktischen Eigenschaften zutreffen? b) Was passiert, wenn keine Regel aus einem bestimmten \isi{Block} angewendet wird?

Frage a) kann mit \textsc{\is{P\=aṇinis Prinzip}P\=aṇinis Prinzip} beantwortet werden, das besagt, dass immer zuerst die engste oder spezifischste Regel für ein bestimmtes Set an morphosyntaktischen Einheiten angewendet wird. Z.B. weist der Dativ Singular der starken Adjektivflexion in der deutschen Standardsprache zwei \isi{Suffixe} auf: -\textit{əm} im Maskulin und im Neutrum, -\textit{ər} im Feminin. Dazu können zwei RR -- nämlich (\ref{ex:key:21}) und (\ref{ex:key:22}) -- formuliert werden:

\ea%21
    \label{ex:key:21}
 RR \textsubscript{A, \{\textsc{case:dat}, \textsc{num:sg}\}, \textsc{adj[strong]}} ($\langle$X,$\sigma$$\rangle$) = \textsubscript{def} $\langle$X\textit{əm}ˊ,$\sigma$$\rangle$
\z
 
\ea%22
    \label{ex:key:22}
RR \textsubscript{A, \{\textsc{case:dat}, \textsc{num:sg}, \textsc{gend:fem}\}, \textsc{adj[strong]}} ($\langle$X,$\sigma$$\rangle$) = \textsubscript{def} $\langle$X\textit{ər}ˊ,$\sigma$$\rangle$
\z

\ea%23
    \label{ex:key:23}
RR \textsubscript{A, \{\textsc{case:acc}, \textsc{num:sg}, \textsc{gend:masc}\}, \textsc{adj[]}} ($\langle$X,$\sigma$$\rangle$) = \textsubscript{def} $\langle$X\textit{ən}ˊ,$\sigma$$\rangle$
\z

Da die RR \REF{ex:key:22} spezifischer ist als die RR \REF{ex:key:21}, wird sie zuerst angewendet und blockiert die Anwendung einer weiteren RR (z.\,B.\ der RR \REF{ex:key:21}) aus demselben \isi{Block}. Somit wird für den Dativ Singular Feminin die RR \REF{ex:key:22} verwendet und danach die RR \REF{ex:key:21} für die übrig gebliebenen Positionen des Dativs Singular, und zwar für das Maskulin und Neutrum. Dieses Beispiel zeigt auch, dass die RRs im Gegensatz zum \is{Content-Paradigm}\textit{Content-Paradigm} unterspezifiziert bleiben können. Im \is{Content-Paradigm}\textit{Content-Paradigm} muss das Set an morphosyntaktischen Eigenschaften komplett sein, d.h. z.\,B.\ für ein \isi{Adjektiv}, dass die Features \isi{Kasus}, \isi{Numerus} und \isi{Genus} spezifiziert sein müssen. In den RRs ist dies nicht nötig, hier können Features auch unterspezifiziert bleiben. In den RRs trifft dies nicht nur auf die morphosyntaktischen Eigenschaften zu, sondern auch auf die Wortklasse. Z.B. wird im Akkusativ Singular Maskulin der starken und schwachen Adjektivflexion -\textit{ən} suffigiert \REF{ex:key:23}. Im Gegensatz zu den RRs in \REF{ex:key:21} und \REF{ex:key:22} muss folglich die Art der Flexion nicht angegeben werden, die eckige Klammer nach ADJ bleibt also leer. \citet{Stump2001} definiert die spezifischste RR, wie in \REF{ex:key:24} wiedergegeben wird. Dabei bezieht sich a. auf die morphosyntaktischen Eigenschaften und b. auf die Wortklasse. Die Anwendung einer RR ist in \REF{ex:key:25} definiert:

\ea%24
    \label{ex:key:24}
\begin{itemize}
\item[a.] RR\textsubscript{n,}\textsubscript{$\sigma $}\textsubscript{,C} is NARROWER than RR\textsubscript{n,}\textsubscript{$\tau $}\textsubscript{,C} iff $\sigma $ is an extension of $\tau $ and \mbox{$\sigma $ ${\neq}$ $\tau $}.
\item[b.] Where C ${\neq}$ Cˊ, RR \textsubscript{n,}\textsubscript{$\sigma $}\textsubscript{,C} is NARROWER than RR\textsubscript{n,}\textsubscript{$\tau $}\textsubscript{,C} iff C ${\subseteq}$ Cˊ. \citep[52]{Stump2001}
\end{itemize}
\z


\ea%25
    \label{ex:key:25}
RR\textsubscript{n,}\textsubscript{$\tau $}\textsubscript{,C} is APPLICABLE TO $\langle$X,$\sigma$$\rangle$ iff RR\textsubscript{n,}\textsubscript{$\tau $}\textsubscript{,C} ($\langle$X,$\sigma$$\rangle$) is defined (according to [(24)]. \citep[52]{Stump2001}
\z

Die Frage b), was geschieht, wenn keine Regel aus einem bestimmten \isi{Block} angewendet wird, kann dadurch beantwortet werden, dass die Wortform nicht verändert wird. Beispielsweise zeichnet sich die \isi{Flexionsklasse} 2\footnote{Zur Definition und Einteilung der \isi{Flexionsklassen} siehe \sectref{5.1.1.}.} der standarddeutschen \isi{Substantive} dadurch aus, dass sie keinen \isi{Umlaut} aufweist (z.\,B.\ \textit{Tag} – \textit{Tagə}). Ist nun die \isi{Wurzel} \textit{tag} mit den morphosyntaktischen Eigenschaften \{\textsc{case:nom}, \textsc{num:pl}\} verbunden, findet sich in \isi{Block} B keine RR für die \isi{Flexionsklasse} 2 (vgl. RR in \ref{ex:key:12}), folglich passiert mit der \isi{Wurzel} nichts. \citet{Stump2001} nimmt für diese Fälle die RR {\textit{Identity Function Default}} an:

\ea%26
    \label{ex:key:26}
\isi{Identity Function Default} […]\\
RR\textsubscript{n,\{\},U} ($\langle$X,$\sigma$$\rangle$) = \textsubscript{def} ($\langle$X,$\sigma$$\rangle$). \citep[53]{Stump2001}
\z

Sie betrifft alle \isi{Blocks} und alle Lexeme einer Sprache und die morphosyntaktischen Eigenschaften sind maximal unterspezifiziert, d.h. ø. Diese RR definiert, dass die \isi{Wurzel} nicht verändert wird. Es handelt sich dabei also um die allgemeinste RR, die nach \is{P\=aṇinis Prinzip}P\=aṇinis Prinzip immer dann angewendet wird, wenn sonst in einem \isi{Block} keine RR für eine bestimmte morphosyntaktische Eigenschaft vorhanden ist.

\subsubsubsection{Realisierungsregeln und Komplexität} Damit möchte ich die Darstellung der RRs basierend auf \citet[40–46 und 50–53]{Stump2001} schließen. Bis hier wurde gezeigt, dass ein Lexem aus einem \is{Content-Paradigm}\textit{Content-Paradigm}, einem \is{Form-Paradigm}\textit{Form-Paradigm} und einer realisierten Wortform besteht. Die Relationen zwischen \is{Content-Paradigm}\textit{Content-Paradigm} und \is{Form-Paradigm}\textit{Form-Paradigm} werden durch \textit{Rules of Paradigm Linkage} ausgedrückt, die Relationen zwischen \is{Form-Paradigm}\textit{Form-Paradigm} und der Realisierung durch RRs (vgl. \tabref{table4.3} in \sectref{4.1.3.1}). Da für die \is{Modifikation}Modifikation einer Wurzel zu einem flektierten Wort die RRs zuständig sind, kann folglich die Flexionsmorphologie im System der RRs verortet werden. Daraus folgt, dass, wenn man die Komplexität der Flexionsmorphologie berechnen möchte, die RRs den Ort der Messung bilden. In \sectref{3.1.2} wurde erörtert, dass die absolute Komplexität eines linguistischen Phänomens durch die Länge der Beschreibung dieses Phänomens gemessen werden kann \citep[24]{Miestamo2008}. Je kürzer diese Beschreibung ist, desto weniger komplex ist das linguistische Phänomen. Übertragen auf RRs heißt das Folgendes: Da die RRs die Beschreibung der Flexionsmorphologie darstellen, ist jenes Flexionssystem weniger komplex, das weniger RRs aufweist. Es wird hier also mit einem formalen Instrument die Komplexität der Flexionsmorphologie gemessen. Dies und besonders die Vorteile dieser Methode werden in \sectref{4.3.2} ausgeführt.

\subsubsection{Variation, Synkretismus und Wurzelalternation}\label{4.1.3.3}

Drei Fragen müssen noch beantwortet werden: 1. Wie wird (freie) \isi{Variation} modelliert?; 2. Wie werden \isi{Synkretismen} durch die RRs erfasst?; 3. Wie werden Stämme von \isi{Wurzeln} abgeleitet und wie wird die Verteilung der Stämme erfasst?

\subsubsubsection{Variation} Bis jetzt wurden nur Beispiele gezeigt, in denen keine Variation vorkommt, d.h., dass eine Zelle eines Paradigmas nur eine Wortform enthält. Jedoch gerade die Nicht-Stan\-dard\-va\-ri\-e\-tä\-ten des in dieser Arbeit untersuchten Samples weisen immer wieder Variation auf. \tabref{table4.6} zeigt die Flexion des bestimmten Artikels in Jaun (Höchstalemannisch). Hier sind drei Arten von Variation zu unterscheiden: 1. phonologisch bedingte Variation, 2. syntaktisch bedingte Variation, 3. freie Variation. Die phonologisch bedingte Variation betrifft die Formen \textit{əm} und \textit{m} im Dativ Singular Maskulin, welche nur nach einer Präposition vorkommen. Endet die Präposition auf einen Vokal, folgt die Form \textit{m}, endet sie auf einen Konsonanten, folgt \textit{əm}. Die Variation ist also rein phonologisch, d.h., das Subsystem Phonologie steuert diese Variation und diese braucht deswegen hier nicht weiter berücksichtigt zu werden. Das Paradigma (vgl. \tabref{table4.6}) enthält also nur die Formen \textit{dəm} und \textit{əm}. Syntaktisch bedingt sind alle Varianten dieses Paradigmas. Die Artikel \textit{ə}\slash\textit{əm} (Akkusativ/Dativ) treten nach einer Präposition auf, die Artikel \textit{dər}\slash\textit{dəm} (Akkusativ/Dativ), wenn ihnen keine Präposition vorangeht. Der Artikel \textit{di} wird vor einem Adjektiv verwendet, der Artikel \textit{t} vor einem Nomen. Auch wenn die Verteilung syntaktisch gesteuert ist, müssen die unterschiedlichen Formen, die die Syntax benutzt, von irgendwo herkommen, und zwar von der Morphologie. Freie Variation findet sich im Dativ Singular Maskulin: Geht dem Artikel keine Präposition voraus, kann sowohl \textit{dəm} als auch \textit{əm} verwendet werden. Sowohl die syntaktisch bedingte als auch die freie Variation sind innerhalb der Morphologie als freie Variation anzusehen, d.h., sie müssen durch RRs produziert werden.

\begin{table}
\caption{Flexion des bestimmten Artikels in Jaun \citep[282]{Stucki1917}}\label{table4.6}
\resizebox{\textwidth}{!}{\begin{tabular}{lllllll} 
\lsptoprule
&  &  & {\NOM} & {\AKK} & {\DAT} & {\GEN}\\
\hline
\multirow{2}{4em}{{\textsc{sg}}} & \multirow{2}{4em}{\textsc{m}} & Artikel & \multirow{2}{4em}{dər} & dər & dəm, əm & \multirow{2}{4em}{ts}\\
&  & \mbox{P + Artikel} &  & ə & əm, (\textit{m}) & \\
\hline
\multirow{2}{4em}{{\textsc{sg}}} & \multirow{2}{4em}{\textsc{n}} & Artikel & \multirow{2}{4em}{ts} & \multirow{2}{4em}{ts} & dəm, əm & \multirow{2}{4em}{ts}\\
&  & \mbox{P + Artikel} &  &  & əm, (\textit{m}) & \\
\hline
\multirow{2}{4em}{{\textsc{sg}}} & \multirow{2}{4em}{\textsc{f}} & \mbox{Artikel + Adj} & di & di & \multirow{2}{4em}{dər} & \multirow{2}{4em}{dər}\\
&  & \mbox{Artikel + N} & t & t &  & \\
\hline
\multirow{2}{4em}{{\textsc{pl}}} &  & \mbox{Artikel + Adj} & di & di & \multirow{2}{4em}{də} & \multirow{2}{4em}{dər}\\
 &  & \mbox{Artikel + N} & t & t &  & \\
\lspbottomrule
\end{tabular}}
\end{table}

Diese Tatsache muss das System an RRs abbilden. Dies bedeutet, dass es in einem \isi{Block} für dieselbe Wortklasse zwei RRs mit denselben morphosyntaktischen Eigenschaften gibt:

\ea%27
    \label{ex:key:27}
RR \textsubscript{A, \{\textsc{case:dat}, \textsc{num:sg}, \textsc{gend:masc}} \textsubscript{${\veebar}$}\textsubscript{ \textsc{neut}\}, \textsc{art[def]}} ($\langle$X,$\sigma$$\rangle$) = \textsubscript{def} $\langle$X\textit{dəm}ˊ,$\sigma$$\rangle$
\z

\ea%28
    \label{ex:key:28}
RR \textsubscript{A, \{\textsc{case:dat}, \textsc{num:sg}, \textsc{gend:masc}} \textsubscript{${\veebar}$}\textsubscript{ \textsc{neut} \}, \textsc{art[def]}} ($\langle$X,$\sigma$$\rangle$) = \textsubscript{def} $\langle$X\textit{əm}ˊ,$\sigma$$\rangle$
    \z


Genau dies ist aber laut \citet{Stump2001} nicht möglich, denn in jedem \isi{Block} gibt es eine RR, die am spezifischsten ist. Die spezifischste RR ist nicht definiert, „if block \textit{n} contained two or more applicable rules no one of which was narrower than all the others“ \citep[53]{Stump2001}. Des Weiteren war keine Stelle in \citet{Stump2001} zu finden, in der \isi{Variation} modelliert wird. Die Frage, die sich hier auftut, ist eine viel grundsätzlichere: Wenn zwei Varianten vorhanden sind, bilden diese den Output einer oder zwei Grammatiken? Traditionell wird in der generativen Grammatik davon ausgegangen, dass \isi{Variation} und Optionalität auf die Präsenz unterschiedlicher Grammatiken zurückgeführt werden kann \citep[385]{Seiler2004}, d.h. also, interne \isi{Variation} und Diglossie werden gleich behandelt \citep[384]{Seiler2004}. Dass dies aus prinzipiellen und empirischen Gründen nur wenig Sinn macht, dafür argumentiert \citet{Seiler2004}, wobei die für diese Arbeit wichtigsten Gründe zitiert werden:

\begin{quote}
First, diglossia and grammar-internal variation are sociolinguistically different facts, and we need a formal instrument in order to account for this difference. Second, we need a ʻmeta-grammarʼ telling us the right arrangement of the two co-present grammars. Third, if two co-present grammars are needed in order to account for one variable feature (i.e., one feature with two options), the number of parallel grammars exponentially increases with every variable feature – which is a highly undesired result. \citep[384]{Seiler2004}
\end{quote}


Geht man davon aus, dass die Artikel \textit{dəm} und \textit{əm} die Outputs von zwei verschiedenen Grammatiken darstellen, bräuchte es für jede dieser Varianten wie auch für jede weitere Variante ein zusätzliches Paradigma. Dies würde zu einer sehr hohen Zahl an Paradigmen führen. Die Morphologie kann ökonomischer, d.h., mit weniger Mitteln modelliert werden, wenn in einer Grammatik Varianten erlaubt sind.

Dazu wird hier eine Modifikation von \citeauthor{Stump2001}s \citeyearpar{Stump2001} Regeln vorgeschlagen, die die Anwendung der spezifischsten RR betrifft. Die Definition der spezifischsten RR \REF{ex:key:24} und ihrer Anwendung \REF{ex:key:25} nach \citet{Stump2001} sind hier wiederholt:

\begin{exe}
\exr{ex:key:24}
\begin{itemize}
\item[a.] RR\textsubscript{n,}\textsubscript{$\sigma $}\textsubscript{,C} is NARROWER than RR\textsubscript{n,}\textsubscript{$\tau $}\textsubscript{,C} iff $\sigma $ is an extension of $\tau $ and \mbox{$\sigma $ ${\neq}$ $\tau $}.
\item[b.] Where C ${\neq}$ Cˊ, RR \textsubscript{n,}\textsubscript{$\sigma $}\textsubscript{,C} is NARROWER than RR\textsubscript{n,}\textsubscript{$\tau $}\textsubscript{,C} iff C ${\subseteq}$ Cˊ. \citep[52]{Stump2001}
\end{itemize}
\z

\begin{exe}
\exr{ex:key:25}
RR\textsubscript{n,}\textsubscript{$\tau $}\textsubscript{,C} is APPLICABLE TO $\langle$X,$\sigma$$\rangle$ iff RR\textsubscript{n,}\textsubscript{$\tau $}\textsubscript{,C} ($\langle$X,$\sigma$$\rangle$) is defined (according to [(24)]. \citep[52]{Stump2001}
\z

\noindent        
\REF{ex:key:25} impliziert, dass immer nur eine RR appliziert werden kann. Diese Beschränkung muss also dahingehend erweitert werden, dass alle RRs, die am spezifischsten sind (also der Definition in \REF{ex:key:24} entsprechen), angewendet werden, egal ob es sich dabei um eine oder mehrere RRs handelt. In \REF{ex:key:29} wird also eine Erweiterung von \REF{ex:key:25} vorgeschlagen:

\ea%29
    \label{ex:key:29}
Every RR\textsubscript{n,}\textsubscript{$\tau $}\textsubscript{,C} is APPLICABLE to $\langle$X,$\sigma$$\rangle$ iff RR\textsubscript{n,}\textsubscript{$\tau $}\textsubscript{,C} ($\langle$X,$\sigma$$\rangle$) is defined (according to \ref{ex:key:24}).

    \z
\noindent
Durch diese Erweiterung können sowohl die RR (27) für \textit{dəm} als auch die RR (28) für \textit{əm} angewendet werden und füllen bzw. definieren die Zelle Dativ Singular Maskulin und Neutrum des \isi{bestimmten Artikels}.

\subsubsubsection{Synkretismus} 
Eine weitere Frage ist, wie mit \isi{Synkretismen} umgegangen wird. \citet{Stump2001} unterscheidet vier Arten von \isi{Synkretismen}, wobei diese mit verschiedenen Typen von Regeln definiert werden, was kurz vorgestellt wird. Anschließend wird gezeigt, weshalb dies zur Komplexitätsmessung dieses Samples nicht übernommen werden kann. Dafür wird ein Vorschlag gemacht, wie die \isi{Synkretismen} in den Varietäten dieses Samples einheitlich erfasst und gemessen werden können.\\

\noindent
\textbf{\isi{Synkretismus} bei \citet{Stump2001}:} \citet{Stump2001} unterscheidet vier Arten von \isi{Synkretismen}, die teils u.a. an \citeauthor{Zwicky1985}s \citeyearpar{Zwicky1985} Unterscheidung zwischen systematischem und zufälligem \isi{Synkretismus} erinnern. Stumps vier Arten von \isi{Synkretismen} und ihre Typen von Regeln sind in \tabref{table4.7} zusammengefasst.

% % \textbf{\tabref{table4.7}: Typen von \isi{Synkretismen} basierend auf \citet[212–217]{Stump2001}}\\

\begin{table}
\caption{Typen von Synkretismen basierend auf \citet[212–217]{Stump2001}}\label{table4.7}
\resizebox{\textwidth}{!}{\begin{tabular}{lllll}
\lsptoprule
& \multicolumn{4}{c}{Synkretismus}\\
\midrule
& \multicolumn{2}{c}{direktional}  & \multicolumn{2}{c}{nicht direktional} \\\cmidrule(lr){2-3}\cmidrule(lr){4-5}
& \multicolumn{1}{c}{unidirektional} & \multicolumn{1}{c}{bidirektional} & \multicolumn{1}{c}{symmetrisch} & \multicolumn{1}{c}{\mbox{nicht stipuliert}}\\
\midrule
\multicolumn{1}{l}{Abhängigkeit} & \multicolumn{1}{c}{A=Determ B=Dep} & \multicolumn{1}{c}{A=Determ B=Dep;} & \multicolumn{2}{c}{ –} \\
\multicolumn{1}{l}{} & \multicolumn{1}{c}{}& \multicolumn{1}{c}{B=Determ A=Dep} & \multicolumn{1}{c} {}& \multicolumn{1}{c}{} \\
\midrule
\multicolumn{1}{l}{teilen natürl.}  & \multicolumn{2}{c}{unwichtig} & \multicolumn{1}{c}{–}  & \multicolumn{1}{c}{+} \\
\multicolumn{1}{l}{Klasse} & \multicolumn{2}{c}{} & \multicolumn{1}{c}{}  & \multicolumn{1}{c}{} \\
\midrule
\multicolumn{1}{l}{Regeltyp} & \multicolumn{2}{c}{rules of referral} & \multicolumn{1}{c}{metarule} & \multicolumn{1}{c}{RR} \\
\lspbottomrule
\end{tabular}}
\end{table}

Ein direktionaler \isi{Synkretismus} ist dann gegeben, wenn eine Form nach dem Vorbild einer anderen gebildet wird, wobei ersterer das abhängige (\textit{dependent}) Mitglied und letzterer das determinierende (\textit{determinant}) Mitglied des Synkretismuspaares bilden \citep[213]{Stump2001}. \citet{Stump2001} illustriert dies am folgenden Beispiel. Im Bulgarischen weisen die Verben der 3. Person Singular unabhängig vom Tempus ein -\textit{e} als \isi{Suffix} auf. In der 2. Person Singular wird ebenfalls ein -\textit{e} suffigiert, jedoch nur in den präteritalen Tempora, d.h. im Imperfekt und Aorist. Die 3. Person Singular bildet also das determinierende Mitglied des Synkretismuspaares, die 2. Person Singular das abhängige Mitglied \citep[212–213]{Stump2001}. Direktionale \isi{Synkretismen} werden durch eine \textit{Rule of Referral} definiert. In \REF{ex:key:30} ist die \textit{Rule of Referral} für das bulgarische Beispiel aufgeführt:

\ea%30
    \label{ex:key:30}
Rule of referral:\\
Where \textit{n} is any of rule \isi{Blocks} \textbf{A} to \textbf{D},\\
RR\textsubscript{n,←\{PRET:yes, \textsc{agr:}\{\textsc{pers:}2, \textsc{num:sg}\}\}→, V}($\langle$X,$\sigma$$\rangle$) = \textsubscript{def} $\langle$Y,$\sigma$$\rangle$, where\\
Nar\textsubscript{n}($\langle$X,$\sigma $/\{\textsc{agr:}\{\textsc{pers:3}\}\}$\rangle$) = $\langle$Y,$\sigma $/\{\textsc{agr:}\{\textsc{pers:3}\}\}$\rangle$. \citep[218]{Stump2001}
    \z

Bei dieser Regel handelt es sich um einen unidirektionalen \isi{Synkretismus}, d.h. für dieses Beispiel, dass immer die 2. Person Singular nach der 3. Person Singular gebildet wird und nicht umgekehrt. Bei einem bidirektionalen \isi{Synkretismus} kommen beide Richtungen vor. Z.B. im Rumänischen (\tabref{table4.8}) weisen die Verben aller Konjugationsklasse mit Ausnahme der ersten Konjugationsklasse einen \isi{Synkretismus} zwischen der 1. Person Singular und der 3. Person Plural auf. Dabei handelt es sich um einen bidirektionalen \isi{Synkretismus}. Im Paradigma von \textit{umplea} bildet die 3. Person Plural die abhängige und die 1. Person Singular die determinierende Form, da das \isi{Suffix} -\textit{u} in der ersten Konjugationsklasse nur in der 1. Person Singular vorkommt. Im Paradigma des Verbs \textit{fi} gilt der umgekehrte Fall: Die 1. Person Singular ist die abhängige Form, da sie auf der Basis des Pluralstamms gebildet wird \citep[213]{Stump2001}. Der bidirektionale \isi{Synkretismus} wird ebenfalls mit einer \textit{Rule of Referral} ausgedrückt, die komplexer ausfällt als die Regel in \REF{ex:key:30} (vgl. \citealt[219]{Stump2001}). Für die Belange dieser Arbeit braucht sie aber nicht weiter ausgeführt zu werden.

% % \textbf{\tabref{table4.8}: Präsens Indikativ rumänischer Verben aus (\citealt[214]{Stump2001}, hier gekürzt)}\\

\begin{table}
\caption{Präsens Indikativ rumänischer Verben aus (\citealt[214]{Stump2001}, hier gekürzt)}\label{table4.8}
\begin{tabularx}{\textwidth}{lXXXX} 
\lsptoprule
 & \textit{a invita}  & \textit{a sufla}  & \textit{a umplea}  & \textit{a fi} \\
 & einladen & atmen & füllen & sein\\\midrule
Konjugationsklasse & 1 & 1 & 2 & 4\\\midrule
\textsc{1.sg} & invít & \textbf{súfl-u} & \textbf{úmpl-u} & \textbf{sínt}\\
\textsc{2.sg} & invíţ-i & súfl-i & úmpl-i & éşt-i\\
\textsc{3.sg} & \textbf{invít-ă} & \textbf{súfl-ă} & úmpl-e & ést-e\\
\textsc{1.pl} & invitắ-m & suflắ-m & úmple-m & \textbf{sínte}-m\\
\textsc{2.pl} & invitá-ţi & suflá-ţi & úmple-ţi & \textbf{sínte}-ţi\\
\textsc{3.pl} & \textbf{invít-ă} & \textbf{súfl-ă} & \textbf{úmpl-u} & \textbf{sínt}\\
\lspbottomrule
\end{tabularx}
\end{table}

Bei den nicht direktionalen \isi{Synkretismen} können keine determinierenden und abhängigen Formen ausgemacht werden, es gibt keine Default-Form. Sie werden weiter in nicht-stipulierte und symmetrische \isi{Synkretismen} eingeteilt. Als Beispiel für den nicht-stipulierten \isi{Synkretismus} nennt \citet{Stump2001} die 3. Person Singular und Plural der ersten Konjugationsklasse im Präsens Indikativ des Rumänischen (vgl. \tabref{table4.8}). Erstens bildet das Set an morphosyntaktischen Einheiten eine natürlich Klasse, nämlich 3. Person Präsens Indikativ \citep[213]{Stump2001}. Zweitens entstehen die beiden Formen durch eine Regel, die -\textit{ă} in der 3. Person suffigiert, die aber für \isi{Numerus} unterspezifiziert ist \citep[213–215]{Stump2001}. Es handelt sich also um eine Arte „poverty in the system of realization rules […] the system happens not to have any rule that is sensitive to number in the inflection of third-person present indicative forms of first-conjugation verbs“ \citep[215]{Stump2001}. Diese Art von \isi{Synkretismus} wird also durch eine für \isi{Numerus} unterspezifizierte RR wie in \REF{ex:key:11} definiert.

Der symmetrische \isi{Synkretismus} zeichnet sich schließlich dadurch aus, dass weder eine Richtung ausgemacht werden kann, noch bilden die Sets an morphosyntaktischen Eigenschaften eine natürliche Klasse \citep[216]{Stump2001}. \citet{Stump2001} veranschaulicht dies am Beispiel der 2. Person Singular und der 1. Person Plural der Verben in Hua (Sprache von Neu Guinea). Die beiden Formen fallen unabhängig vom Modus immer zusammen, es ist aber weder eine Richtung zu identifizieren, noch formen die Sets an morphosyntaktischen Eigenschaften eine natürliche Klasse \citep[216–217]{Stump2001}. Der symmetrische \isi{Synkretismus} kann also nicht durch eine \textit{Rule of Referral} und auch nicht durch eine RR nach \REF{ex:key:11} ausgedrückt werden. Um den symmetrischen \isi{Synkretismus} zu definieren, schlägt Stump die \textit{Symmetrical Syncretism Metarule} vor \citep[222]{Stump2001}. Auf das diskutierte Beispiel bezogen, sieht diese Regel wie folgt aus:

\ea%31
    \label{ex:key:31}
Symmetrical Syncretism Metarule\\
Where $\tau $ is an extension of \{\textsc{agr}(su):\{\textsc{per:2}, \textsc{num:sg}\}\},\\
RR\textsubscript{II,}\textsubscript{$\tau $}\textsubscript{,V}($\langle$X,$\sigma$$\rangle$) = \textsubscript{def} $\langle$Y,$\sigma$$\rangle$\\
$\updownarrow $\\
RR\textsubscript{II,}\textsubscript{$\tau $}\textsubscript{/\{\textsc{agr}(su):\{\textsc{per:1}, \textsc{num:pl}\}\}, V}($\langle$X,$\sigma$$\rangle$) = \textsubscript{def} $\langle$Y,$\sigma$$\rangle$. \citep[223]{Stump2001}
\z

Wie die \textit{Rule of Referral} muss auch die \textit{Symmetrical Syncretism Metarule} nicht näher besprochen werden. Für beide Regeln ist aber festzustellen, dass sie von der RR wie in \REF{ex:key:11} abweichen, folglich einen anderen Typ von Regeln darstellen und dass sie komplexer sind als die RRs. Möchte man also mit diesen Regeln die Komplexität der Flexion messen, stellt sich sehr schnell das Problem der Vergleichbarkeit. Können verschiedene Typen von Regeln miteinander verrechnet werden, und wenn ja, wie? Wenn, wie bereits konstatiert wurde, die \textit{Rules of Referral} und die \textit{Symmetrical Syncretism Metarules} in ihrer Form deutlich komplexer sind als die RRs, wie viel komplexer als die RRs sind sie? Sind die \textit{Rules of Referral} und die \textit{Symmetrical Syncretism Metarules} gleich komplex? In den hier vorgestellten Beispielen fallen immer zwei Zellen des Paradigmas zusammen. Je mehr Zellen jedoch zusammenfallen, desto umfangreicher werden die \textit{Rules of Referral} und die \textit{Symmetrical Syncretism Metarules}. Bildet eine komplexere Regel auch einen komplexeren Sachverhalt ab? Es ist kontraintuitiv zu sagen, dass, wenn z.\,B.\ drei von vier Zellen eines Paradigmas zusammenfallen, dies komplexer ist, als wenn zwei von vier Zellen zusammenfallen. Es zeigt sich hier, dass, obwohl Stumps Analyse sehr reizvoll erscheint, sie zur Komplexitätsmessung nicht verwendet werden kann. Zur Komplexitätsmessung ist es also fundamental wichtig, ein einheitliches Messinstrument zu benutzen, damit die Objektivität, Reliabilität und besonders die Validität gewährleistet werden können.

Im vorangehenden Abschnitt \sectref{4.1.3.2} wurde argumentiert, dass die Komplexität durch die Anzahl RRs gemessen werden kann. Würde man dies erweitern, folglich durch unterschiedliche Regeln (RR, \textit{Rules of Referral} etc.) die Komplexität messen und würde man annehmen, dass alle Regeln unabhängig von ihrem Typ gleich komplex sind, wäre ein \isi{Synkretismus} genauso komplex wie wenn zwei unterschiedliche \isi{Suffixe} vorkommen würden. Man vergleiche dazu die starke Adjektivflexion in der deutschen Standardsprache (\tabref{table4.12} unten). Das Maskulin unterscheiden zwei Formen für den Nominativ (-\textit{ər}) und für den Akkusativ (-\textit{ən}). Das Neutrum weist für die beiden \isi{Kasus} ein \isi{Suffix} auf (-\textit{əs}). Für das Maskulin bräuchte man also zwei RRs, für das Neutrum eine RR für den Nominativ und eine Synkretismusregel für den Akkusativ. Damit würden Maskulin und Neutrum die gleiche Komplexität aufweisen, obwohl das Maskulin zwei unterschiedliche \isi{Suffixe} hat und das Neutrum nur eines. Es muss also eine Messmethode gefunden werden, die genau diese Beobachtung abbildet.\\

\noindent
\textbf{\isi{Synkretismus} in dieser Arbeit:} Da zur Komplexitätsmessung, wie bereits argumentiert wurde, eine einheitliche Methode verwendet werden muss, soll die RR so erweitert werden, dass sie auch \isi{Synkretismen} abbilden kann. Dazu wird hier das Einführen eines Junktors vorgeschlagen, nämlich der ausschließenden Disjunktion (auch Kontravalenz genannt). Es handelt sich dabei um die Entweder-Oder-Verbindung, d.h., die ausschließenden Disjunktion ist wahr, wenn entweder die eine Aussage oder die andere wahr ist, aber nicht, wenn beide Aussagen wahr oder falsch sind \citep[38]{Zoglauer2008}. Als Operator wird in dieser Arbeit das Symbol ${\veebar}$ verwendet. Dadurch können der Nominativ und Akkusativ Maskulin und Neutrum der starken Adjektivflexion adäquat abgebildet werden, d.h. zwei RRs für das Maskulin (-\textit{ər}, -\textit{ən}) (\ref{ex:key:32}–\ref{ex:key:33}) und eine RR für das Neutrum (-\textit{əs}) (\ref{ex:key:34}):

\ea%32
    \label{ex:key:32}
RR \textsubscript{A, \{\textsc{case:nom}, \textsc{num:sg}\}, \textsc{adj[strong]}} ($\langle$X,$\sigma$$\rangle$) = \textsubscript{def} $\langle$X\textit{ər}ˊ,$\sigma$$\rangle$
\z         
\ea%33
    \label{ex:key:33}
RR \textsubscript{A, \{\textsc{case:acc}, \textsc{num:sg}\}, \textsc{adj[strong]}} ($\langle$X,$\sigma$$\rangle$) = \textsubscript{def} $\langle$X\textit{ən}ˊ,$\sigma$$\rangle$
\z
\ea%34
    \label{ex:key:34}
 RR \textsubscript{A, \{\textsc{case:nom}} \textsubscript{${\veebar}$}\textsubscript{ \textsc{acc}, \textsc{num:sg}\}, \textsc{adj[strong]}} ($\langle$X,$\sigma$$\rangle$) = \textsubscript{def} $\langle$X\textit{əs}ˊ,$\sigma$$\rangle$
\z

Es stellt sich nun die Frage, ob das vorgestellte Modell von \citet{AckermanStump2004} mit der Einführung einer ausschließenden Disjunktion vereinbart werden kann. \citet{AckermanStump2004} plädieren dafür, Form und Bedeutung klar voneinander zu trennen. Wie bereits erörtert wurde, stellen dann beispielsweise lateinische \isi{Deponentia} kein Problem mehr dar (\tabref{table4.9}). Auch wenn sie eine passivische Form haben (FC und Realisierung), weisen sie eine aktivische Bedeutung auf (C-P). Genauso problemlos fügen sich \isi{Synkretismen} in dieses Modell ein (\tabref{table4.10}): Auf der Bedeutungsebene macht es einen Unterschied, ob ein Wort im Nominativ oder Akkusativ steht (C-P), auf der Ebene der Form hingegen können eben diese Formen auch zusammenfallen (FC und Realisierung). Es soll hier kurz daran erinnert werden, dass nur von den Informationen des C-P aus die f-Struk\-tur projiziert wird, während die Realisierungen lediglich den c-struk\-tu\-rel\-len Ausdruck bilden.

%\textbf{\tabref{table4.9}: Flexion der lateinischen Verben \textit{fat\=er\=\i}} \textbf{und \textit{mon\=ere}, \citet[122]{AckermanStump2004}}\\

\begin{table}
\caption{Flexion der lateinischen Verben \textit{fat\=er\=\i} und \textit{mon\=ere}, \citet[122]{AckermanStump2004}}\label{table4.9}
\resizebox{\textwidth}{!}{\begin{tabular}{lll}
\lsptoprule
{\mbox{Cells in content-paradigm}} & {Form-correspondents} & {Realizations}\\\midrule
\mbox{<\textsc{fat\=er\=i}, \{\textsc{1 sg pres act indic}\}>} & <fat,  \{\textsc{1 sg pres pass indic}\}> & \textit{fateor}\\
\mbox{\textsc{<mon\=ere},    \{\textsc{1 sg pres pass indic}\}>} & <mon, \{\textsc{1 sg pres pass indic}\}> & \textit{moneor}\\
\lspbottomrule
\end{tabular}}
\end{table}

%\textbf{\tabref{table4.10}: C-P, FC und Realisierung des Nominativs und Akkusativs Maskulin und Neutrum}\\

\begin{table}
\caption{C-P, FC und Realisierung des Nominativs und Akkusativs Maskulin und Neutrum}\label{table4.10}
\begin{tabular}{lll}
\lsptoprule
{Cells in content-paradigm} & {Form-correspondents} & {Realizations}\\
\midrule
<\textsc{schön}, \{\textsc{nom sg m}\}> & <schön, \{\textsc{nom sg m}\}> & \textit{schöner}\\
\midrule
<\textsc{schön}, \{\textsc{acc sg m}\}> & <schön, \{\textsc{acc sg m}\}> & \textit{schönen}\\
\midrule
<\textsc{schön}, \{\textsc{nom sg n}\}> & \multirow{2}{*}{\mbox{<schön, \{\textsc{nom} {\tiny ${\veebar}$} \textsc{acc sg n}\}>}} & \multirow{2}{*}{\textit{schönes}}\\\cmidrule(r){1-1}
<\textsc{schön}, \{\textsc{acc sg n}\}> &  & \\
\lspbottomrule
\end{tabular}
\end{table}

Die Herausforderung liegt vielmehr in der Vereinbarkeit mit der Bedingung der \isi{Wohlgeformtheit}. Es wurde gezeigt, dass die RRs nur unter bestimmten Bedingungen definiert sind: \textit{Persistence of L-indexing} \REF{ex:key:15}, \isi{Wohlgeformtheit} \REF{ex:key:16}, \isi{Extension} \REF{ex:key:17} und \isi{Unifikation} \REF{ex:key:19}. Die Bedingung der \isi{Wohlgeformtheit} ist hier in gekürzter Version wiederholt:

\begin{exe}%16
    \exr{ex:key:16}
A set $\tau $ of morphosyntactic properties for a Lexeme of category C is WELL-FORMED in some language [l] only if $\tau $ satisfies the following conditions in [l]:\\
\begin{itemize}
\item[b.] For any morphosyntactic feature F having v\textsubscript{1}, v\textsubscript{2} as permissible values, if v\textsubscript{1} ${\neq}$ v\textsubscript{2} and   F:v\textsubscript{1} ${\in}$ $\tau $, then F:v\textsubscript{2} ${\notin}$ $\tau $. \citep[41]{Stump2001}
\end{itemize}\end{exe}

Auf den ersten Blick scheint also eine RR wie in \REF{ex:key:34} nicht wohlgeformt zu sein. Nimmt man aber die Trennung von Bedeutung und Form ernst, ist die vorgeschlagene Erweiterung der RRs durch eine ausschließende Disjunktion mit der Bedingung der \isi{Wohlgeformtheit} kompatibel. Erstens bezieht sich die \isi{Wohlgeformtheit} primär auf Lexeme, d.h. auf das C-P, von dem aus die f-Struk\-tur projiziert wird. Die Bedingung der \isi{Wohlgeformtheit} ist folglich besonders wichtig, denn sie verhindert, dass im Satz eine Konstituente z.\,B.\ gleichzeitig im Nominativ und im Akkusativ steht, was einen Satz uninterpretierbar macht. Die RR dagegen stellt nur eine Form zur Verfügung. Zweitens lässt die ausschließende Disjunktion offen, ob es sich z.\,B.\ um einen Nominativ oder Akkusativ handelt. Sie beschränkt nur, dass eine bestimmte Form entweder im Nominativ oder im Akkusativ, aber in keinem anderen \isi{Kasus} (z.\,B.\ Dativ) stehen kann. Ob eine bestimmte Form im Satz schließlich im Nominativ oder Akkusativ steht, hängt von der Valenz des Verbs, von der f-Struk\-tur und dem C-P ab. Einfach ausgedrückt auf die RR \REF{ex:key:34} übertragen heißt das, dass die Form \textit{schönəs} verwendet werden kann, wenn in einem Satz ein Nominativ oder ein Akkusativ, aber nicht wenn ein Dativ gebraucht wird.

In der Folge werden zwei weitere wichtige Vorteile der Implementierung der ausschließenden Disjunktion vorgestellt. Erstens wurde bezüglich der \textit{Rules of Referral} und der \textit{Symmetrical Syncretism Metarules} darauf hingewiesen, dass diese Typen von Regeln umfangreicher werden, je mehr Zellen eines Paradigmas zusammenfallen. Dies impliziert, dass der Zusammenfall von drei Zellen ein System komplexer macht als der Zusammenfall von zwei Zellen. Dies ist kontraintuitiv, denn je mehr Zellen zusammenfallen, desto weniger unterschiedliche Formen weist ein Paradigma auf und desto kürzer sollte also die Beschreibung des Systems ausfallen. Dieses Problem kann vermieden werden, indem \isi{Synkretismen} mit RRs abgebildet werden, welche eine ausschließende Disjunktion zulassen. Als Beispiel dient die \isi{Flexionsklasse} 4 der standarddeutschen Substantivflexion (\tabref{table4.11}). Im Singular fallen Akkusativ, Dativ und Genitiv zusammen, was mit einer RR definiert werden kann:

\ea%35
    \label{ex:key:35}
RR \textsubscript{C, \{\textsc{case:acc}} \textsubscript{${\veebar}$ \textsc{dat} ${\veebar}$ \textsc{gen}}\textsubscript{, \textsc{num:sg}\}, N[IC:4]} ($\langle$X,$\sigma$$\rangle$) = \textsubscript{def} $\langle$X\textit{ən}ˊ,$\sigma$$\rangle$
\z


Es ist also egal, ob von vier Formen zwei oder drei zusammenfallen. In beiden Fällen kann derselbe Typ von RR verwendet werden, womit die einzelnen RRs miteinander vergleichbar sind und zur Komplexitätsmessung miteinander verrechnet werden können. Auf die gleiche Art und Weise können auch Genussynkretismen abgebildet werden. Fallen Singular und Plural zusammen, bleibt der \isi{Numerus} unterspezifiziert.

Zweitens ist zu beobachten, dass dasselbe \isi{Suffix} über die \isi{Flexionsklassen} hinweg wiederverwendet wird. Beispielsweise wird in der Substantivflexion der\linebreak deutschen Standardsprache -\textit{əs} als Genitiv Singular und -\textit{ən} als Dativ Plural in mehreren \isi{Flexionsklassen} benutzt (\tabref{table4.11}). Angenommen jede \isi{Flexionsklasse} weist im Dativ Plural ein anderes \isi{Suffix} auf, würde die Beschreibung jenes Systems länger ausfallen als die Beschreibung des Systems der deutschen Standardsprache. Auch dies kann durch die folgende RR adäquat abgebildet werden, welche bereits weiter oben eingeführt wurde:

\begin{exe}%20
    \exr{ex:key:20}
RR \textsubscript{C,} \textsubscript{\{\textsc{case:dat}, \textsc{num:pl}\},} \textsubscript{N[IC: 1} \textsubscript{${\veebar}$}\textsubscript{ 2} \textsubscript{${\veebar}$}\textsubscript{ 3} \textsubscript{${\veebar}$}\textsubscript{ 4} \textsubscript{${\veebar}$}\textsubscript{ 5} \textsubscript{${\veebar}$}\textsubscript{ 6} \textsubscript{${\veebar}$}\textsubscript{ 7} \textsubscript{${\veebar}$}\textsubscript{ 8]} ($\langle$X,$\sigma$$\rangle$) = \textsubscript{def} $\langle$X\textit{ən}ˊ,$\sigma$$\rangle$
\end{exe}

% % \textbf{\tabref{table4.11}: Flexion der \isi{Substantive} in der deutschen Standardsprache basierend auf \citet[158–167]{Eisenberg2006}}\\

\begin{table}
\caption{Flexion der Substantive in der deutschen Standardsprache basierend auf \citet[158–167]{Eisenberg2006}}\label{table4.11}
\resizebox{\textwidth}{!}{\begin{tabular}{r*{8}{l}} 
\lsptoprule
& \multicolumn{4}{c}{Singular}  & \multicolumn{4}{c}{Plural} \\\cmidrule(lr){2-5}\cmidrule(lr){6-9}
{FK} & {\NOM} & {\AKK} & {\DAT} & {\GEN} & {\NOM} & {\AKK} & {\DAT} & {\GEN}\\\midrule
1 & gast & gast & gast & \textbf{gast-əs} & gäst-ə & gäst-ə & \textbf{gäst-ən} & gäst-ə\\
2 & tag & tag & tag & \textbf{tag-əs} & tag-ə & tag-ə & \textbf{tag-ən} & tag-ə\\
3 & wald & wald & wald & \textbf{wald-əs} & wäld-ər & wäld-ər & \textbf{wäld-ər-n} & wäld-ər\\
4 & matrosə & \textbf{matrosə-n} & \textbf{matrosə-n} & \textbf{matrosə-n} & matrosə-n & matrosə-n & \textbf{matrosə-n} & matrosə-n\\
5 & staat & staat & staat & \textbf{staat-s} & staat-ən & staat-ən & \textbf{staat-ən} & staat-ən\\
6 & blumə & blumə & blumə & blumə & blumə-n & blumə-n & \textbf{blumə-n} & blumə-n\\
7 & stadt & stadt & stadt & stadt & städt-ə & städt-ə & \textbf{städt-ən} & städt-ə\\
8 & muttər & muttər & muttər & muttər & müttər & müttər & \textbf{müttər-n} & müttər\\
9 & zoo & zoo & zoo & \textbf{zoo-s} & zoo-s & zoo-s & zoo-s & zoo-s\\
10 & pizza & pizza & pizza & pizza & pizza-s & pizza-s & pizza-s & pizza-s\\
\lspbottomrule
\end{tabular}}
\end{table}

% % \textbf{\tabref{table4.12}: Flexion der starken \isi{Adjektive} in der deutschen Standardsprache basierend auf \citet[178]{Eisenberg2006}}\\

\begin{table}
\caption{Flexion der starken Adjektive in der deutschen Standardsprache basierend auf \citet[178]{Eisenberg2006}}\label{table4.12}
\begin{tabular}{*{6}{l}}
\lsptoprule
&  & {\NOM} & {\AKK} & {\DAT} & {\GEN}\\\midrule
{\textsc{sg}} & \textsc{m} & \textbf{–ər} & \textbf{–ən} & –əm & –ən\\
& \textsc{n} & \textbf{–əs} & \textbf{–əs} & –əm & –ən\\
& \textsc{f} & –ə & –ə & \textbf{–ər} & \textbf{–ər}\\
\textsc{pl} &  & –ə & –ə & –ən & \textbf{–ər}\\
\lspbottomrule
\end{tabular}
\end{table}

Die \isi{Synkretismen} betreffend bleibt noch ein letzter Fall zu besprechen. In der starken Adjektivflexion der deutschen Standardsprache (\tabref{table4.12}) weisen der Dativ und der Genitiv Feminin Singular sowie der Genitiv Plural dieselbe Form auf, nämlich -\textit{ər}. Man könnte nun auf die Idee kommen, das \isi{Suffix} -\textit{ər} der drei Zellen durch die RR \REF{ex:key:36} zu definieren. Dabei kommen jedoch falsche Formen heraus, denn die RR \REF{ex:key:36} stellt die Formen für die folgenden Bündel an syntaktischen Eigenschaften zur Verfügung: Dativ Singular Feminin, Genitiv Singular Feminin, Dativ Plural Feminin, Genitiv Plural Feminin. Die RR \REF{ex:key:36} produziert also fälschlicherweise ein -\textit{ər} für den Dativ Plural und ausschließlich feminine Formen für den Dativ und Genitiv Plural, obwohl im Plural \isi{Genus} nicht unterschieden wird.

\ea[*]{%36
    \label{ex:key:36}
 RR \textsubscript{A, \{\textsc{case:dat}} \textsubscript{${\veebar}$}\textsubscript{ \textsc{gen}, \textsc{num:sg}} \textsubscript{${\veebar}$ PL,}\footnote{Der \isi{Numerus} könnte unterspezifiziert bleiben. Zur besseren Verständlichkeit werden Singular und Plural explizit genannt.} \textsubscript{\textsc{gend:fem}\}, \textsc{adj[strong]}} ($\langle$X,$\sigma$$\rangle$) = \textsubscript{def} $\langle$X\textit{ər}ˊ,$\sigma$$\rangle$}
\z

\noindent
Es werden hier also zwei RRs benötigt, nämlich eine für den Dativ und den Genitiv Singular Feminin \REF{ex:key:37} und eine für den Genitiv Plural \REF{ex:key:38}, wobei \isi{Genus} unterspezifiziert ist.

\ea%37
    \label{ex:key:37}
RR \textsubscript{A, \{\textsc{case:dat}} \textsubscript{${\veebar}$}\textsubscript{ \textsc{gen}, \textsc{num:sg}}\textsubscript{,} \textsubscript{\textsc{gend:fem}\}, \textsc{adj[strong]}} ($\langle$X,$\sigma$$\rangle$) = \textsubscript{def} $\langle$X\textit{ər}ˊ,$\sigma$$\rangle$
\z

\ea%38
    \label{ex:key:38}
 RR \textsubscript{A, \{\textsc{case:gen}, \textsc{num:pl}\}, \textsc{adj[strong]}} ($\langle$X,$\sigma$$\rangle$) = \textsubscript{def} $\langle$X\textit{ər}ˊ,$\sigma$$\rangle$
\z
\ea%39
    \label{ex:key:39}
RR \textsubscript{A, \{\textsc{case:dat}, \textsc{num:sg}, \textsc{gend}: \textsc{masc}} \textsubscript{${\veebar}$ \textsc{neut}}\textsubscript{\}, \textsc{adj[strong]}} ($\langle$X,$\sigma$$\rangle$) = \textsubscript{def} $\langle$X\textit{əm}ˊ,$\sigma$$\rangle$
\z

Auf den ersten Blick erscheint diese Lösung etwas ungünstig. Mit \citeauthor{Stump2001}s \citeyearpar{Stump2001} Analyse bräuchte es jedoch für diesen Fall auch zwei Regeln, und zwar eine RR für das \isi{Suffix} -\textit{ər} und eine \textit{Symmetrical Syncretism Metarule} für die synkretisierten Formen. Die RRs \REF{ex:key:37} und \REF{ex:key:38} machen das System also nicht komplexer als Stumps Analyse. Dafür stellen sich die erörterten Probleme der Vergleichbarkeit bei verschiedenen Regeltypen nicht, wie sie \citet{Stump2001} vorschlägt. Außerdem handelt es sich hier eventuell doch um einen komplexeren \isi{Synkretismus} im Vergleich z.\,B.\ zum Dativ Singular Maskulin und Neutrum der stark flektierten \isi{Adjektive} (vgl. RR \REF{ex:key:39}). Erstens wird im Plural im Gegensatz zum Singular kein \isi{Genus} unterschieden. Zweitens fallen der Dativ und Genitiv Singular Feminin nur mit den Genitiv Plural und nicht mit dem Genitiv und Dativ Plural zusammen. Diese höhere Komplexität des Dativ/Genitiv Singular Feminin und des Genitiv Plural verglichen mit dem Dativ Singular Maskulin und Neutrum kann folglich durch die RRs abgebildet werden.

\subsubsubsection{Wurzel-/Stammalternation} Schließlich ist die Frage zu klären, wie Wurzeln zu Stämmen werden und wie die Verteilung der Stämme erfasst werden kann. \citeauthor{Stump2001}s \citeyearpar{Stump2001} Modell bezüglich der Stämme kann aus unterschiedlichen Gründen, die erörtert werden, nicht übernommen werden. Dafür wird ein Vorschlag gemacht, der mit dem bisher vorgestellten Modell kompatibel ist und mit dem Komplexität gemessen werden kann.

Es soll an dieser Stelle außerdem vorausgeschickt werden, was unter \isi{Wurzel} und Stamm verstanden wird. Wie in \sectref{4.1.3.1} dargestellt wurde, stammt die \isi{Wurzel} aus dem \isi{Radikon}. Durch RRs werden von \isi{Wurzeln} neue Formen abgeleitet, indem phonologisches Material affigiert oder die \isi{Wurzel} modifiziert wird (z.\,B.\ \isi{Umlaut}). Dabei können Stämme entstehen, wie z.\,B.\ die Pluralstämme \textit{äpfəl}- und \textit{kind}-\textit{ər}-. Es muss aber nicht zwangsläufig zuerst ein Stamm gebildet werden, bevor ein Flexionssuffix angehängt wird. Beispielsweise macht es sehr wenig Sinn, davon auszugehen, dass von der \isi{Wurzel} \textit{kind} ein Stamm \textit{kind} geformt wird,\footnote{In der in\-fe\-ren\-tiel\-len-re\-a\-li\-sie\-ren\-den Morphologie gibt es keine Nullmorpheme (vgl. \sectref{4.1.2}).} an den -\textit{əs} suffigiert wird (z.\,B.\ \textit{Kind}-\textit{əs}). Durch RRs können Stämme von anderen Stämmen (Stammalternation) wie auch von \isi{Wurzeln} abgeleitet werden. In der Folge wird also von Wurzel- und Stammalternation gesprochen, da zwei Stämme, aber ebenfalls eine \isi{Wurzel} und ein Stamm in einem Paradigma systematisch miteinander alternieren können.\\

\noindent
\textbf{\isi{Wurzel-/Stammalternationen} bei \citet{Stump2001}:} \citet{Stump2001} unterscheidet u.a. \textit{Stem-Formation Rules} und \textit{Stem-Selection Rules}. Eine \textit{Stem-Formation Rule} definiert die Form eines Stammes. Folgendes Beispiel aus \citet{Stump2001} beschreibt die starken und schwachen Stämme der Possessivadjektive (Maskulin und Neutrum) \textit{bhágavant} und \textit{bhágavat} im Sanskrit:

\ea%40
    \label{ex:key:40}
Stem-formation rule:\\
A possessive adjective has a Strong stem in -\textit{vant} iff it has a Weak stem in -\textit{vat}. \citep[172]{Stump2001}
\z

Die Verteilung der Stämme wird über \textit{Stem-Selection Rules} definiert. Bezogen auf die Stämme in \REF{ex:key:40} beschreibt die \textit{Stem-Selection Rule} in \REF{ex:key:41}, dass der starke Stamm in den direkten \isi{Kasus} des Maskulinums verwendet wird:\\

\ea%41
    \label{ex:key:41}
Stem-selection rules:\\\relax
[…] RR\textsubscript{0 \{\textsc{gen}:masc, \textsc{dir:}yes\}, [C-stem nominal]} ($\langle$X,$\sigma$$\rangle$) = def $\langle$Y,$\sigma$$\rangle$, where Y is X’s Strong stem […]. \citep[179]{Stump2001}
\z

\citeauthor{Stump2001}s \citeyearpar{Stump2001} Modell ist in zweierlei Hinsicht problematisch. Erstens\linebreak beschreibt zwar die \textit{Stem-Formation Rule} die Form des Stammes, aber auf eine eher informelle Weise. Vielmehr ähneln die \textit{Stem-Formation Rules} einer Listung von Stämmen, was grundsätzlich unbefriedigend ist. Es wurde gezeigt, dass laut \citet{AckermanStump2004} lediglich \isi{Wurzeln} im \isi{Radikon} gelistet sind und alle anderen Formen von diesen \isi{Wurzeln} durch RR abgeleitet werden. Daraus kann geschlossen werden, dass auch die Stämme von den \isi{Wurzeln} deriviert werden sollen. Der zweite Grund, weshalb \citeauthor{Stump2001}s \citeyearpar{Stump2001} Vorschlag nicht übernommen werden kann, betrifft die Komplexitätsmessung. Da die Stämme von den \isi{Wurzeln} abgeleitet werden müssen, erhöht die Anzahl der Stämme (wie auch der \isi{Suffixe}) die Komplexität der Flexion. Es wurde bereits dafür argumentiert, dass zur Komplexitätsmessung ein einheitliches und vergleichbares Instrument zentral ist. Die \textit{Stem-Formation Rules} und die \textit{Stem-Selection Rules} können jedoch miteinander nicht verglichen bzw. verrechnet werden, da sie völlig unterschiedliche Formen aufweisen (vgl. Diskussion zu den \isi{Synkretismen}). Des Weiteren sind diese beiden Regeltypen nicht vergleichbar mit dem bisher vorgeschlagenen Regeltyp, nämlich der RRs. Zwar hat \REF{ex:key:41} die Form einer RR, der Nebensatz ist aber zu informell und beliebig erweiterbar, was für die Vergleichbarkeit problematisch ist. Aus diesen Gründen wird hier vorgeschlagen, die Form und Verteilung der Stämme über RRs zu definieren.\\

\noindent
\textbf{\isi{Wurzel-/Stammalternationen} in dieser Arbeit:} In den Abschnitten  \sectref{4.1.2}, \sectref{4.1.3.1} und \sectref{4.1.3.2} wurde gezeigt (basierend auf \citealt{Anderson1992}, \citealt{Stump2001}, \citealt{AckermanStump2004}), dass ein Lexem mit seinen morphosyntaktischen Eigenschaften verbunden wird und dass erst diese Verknüpfung es den RRs erlaubt, die Flexionsaffixe anzubringen. Flexion ist also ein rein phonologischer Prozess, der aus einer \isi{Wurzel} ein flektiertes Wort macht. Diese Idee soll hier auf die Stammbildung übertragen werden. Denn auch die Stammbildung ist nichts anderes als das Ableiten einer neuen Form von einer \isi{Wurzel}. Auf das Resultat dieser Ableitung, d.h. auf den Stamm, können weitere RRs angewendet werden (z.\,B.\ für \isi{Suffixe}), wie dies auch bei \isi{Suffixen} möglich ist, wo mehrere \isi{Suffixe} nacheinander angebracht werden können. So entstehen Stämme nicht durch Auflistung, sondern werden analog zu den \isi{Suffixen} behandelt.\largerpage

Im Gegensatz zu \citet{Stump2001} wird also in dieser Arbeit, streng genommen, das Konzept des Stammes obsolet. Ein Lexem hat eine oder mehrere \isi{Wurzeln} (im \isi{Radikon}), die durch RRs modifiziert werden, wobei diese RRs wiederum in \isi{Blöcke} eingeteilt sind, um die richtige Abfolge der RRs zu definieren. Trotzdem wird in der Folge zum einfacheren Verständnis von Wur\-zel-/Stamm\-al\-ter\-na\-tio\-nen gesprochen, da diese Konzepte in der Morphologie weitverbreitet sind.

Affigierung und Stammbildung als phonologische Prozesse zu betrachten, hat den weiteren Vorteil, dass Operationen der \is{nicht-konkatenative Flexion}nicht-konkatenativen Flexion wie Ablaut, \isi{Umlaut}, \is{Subtraktion}Subtraktion etc. problemlos abgebildet werden können. Ein zentraler Anspruch der in\-fe\-ren\-tiel\-len-re\-a\-li\-sie\-ren\-den Morphologie ist, durch die strikte Trennung von Form und Funktion/Bedeutung auch Phänomene der \is{nicht-konkatenative Flexion}nicht-konkatenativen Flexion adäquat abzubilden. Es wurde bereits gezeigt, dass RRs \isi{Umlaute} produzieren können (vgl. RR in \REF{ex:key:12}). Hier wird dargestellt werden, wie eine RR aussieht, die \is{Subtraktion}Subtraktion definiert.

Die \isi{Wurzel-/Stammalternationen} der in dieser Arbeit untersuchten Varietäten können alle durch die phonologische Umgebung erklärt werden. Wenn von einer \isi{Wurzel} Stämme abgeleitet werden, gibt es drei logisch mögliche Operationen: Entweder wird der \isi{Wurzel} phonologisches Material hinzugefügt oder getilgt oder die \isi{Wurzel} wird modifiziert (z.\,B.\ \isi{Umlaut}). Beides kommt in den hier untersuchten Varietäten vor. Das \isi{Possessivpronomen} der 1. Person Singular im Alemannischen von Bern weist zwei Stämme auf: \textit{mi}- und \textit{min}- \citep[99]{Marti1985}. In dieser Arbeit wird davon ausgegangen, dass, wenn die \isi{Wurzel} und ein Stamm homophon sind, dieser Stamm den Defaultstamm bildet und mindestens im Nominativ verwendet wird. Im Alemannischen von Bern steht der Defaultstamm \textit{mi}- wenn kein oder ein konsonantisch anlautendes \isi{Suffix} folgt, der Stamm \mbox{\textit{min}-,} wenn ein vokalisch anlautendes \isi{Suffix} folgt (vgl. Paradigma 111). Durch eine RR muss also von der \isi{Wurzel} \textit{mi} der Stamm \textit{min}- abgeleitet werden:

\ea%42
    \label{ex:key:42}
RR \textsubscript{B, \{ \}, \textsc{pron.poss}[\textsc{pers:1}} \textsubscript{${\veebar}$} \textsubscript{2, \textsc{num:sg}]} ($\langle$X,$\sigma$$\rangle$) = \textsubscript{def} $\langle$X\textit{n}/\_Vˊ,$\sigma$$\rangle$
\z 

Diese RR besagt, dass für das \isi{Possessivpronomen} der 1. und 2. Person Singular der \isi{Wurzel} ein \textit{n} suffigiert wird, wenn ein Vokal folgt. Die RRs müssen also nur durch Kontextbedingungen erweitert werden, die durch phonologische Regeln formalisiert werden können. Dies ist mit dem bisherigen Modell kompatibel, da auch die Suffigierungsregeln ohne Kontextbedingungen nichts anderes als phonologische Prozesse sind, die durch phonologische Regeln definiert werden.\largerpage

Ein Beispiel für \is{Subtraktion}Subtraktion findet sich im Alemannischen von Zürich. Auch hier weist das \isi{Possessivpronomen} der 1. Person Singular zwei Stämme auf: \textit{m\=in}- und \textit{m\=i}- \citep[135]{Weber1987}. \textit{M\=in}- bildet den Defaultstamm (also \isi{Wurzel}), \textit{m\=i}- den abgeleiteten Stamm, der dann auftritt, wenn das folgende \isi{Suffix} konsonantisch anlautet (vgl. Paradigma 110). Dies drückt folgende RR aus (auch für die 2. Person Singular, die gleich variiert), indem sie besagt, dass \textit{n} getilgt wird, wenn ein Konsonant folgt.

\ea%43
    \label{ex:key:43}
RR \textsubscript{B, \{ \}, \textsc{pron.poss}[\textsc{pers:1}} \textsubscript{${\veebar}$} \textsubscript{2, \textsc{num:sg}]} ($\langle$X,$\sigma$$\rangle$) = \textsubscript{def} $\langle$X *\textit{n}→ø/\_Kˊ,$\sigma$$\rangle$
\z

\isi{Wurzel-/Stammalternationen} können auch mit morphosyntaktischen Eigenschaften in Zusammenhang stehen, z.\,B.\ wenn ein bestimmter Stamm nur in einem bestimmten \isi{Kasus} oder \isi{Numerus} auftritt. Auch dies kann dargestellt werden. Einige der untersuchten Varietäten zeigen in der Substantivflexion einen Singular- und einen Pluralstamm. Dies ist der Fall z.\,B.\ in der deutschen Standardsprache, in der in einigen \isi{Flexionsklassen} der Plural mit dem \isi{Umlaut} markiert wird. Dies ist also gleichzeitig ein Beispiel für Stammmodifikation. Folgende RR, die bereits oben eingeführt wurde, definiert den \isi{Umlaut} für das Substantivparadigma der deutschen Standardsprache:

\begin{exe}%12
    \exr{ex:key:12}
RR \textsubscript{A,} \textsubscript{\{\textsc{num:pl}\},} \textsubscript{N[IC: 1} \textsubscript{${\veebar}$}\textsubscript{ 3} \textsubscript{${\veebar}$}\textsubscript{ 7} \textsubscript{${\veebar}$}\textsubscript{ 8]} ($\langle$X,$\sigma$$\rangle$) = \textsubscript{def} $\langle$Ẍˊ,$\sigma$$\rangle$
\end{exe}

Ein weiteres Beispiel für den Zusammenhang zwischen Wurzelmodifikation und morphosyntaktischen Eigenschaften findet sich im Alemannischen von Huzenbach. Hier lautet der Singular des Wortes \textit{Häuschen} \textit{heisle}, der Plural \textit{heisl-ə} \citep[98]{Baur1967}. Da \isi{Kasus} nicht am \isi{Substantiv} markiert wird, gibt es keinen Grund das \textit{e} in \textit{heisle} als \isi{Suffix} zu analysieren. Vielmehr weisen auf \textit{e} endende \isi{Wurzeln} einen Singularstamm auf, der gleich lautet wie die \isi{Wurzel}, und einen Pluralstamm, bei dem das \textit{e} getilgt wird. Auch der Plural zeigt keine Kasusmarkierung, sondern nur den Pluralmarker -\textit{ə}. Die RR in \REF{ex:key:44} definiert genau dies: Stehen die \isi{Substantive} der \isi{Flexionsklassen} 4 und 5 im Plural, wird \textit{e} vor \textit{ə} getilgt.\largerpage[2]

\ea%44
    \label{ex:key:44}
 RR \textsubscript{C, \{\textsc{num:pl}\}, N[IC: 4} \textsubscript{${\veebar}$} \textsubscript{5]} ($\langle$X,$\sigma$$\rangle$) = \textsubscript{def} $\langle$X *\textit{e}→ø/\_əˊ,$\sigma$$\rangle$
\z

Als Letztes muss noch überlegt werden, in welchen \isi{Blöcken} die RRs zur Wur\-zel-/Stamm\-al\-ter\-na\-tion zu verorten sind. Für die Substantivflexion der deutschen Standardsprache wurden drei \isi{Blöcke} vorgeschlagen: \isi{Block} A für die \is{Modifikation}Modifikationen an der \isi{Wurzel} (\isi{Umlaut}), \isi{Block} B für die Pluralsuffixe, \isi{Block} C für die Kasussuffixe. So können die flektierten \isi{Substantive} adäquat aufgebaut werden. Auf den ersten Blick scheint auch jede andere Abfolge unplausibel. Dies ändert sich jedoch, betrachtet man die Nicht-Stan\-dard\-va\-ri\-e\-tä\-ten, was am Alemannischen von Huzenbach illustriert wird. Dieser Dialekt weist drei Typen von Flexion auf: Der Plural kann durch \isi{Suffixe} oder \isi{Umlaut} markiert werden und die \isi{Flexionsklassen} 4 und 5 weisen eine Wurzelalternation auf. Man könnte also drei \isi{Blöcke} annehmen, wobei \isi{Block} A die Wurzelalternationen beinhaltet, \isi{Block} B den \isi{Umlaut} und \isi{Block} C die \isi{Suffixe}, was die Daten aber nicht genau erfasst. Denn die Wurzelalternation wird durch die Präsenz des Pluralsuffixes verursacht und nicht umgekehrt, es wird zuerst suffigiert und dann variiert die \isi{Wurzel} je nach phonologischer Umgebung. Folglich müssen diese drei \isi{Blöcke} angenommen werden: \isi{Block} A für den \isi{Umlaut}, \isi{Block} B für die \isi{Suffixe}, \isi{Block} C für die Wurzelalternation. Es kann somit dargestellt werden, dass die Präsenz eines \isi{Suffixes} der Auslöser der Wurzelalternation ist. Übrigens passiert diese \isi{Variation} in der Morphologie und nicht in der Phonologie. Bei der Pluralsuffigierung entsteht zwar \textit{heisle-ə}. Würde man aber die Tilgung in der Phonologie verorten, würde sie im gesamten Dialekt angewendet. Zur Vermeidung eines \isi{Hiats} ist jedoch in den alemannischen Dialekten das Einfügen eines \textit{n} zu erwarten. Da die Tilgung nur in diesem Paradigma gilt, ist die Regel dazu in der Flexionsmorphologie anzusetzen.\largerpage

\section{Was macht ein System komplexer oder simpler?}\label{4.2}

Im vorangehenden Kapitel wurde gezeigt, dass die Flexionsmorphologie im System der RRs zu verorten ist. Des Weiteren wurde in \sectref{3.1.2} basierend auf \citet{Miestamo2008} die absolute Komplexität eines linguistischen Phänomens definiert als die Länge der Beschreibung dieses Phänomens. Beides zusammengenommen bedeutet dies, dass je mehr RRs benötigt werden, um das System der Flexion zu beschreiben, desto komplexer ist dieses System. \citet{Stump2001} schlägt neben den RRs weitere Regeltypen für \isi{Synkretismen} und Wur\-zel-/Stamm\-al\-ter\-na\-tion vor. Dass diese aber zur Komplexitätsmessung nicht übernommen werden können, da sie miteinander nicht vergleichbar sind, wurde bereits in \sectref{4.1.3.3} erörtert. Was also ein System mehr oder weniger komplex macht und was ein System um wie viel mehr oder weniger komplex macht, wird ausschließlich aus dem System der RRs deduziert. Dadurch, dass die Definition und Messung der Komplexität theoriegeleitet und nicht durch Introspektion o.Ä. vonstattengeht, wird eine maximal mögliche Objektivität gewährleistet. 

Was die Flexion \textsc{komplexer} macht, ist folglich die Menge der \isi{Affixe}, \mbox{Wur\-zel-/} Stamm\-al\-ter\-na\-tio\-nen (z.\,B.\ \isi{Umlaute} und \is{Subtraktion}Subtraktion) und freien Varianten, die alle durch RRs abgebildet werden. So wie die RRs hier definiert sind, resultiert bezüglich der Erhöhung der Komplexität des Flexionssystems automatisch Folgendes. Erstens erhöht die Anzahl der grammatischen Eigenschaften, die in der Flexion unterschieden werden, die Komplexität der Flexion, wie z.\,B.\ die Anzahl \isi{Kasus}, die durch unterschiedliche Flexionsaffixe kodiert werden. Beispielsweise wird in der deutschen Standardsprache neben Plural auch der Dativ Plural markiert, während in den meisten alemannischen Dialekten nur der Plural markiert wird. Für die deutsche Standardsprache braucht es also eine RR für den Dativ Plural, dagegen ist dies in den meisten alemannischen Dialekten nicht nötig. Zweitens wird die Flexion komplexer, je mehr Allomorphe gezählt werden, z.\,B.\ die Anzahl unterschiedlicher Pluralaffixe. Für die Varietäten, die hier untersucht werden, bedeutet dies die Anzahl RRs für \isi{Suffixe} und Wur\-zel-/Stamm\-al\-ter\-na\-tio\-nen (\isi{Umlaut} und \is{Subtraktion}Subtraktion), die durch die Verknüpfung einer \isi{Wurzel} mit der Bedeutung Plural lizenziert werden. Drittens ergibt sich daraus, dass auch der Mehrfachausdruck zu höherer Komplexität führt. Während der Plural \textit{Tag-ə} mit einer RR kodiert wird \REF{ex:key:45}, werden für den Plural \textit{Wäld-ər} zwei RRs benötigt, nämlich eine für das \isi{Suffix} \REF{ex:key:12} und eine für den \isi{Umlaut} \REF{ex:key:13}.

\ea%45
    \label{ex:key:45}
 RR \textsubscript{B,} \textsubscript{\{\textsc{num:pl}\},} \textsubscript{N[IC: 1} \textsubscript{${\veebar}$}\textsubscript{ 2} \textsubscript{${\veebar}$}\textsubscript{ 7]} ($\langle$X,$\sigma$$\rangle$) = \textsubscript{def} $\langle$X\textit{ə}ˊ,$\sigma$$\rangle$
\z

\begin{exe}%12
    \exr{ex:key:12}
RR \textsubscript{A,} \textsubscript{\{\textsc{num:pl}\},}
\textsubscript{N[IC: 1} \textsubscript{${\veebar}$}\textsubscript{ 3} \textsubscript{${\veebar}$}\textsubscript{ 7} \textsubscript{${\veebar}$}\textsubscript{ 8]} ($\langle$X,$\sigma$$\rangle$) = \textsubscript{def} $\langle$Ẍˊ,$\sigma$$\rangle$
\end{exe}

\begin{exe}%13
\exr{ex:key:13} RR \textsubscript{B,} \textsubscript{\{\textsc{num:pl}\},} \textsubscript{N[IC: 3]} ($\langle$X,$\sigma$$\rangle$) = \textsubscript{def} $\langle$X\textit{ər}ˊ,$\sigma$$\rangle$
\end{exe}

Viertens erhöht eine bestimmte Art von \isi{Synkretismus} die Komplexität der Flexion, der in \sectref{4.1.3.3} ausführlich erörtert wurde. Kurz gesagt handelt es sich darum, dass, wenn die Werte von mehr als zwei Features variieren, der \isi{Synkretismus} durch die RRs nicht einheitlich erfasst werden kann. Dargestellt wurde dies am Beispiel des Dativs und Genitivs Feminin Singular und des Dativs und Genitivs Plural der starken Adjektivflexion (vgl. \tabref{table4.7}). Der Dativ und Genitiv Feminin Singular weisen das \isi{Suffix} -\textit{ər} auf, der Genitiv Plural ebenfalls das \isi{Suffix} -\textit{ər}, aber der Dativ Plural das \isi{Suffix} -\textit{ən}. Die \isi{Suffixe} -\textit{ər} können nicht durch eine RR erfasst werden, da der Plural im Gegensatz zum Singular kein \isi{Genus} unterscheidet und der Dativ und Genitiv Plural nicht wie im Feminin Singular zusammenfallen. Es braucht also nicht eine, sondern zwei RRs, die in (\REF{ex:key:37}, Dativ und Genitiv Singular Feminin) und (\REF{ex:key:38}, Genitiv Plural) aufgezeigt und hier wiederholt werden.\largerpage

\begin{exe}%37
    \exr{ex:key:37}
 RR \textsubscript{A, \{\textsc{case:dat}} \textsubscript{${\veebar}$}\textsubscript{ \textsc{gen}, \textsc{num:sg}}\textsubscript{,} \textsubscript{\textsc{gend:fem}\}, \textsc{adj[strong]}} ($\langle$X,$\sigma$$\rangle$) = \textsubscript{def} $\langle$\textit{ər}ˊ,$\sigma$$\rangle$
\end{exe}

\begin{exe}%38
    \exr{ex:key:38}
RR \textsubscript{A, \{\textsc{case:gen}, \textsc{num:pl}\}, \textsc{adj[strong]}} ($\langle$X,$\sigma$$\rangle$) = \textsubscript{def} $\langle$X\textit{ər}ˊ,$\sigma$$\rangle$
\end{exe}

Was das System \textsc{simpler} macht, sind erstens alle anderen \isi{Synkretismen}, bei denen nur die Werte eines Features variierten. Solche \isi{Synkretismen} finden sich bezüglich aller Features: \isi{Kasus}, \isi{Numerus}, \isi{Genus}, Wortart, \isi{Flexionsklasse}. Ein Kasussynkretismus ist in der RR \REF{ex:key:37} dargestellt. In der starken Adjektivflexion fallen Dativ und Genitiv im Singular Feminin zusammen. Auf dieselbe Weise werden auch Genussynkretismen erfasst. Fallen Singular und Plural zusammen, bleibt das Numerusfeature unterspezifiziert. Auch können Unterscheidungen innerhalb einer Wortart aufgehoben werden. Beispielsweise wird das \isi{Suffix} -\textit{ən} (Akkusativ Singular Maskulin) sowohl in der starken als auch in der schwachen Adjektivflexion verwendet. Dies zeigt die RR \REF{ex:key:23}, die im Gegensatz zu den RRs \REF{ex:key:37} und \REF{ex:key:38} die Wortart nicht weiter spezifiziert.\\

\begin{exe}%23
    \exr{ex:key:23}
RR \textsubscript{A, \{\textsc{case:acc}, \textsc{num:sg}, \textsc{gend:masc}\}, \textsc{adj[]}} ($\langle$X,$\sigma$$\rangle$) = \textsubscript{def} $\langle$X\textit{ən}ˊ,$\sigma$$\rangle$
    \end{exe}
   

Auch \isi{Synkretismen} zwischen den \isi{Flexionsklassen} können erfasst werden. Das \isi{Suffix} -\textit{ən} für den Dativ Plural der \isi{Substantive} in der deutschen Standardsprache wird in verschiedenen \isi{Flexionsklassen} wiederverwendet. Dazu wird nur eine RR \REF{ex:key:20} gebraucht:

\begin{exe}%20
    \exr{ex:key:20}
RR \textsubscript{C,} \textsubscript{\{\textsc{case:dat}, \textsc{num:pl}\},} \textsubscript{N[IC: 1} \textsubscript{${\veebar}$}\textsubscript{ 2} \textsubscript{${\veebar}$}\textsubscript{ 3} \textsubscript{${\veebar}$}\textsubscript{ 4} \textsubscript{${\veebar}$}\textsubscript{ 5} \textsubscript{${\veebar}$}\textsubscript{ 6} \textsubscript{${\veebar}$}\textsubscript{ 7} \textsubscript{${\veebar}$}\textsubscript{ 8]} ($\langle$X,$\sigma$$\rangle$) = \textsubscript{def} $\langle$X\textit{ən}ˊ,$\sigma$$\rangle$
\end{exe}

Zweitens vereinfachen Wörter, die keine overte Markierung (Wur\-zel-/Stamm\-al\-ter\-na\-tion oder \isi{Affix}) haben, das System der Flexion. Dies kommt gerade in der Substantivflexion der modernen deutschen Varietäten besonders häufig vor. Z.B. weisen in der deutschen Standardsprache im Singular nur der Genitiv Maskulin und Neutrum sowie die schwache Flexion Kasussuffixe auf (vgl. \tabref{table4.11}). Diese Zellen werden durch unterschiedliche RRs gefüllt. Für alle anderen Zellen, d.h. für jene ohne overte Markierung, wird nach \citet{Stump2001} nur eine RR benötigt, nämlich die RR \textit{Identity Function Default}:

\begin{exe}%26
    \exr{ex:key:26}
\isi{Identity Function Default} […]\\
RR\textsubscript{n,\{\},U} ($\langle$X,$\sigma$$\rangle$) = \textsubscript{def} ($\langle$X,$\sigma$$\rangle$). \citep[53]{Stump2001}
\end{exe}

\noindent
Diese RR besagt, dass mit der \isi{Wurzel} nichts passiert, wenn keine RR gefunden wird. Da es diese RR in allen Varietäten braucht, sie also keine Varietät im Vergleich mehr oder weniger komplex macht, muss sie bei der Messung der Komplexität nicht berücksichtigt werden.

Drittens werden Unterscheidungen, die in einer Wortart nicht gemacht werden, jedoch in einer anderen Wortart ausgedrückt werden, nicht berücksichtigt. Zum Beispiel unterscheiden die meisten alemannischen Dialekte zwar \isi{Kasus} an den Determinierern und Pronomen, aber nicht am \isi{Substantiv}. Das Pronomen braucht also RRs für die Kasusunterscheidung, das \isi{Substantiv} hingegen benötigt keine. Viertens wird alles, was nicht der Morphologie zugeordnet werden kann, nicht berücksichtigt. Dies betrifft u.a. Allomorphie, die phonologisch erklärt werden kann. Z.B. ist die -\textit{əs}/-\textit{s}-\isi{Variation} der \isi{Substantive} im Genitiv Singular Maskulin und Neutrum der deutschen Standardsprache phonologisch distribuiert. Die Verteilung der Varianten hängt vom Auslaut, der Betonung und der Anzahl Silben ab \citep[224-225]{EisenbergGelhausHenneSittaWellmann1998}. Diese phonotaktischen Regeln gelten für das gesamte System und nicht nur für die \isi{Substantive}, die \isi{Variation} wird also nicht durch RRs, sondern durch phonotaktische Regeln ausgedrückt, weshalb sie die Komplexität der Phonologie und nicht der Morphologie erhöhen.

\section{Messmethode}\label{4.3}

Dieses Kapitel widmet sich den Methoden zur Messung struktureller Komplexität. In \sectref{4.3.1} werden die bis jetzt entworfenen Messmethoden vorgestellt. Es wird gezeigt, dass keine dieser Methoden auf stark flektierende, eng verwandte Sprachen übertragen werden und die Komplexität in der Nominalflexion adäquat erfassen kann. In \sectref{4.3.2} wird die hier entwickelte und angewandte Messmethode präsentiert, welche auf der in \sectref{4.1} dargestellten Theorie basiert.

\subsection{Bisherige Messmethoden}\label{4.3.1}

Es wird nun eine Übersicht über die verschiedenen Typen an Methoden gegeben, die bis jetzt verwendet wurden, um strukturelle Komplexität zu messen. Jeder Typ an Methode wird kurz anhand eines Beispiels illustriert, wobei die meisten Beispiele aus Arbeiten stammen, die in den Abschnitten \sectref{2.1.2} und \sectref{2.2} vorgestellt wurden. Ziel ist es, das breite Spektrum an Methoden aufzuzeigen, und nicht, alle Methoden zu diskutieren. In der Folge werden zuerst die qualitativen und dann die quantitativen Methoden vorgestellt.

\subsubsection{Qualitative Methoden}\label{4.3.1.1}

{\citet{DammelKürschner2008}} untersuchen die Komplexität der nominalen Pluralmorphologie in zehn germanischen Standardsprachen, und zwar an den folgenden fünf Parametern: Anzahl Allomorphe, formale Kodierung (Stammmodifikation, Redundanz, Null-Markierung, \is{Subtraktion}Subtraktion und Fusion \isi{Numerus}/\isi{Genus}), Zuweisung der Allomorphe zu Stämmen, Richtung der Bestimmung formaler Charakteristika zwischen Stamm und \isi{Suffix}, Regularität \citep[244, 257]{DammelKürschner2008}. Für all diese fünf Parameter wird bestimmt, was die Pluralmarkierung mehr oder weniger komplex macht, wobei qualitativ vorgegangen wird. Zum Beispiel werden für den Parameter \textit{Redundanz} zwei Gruppen an Sprachen gebildet: Sprachen, welche Mehrfachausdruck von Plural aufweisen (= komplex) oder nicht (= simpel) \citep[257]{DammelKürschner2008}. Wird mehr als einmal der Plural am \isi{Substantiv} ausgedrückt, gilt dies als komplex, da es Ikonizität verletzt \citep[255-251]{DammelKürschner2008}. Der Parameter \textit{Stammmodifikation} wird anhand von vier weiteren Parametern untersucht: Suffigierung ohne Stammmodifikation, \is{Modifikation}Modifikation des Stamm auslautenden Konsonanten, \is{Modifikation}Modifikation des Wurzelvokals und Suppletion. Für jede Sprache wird geprüft, wie sehr sie diese Kodierungstechniken verwendet (bezüglich Frequenz und Produktivität), wozu Werte von 1–4 vergeben werden \citep[250]{DammelKürschner2008}. Auf der Grundlage dieser Werte werden die Sprachen in fünf Gruppen auf einem Kontinuum eingeteilt, auf dem einfach und komplex die beiden Extreme bilden \citep[257]{DammelKürschner2008}. Auch der quantifizierbare Parameter \textit{Anzahl Allomorphe} wird qualitativ ausgewertet. Es werden vier Gruppen von einfach bis komplex gebildet: 1 Morph, 2–4 Allomorphe, 5–7 Allomorphe, Fusion von Kasus- und Numerusmarkierung \citep[247]{DammelKürschner2008}. Für jeden der fünf eingangs aufgezählten Parameter werden die Sprachen in zwei bis fünf Gruppen eingeteilt, welche auf einem Kontinuum von einfach bis komplex verortet werden (vgl. Tabelle 5 in \citealt[257]{DammelKürschner2008}). Mit dieser Methode ist es zwar möglich, die Sprachen innerhalb eines Parameters miteinander zu vergleichen. Da jedoch nicht quantifiziert wird, können die Parameter nicht miteinander verrechnet werden, d.h., dass keine genauen Aussagen über die Unterschiede in der Gesamtkomplexität der Pluralmarkierung zwischen den untersuchten Sprachen möglich sind. Des Weiteren lässt sich diese Methode nicht auf andere nominale Wortarten übertragen.

Auch {\citet{Camilleri2012}} analysiert morphologische Komplexität aus einer qualitativen Perspektive. Die Grundlage bietet die kanonische Typologie und morphologische Komplexität wird definiert als das Ergebnis „of the divergence from the \textit{canon}, where the further away from the canonical requirement a given example is, the more non-canonical, and the more morphologically complex it is“ \citep[93]{Camilleri2012}. Zur Analyse eines Paradigmas werden zwei Perspektiven unterschieden: Erstens werden die verschiedenen Zellen im Paradigma eines Lexems verglichen, zweitens die Lexeme miteinander. Beide Perspektiven werden anhand von drei Parametern untersucht: 1) Zusammensetzung/Struktur, 2) lexikalisches Material (Form des Stammes), 3) Form der \isi{Affixe} \citep[94]{Camilleri2012}. Ein kanonisches Paradigma wird wie folgt definiert. Vergleicht man die Zellen eines Lexems, dann sind die Zusammensetzung/Struktur und der Stamm des Lexems immer gleich, die Form des \isi{Affixes} ändert sich in Abhängigkeit der morphosyntaktischen Eigenschaften. Vergleicht man die Lexeme miteinander, weisen sie dieselbe Zusammensetzung/Struktur und dieselbe Form des \isi{Affixes} für dieselbe morphosyntaktische Eigenschaft auf, während sich die Form der Stämme unterscheidet \citep[94]{Camilleri2012}. Alles, was von einem solchen kanonischen Paradigma abweicht, gilt als komplex, wie z.\,B.\ \isi{Synkretismen} und \isi{Flexionsklassen} bezüglich des dritten Parameters. Einen Überblick, welche Phänomene als nicht-kanonisch gelten (nicht vollständige Liste), gibt \tabref{table4.13}.

% % \textbf{\tabref{table4.13}: Nicht-kanonische Phänomene in einem Paradigma (übernommen aus \citealt[95]{Camilleri2012})}\\

\begin{table}
\caption{Nicht-kanonische Phänomene in einem Paradigma (übernommen aus \citealt[95]{Camilleri2012})}\label{table4.13}
\begin{tabularx}{\textwidth}{QQQQQ} 
\lsptoprule
& {The content of the paradigmatic cell} & {Deviations} & {Comparisons across different lexical paradigms} & {Deviations}\\
\midrule
{Composition/ structure} & different & fused exponence periphrasis & different & defectiveness
overdifferentiation\\
\midrule
{Lexical material (stem-shape)} & different & stem-alternation
suppletion & same & heteroclisis\\
\midrule
{Affixal material (affix-shapes\slash forms)} & same & syncretism uninflectability & different & deponency inflectional classes\\
\lspbottomrule
\end{tabularx}
\end{table}

Der Vorteil dieser Methode ist, dass ein Bezugspunkt bzw. ein Ideal aus rein formalen Überlegungen definiert wird, womit im Prinzip jedes Paradigma jeder Sprache verglichen werden kann. Damit können aber nur Aussagen der Art gemacht werden, inwiefern ein Phänomen von diesem Ideal abweicht. \citet{Camilleri2012} zeigt nicht, wie diese Abweichungen quantifiziert werden könnten. Es ist folglich nicht möglich, den Grad der Komplexität von unterschiedlichen Sprachen zu vergleichen.

\subsubsection{Quantitative Methoden}\label{4.3.1.2}
\begin{description}
\item[Anzahl Sprachen:] Die Messmethode von \citet{Sinnemäki2009} weicht ganz grundsätzlich von allen Methoden ab, die in der Folge noch vorgestellt werden. Untersucht wird die Markierung von Agens und Patiens, welche als komplex gilt, wenn sie das Eine-Bedeutung-Eine-Form-Prinzip verletzt. Dies ist der Fall, wenn eine Sprache mehr als eine Strategie zur Markierung verwendet (verletzt Ökonomie) oder wenn zu wenig markiert wird (verletzt Distinktheit) (\citealt[130–133]{Sinnemäki2009}, vorgestellt in \sectref{2.2.2}). Es wird nicht die Komplexität an und für sich gemessen, sondern die Anzahl Sprachen, die dem Eine-Bedeutung-Eine-Form-Prinzip entsprechen und solche, die das nicht tun. Ein Resultat ist, dass die Anzahl Sprachen, die dem Eine-Bedeutung-Eine-Form-Prinzip folgen, parallel zur Größe der Sprachge-\linebreak meinschaft zunimmt \citep[135]{Sinnemäki2009}. Diese Methode misst also\linebreak streng genommen nicht die Komplexität des Phänomens selbst, sondern beschreibt sie qualitativ (+/– Verletzung des Eine-Bedeutung-Eine-Form-Prinzips). Quantitativ ist nur die Anzahl Sprachen, die dem qualitativen Kriterium entsprechen bzw. nicht entsprechen.

\item[Inventargröße--einzelne Phänomene und Kategorien] Eine weitere quantitative Methode, die ebenfalls die Komplexität nicht durch eine Zahl misst, ist sehr weit verbreitet und bezieht sich auf die \isi{Inventargröße}. Es werden zumeist Aussagen gemacht, dass eine Kategorie mehr oder weniger grammatikalisiert ist und overt markiert wird. Die Komplexität der untersuchten Phänomene wird jedoch nicht durch Zahlen quantifiziert. Ein Beispiel dafür ist \citet{Schreier2016}, dessen Arbeit in \sectref{2.2.3} vorgestellt wurde. Er stellt u.a. fest, dass Tok Pisin im Gegensatz zu anderen englischen Varietäten im \isi{Personalpronomen} der 1. Person Plural die Unterscheidung inklusiv/exklusiv grammatikalisiert hat, also mehr Formen hat als andere englische Varietäten in der 1. Person Plural, und in dieser Hinsicht folglich komplexer ist \citep[147]{Schreier2016}. Oder das Verbalparadigma in Tristan da Cunha Englisch hat durch Nivellierung Komplexität abgebaut, da \textit{is} und \textit{was} unabhängig von Person und \isi{Numerus} verwendet werden \citep[149]{Schreier2016}. Ähnlich verfahren beispielsweise \citet{Braunmüller1984}, \citet{McWhorter2001}, \citet{Kusters2003} und \citet{Trudgill2011}. Mit dieser Methode kann die Komplexität einzelner Phänomene und Kategorien beschrieben werden. Es ist jedoch nicht möglich, die Komplexität einer Kategorie oder eines Teilsystems zu messen und Aussagen darüber zu machen, wie viel komplexer oder weniger komplex eine Kategorie einer Varietät A im Vergleich zur selben Kategorie einer Varietät B ist. Auch kann die Komplexität verschiedener Komponenten eines Teilsystems nicht miteinander verrechnet werden (z.\,B.\ verschiedene Wortarten in der Nominalflexion).

\item[Inventargröße--Präsenz/Absenz eines Phänomens oder einer Kategorie:] In eine ähnliche Richtung geht ein Teil der Messmethode von \citet{SzmrecsanyiKortmann2009} (vgl. \sectref{2.2.3}), welche jedoch klar quantifizierend ist. Als strukturell komplex gelten u.a. ornamentale Regeln und jene Regeln, die beim L2-Er\-werb Schwierigkeiten verursachen \citep[64–65]{SzmrecsanyiKortmann2009}. Als Datengrundlage dient der \textit{World Atlas of Morphosyntactic \isi{Variation} in English}, der 76 Charakteristika enthält. Von diesen 76 Charakteristika werden jene ausgesucht, welche dem linguistischen System Unterscheidungen und Asymmetrien hinzufügen, ohne dass diese kommunikative oder funktionale Vorteile haben (= ornamentale Regeln), wie z.\,B.\ \isi{Genus} \citep[68]{SzmrecsanyiKortmann2009}. Analog dazu wird bezüglich der L2-Schwie\-rig\-keit vorgegangen, nur dass hier von den 76 jene Eigenschaften extrahiert werden, welche in Intersprachen auftreten, z.\,B.\ fehlende Markierung der Vergangenheit \citep[69]{SzmrecsanyiKortmann2009}. Anschließend wird für jede Varietät geprüft, wie viele der ausgesuchten Eigenschaften sie aufweist. Ein Beispiel: Hat eine Varietät zwei ornamentale Regeln, ist ihr Grad an Komplexität zwei, wobei eine Sprache komplexer ist, je mehr ornamentale Regeln sie aufweist. Das Gegenteil gilt für die L2-Schwie\-rig\-keit: Je mehr intersprachliche Eigenschaften eine Varietät hat, desto weniger komplex ist sie. Auch ein Teil der Methode von \citet{Nichols2016} geht so vor, wobei ebenfalls den binären Eigenschaften die Zahl 1 vergeben wird, wenn eine bestimmte Eigenschaft in einer Sprache vorkommt, und die Zahl 0, wenn dies nicht der Fall ist \citep[137]{Nichols2016}. Diese Methode basiert also auf der Präsenz bzw. Absenz gewisser Phänomene oder Kategorien, die als komplex gelten, wobei gilt, je mehr solche Phänomene/Kategorien vorkommen, desto komplexer ist eine Sprache. Der Vorteil dieser Methode ist, dass sie im Gegensatz zu den bis hierhin vorgestellten Methoden klar quantifiziert. Durch ihre Binarität (Absenz/Präsenz einer Kategorie oder eines Phänomens) ist sie jedoch nicht übertragbar auf die Messung der Komplexität in der Flexion. Zwar wäre es möglich, z.\,B.\ für die Pluralmarkierung 1 zu vergeben und für das Fehlen der Pluralmarkierung 0. Es macht jedoch einen Unterschied, ob der Plural immer mit derselben Form markiert wird oder nicht und ob der Plural mehrfach am selben Wort markiert wird oder nicht. Allomorphie und Mehrfachausdruck können mit dieser Methode folglich nicht erfasst werden, was jedoch bei stark flektierenden und eng verwandten Sprachen mit sehr feinen Unterschieden wünschenswert ist. Schließlich kann diese Methode nicht die höhere bzw. geringere Komplexität innerhalb eines Teilsystems, einer Kategorie oder eines Phänomens messen.

\item[Inventargröße--Anzahl Unterscheidungen innerhalb einer Kategorie:] Vorschläge dazu kommen aus Arbeiten, in denen strukturelle Komplexität über die \isi{Inventargröße} definiert wird: Je größer das Inventar ist, desto höher fällt die strukturelle Komplexität aus. Diese Grundidee wurde bereits auf verschiedene Teilsysteme der Grammatik angewandt, was hier exemplarisch veranschaulicht werden soll.\footnote{Die folgenden Studien wurden bereits in den Abschnitten \sectref{2.2.2} und \sectref{2.2.3} ausführlicher vorgestellt, weshalb hier nur die Messmethoden erörtert werden.} \citet{HayBauer2007} berechnen die Größe des Phoneminventars durch die Anzahl Basismonophthonge, Extramonophthonge (Unterscheidung der Länge und Nasalierung), Diphthonge, Obstruenten und Konsonanten \citep[389]{HayBauer2007}. Die ersten Ideen in diese Richtung stammen bereits von \citet{Jakobson1929}. \citet{Nichols2016} misst die \isi{Inventargröße} von Komponenten unterschiedlicher linguistischer Beschreibungsebenen, wie z.\,B.\ die Anzahl der kontrastiven Obstruenten, die Anzahl der Wortabfolgen in den Basissatztypen sowie die Anzahl der unterschiedlichen Typen an \isi{Präfixen} \citep[137]{Nichols2016}. \citet{Shosted2006} zählt die Anzahl der möglichen Silben und der Marker in der Verbflexion in einer Sprache \citep[9–17]{Shosted2006}. Seine Methode kann nur auf die verbale Flexion angewendet und nicht auf die nominale Flexion übertragen werden \citep[16]{Shosted2006}. Auch ein Teil der Messmethode von \citet{Garzonio2016} kann hier dazugerechnet werden. Er untersucht die Komplexität von Entscheidungsfragesätzen und w-Fra\-ge\-sät\-zen in drei italienischen Dialekten. Dabei stellt er fest, dass in zwei Dialekten das w-E\-le\-ment verdoppelt werden kann und dass in einem das w-E\-le\-ment verschiedene Formen hat \citep[105, 109–110]{Garzonio2016}. Diese freien Varianten werden gezählt und erhöhen folglich die Größe des Inventars. Trotz der großen Vielfalt an Messmethoden in diesem Bereich wurde noch keine für die Nominalflexion von stark flektierenden und eng verwandten Sprachen vorgeschlagen.

\item[\isi{Stammformen}:] In eine völlig andere Richtung zielt die Typologie der \isi{Stammformen} von \citet{FinkelStump2007}. Zwar geht es nicht darum, strukturelle Komplexität zu messen, ihre Typologie ließe sich aber einfach quantifizieren. Zuerst ist jedoch festzuhalten, dass sie drei verschiedene Schemen an Stammformsystemen unterscheiden: statische, adaptive und dynamische \isi{Stammformen}. Dargestellt sind diese in \tabref{table4.14}: I–VI stehen für \isi{Flexionsklassen}, W–Z sind Sets an morphosyntaktischen Eigenschaften, a–o die Flexionsmarker. Werden in jeder \isi{Flexionsklasse} die \isi{Stammformen} durch dieselben Sets an morphosyntaktischen Eigenschaften ermittelt, so spricht man von statischen \isi{Stammformen} \citep[2]{FinkelStump2007}. In adaptiven Systemen teilen die \isi{Stammformen} ein Set an morphosyntaktischen Eigenschaften, die anderen können unterschiedlich sein \citep[2]{FinkelStump2007}. Lautet der Marker von W a, dann gehört dieses Lexem zur \isi{Flexionsklasse} I. Lautet der Marker von W jedoch c, dann kann die \isi{Flexionsklasse} nur durch den Marker von X bestimmt werden. Heißt dieser f, gehört das Lexem zur \isi{Flexionsklasse} III, heißt dieser g, gehört das Lexem zur \isi{Flexionsklasse} IV. In einem dynamischen System hingegen müssen die \isi{Stammformen} keine morphosyntaktischen Eigenschaften teilen \citep[4]{FinkelStump2007}. Lautet der Marker von W b, ist das Lexem Teil der \isi{Flexionsklasse} II, lautet der Marker von X f, ist das Lexem Teil der \isi{Flexionsklasse} III. Es kann festgehalten werden, dass dynamische Schemen zu bevorzugen sind, da sie mit kleineren Systemen an \isi{Stammformen} auskommen als statische oder adaptive Schemen \citep[4]{FinkelStump2007}.

% % \textbf{\tabref{table4.14}: Typen an Stammformsystemen \citep[2-4]{FinkelStump2007}}\\

\begin{table}
\caption{Typen an Stammformsystemen \citep[2-4]{FinkelStump2007}}\label{table4.14}
\begin{tabular}{*{13}{c}}
\lsptoprule
& \multicolumn{4}{c}{statisch} & \multicolumn{4}{c}{adaptiv} & \multicolumn{4}{c}{dynamisch}\\\cmidrule(lr){2-5}\cmidrule(lr){6-9}\cmidrule(lr){10-13}
& W & X & Y & Z & W & X & Y & Z & W & X & Y & Z\\\midrule
I & \cellcolor[gray]{0.9}a & \cellcolor[gray]{0.9}e & \cellcolor[gray]{0.9}i & m & \cellcolor[gray]{0.9}a & e & i & m & \cellcolor[gray]{0.9}a & e & i & m\\
\midrule
II & \cellcolor[gray]{0.9}b & \cellcolor[gray]{0.9}e & \cellcolor[gray]{0.9}i & m & \cellcolor[gray]{0.9}b & e & i & m & \cellcolor[gray]{0.9}b & e & i & m\\
\midrule
III & \cellcolor[gray]{0.9}c & \cellcolor[gray]{0.9}f & \cellcolor[gray]{0.9}j & n & \cellcolor[gray]{0.9}c & \cellcolor[gray]{0.9}f & j & n & c & \cellcolor[gray]{0.9}f & j & n\\
\midrule
IV & \cellcolor[gray]{0.9}c & \cellcolor[gray]{0.9}g & \cellcolor[gray]{0.9}j & n & \cellcolor[gray]{0.9}c & \cellcolor[gray]{0.9}g & j & n & c & \cellcolor[gray]{0.9}g & j & n\\
\midrule
V & \cellcolor[gray]{0.9}d & \cellcolor[gray]{0.9}h & \cellcolor[gray]{0.9}k & o & \cellcolor[gray]{0.9}d & h & \cellcolor[gray]{0.9}k & o & d & h & \cellcolor[gray]{0.9}k & o\\
\midrule
VI & \cellcolor[gray]{0.9}d & \cellcolor[gray]{0.9}h & \cellcolor[gray]{0.9}l & o & \cellcolor[gray]{0.9}d & h & \cellcolor[gray]{0.9}l & o & d & h & \cellcolor[gray]{0.9}l & o\\
\lspbottomrule
\end{tabular}
\end{table}

Um die typologische \isi{Variation} zwischen den Systemen von \isi{Stammformen} verschiedener Sprachen darzustellen, werden fünf Kriterien vorgeschlagen \citep[9]{FinkelStump2007}, wovon zwei zur Messung der Komplexität quantifiziert werden können: 1) Wie viele \isi{Stammformen} sind nötig, um das Paradigma eines Lexems zu bestimmen; 2) Wie viele dynamische \isi{Stammformen} werden gebraucht, um eine bestimmte Wortform im Paradigma eines bestimmten Lexems zu bestimmen \citep[9, 16]{FinkelStump2007}. Für 1) wird die durchschnittliche Anzahl \isi{Stammformen} für das gesamte Flexionssystem (also für alle \isi{Flexionsklassen} zusammen) berechnet \citep[11]{FinkelStump2007}. 2) geht von der Beobachtung aus, dass es Formen mit bestimmten morphosyntaktischen Eigenschaften gibt, die von mehr als einer dynamischen \isi{Stammform} abgeleitet werden müssen. Berechnet wird also die durchschnittlich Anzahl an dynamischen \isi{Stammformen}, die benötigt werden, um den Marker einer bestimmten Wortform ableiten zu können \citep[18]{FinkelStump2007}.

Würde man mit den Kriterien 1) und 2) die Komplexität eines Paradigmas messen, zeichnet sich diese Methode durch ihre radikale Objektivität aus. Annahmen folgender Art müssen nicht gemacht werden: Was vom Eine-Form-Eine-Bedeutung-Prinzip abweicht oder als L2 schwierig gilt, ist komplex. Der Nachteil an dieser Methode ist, dass keine Aussagen über das Inventar der Formen und deren Verteilung (mit Ausnahme der \isi{Stammformen}) auf die morphosyntaktischen Eigenschaften gemacht werden können. Auch jene Formen, die keine \isi{Stammformen} sind, müssen von der Morphologie definiert werden. Diese Methode könnte also jene Fälle nicht adäquat erfassen, wenn z.\,B.\ zwei Flexionssysteme gleich viele \isi{Stammformen}, aber unterschiedlich viele \isi{Synkretismen} haben. Ein Beispiel (vgl. \tabref{table4.14}, dynamisch): Von der Form a für W leite ich ab, dass dieses Lexem zur \isi{Flexionsklasse} I gehört. Um das Paradigma zu vervollständigen, muss ich wissen, dass in der \isi{Flexionsklasse} I die Form für X e lautet, für Y i und für Z m (also drei Regeln). Angenommen die Formen von X, Y und Z sind zu m zusammengefallen, ist der Aufwand jedoch kleiner, um das Paradigma zu vervollständigen, weil ich mir nur eine Regel merken muss. Dieser Tatsache würde diese Methode nicht Rechnung tragen. Es sei hier nochmal darauf hingewiesen, dass \citet{FinkelStump2007} keine Messmethode, sondern eine Typologisierung vorschlagen. Mit der skizzierten, hypothetischen Messmethode ließe sich jedoch die Komplexität der Stammformensysteme berechnen.

\item[Anzahl Move- und Mergeoperationen:] Eine rein theoriegeleitete Messmethode verwendet \citet{Garzonio2016}, dessen Arbeit bereits in \sectref{2.2.4} vorgestellt wurde. Es sei hier nur die Messmethode wiederholt. Sie ist im Minimalismus verortet und gemessen wird die syntaktische Komplexität bzw. die Komplexität der Derivation: Je mehr Move- und Mergeoperationen benötigt werden, um eine bestimmte Struktur abzuleiten, desto komplexer ist die Derivation \citep[99]{Garzonio2016}. Wie jede Theorie geht auch der Minimalismus von bestimmten Annahmen über die Architektur der Grammatik aus. Dies trifft jedoch nicht auf die Messmethode selbst zu, wodurch sie sich durch ein hohes Maß an Objektivität auszeichnet. Auf die Morphologie kann \citeauthor{Garzonio2016}s Messmethode aber naturgemäß nicht übertragen werden.

\item[Korpusbasiert--\isi{Textfrequenz} von Morphemen:] Schließlich sollen noch zwei kor-\linebreak pusbasierte Messmethoden vorgestellt werden. \citet{SzmrecsanyiKortmann2009} wie auch \citet{MaitzNémeth2014} berechnen die \isi{Textfrequenz} bestimmter Morpheme (vgl. \sectref{2.2.3}). Die Komplexität der Grammatizität wird durch die Summe des Synthese- und Analyseindex ermittelt \citep[72]{SzmrecsanyiKortmann2009}: Der Syntheseindex ist die Anzahl (in Prozent) der gebundenen Morpheme pro 1.000 Tokens, der Analyseindex die Anzahl der freien Morpheme pro 1.000 Tokens \citep[72]{SzmrecsanyiKortmann2009}. Des Weiteren wird ein Transparenzindex berechnet, und zwar durch den prozentualen Anteil der gebundenen regelmäßigen Allomorphe an allen gebundenen Allomorphen \citep[74]{SzmrecsanyiKortmann2009}. Je höher die Grammatizität und je tiefer der Transparenzindex sind, desto komplexer ist eine Varietät. Diese Methode hat einen praktischen und prinzipiellen Nachteil. Erstens ist sie nur auf jene Sprachen und Varietäten anwendbar, für die große, annotierte Korpora existieren. Zweitens ist nur eine eher oberflächliche Analyse möglich, da wir nichts über freie Varianten, \isi{Synkretismen}, Mehrfachausdruck, \isi{Flexionsklassen} oder Stammalternationen erfahren. Präzise und detaillierte Aussagen über Zusammenhänge oder eventuelle Ausgleichsmechanismen innerhalb der Flexion sind nicht möglich. Jedoch zeichnet sich diese Methode durch eine hohe Objektivität aus und eignet sich durchaus, um einen ersten Überblick über die Flexionsmorphologie zu erhalten.

\item[Korpusbasiert--\isi{Informationstheorie}:] Als letzte Messmethode sollen noch jene präsentiert werden, die auf einer informationstheoretischen Grundlage basieren, z.\,B.\ Kolmogorov-Komplexität, Ziv-Lempel-Komplexitat etc. Arbeiten auf dieser Basis stammen u.a. von \citet{Juola2008}, \citet{Bane2008} und \citet{EhretSzmrecsanyi2016}, um nur einige wenige zu nennen. Es soll hier zur Illustration nur eine Operationalisierung der Informationstheorien vorgestellt werden, nämlich jene von \citet{EhretSzmrecsanyi2016}. Allgemein geht es darum zu prüfen, wie stark Kompressionsprogramme (z.\,B.\ gzip) Texte komprimieren, wodurch die Kolmogorov-Komplexität approximiert werden kann. Dazu werden dieselben Texte in verzerrter und unverzerrter Form komprimiert und die Resultate dieser Komprimierung verglichen. Um morphologische Komplexität zu messen, werden 10\% der Buchstaben gelöscht, wodurch neue Wortformen entstehen und so Regularitäten vermindert werden. Anschließend wird dieser verzerrte Text komprimiert, wobei eine schlechte Kompressionsrate des verzerrten Textes niedrige morphologische Komplexität bedeutet. Dies wird dadurch begründet, dass morphologisch komplexe Sprachen auf jeden Fall viele verschiedene Wortformen aufweisen, eine Verzerrung richtet also weniger Schaden an als in morphologisch einfachen Sprachen, in denen die Verzerrung verhältnismäßig mehr statistisches Rauschen verursacht \citep[75–76]{EhretSzmrecsanyi2016}. Der Grad der morphologischen Komplexität wird dadurch berechnet, dass die Größe der komprimierten Datei nach der Verzerrung durch die Größe der komprimierten Datei vor der Verzerrung geteilt wird \citep[76]{EhretSzmrecsanyi2016}. Auf eine analoge Weise wird syntaktische Komplexität gemessen, wobei zur Verzerrung des Textes 10\% der Wörter gelöscht werden \citep[76]{EhretSzmrecsanyi2016}. Auch hier können dieselben praktischen und prinzipiellen Nachteile genannt werden, welche bereits oben bezüglich der Methode von \citet{SzmrecsanyiKortmann2009} moniert wurden. Außerdem bilden Texte die Datengrundlage, welche den Output eines grammatischen Systems darstellen und somit nur einen indirekten Blick auf die Komplexität des Systems bieten \citep[19]{Sinnemäki2011}. Es stellt sich also die Frage „what the application of a mathematical algorithm on linguistic products (texts) can reveal about the complexities of the underlying systems (grammar, lexicon) that are needed to produce these texts“ \citep[28]{Miestamo2008}.\\

\item[Zusammenfassung:] Wie dargestellt wurde, kann keine der bisherigen Messmethode in dieser Arbeit übernommen werden. Zusammenfassen lassen sich folgende Gründe nennen:

\begin{itemize}
\item 
Die Messmethode ist nicht auf die Nominalflexion übertragbar (z.\,B.\ \citealt{Shosted2006}, \citealt{Garzonio2016}).
\item 
Die Messmethode sammelt, beschreibt und vergleicht nur einzelne Phänomene (z.\,B.\ \citealt{McWhorter2001}, \citealt{Schreier2016}). Dieses Vorgehen kann zwar erste Hinweise geben, ermöglicht aber keine systematische Messung struktureller Komplexität.
\item 
Mit der Quantifizierung der Präsenz/Absenz einer Kategorie kann strukturelle Komplexität nur grob gemessen werden. Zudem funktioniert dieses Verfahren bei feinen Unterschieden zwischen eng verwandten Sprachen und Varietäten nicht (z.\,B.\ \citealt{Nichols2009}).
\item 
Die korpusbasierten Methoden sind zu oberflächlich, d.h., es können keine präzisen und detaillierten Aussagen über Zusammenhänge oder eventuelle Ausgleichsmechanismen innerhalb des Systems (z.\,B.\ Flexion) gemacht werden (z.\,B.\ \citealt{EhretSzmrecsanyi2016}).
\end{itemize}

\noindent
In der vorliegenden Arbeit soll die Komplexität der Nominalflexion in deutschen Varietäten gemessen werden. Die Messmethode muss also folgende Kriterien erfüllen:

\begin{itemize}
\item 
Die Messmethode muss strukturelle Komplexität quantifizieren.
\item 
Die Messmethode soll auf die Flexionsmorphologie anwendbar sein, und zwar unabhängig von der Wortart. Dies gilt innerhalb der Nominalflexion, z.\,B.\ für \isi{Substantive}, Artikel etc. Sie soll aber grundsätzlich die Komplexität der gesamten Flexionsmorphologie messen können, also auch der Verbalflexion. Denn nur mit einer einheitlichen Messmethode kann zukünftig die Komplexität der ganzen Flexionsmorphologie ermittelt werden.
\item 
Die Messmethode muss kleinste Unterschiede erfassen und quantifizieren. Nur so können diese zwischen eng verwandten Sprachen und Varietäten gemessen werden.
\item 
Die Messmethode muss auch in der Flexion häufig auftretende Phänomene erfassen, wie \isi{Synkretismen}, Allomorphie, Mehrfachausdruck an einem Wort, Stammalternationen (z.\,B.\ \isi{Umlaut}, Ablaut, Subtraktion), \isi{Flexionsklassen}, freie Varianten etc. Inwiefern diese Phänomene die strukturelle Komplexität erhöhen oder senken, soll theoretisch hergeleitet werden, um eine möglichst hohe Objektivität zu gewährleisten. Die Messmethode muss diese Herleitung quantifizieren.
\end{itemize}
\end{description}

\subsection{Methode zur Komplexitätsmessung in der Flexion}\label{4.3.2}

Im vorangehenden Kapitel wurde gezeigt, dass keine der bisherigen Messmethoden für die hier untersuchten Varietäten verwendet werden kann. Entweder quantifizieren die Messmethoden nicht systematisch genug oder sie können kleinste Unterschiede stark flektierender Varietäten nicht erfassen (vgl. \sectref{4.3.1.2}). In \sectref{4.1.3.2} wurde dargestellt, dass die Flexion in den RRs zu verorten ist, folglich kann die Komplexität in der Flexion anhand der Anzahl RRs gemessen werden. Davon wurde in \sectref{4.2} abgeleitet, was das System der Flexion mehr oder weniger komplex macht. In der Folge wird zuerst vorgestellt, wie hier Komplexität in der Flexion von nominalen Wortarten gemessen wird und dann, welches die Vorteile dieser Messmethode sind.

\subsubsection{Messmethode} Gemessen werden folgende nominale Wortarten, die in sechs Kategorien eingeteilt sind: Substantive (Kategorie A), starke und schwache Adjektive (Kategorie B), Personalpronomen (Kategorie C), Interrogativpronomen \textit{wer}/\textit{was} (Kategorie D), bestimmter Artikel und einfaches Demonstrativpronomen \textit{der} (Kategorie E), unbestimmter Artikel und Possessivpronomen (Kategorie F). Die Wahl für diese Wortarten fiel vorwiegend aus praktischen Gründen. Denn es handelt sich dabei um jene Wortarten, die in allen Grammatiken beschrieben werden. Es wäre sicher gerade für die Nicht-Stan\-dard\-va\-ri\-e\-tä\-ten wünschenswert gewesen, z.\,B.\ auch die Numeralia zu berücksichtigen, die in vielen alemannischen Dialekten flektiert werden. Da aber nicht für jede untersuchte Varietät eine Beschreibung der Numeralia zur Verfügung steht, wäre die Gesamtkomplexität der Varietäten nicht mehr vergleichbar, hätte man auch die diese berücksichtigt.

Die Grammatiken beschreiben nicht nur verschiedene Wortarten, sondern sie bauen die Paradigmen auch nach unterschiedlichen Kriterien auf. Würde man also die Paradigmen so übernehmen, wie sie in den grammatischen Beschreibungen stehen, könnte man die Varietäten nicht vergleichen. Deswegen werden zwar die Informationen aus den Grammatiken entnommen. Anhand dieser Informationen werden aber nach einheitlichen Kriterien Paradigmen erstellt, damit sie dann auch verglichen werden können. Die Details dazu werden im \chapref{5} vorgestellt. Das allgemeinste Kriterium jedoch ist, dass die Paradigmen auf ein Minimum reduziert werden, d.h., es werden nur jene Unterscheidungen gemacht, die unbedingt nötig sind. Beispielsweise unterscheiden die alemannischen Dialekte \isi{Kasus} an den Pronomen und Determinierern, aber die meisten nicht am \isi{Substantiv}. In diesen Dialekten hat also im Substantivparadigma jede \isi{Flexionsklasse} nur zwei Zellen, und zwar eine für den Singular und eine für den Plural. Dieses Vorgehen hat zwei Vorteile, die miteinander zusammenhängen. Erstens, wenn jedes Paradigma auf sein Minimum reduziert wird, sind diese Paradigmen auch miteinander vergleichbar. Zweitens soll hier absolute Komplexität gemessen werden. Nach \citet[24]{Miestamo2008} kann die absolute Komplexität eines linguistischen Phänomens durch die Länge der Beschreibung dieses linguistischen Phänomens gemessen werden. Dabei gilt, dass je kürzer diese Beschreibung ausfällt, desto weniger komplex ist dieses Phänomen. Auf das Modell übertragen, das hier verwendet wird, heißt das, dass je weniger RRs für ein Paradigma benötigt werden, desto weniger komplex ist dieses Paradigma. Um dies zu ermöglichen und die Anzahl RRs der verschiedenen Varietät vergleichen zu können, müssen aber auch die Informationen aus den Grammatiken zuerst auf ein Minimum reduziert werden. Nur so kann mit den RRs gemessen werden, wie stark ein Paradigma komprimiert werden kann.

Oben wurden die sechs Kategorien vorgestellt, in die die untersuchten Wortarten eingeteilt sind. Die Kategorien E und F beinhalten je zwei Wortarten: \isi{bestimmter Artikel} und einfaches \isi{Demonstrativpronomen} \textit{der} (Kategorie E), \isi{unbestimmter Artikel} und \isi{Possessivpronomen} (Kategorie F). Dass gerade diese Wortarten in die zwei Kategorien eingeteilt werden, hat zwei Gründe. Erstens weisen der \isi{bestimmte Artikel} und das einfache \isi{Demonstrativpronomen} auf der einen Seite und der \isi{unbestimmte Artikel} und das \isi{Possessivpronomen} auf der anderen Seite jeweils die größten Ähnlichkeiten in ihrer Flexionsmorphologie auf. In der deutschen Standardsprache geht das sogar so weit, dass die Paradigmen vollständig zusammenfallen (mit der Ausnahme, dass das \isi{Possessivpronomen} einen Plural hat, der \isi{unbestimmte Artikel} nicht). Im vorangehenden Absatz wurde argumentiert, dass die Paradigmen auf ein Minimum komprimiert werden müssen. Dies ist am besten zu erreichen, indem ähnliche oder gar gleiche Paradigmen in derselben Kategorie stehen. Wichtig ist jedoch auch hier, dass für alle Varietäten die gleichen Kategorien gebildet werden, damit die Kategorien über die Varietäten hinweg miteinander verglichen werden können. Der zweite Grund für diese Einteilung findet sich in der diachronen Entwicklung. Erstens ist der \isi{bestimmte Artikel} aus dem einfachen \isi{Demonstrativpronomen} entstanden \citep[23]{Schrodt2004}. Zweitens entwickelte sich die Flexion des \isi{unbestimmten Artikels} und des \isi{Possessivpronomens} parallel. Während im Althochdeutschen das Zahlwort und das \isi{Possessivpronomen} stark flektiert wurden \citep[234, 245]{Braune2004}, konnte im Mittelhochdeutschen der \isi{unbestimmte Artikel} stark \citep[217]{Paul2007}, das \isi{Possessivpronomen} stark und schwach flektiert werden \citep[216, 369]{Paul2007}. Außerdem sind im Mittelhochdeutschen der Nominativ Singular aller \isi{Genera}, der Akkusativ Singular Neutrum und oft auch der Akkusativ Singular Feminin in beiden Wortarten endungslos \citep[216-217]{Paul2007}. In der modernen deutschen Standardsprache flektieren beide Wortarten stark. Davon weichen der Nominativ Singular Maskulin und Neutrum und der Akkusativ Singular Neutrum ab, die endungslos sind (ähnlich wie im Mittelhochdeutschen), wie auch der Genitiv Singular Maskulin und Neutrum, der auf -\textit{əs} endet \citep[176]{Eisenberg2006}. Dies gilt ebenfalls für beide Wortarten.

Bei der Art der Flexion, die in den hier untersuchten Varietäten vorkommt, handelt es sich um \isi{Suffixe} und Wur\-zel-/Stamm\-al\-ter\-na\-tio\-nen (z.\,B.\ \isi{Umlaut}). In den Abschnitten \sectref{4.1.3.2} und \sectref{4.1.3.3} wurde gezeigt, dass diese durch RRs erfasst werden können. Die Höhe der Komplexität der Nominalflexion von allen Kategorien kann also durch die Anzahl RRs berechnet werden. Dabei gilt, je mehr RRs gezählt werden, desto höher ist die Komplexität. Eine Ausnahme bilden die \isi{Substantive}. Zwar werden die \isi{Suffixe} und \isi{Wurzel}/-Stammalternationen der \isi{Substantive} auch anhand der RRs abgebildet. Bei den \isi{Substantiven} kommen aber noch die \isi{Flexionsklassen} dazu. In dieser Arbeit wird \isi{Flexionsklasse} als eine Art Instruktion verstanden, die die verschiedenen RRs miteinander kombiniert. Zum Beispiel weisen das Alemannische von Zürich und des Sensebezirks acht RRs auf, aber das Alemannische von Zürich hat acht \isi{Flexionsklassen} und das Alemannische des Sensebezirks zehn \isi{Flexionsklassen}. Das Alemannische des Sensebezirks enthält also zwei Kombinationen mehr als das Alemannische von Zürich bei gleich vielen RRs, dieselbe Größe des Inventars an RRs wird im Alemannischen des Sensebezirks öfter kombiniert als im Alemannischen von Zürich. Um die Komplexität der \isi{Substantive} adäquat zu erfassen, müssen also neben der Anzahl RRs auch die Anzahl \isi{Flexionsklassen} berücksichtigt werden. Es stellt sich nun die Frage, wie die RRs und die \isi{Flexionsklassen} miteinander verrechnet werden können. Logisch möglich sind zwei Operationen, nämlich Addition und Multiplikation. Für die Addition sprechen zwei Beobachtungen. Erstens werden die RRs miteinander addiert, was auf die \isi{Flexionsklassen} übertragen werden soll, damit mit einer einheitlichen Messmethode die Vergleichbarkeit gewahrt bleibt. Zweitens würde bei der Multiplikation von RRs und \isi{Flexionsklassen} die Komplexität der \isi{Substantive} im Vergleich zu den anderen Kategorien überproportional hoch.

Neben den RRs für die \isi{Suffixe} werden auch die RRs für die Wur\-zel-/Stamm\-al\-ter\-na\-tio\-nen gezählt. Eine Ausnahme bildet die RR \textit{Identity Function Default} (RR \REF{ex:key:26}, \sectref{4.1.3.2}). Sie definiert, dass, wenn in einem beliebigen \isi{Block} keine RR gefunden wird, mit der \isi{Wurzel} nichts passiert. Dies kommt gerade in der Substantivflexion moderner deutscher Varietäten besonders häufig vor, z.\,B.\ \textit{Tag} (Nominativ, Akkusativ, Dativ Singular). In \sectref{4.2} wurde erörtert, dass diese RR in allen Varietäten anzunehmen ist, weshalb sie keine Varietät mehr oder weniger komplex macht. Ob sie also bei der Komplexitätsmessung mitgezählt wird oder nicht, macht keinen Unterschied. Deshalb wird sie der Einfachheit halber in der Folge bei der Komplexitätsmessung nicht berücksichtigt.

Praktisch wird nun so vorgegangen: Für jede Kategorie sind die RRs, für die \isi{Substantive} zusätzlich die \isi{Flexionsklassen} zu finden. Die Anzahl RRs für eine bestimmte Kategorie entspricht der Komplexität dieser Kategorie. Hat also z.\,B.\ eine Kategorie 10 RRs, beträgt ihre Komplexität 10. Weist eine Kategorie einer Varietät A 10 RRs auf und dieselbe Kategorie einer Varietät B 20 RR, dann ist jene der Varietät B komplexer als jene der Varietät A. Nachdem die Komplexität jeder Kategorie berechnet wurde, kann die Gesamtkomplexität der Nominalflexion ermittelt werden. Die Gesamtkomplexität der Nominalflexion einer Varietät wird durch die Addition der Komplexität aller Kategorien definiert: Gesamtkomplexität der Nominalflexion = Komplexität der \isi{Substantive} (Kategorie A) + Komplexität der starken und schwachen \isi{Adjektive} (Kategorie B) + Komplexität des \isi{Personalpronomens} (Kategorie C) + Komplexität des \isi{Interrogativpronomens} (Kategorie D) + Komplexität des \isi{bestimmten Artikels} und des einfachen \isi{Demonstrativpronomens} (Kategorie E) + Komplexität des \isi{unbestimmten Artikels} und des \isi{Possessivpronomens} (Kategorie F).

\subsubsection{Vorteile} Diese Messmethode zeichnet sich durch vier Vorteile aus: Ein einheitliches und sprachunabhängiges Messgerät, das quantifiziert und auch kleinste Unterschiede messen kann. Erstens ist die Einheitlichkeit, die durch die Messung anhand RRs gewährleistet wird, wohl der wichtigste Vorteil. Denn nur so können Wortarten oder Kategorien innerhalb einer Varietät und zwischen verschiedenen Varietäten verglichen werden, nur so ist ein Vergleich der Komplexität der Nominalflexion zwischen unterschiedlichen Varietäten möglich. Des Weiteren kann die Gesamtkomplexität der Nominalflexion nur ermittelt werden, indem die Komplexität jeder Wortart oder Kategorie mit demselben Messgerät berechnet wird. Einerseits geht es also um die Vergleichbarkeit der Komplexität von Wortarten und der gesamten Nominalflexion innerhalb einer Varietät und zwischen unterschiedlichen Varietäten, andererseits um die Möglichkeit, Wortarten miteinander zu verrechnen, um die Gesamtkomplexität ermitteln zu können. Zweitens ist diese Messmethode auf die gesamte Flexionsmorphologie übertragbar. Damit ist gemeint, a) dass sie z.\,B.\ auch auf die Verbflexion übertragen werden kann, b) dass es möglich ist, damit auch andere Flexionstypen zu erfassen, z.\,B.\ Präfigierung, Infigierung etc., c) dass diese Methode nicht nur die Komplexität in der Flexionsmorphologie des Deutschen berechnen kann, sondern im Prinzip für jede flektierende Sprache verwendet werden kann. Drittens ermöglicht es diese Messmethode, auch kleinste Unterschiede in der Flexion zu messen. Sie ist also eine sehr genaue Methode, mit der auch eng verwandte Varietäten untersucht werden können. Viertens ist sie klar quantifizierend, d.h., es können problemlos Aussagen gemacht werden, wie z.\,B.\ eine Sprache mit 150 RRs ist komplexer als jene mit 100 RRs.