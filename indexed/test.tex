%Questions to Language Science Press

\textbf{Annotierte LFG Bäume:} \\
%Leider entspricht dieser Baum überhaupt nicht dem, wie er aussehen soll (siehe an E-Mail angehängte Word-Datei). Wie können annotierte LFG Bäume in Latex geschrieben werden? 
\begin{forest} for tree={align=left,base=top}
	[S\textsubscript{f1}
    	[(f\textsubscript{1} SUBJ) {=} f\textsubscript{2}\\NP\textsubscript{f2}
        [f\textsubscript{2} {=} f\textsubscript{4}\\N\textsubscript{f4}
        [Max\\
        (f\textsubscript{4} NUM) {=} SG \\
        (f\textsubscript{4} PRED) {=} ʻMaxʼ]]]
        [(f\textsubscript{1} {=} f\textsubscript{3}\\
        VP \textsubscript{f3}
        [f\textsubscript{3} {=} f\textsubscript{5}\\V\textsubscript{f5}
        [live   s \textsubscript{f6}\\
        (f\textsubscript{5} PRED) {=} ʻlive ˂…˃ʼ\\
        (f\textsubscript{5} TENSE) {=} PRES\\
        (f\textsubscript{5} SUBJ) {=} f\textsubscript{6}\\
        (f\textsubscript{6} PERS) {=} 3\\
        (f\textsubscript{6} NUM) {=} SG]]]
        ]
\end{forest}
\\ \\

\textbf{AVM:}\\
%Im Language Science Press template bleibt in AVMs alles auf einer Zeile. In anderen Templates war dies kein Problem. Hier habe ich das Beispiel aus Ihrem Manual kopiert.
\begin{avm}
\[phon & \< {\itshape procupine} \>\\
feat-a & \@10
\]
\end{avm} 
\\
\\

\textbf{Griechisches Wort:} \\
%Auf dem Eta soll ein akuter Akzent stehen. \'{\eta} funktioniert leider nicht.
χoɩν\'η
\\
\\

\textbf{In-text citations:} \\
%Im Text sollte dies wie folgt stehen: (Historisches Lexikon der Schweiz 2011, Gotthardpass) bzw. (Historisches Lexikon der Schweiz 2011, Altdorf). Ist dies möglich, und wenn ja, wie? Ist dies von Language Science Press erlaubt, und falls nicht, wie hätte die Language Science Press dies gerne?
% @electronic{LexG2011,
%   Author = {{Historisches Lexikon der Schweiz}},
%   title = {Gotthardpass},
%   Url = {http://www.hls-dhs-dss.ch},
%   Urldate = {2015-10-09},
%   Year = {2011}
% }
% \\
% @electronic{LexA2011,
%   Author = {{Historisches Lexikon der Schweiz}},
%   title = {Altdorf},
%   Url = {http://www.hls-dhs-dss.ch},
%   Urldate = {2015-10-09},
%   Year = {2011}
% }
\\
\\
\textbf{Bibliografie:} \\
%Wie muss der Bibliografie-Eintrag in Latex aussehen, damit man folgendes Resultat erhält? Es geht hier um a) Facsimile, b) Repr. der Ausg. Prague 1929., c) Masch.
Humboldt, Wilhelm. 1836. Über die Verschiedenheit des menschlichen Sprachbaues und ihren Einfluss auf die geistige Entwickelung des Menschengeschlechts. Berlin: Dümmler (Facsimile 1960).\\
Jakobson, Roman. 1929/1968. Remarques sur l’évolution phonologique du russe compare à celle des autres langues slaves. Repr. der Ausg. Prague 1929. Nendel: Kraus.\\
Perinetto, Renato. 1981. Eischemer’s Büjie. Masch. Issime.
\\
\\
\textbf{Appendix:} \\
%Gibt es ein Styefile für den Anhang? Im Text wie auch in Literaturverzeichnis sollte es ein Kapitel "Anhang" geben und anschließend Kapitel mit dem Titel "Anhang 1: Paradigmen", "Anhang 2: \isi{Realisierungsregeln}", "Anhang 3: Verzeichnis der Tabellen", etc. Dies sowohl im Literaturverzeichnis als auch im Text selbst.
Anhang\\
Anhang 1: Paradigmen\\
Anhang 2: \isi{Realisierungsregeln}

\nocite{Humboldt1836,Perinetto1981,Jakobson1929}

