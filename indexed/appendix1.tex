\chapter{Paradigmen}\label{appendix1}

\section{Substantive}

% \textbf{Paradigma 1: Althochdeutsches \isi{Substantiv} \citep[183-217]{Braune2004}}

\begin{table}[H]
	\resizebox{\textwidth}{!}{%
	\begin{tabular}{rlllllllll}
	 	\caption{Althochdeutsches Substantiv \citep[183-217]{Braune2004}\label{table1}}\\
		\lsptoprule
		& \multicolumn{5}{c}{\textsc{sg}} & \multicolumn{4}{c}{\textsc{pl}}\\\cmidrule(lr){2-6}\cmidrule(lr){7-10}
		\textsc{fk} & \NOM    & \AKK & \DAT & \GEN & \INSTR                  &	Nom & \AKK & \DAT & \GEN\\\midrule
		1  & tag    & tag & tag-e & tag-es & tag-u                       &        tag-a & tag-a & tag-um & tag-o\\
		2  & hirt-i & hirt-i & hirt-e & hirt-es & hirt-u                 & hirt-a & hirt-a & hirt-um & hirt-o\\
		3  & gast   & gast & gast-e & gast-es & gast-u                   & gest-i & gest-i & gest-im & gest-o\\
		4  & win-i  & win-i & win-e & win-es &                           & win-i  & win-i & win-im & win-o\\
		5  & sit-u  & sit-u & sit-e & sit-es & sit-u                     & sit-i & sit-i & sit-im & sit-o\\
		6  & han-o  & han-un & han-in & han-in &                         & han-un & han-un & han-\=om & han-\=ono\\
		7  & fater  & fater & fater-ø/-e & fater-ø/-es &                 & fater-a & fater-a & fater-um & fater-o\\
		8  & wort   & wort & wort-e & wort-es & wort-u                   & wort & wort & wort-um & wort-o\\
		9  & lamb   & lamb & lamb-e & lamb-es & lamb-u                   & lemb-ir & lemb-ir & lemb-ir-um & lemb-ir-o\\
		10 & kunn-i & kunn-i & kunn-e & kunn-es & kunn-u                 & kunn-i & kunn-i & kunn-im & kunn-o\\
		11 & herz-a & herz-a & herz-in & herz-in &                       & herz-un & herz-un & herz-\=om & herz-\=ono\\
		12 & geb-a  & geb-a & geb-u & geb-a &                            & geb-a & geb-a & geb-\=om & geb-\=ono\\
		13 & kunningin  & kunninginn-a & kunninginn-u & kunninginn-a &   & kunninginn-a & kunninginn-a & kunninginn-\=om & kunninginn-\=ono\\
		14 & anst       & anst & enst-i & enst-i &                       & enst-i & enst-i & enst-im & enst-o\\
		15 & zung-a     & zung-un & zung-un & zung-un &                  & zung-\=un & zung-\=un & zung-\=om & zung-\=ono\\
		16 & hoh-\=i    & hoh-\=i & hoh-\=i & hoh-\=i &                  & hoh-\=i & hoh-\=i & hoh-\=im & hoh-\=ino\\
		17 & muoter     & muoter & muoter & muoter &                     & muoter & muoter & muoter-um & muoter-o\\
		18 & naht       & naht & naht & naht &                           & naht & naht & naht-um & naht-o\\
		19 & chindil\=i & chindil\=i & chindil\=in-e & chindil\=in-es &  &  chindil-iu & chindil-iu & chindil\=in-um & chindil\=in-o\\
\lspbottomrule
	\end{tabular}}
\end{table}

% \textbf{Paradigma 2: Mittelhochdeutsches \isi{Substantiv} \citep[183-199]{Paul2007}}

\begin{table}[H]
	\caption{Mittelhochdeutsches Substantiv \citep[183-199]{Paul2007}}\label{table2}
	\resizebox{\textwidth}{!}{\begin{tabular}{rllllllll}
		\lsptoprule
		& \multicolumn{4}{c}{\textsc{sg}}  &  \multicolumn{4}{c}{\textsc{pl}} \\\cmidrule(lr){2-5}\cmidrule(lr){6-9}
		\textsc{fk} & \NOM & \AKK & \DAT & \GEN & \NOM & \AKK & \DAT & \GEN\\\midrule
		1 & tak & tak & tag-ə & tag-əs & tag-ə & tag-ə & tag-ən & tag-ə\\
		2 & gast & gast & gast-ə & gast-əs & gest-ə & gest-ə & gest-ən & gest-ə\\
		3 & wort & wort & wort-ə & wort-əs & wort & wort & wort-ən & wort-ə\\
		4 & lamp & lamp & lamb-ə & lamb-əs & lemb-ər & lemb-ər & lemb-ər-ən & lemb-ər\\
		5 & botə & botə-n & botə-n & botə-n & botə-n & botə-n & botə-n & botə-n\\
		6 & hɛrzə & hɛrzə & hɛrzə-n & hɛrzə-n & hɛrzə-n & hɛrzə-n & hɛrzə-n & hɛrzə-n\\
		7 & gɛbə & gɛbə & gɛbə & gɛbə & gɛbə & gɛbə & gɛbə-n & gɛbə-n\\
		8 & man & man & man & man & man & man & man & man\\
		9 & kraft & kraft & kreft-ə/kraft & kreft-ə/kraft & kreft-ə & kreft-ə & kreft-ən & kreft-ə\\
		10 & zungə & zungə & zungə & zungə & zungə-n & zungə-n & zungə-n & zungə-n\\
		11 & muotər & muotər & muotər & muotər & müətər & müətər & müətər-ən & müətər\\
		\lspbottomrule
	\end{tabular}}
\end{table}

% \textbf{Paradigma 3: Substantivflexion der deutschen Standardsprache \citep[158-169]{Eisenberg2006}}

\begin{table}[H]
	\caption{Substantivflexion der deutschen Standardsprache \citep[158-169]{Eisenberg2006}}\label{table3}
	\resizebox{\textwidth}{!}{\begin{tabular}{rllllllll}
		\lsptoprule
		& \multicolumn{4}{c}{\textsc{sg}}  &  \multicolumn{4}{c}{\textsc{pl}} \\\cmidrule(lr){2-5}\cmidrule(lr){6-9}
		\textsc{fk} & \NOM & \AKK & \DAT & \GEN & \NOM & \AKK & \DAT & \GEN\\\midrule
		1 & gast & gast & gast & gast-əs & gäst-ə & gäst-ə & gäst-ən & gäst-ə\\
		2 & tag & tag & tag & tag-əs & tag-ə & tag-ə & tag-ən & tag-ə\\
		3 & wald & wald & wald & wald-əs & wäld-ər & wäld-ər & wäld-ər-n & wäld-ər\\
		4 & matrosə & matrosə-n & matrosə-n & matrosə-n & matrosə-n & matrosə-n & matrosə-n & matrosə-n\\
		5 & staat & staat & staat & staat-s & staat-ən & staat-ən & staat-ən & staat-ən\\
		6 & blumə & blumə & blumə & blumə & blumə-n & blumə-n & blumə-n & blumə-n\\
		7 & stadt & stadt & stadt & stadt & städt-ə & städt-ə & städt-ən & städt-ə\\
		8 & muttər & muttər & muttər & muttər & müttər & müttər & müttər-n & müttər\\
		9 & zoo & zoo & zoo & zoo-s & zoo-s & zoo-s & zoo-s & zoo-s\\
		10 & pizza & pizza & pizza & pizza & pizza-s & pizza-s & pizza-s & pizza-s\\
		& \multicolumn{4}{c}{Possessiv-S} & \multicolumn{4}{c}{} \\
		\lspbottomrule
	\end{tabular}}
\end{table}

% \textbf{Paradigma 4: Substantivflexion von Issime \citep[144-205]{Zürrer1999}}

\begin{table}[H]
	\caption{Substantivflexion von Issime \citep[144-205]{Zürrer1999}}\label{table4}
	\resizebox{\textwidth}{!}{\begin{tabular}{rllllllll}
\lsptoprule
& \multicolumn{4}{c}{\textsc{sg}} & \multicolumn{4}{c}{\textsc{pl}} \\\cmidrule(lr){2-5}\cmidrule(lr){6-9}
		\textsc{fk} & \NOM & \AKK & \DAT & \GEN & \NOM & \AKK & \DAT & \GEN\\\midrule
		1 & weg & weg & weg & weg-sch & weg-a & weg-a & weg-e & weg-u\\
		2 & uav-e & uav-e & uav-e & uav-endsch & uav-n-a & uav-n-a & uav-n-e & uav-n-u\\
		3 & noam-e & noam-e & noam-e & noam-endsch & noam-i & noam-i & noam-e & noam-u\\
		4 & hoan-u & hoan-u & hoan-e & hoan-endsch & hoan-i & hoan-i & hoan-u & hoan-u\\
		5 & vus & vus & vus & vus-sch & vüs & vüs & vüs-e & vüs-u\\
		6 & att-u & att-u & att-e & att-e & att-i & att-i & att-e & att-e\\
		7 & schu & schu & schu & schu-sch & schu & schu & schun-e & schun-u\\
		8 & sia & sia & sia & sia-sch & sia-w-a & sia-w-a & sia-w-e & sia-w-u\\
		9 & bet & bet & bet & bet-sch & bet-i & bet-i & bet-u & bet-u\\
		10 & lam & lam & lam & lam-sch & lam-er & lam-er & lam-er-e & lam-er-u\\
		11 & lan & lan & lan & lan-sch & len-er & len-er & len-er-e & len-er-u\\
		12 & matt-u & matt-u & matt-u & matt-u & matt-i & matt-i & matt-u & matt-u\\
		13 & mum-a & mum-a & mum-u & mum-u & mum-i & mum-i & mum-u & mum-u\\
		14 & aksch & aksch & aksch & aksch & aksch-i & aksch-i & aksch-u & aksch-u\\
		15 & schuld & schuld & schuld & schuld & schuld-in-i & schuld-in-i & schuld-in-u & schuld-in-u\\
		16 & nacht & nacht & nacht & nacht & necht-in-i & necht-in-i & necht-in-u & necht-in-u\\
		17 & han & han & han & han & hen & hen & hen-e & hen-u\\
		18 & geiss & geiss & geiss & geiss & geiss & geiss & geiss-e & geiss-u\\
		\lspbottomrule
	\end{tabular}}
\end{table}

% \textbf{Paradigma 5: Substantivflexion von Visperterminen \citep[119-134]{Wipf1911}}

\begin{table}[H]
	\caption{Substantivflexion von Visperterminen \citep[119-134]{Wipf1911}}\label{table5}
	\resizebox{\textwidth}{!}{\begin{tabular}{rllllllll}
\lsptoprule
& \multicolumn{4}{c}{\textsc{sg}}  & \multicolumn{4}{c}{\textsc{pl}} \\\cmidrule(lr){2-5}\cmidrule(lr){6-9}
		\textsc{fk} & \NOM & \AKK & \DAT & \GEN & \NOM & \AKK & \DAT & \GEN\\\midrule
		1 & tag & tag & tag & tag-sch & tag-a & tag-a & tag-u & tag-o\\
		2 & chopf & chopf & chopf & chopf-sch & chepf & chepf & chepf-u & chepf-o\\
		3 & ar-o & ar-o & ar-u & ar-u & ar-m-a & ar-m-a & ar-m-u & ar-m-o\\
		4 & santim & santim & santim & santim-sch & santim & santim & santim & santim\\
		5 & han-o & han-o & han-u & han-u & han-e & han-e & han-u & han-o\\
		6 & bog-o & bog-o & bog-u & bog-u & beg-e & beg-e & beg-u & beg-o\\
		7 & sɛnn-o & sɛnn-o & sɛnn-u & sɛnn-u & sɛnn-u & sɛnn-u & sɛnn-u & sɛnn-o\\
		8 & jar & jar & jar & jar-sch & jar & jar & jar-u & jar-o\\
		9 & hor-u & hor-u & hor & hor-sch & hor-u & hor-u & hor-n-u & hor-o\\
		10 & chrut & chrut & chrut & chrut-sch & chrit-er & chrit-er & chrit-er-u & chrit-er-o\\
		11 & lamm & lamm & lamm & lamm-sch & lamm-er & lamm-er & lamm-er-u & lamm-er-o\\
		12 & ber & ber & ber & ber-sch & ber-i & ber-i & ber-u & ber-o\\
		13 & öig & öig & öig & öig-sch & öig-u & öig-u & öig-u & öig-o\\
		14 & farb & farb & farb & farb & farb-e & farb-e & farb-u & farb-o\\
		15 & bon & bon & bon & bon & bon-a & bon-a & bon-u & bon-o\\
		16 & sach & sach & sach & sach & sach-u & sach-u & sach-u & sach-o\\
		17 & mus & mus & mus & mus & mis & mis & mis-u & mis-o\\
		18 & tsung-a & tsung-a & tsung-u & tsung-u & tsung-e & tsung-e & tsung-u & tsung-o\\
		\lspbottomrule
	\end{tabular}}
\end{table}

% \textbf{Paradigma 6: Substantivflexion von Jaun \citep[255-272]{Stucki1917}}

\begin{table}[H]
	\caption{Substantivflexion von Jaun \citep[255-272]{Stucki1917}}\label{table6}
	\resizebox{\textwidth}{!}{\begin{tabular}{rllllllll}
\lsptoprule
& \multicolumn{4}{c}{\textsc{sg}} &  \multicolumn{4}{c}{\textsc{pl}} \\\cmidrule(lr){2-5}\cmidrule(lr){6-9}
		\textsc{fk} & \NOM & \AKK & \DAT & \GEN & \NOM & \AKK & \DAT & \GEN\\\midrule
		1 & achər & achər & achər & achər-s & æchər-a & æchər-a & æchər-ə & æchər-ə\\
		2 & apətiəkər & apətiəkər & apətiəkər & apətiəkər-s & apətiəkər & apətiəkər & apətiəkər{}-ə & apətiəkər{}-ə\\
		3 & gascht & gascht & gascht & gascht-s & gɛscht & gɛscht & gɛscht{}-ə & gɛscht{}-ə\\
		4 & h\=ar & h\=ar & h\=ar & h\=ar{}-s & h\=ar{}-ən-i & h\=ar{}-ən-i & h\=ar{}-ən-ə & h\=ar{}-ən-ə\\
		5 & buəb & buəb & buəb & buəb-s & buəb-ə & buəb-ə & buəb-nə & buəb-nə\\
		6 & chaschtə & chaschtə & chaschtə & chaschtə-s & chæscht-ə & chæscht-ə & chæschtə-nə & chæschtə-nə\\
		7 & chund & chund & chund & chund-s & chund-ə & chund-ə & chund-ə & chund-ə\\
		8 & sch\=af & sch\=af & sch\=af & sch\=af{}-s & sch\=af & sch\=af & sch\=af{}-ə & sch\=af{}-ə\\
		9 & h\=us & h\=us & h\=us & h\=us{}-s & hǖs-ər & hǖs-ər & hǖs-ər-ə & hǖs-ər-ə\\
		10 & lamp & lamp & lamp & lamp-s & lamp-ər & lamp-ər & lamp-ər-ə & lamp-ər-ə\\
		11 & bet & bet & bet & bet-s & bet-i & bet-i & bet-ə & bet-ə\\
		12 & fr\=ag & fr\=ag & fr\=ag & fr\=ag & fr\=ag{}-i & fr\=ag{}-i & fr\=ag{}-ə & fr\=ag{}-ə\\
		13 & f\=uscht & f\=uscht & f\=uscht & f\=uscht & fǖscht & fǖscht & fǖscht{}-ə & fǖscht{}-ə\\
		14 & tǖr & tǖr & tǖr & tǖr & tǖr{}-ən-i & tǖr{}-ən-i & tǖr{}-ən-ə & tǖr{}-ən-ə\\
		15 & tsung-a & tsung-a & tsung-ə & tsung-ə & tsung-i & tsung-i & tsung-ə & tsung-ə\\
		16 & matt-a & matt-a & matt-ə & matt-ə & matt-ən-i & matt-ən-i & matt-ən-ə & matt-ən-ə\\
		\lspbottomrule
	\end{tabular}}
\end{table}

% \textbf{Paradigma 7: Substantivflexion des Sensebezirks \citep[179-190]{Henzen1927}}

\begin{table}[H]
	\caption{Substantivflexion des Sensebezirks \citep[179-190]{Henzen1927}}\label{table7}
	\begin{tabular}{rll}
		\lsptoprule
		\textsc{fk} & \textsc{sg} & \textsc{pl}\\\midrule
		1 & ascht & ɛscht\\
		2 & bach & bæch\\
		3 & bærg & bærg-ə\\
		4 & pfana & pfan-ə\\
		5 & malər & malər\\
		6 & nɛts & nɛts-əni\\
		7 & blati & blatə{}-(n)i\\
		8 & rad & rɛd-ər\\
		9 & tach & tæch-ər\\
		10 & lamm & lamm-ər\\
		& \multicolumn{2}{c}{Possessiv-S}\\
		\lspbottomrule
	\end{tabular}
\end{table}


% \textbf{Paradigma 8: Substantivflexion von Uri \citep[173-185]{Clauß1929}}

\begin{table}[H]
	\caption{Substantivflexion von Uri \citep[173-185]{Clauß1929}}\label{table8}
	\begin{tabular}{rlll}
		\lsptoprule
		& \textsc{sg} & \multicolumn{2}{c}{\textsc{pl}}\\\cmidrule(lr){3-4}
		\textsc{fk} &  & \NOM\slash\AKK & \DAT\\\midrule
		1 & ɑscht & escht & escht-ɐ\\
		2 & chrɑmpf & chrampf & chrampf-ɐ\\
		3 & mɑlər & mɑlər & mɑlər-ɐ\\
		4 & chnacht & chnacht-ɐ & chnacht-ɐ\\
		5 & fɑttər & fattər-ɐ & fattər-ɐ\\
		6 & fadɐ & fad-əm & fap-m-ɐ\\
		7 & schnidəri & schnidər-ɐ & schnidər-ɐ\\
		8 & kcharli & kcharl-əs-ɐ & kcharl-əs-ɐ\\
		9 & blɑt & blet-ǝr & blet-ǝr-ɐ\\
		10 & tɑch & tach-ər & tach-ər-ɐ\\
		11 & bet & bet-i & bet-ənɐ\\
		12 & nets & nets-i & nets-ɐ\\
		& \multicolumn{3}{c}{Possessiv-S}\\
		\lspbottomrule
	\end{tabular}
\end{table}

% \textbf{Paradigma 9: Substantivflexion von Vorarlberg \citep[231-261]{Jutz1925}}

\begin{table}[H]
	\caption{Substantivflexion von Vorarlberg \citep[231-261]{Jutz1925}}\label{table9}
	\begin{tabular}{rlll}
		\lsptoprule
		& \textsc{sg} & \multicolumn{2}{c}{\textsc{pl}}\\\cmidrule(lr){3-4}
		\textsc{fk} &  & \NOM\slash\AKK & \DAT\\\midrule
		1 & schlag & schleg & \\
		2 & bǣrg & bǣrg & bǣrg-ə\\
		3 & kchind & kchind & \\
		4 & wart & wart & wart-ə\\
		5 & bett & bett-ər & \\
		6 & \=arm & \=arm-ə & \\
		7 & fuədər & füədər-ə & \\
		& \multicolumn{3}{c}{Possessiv-S}\\
		\lspbottomrule
	\end{tabular}
\end{table}

% \textbf{Paradigma 10: Substantivflexion von Zürich \citep[108-119]{Weber1987}}

\begin{table}[H]
	\caption{Substantivflexion von Zürich \citep[108-119]{Weber1987}}\label{table10}
	\begin{tabular}{rlll}\lsptoprule
		& \textsc{sg} & \multicolumn{2}{c}{\textsc{pl}}\\\cmidrule(lr){3-4}
		\textsc{fk} &  & \NOM\slash\AKK & \DAT\\\midrule
		1 & gascht & gescht & {}-ə\\
		2 & bank & bænk & {}-ə\\
		3 & h\=aggə & hȫggə & {}-ə\\
		4 & rad & red-ər & {}-ə\\
		5 & fass & fæss-ər & \\
		6 & vattər & vætter-ə & {}-ə\\
		7 & p\=ur & p\=ur-ə & {}-ə\\
		8 & fisch & fisch & {}-ə\\
		& \multicolumn{3}{c}{Possessiv-S}\\
		\lspbottomrule
	\end{tabular}
\end{table}

% \textbf{Paradigma 11: Substantivflexion von Bern \citep[82-90]{Marti1985}}

\begin{table}[H]
	\caption{Substantivflexion von Bern \citep[82-90]{Marti1985}}\label{table11}
	\begin{tabular}{rll}
		\lsptoprule
		\textsc{fk} & \textsc{sg} & \textsc{pl}\\\midrule
		1 & tannə & tannə\\
		2 & darm & dærm\\
		3 & gascht & gescht\\
		4 & tal & tæl-ər\\
		5 & gl\=as & glɛs-ər\\
		6 & h\=as & has-ə\\
		7 & tochtər & töchtər-ə\\
		& \multicolumn{2}{c}{Possessiv-S}\\
		\lspbottomrule
	\end{tabular}
\end{table}

% \textbf{Paradigma 12: Substantivflexion von Huzenbach \citep[92-98]{Baur1967}}

\begin{table}[H]
	\caption{Substantivflexion von Huzenbach \citep[92-98]{Baur1967}}\label{table12}
	\begin{tabular}{rll}
		\lsptoprule
		\textsc{fk} & \textsc{sg} & \textsc{pl}\\\midrule
		1 & schuə & schuə\\
		2 & schdal & schdel\\
		3 & wald & weld-ǝr\\
		4 & dan & dan-ǝ\\
		5 & muədər & miədər-ə\\
		6 & wiǝrde & wiǝrd-ǝnǝ\\
		& \multicolumn{2}{c}{Possessiv-S}\\
		\lspbottomrule
	\end{tabular}
\end{table}

% \textbf{Paradigma 13: Substantivflexion von Saulgau \citep[100-109]{Raichle1932}}

\begin{table}[H]
	\caption{Substantivflexion von Saulgau \citep[100-109]{Raichle1932}}\label{table13}
\begin{tabular}{rll}
	\lsptoprule
	\textsc{fk} & \textsc{sg} & \textsc{pl}\\\midrule
	1 & molr & molr\\
	2 & gascht & gescht\\
	3 & arm & ɛrm\\
	4 & bot & bot-ǝ\\
	5 & {}-le & {}-lǝ\\
	6 & blat & blet-r\\
	& \multicolumn{2}{c}{Possessiv-S}\\
	\lspbottomrule
\end{tabular}
\end{table}
% \textbf{Paradigma 14: Substantivflexion von Stuttgart \citep[149-152]{Frey1975}}

\begin{table}[H]
	\caption{Substantivflexion von Stuttgart \citep[149-152]{Frey1975}}\label{table14}
	\begin{tabular}{rll}
		\lsptoprule
		\textsc{fk} & \textsc{sg} & \textsc{pl}\\\midrule
		1 & balgǝ & balgǝ\\
		2 & dischle & dischl-ǝ\\
		3 & bal & bɛl\\
		4 & wald & wɛld-ǝr\\
		5 & mensch & mensch-ǝ\\
		\lspbottomrule
	\end{tabular}
\end{table}

% \textbf{Paradigma 15: Substantivflexion von Petrifeld \citep[59-62]{Moser1937}}

\begin{table}[H]
	\caption{Substantivflexion von Petrifeld \citep[59-62]{Moser1937}}\label{table15}
	\begin{tabular}{rlll}
		\lsptoprule
		& \multicolumn{2}{c}{\textsc{sg}} & \textsc{pl}\\\cmidrule(lr){2-3}
		\textsc{fk} & \NOM\slash\AKK & \DAT\slash\GEN & \\\midrule
		1 & hund & & hund\\
		2 & nagl & & negl\\
		3 & akr & & ɛkr\\
		4 & bek & bek-ǝ & bek-ǝ\\
		5 & fas & & fes-r\\
		6 & khar & & khɛr-ǝr\\
		7 & {}-le & & -lǝ\\
		8 & kh\=inege & &kh\=ineg-inǝ\\
		9 & khexe & & khex-ǝnǝ\\
		& \multicolumn{3}{c}{Possessiv-S} \\
		\lspbottomrule
	\end{tabular}
\end{table}

% \textbf{Paradigma 16: Substantivflexion von Elisabethtal \citep[50-52]{Žirmunskij1928/29}}

\begin{table}[H]
	\caption{Substantivflexion von Elisabethtal \citep[50-52]{Žirmunskij1928/29}}\label{table16}
	\begin{tabular}{rlll}
		\lsptoprule
		& \multicolumn{2}{c}{\textsc{sg}} & \textsc{pl}\\\cmidrule(lr){2-3}
		\textsc{fk} & \NOM & \AKK\slash\DAT & \\\midrule
		1 & bɛrg &  & bɛrg\\
		2 & gaschd &  & geschd\\
		3 & darm &  & dɛrm\\
		4 & d\=ag &  & dǣg\\
		5 & fas &  & fɛs-r\\
		6 &  &  & Sekundär (lang) + r\\
		7 & h\=as & h\=as-ɐ & h\=as-ɐ\\
		8 & schuld &  & schuld-ɐ\\
		9 & machd &  & mɛchd-ɐ\\
		10 &  &  & Sekundär (lang) + ɐ\\
		11 & hendlǝ &  & hendl-ɐ\\
		\lspbottomrule
	\end{tabular}
\end{table}

% \textbf{Paradigma 17: Substantivflexion des Kaiserstuhls \citep[359-373]{Noth1993}}

\begin{table}[H]
	\caption{Substantivflexion des Kaiserstuhls \citep[359-373]{Noth1993}}\label{table17}
	\begin{tabular}{rll}
		\lsptoprule
		\textsc{fk} & \textsc{sg} & \textsc{pl}\\\midrule
		1 & briaf & briaf\\
		2 & gumb & gimb\\
		3 & schdai & schdai-n-ər\\
		4 & grab & grab-ɐ\\
		5 & ghuchi & ghuch-ɐnɐ\\
		& \multicolumn{2}{c}{Possessiv-S}\\
		\lspbottomrule
	\end{tabular}
\end{table}

% \textbf{Paradigma 18: Substantivflexion des Münstertals \citep[40-44]{Mankel1886}}

\begin{table}[H]
	\caption{Substantivflexion des Münstertals \citep[40-44]{Mankel1886}}\label{table18}
	\begin{tabular}{rll}
		\lsptoprule
		\textsc{fk} & \textsc{sg} & \textsc{pl}\\\midrule
		1 & prief & prief\\
		2 & sɑk & sɛk\\
		3 & nɑmə & namə\\
		4 & hɑs & hɑs-ə\\
		5 & hamp & hamp-ər\\
		6 & glɑs & glɛs-ər\\
		7 & p\~{\=a}t & pain\\
		& \multicolumn{2}{c}{Possessiv-S}\\
		\lspbottomrule
	\end{tabular}
\end{table}

% \textbf{Paradigma 19: Substantivflexion von Colmar \citep[71-76]{Henry1900}}

\begin{table}[H]
	\caption{Substantivflexion von Colmar \citep[71-76]{Henry1900}}\label{table19}
	\begin{tabular}{rll}
		\lsptoprule
		\textsc{fk} & \textsc{sg} & \textsc{pl}\\\midrule
		1 & fɑtr & fatr\\
		2 & nɑil & neil\\
		3 & v\=ai & v\=ai\\
		4 & mansch & mansch-ə\\
		5 & tɑch & tech-r\\
		\lspbottomrule
	\end{tabular}
\end{table}

% \textbf{Paradigma 20: Substantivflexion des Elsass (Ebene) \citep[25-71]{Beyer1963}}

\begin{table}[H]
	\caption{Substantivflexion des Elsass (Ebene) \citep[25-71]{Beyer1963}}\label{table20}
	\begin{tabular}{rll}
		\lsptoprule
		\textsc{fk} & \textsc{sg} & \textsc{pl}\\\midrule
		1 & gɑst & gest\\
		2 & zɑl & zɑl-ə\\
		3 & müətər & miətər-ə\\
		4 & dɑch & dech-ər\\
		5 & dɑch & dech-ərə\\
		6 & hund & hung\\
		7 & hand & hæng\\
		8 & stæin & stæin\\
		\lspbottomrule
	\end{tabular}
\end{table}

\section{Stark und schwach flektierte Adjektive}

% \textbf{Paradigma 21: Starke und schwache \isi{Adjektive} im Althochdeutschen \citep[217-227]{Braune2004}}

\begin{table}[H]
	\caption{Starke und schwache Adjektive im Althochdeutschen \citep[217-227]{Braune2004}}\label{table21}
	\begin{tabular}{>{\scshape}llllllllll}
		\lsptoprule
		\multicolumn{10}{c}{{stark}} \\
		& \multicolumn{5}{c}{\textsc{sg}} &  \multicolumn{4}{c}{\textsc{pl}} \\\cmidrule(lr){2-6}\cmidrule(lr){7-10}
		& \NOM & \AKK & \DAT & \GEN & \INSTR & \NOM & \AKK & \DAT & \GEN\\\midrule
		m & {}-\=er/-ø & {}-an & {}-emu & {}-es & {}-u & {}-e & {}-e & {}-\=em & {}-ero\\
		& {}-\=er/-i & {}-an &  &  &  &  &  &  & \\
		n & {}-az/-ø & {}-az/-ø & {}-emu & {}-es & {}-u & {}-iu & {}-iu & {}-\=em & {}-ero\\
		& {}-az/-i & {}-az/-i &  &  &  &  &  &  & \\
		f & {}-iu/-ø & {}-a & {}-eru & {}-era & - & {}-o & {}-o & {}-\=em & {}-ero\\
		& {}-iu/-i & {}-a &  &  &  &  &  &  & \\\midrule
		\multicolumn{10}{c}{{schwach}}  \\
		& \multicolumn{4}{c}{\textsc{sg}}&  & \multicolumn{4}{c}{\textsc{pl}} \\\cmidrule(lr){2-5}\cmidrule(lr){7-10}
		& \NOM & \AKK & \DAT & \GEN &  & \NOM & \AKK & \DAT & \GEN\\\midrule
		m & {}-o & {}-un & {}-in & {}-in &  & {}-un & {}-un & {}-\=om & {}-\=ono\\
		n & {}-a & {}-a & {}-in & {}-in &  & {}-un & {}-un & {}-\=om & {}-\=ono\\
		f & {}-a & {}-\=un & {}-\=un & {}-\=un &  & {}-\=un & {}-\=un & {}-\=om & {}-\=ono\\
		\lspbottomrule
	\end{tabular}
\end{table}

% \textbf{Paradigma 22: Starke und schwache \isi{Adjektive} im Mittelhochdeutschen \citep[200-203]{Paul2007}}

\begin{table}[H]
	\caption{Starke und schwache Adjektive im Mittelhochdeutschen \citep[200-203]{Paul2007}}\label{table22}
	\begin{tabular}{>{\scshape}llllllllll}
		\lsptoprule
		\multicolumn{9}{c}{{stark}} \\
		& \multicolumn{4}{c}{\textsc{sg}} &  \multicolumn{4}{c}{\textsc{pl}} \\\cmidrule(lr){2-5}\cmidrule(lr){6-9}
		& \NOM & \AKK & \DAT & \GEN & \NOM & \AKK & \DAT & \GEN\\\midrule
		m & ø/-ər & {}-ən & {}-əm & {}-əs & {}-ə & {}-ə & {}-ən & {}-ər\\
		n & ø/-əs & ø/-əs & {}-əm & {}-əs & {}-iu & {}-iu & {}-ən & {}-ər\\
		f & ø/-iu & {}-ə & {}-ər & {}-ər & {}-ə & {}-ə & {}-ən & {}-ər\\\midrule
		\multicolumn{9}{c}{{schwach}}  \\
		& \multicolumn{4}{c}{\textsc{sg}}&  \\\cmidrule(lr){2-5}
		& \NOM & \AKK & \DAT & \GEN &  \\\midrule
		\textsc{m.sg.} & {}-ə & {}-ən & {}-ən & {}-ən &  &  &  & \\
		\textsc{n.sg.} & {}-ə & {}-ə & {}-ən & {}-ən &  &  &  & \\
		\textsc{f.sg.} & {}-ə & {}-ən & {}-ən & {}-ən &  &  &  & \\
		\textsc{pl} & {}-ən & {}-ən & {}-ən & {}-ən &  &  &  & \\
		\lspbottomrule
	\end{tabular}
\end{table}

% \textbf{Paradigma 23: Starke und schwache \isi{Adjektive} in der Standardsprache \citep[177-184]{Eisenberg2006}}

\begin{table}[H]
	\caption{Starke und schwache Adjektive in der Standardsprache \citep[177-184]{Eisenberg2006}}\label{table23}
	\begin{tabular}{lllll}
		\lsptoprule
		\multicolumn{5}{c}{stark}\\
		& \NOM & \AKK & \DAT & \GEN\\\midrule
		\textsc{m.sg.} & {}-ər & {}-ən & {}-əm & {}-ən\\
		\textsc{n.sg.} & {}-əs & {}-əs & {}-əm & {}-ən\\
		\textsc{f.sg.} & {}-ə & {}-ə & {}-ər & {}-ər\\
		\textsc{pl} & {}-ə & {}-ə & {}-ən & {}-ər\\\midrule
		\multicolumn{5}{c}{schwach}\\
		& \NOM & \AKK & \DAT & \GEN\\\midrule
		\textsc{m.sg.} & {}-ə & {}-ən & {}-ən & {}-ən\\
		\textsc{n.sg.} & {}-ə & {}-ə & {}-ən & {}-ən\\
		\textsc{f.sg.} & {}-ə & {}-ə & {}-ən & {}-ən\\
		\textsc{pl} & {}-ən & {}-ən & {}-ən & {}-ən\\
		\lspbottomrule
	\end{tabular}
\end{table}

% \textbf{Paradigma 24: Starke und schwache \isi{Adjektive} in Issime (\citealt[267-268]{Zürrer1999}, \citealt[90-97]{Perinetto1981})}

\begin{table}[H]
	\caption{Starke und schwache Adjektive in Issime (\citealt[267-268]{Zürrer1999}, \citealt[90-97]{Perinetto1981})}\label{table24}
	\begin{tabular}{>{\scshape}lllllllll}
		\lsptoprule
		\multicolumn{9}{c}{{stark}} \\
		& \multicolumn{4}{c}{\textsc{sg}} &  \multicolumn{4}{c}{\textsc{pl}} \\\cmidrule(lr){2-5}\cmidrule(lr){6-9}
		& \NOM & \AKK & \DAT & \GEN & \NOM & \AKK & \DAT & \GEN\\\midrule
		m & {}-e & {}-e & {}-e & {}-s & {}-ø & {}-ø & {}-ø & {}-er\\
		n & {}-s & {}-s & {}-s & {}-s & {}-i & {}-i & {}-i & {}-er\\
		f & {}-ø & {}-ø & {}-ø & {}-er & {}-ø & {}-ø & {}-ø & {}-er\\\midrule
		\multicolumn{9}{c}{{schwach}} \\		
		& \NOM & \AKK & \DAT & \GEN &  &  &  & \\\midrule
		\textsc{m.sg.} & {}-e & {}-e & {}-e & {}-e &  &  &  & \\
		\textsc{n.sg.} & {}-ø & {}-ø & {}-e & {}-e &  &  &  & \\
		\textsc{f.sg.} & {}-u & {}-u & {}-u & {}-u &  &  &  & \\
		\textsc{pl} & {}-u & {}-u & {}-e & {}-u &  &  &  & \\
		\lspbottomrule
	\end{tabular}
\end{table}

% \textbf{Paradigma 25: Starke und schwache \isi{Adjektive} in Visperterminen \citep[134-135]{Wipf1911}}

\begin{table}[H]
	\caption{Starke und schwache Adjektive in Visperterminen \citep[134-135]{Wipf1911}}\label{table25}
	\begin{tabular}{lllll}
		\lsptoprule
		\multicolumn{5}{c}{stark}\\
		& \NOM & \AKK & \DAT & \GEN\\\midrule
		\textsc{m.sg.} & {}-e & {}-e & {}-um & {}-s\\
		\textsc{n.sg.} & {}-s & {}-s & {}-um & {}-s\\
		\textsc{f.sg.} & {}-i & {}-i & {}-er & {}-er\\
		\textsc{pl} & {}-i & {}-i & {}-e & {}-er\\\midrule 		
		\multicolumn{5}{c}{schwach}\\
		& \NOM & \AKK & \DAT & \GEN\\\midrule
		\textsc{m.sg.} & {}-o & {}-o & {}-u & {}-u\\
		\textsc{n.sg.} & {}-a & {}-a & {}-u & {}-u\\
		\textsc{f.sg.} & {}-a & {}-a & {}-u & {}-u\\
		\textsc{pl} & {}-u & {}-u & {}-u & {}-o\\
		\lspbottomrule
	\end{tabular}
\end{table}

% \textbf{Paradigma 26: Starke und schwache \isi{Adjektive} in Jaun \citep[272-275]{Stucki1917}}

\begin{table}[H]
	\caption{Starke und schwache Adjektive in Jaun \citep[272-275]{Stucki1917}}\label{table26}
	\begin{tabular}{*{8}{l}}
		\lsptoprule
		\multicolumn{8}{c}{stark}\\
 & \multicolumn{4}{c}{\textsc{sg}} & \multicolumn{3}{c}{\textsc{pl}}\\\cmidrule(lr){2-5}\cmidrule(lr){6-8}
 & \NOM & \AKK & \DAT & \GEN & \NOM & \AKK & \DAT \\\midrule
		\scshape m & {}-a & {}-a & {}-əm & {}-s & {}-ø & {}-ø & {}-ə\\
		\scshape n & {}-s & {}-s & {}-əm & {}-s & {}-i & {}-i & {}-ə\\
		\scshape f & {}-i & {}-i & {}-ər & {}- & {}-u/-ø & {}-u/-ø & {}-ə\\ \midrule
		\multicolumn{8}{c}{schwach}\\
		& \NOM & \AKK & \DAT &  &  &  & \\\midrule
		\textsc{m.sg.} & {}-u/-ø & {}-u/-ø & {}-ə &  &  &  & \\
		\textsc{n.sg.} & {}-a-/ø & {}-a-/ø & {}-ə &  &  &  & \\
		\textsc{f.sg.} & {}-i & {}-i & {}-ə &  &  &  & \\
		\textsc{pl} & {}-ə & {}-ə & {}-ə &  &  &  & \\
		\lspbottomrule
	\end{tabular}
\end{table}

% \textbf{Paradigma 27: Starke und schwache \isi{Adjektive} im Sensebezirk \citep[190-192]{Henzen1927}}

\begin{table}[H]
	\caption{Starke und schwache Adjektive im Sensebezirk \citep[190-192]{Henzen1927}}\label{table27}
	\begin{tabular}{*{7}{l}}
		\lsptoprule
		\multicolumn{7}{c}{stark}\\
 & \multicolumn{3}{c}{\textsc{sg}} & \multicolumn{3}{c}{\textsc{pl}}\\\cmidrule(lr){2-4}\cmidrule(lr){5-7}
 & \NOM & \AKK & \DAT & \NOM & \AKK & \DAT\\\midrule
		\scshape m & {}-a & {}-a & {}-um & {}-ø & {}-ø & {}-ə\\
		\scshape n & {}-s & {}-s & {}-um & {}-i & {}-i & {}-ə\\
		\scshape f & {}-i & {}-i & {}-ər & {}-ø/-u & {}-ø/-u & {}-ə\\\midrule
		\multicolumn{7}{c}{schwach}\\		
		& \NOM & \AKK & \DAT &  &  & \\\midrule
		\textsc{m.sg.} & {}-ø & {}-ø & {}-ə &  &  & \\
		\textsc{n.sg.} & {}-ø & {}-ø & {}-ə &  &  & \\
		\textsc{f.sg.} & {}-i & {}-i & {}-ə &  &  & \\
		\textsc{pl} & {}-ə & {}-ə & {}-ə &  &  & \\
		\lspbottomrule
	\end{tabular}
\end{table}

% \textbf{Paradigma 28: Starke und schwache \isi{Adjektive} in Uri \citep[185-187]{Clauß1929}}

\begin{table}[H]
	\caption{Starke und schwache Adjektive in Uri \citep[185-187]{Clauß1929}}\label{table28}
	\begin{tabular}{*{7}{l}}
		\lsptoprule
		\multicolumn{7}{c}{stark}\\
 & \multicolumn{3}{c}{\textsc{sg}} & \multicolumn{3}{c}{\textsc{pl}}\\\cmidrule(lr){2-4}\cmidrule(lr){5-7}
 & \NOM & \AKK & \DAT & \NOM & \AKK & \DAT\\\midrule
		\scshape m & {}-ɐ & {}-ɐ & {}-əm & {}-ø & {}-ø & {}-ɐ\\
		\scshape n & {}-s & {}-s & {}-əm & {}-i & {}-i & {}-ɐ\\
		\scshape f & {}-i & {}-i & {}-ər & {}-ø & {}-ø & {}-ɐ\\\midrule
		\multicolumn{7}{c}{schwach}\\
		& \NOM & \AKK & \DAT &  &  & \\\midrule
		\textsc{sg} & {}-ø & {}-ø & {}-ɐ &  &  & \\
		\textsc{pl} & {}-ɐ & {}-ɐ & {}-ɐ &  &  & \\
		\lspbottomrule
	\end{tabular}
\end{table}

% \textbf{Paradigma 29: Starke und schwache \isi{Adjektive} in Vorarlberg \citep[261-267]{Jutz1925}}

\begin{table}[H]
	\caption{Starke und schwache Adjektive in Vorarlberg \citep[261-267]{Jutz1925}}\label{table29}
	\begin{tabular}{lccc}
\lsptoprule
 & \multicolumn{3}{c}{stark}\\
 & \NOM & \AKK & \DAT\\\midrule
		\textsc{m.sg.} & {}-ə & {}-ə & {}-m\\
		\textsc{n.sg.} & {}-s & {}-s & {}-m\\
		\textsc{f.sg.} & {}-i & {}-i & {}-r\\
		\textsc{pl} & {}-i & {}-i & {}-ə\\\midrule
 & \multicolumn{3}{c}{schwach}\\
 & \NOM & \AKK & \DAT\\\midrule
		\textsc{m.sg.} & {}-ø & {}-ə & {}-ə\\
		\textsc{n.sg.} & {}-ø & {}-ø & {}-ə\\
		\textsc{f.sg.} & {}-ø & {}-ø & {}-ə\\
		\textsc{pl} & {}-ə & {}-ə & {}-ə\\
		\lspbottomrule
	\end{tabular}
\end{table}

% \textbf{Paradigma 30: Starke und schwache \isi{Adjektive} in Zürich \citep[121-126]{Weber1987}}

\begin{table}[H]
	\caption{Starke und schwache Adjektive in Zürich \citep[121-126]{Weber1987}}\label{table30}
	\begin{tabular}{*{7}{l}}
		\lsptoprule
		\multicolumn{7}{c}{stark}\\
 & \multicolumn{3}{c}{\textsc{sg}} & \multicolumn{3}{c}{\textsc{pl}}\\\cmidrule(lr){2-4}\cmidrule(lr){5-7}
 & \NOM & \AKK & \DAT & \NOM & \AKK & \DAT\\\midrule
		\scshape m & {}-ə & {}-ə & {}-əm & {}-ø & {}-ø & {}-ə\\
		\scshape n & {}-s & {}-s & {}-əm & {}-i & {}-i & {}-ə\\
		\scshape f & {}-i & {}-i & {}-ər & {}-ø & {}-ø & {}-ə\\\midrule
		\multicolumn{7}{c}{schwach}\\
		& \NOM & \AKK & \DAT &  &  & \\\midrule
		\textsc{m.sg.} & {}-ø & {}-ø & {}-ə &  &  & \\
		\textsc{n.sg.} & {}-ø & {}-ø & {}-ə &  &  & \\
		\textsc{f.sg.} & {}-ø & {}-ø & {}-ə &  &  & \\
		\textsc{pl} & {}-ə & {}-ə & {}-ə &  &  & \\
		\lspbottomrule
	\end{tabular}
\end{table}

% \textbf{Paradigma 31: Starke und schwache \isi{Adjektive} in Bern \citep[117-120]{Marti1985}}

\begin{table}[H]
	\caption{Starke und schwache Adjektive in Bern \citep[117-120]{Marti1985}}\label{table31}
	\begin{tabular}{lccc}
\lsptoprule
 & \multicolumn{3}{c}{stark}\\
 & \NOM & \AKK & \DAT\\\midrule
		\textsc{m.sg.} & {}-ə & {}-ə & {}-əm\\
		\textsc{n.sg.} & {}-s & {}-s & {}-əm\\
		\textsc{f.sg.} & {}-i & {}-i & {}-ər\\
		\textsc{pl} & {}-i & {}-i & {}-ə\\\midrule
 & \multicolumn{3}{c}{schwach}\\
 & \NOM & \AKK & \DAT\\\midrule
		\textsc{m.sg.} & {}-ø & {}-ø & {}-ə\\
		\textsc{n.sg.} & {}-ə & {}-ə & {}-ə\\
		\textsc{f.sg.} & {}-i & {}-i & {}-ə\\
		\textsc{pl} & {}-ə & {}-ə & {}-ə\\
		\lspbottomrule
	\end{tabular}
\end{table}

% \textbf{Paradigma 32: Starke und schwache \isi{Adjektive} in Huzenbach \citep[98-99]{Baur1967}}

\begin{table}[H]
	\caption{Starke und schwache Adjektive in Huzenbach \citep[98-99]{Baur1967}}\label{table32}
	\begin{tabular}{lccc}
\lsptoprule
 & \multicolumn{3}{c}{stark}\\
 & \NOM & \AKK & \DAT\\\midrule
		\textsc{m.sg.} & {}-ǝ/-ǝr & {}-ǝ & {}-ǝ\\
		\textsc{n.sg.} & {}-s & {}-s & {}-ǝ\\
		\textsc{f.sg.} & {}-e & {}-e & {}-ǝ\\
		\textsc{pl} & {}-e & {}-e & {}-e\\\midrule
 & \multicolumn{3}{c}{schwach}\\
 & \NOM & \AKK & \DAT\\\midrule
		\textsc{m.sg.} & {}-ø & {}-ø & {}-ǝ\\
		\textsc{n.sg.} & {}-ø & {}-ø & {}-ǝ\\
		\textsc{f.sg.} & {}-ø & {}-ø & {}-ǝ\\
		\textsc{pl} & {}-e & {}-e & {}-e\\
		\lspbottomrule
	\end{tabular}
\end{table}

% \textbf{Paradigma 33: Starke und schwache \isi{Adjektive} in Saulgau \citep[109-113]{Raichle1932}}

\begin{table}[H]
	\caption{Starke und schwache Adjektive in Saulgau \citep[109-113]{Raichle1932}}\label{table33}
	\begin{tabular}{lccc}
\lsptoprule
 & \multicolumn{3}{c}{stark}\\
 & \NOM & \AKK & \DAT\\\midrule
		\textsc{m.sg.} & {}-ǝ & {}-ǝ & {}-m\\
		\textsc{n.sg.} & {}-s & {}-s & {}-m\\
		\textsc{f.sg.} & {}-e & {}-e & {}-r\\
		\textsc{pl} & {}-e & {}-e & {}-e\\\midrule
 & \multicolumn{3}{c}{schwach}\\
 & \NOM & \AKK & \DAT\\\midrule
		\textsc{m.sg.} & {}-ø & {}-ǝ & {}-ǝ\\
		\textsc{n.sg.} & {}-ø & {}-ø & {}-ǝ\\
		\textsc{f.sg.} & {}-ø & {}-ø & {}-ǝ\\
		\textsc{pl} & {}-e & {}-e & {}-e\\
		\lspbottomrule
	\end{tabular}
\end{table}

% \textbf{Paradigma 34: Starke und schwache \isi{Adjektive} in Stuttgart \citep[157-159]{Frey1975}}

\begin{table}[H]
	\caption{Starke und schwache Adjektive in Stuttgart \citep[157-159]{Frey1975}}\label{table34}
	\begin{tabular}{lccc}
\lsptoprule
 & \multicolumn{3}{c}{stark}\\
 & \NOM & \AKK & \DAT\\\midrule
		\textsc{m.sg.} & {}-ǝr & {}-ǝ & {}-ǝm\\
		\textsc{n.sg.} & {}-s & {}-s & {}-ǝm\\
		\textsc{f.sg.} & {}-e & {}-e & {}-ǝr\\
		\textsc{pl} & {}-e & {}-e & {}-e\\\midrule
 & \multicolumn{3}{c}{schwach}\\
 & \NOM & \AKK & \DAT\\\midrule
		\textsc{m.sg.} & {}-e & {}-e & {}-ǝ\\
		\textsc{n.sg.} & {}-e & {}-e & {}-ǝ\\
		\textsc{f.sg.} & {}-e/-ø & {}-e/-ø & {}-ǝ\\
		\textsc{pl} & {}-e & {}-e & {}-e\\
		\lspbottomrule
	\end{tabular}
\end{table}

% \textbf{Paradigma 35: Starke und schwache \isi{Adjektive} in Stuttgart \citep[157-159]{Frey1975}}

\begin{table}[H]
	\caption{Starke und schwache Adjektive in Stuttgart \citep[157-159]{Frey1975}}\label{table35}
	\begin{tabular}{lccc}
\lsptoprule
 & \multicolumn{3}{c}{stark}\\
 & \NOM & \AKK & \DAT\\\midrule
		\textsc{m.sg.} & {}-ǝ & {}-ǝ & {}-ǝm\\
		\textsc{n.sg.} & {}-s & {}-s & {}-ǝm\\
		\textsc{f.sg.} & {}-e & {}-e & {}-r\\
		\textsc{pl} & {}-e & {}-e & {}-e\\\midrule
 & \multicolumn{3}{c}{schwach}\\
 & \NOM & \AKK & \DAT\\\midrule
		\textsc{m.sg.} & {}-ø & {}-ǝ & {}-ǝ\\
		\textsc{n.sg.} & {}-ø & {}-ǝ & {}-ǝ\\
		\textsc{f.sg.} & {}-ø & {}-ǝ & {}-ǝ\\
		\textsc{pl} & {}-e & {}-e & {}-e\\
		\lspbottomrule
	\end{tabular}
\end{table}

% \textbf{Paradigma 36: Starke und schwache \isi{Adjektive} in Elisabethtal \citep[52]{Žirmunskij1928/29}}

\begin{table}[H]
	\caption{Starke und schwache Adjektive in Elisabethtal \citep[52]{Žirmunskij1928/29}}\label{table36}
	\begin{tabular}{lccc}
\lsptoprule
 & \multicolumn{3}{c}{stark}\\
 & \NOM & \AKK & \DAT\\\midrule
		\textsc{m.sg.} & {}-r & {}-ɐ & {}-ɐ\\
		\textsc{n.sg.} & {}-s & {}-s & {}-ɐ\\
		\textsc{f.sg.} & {}-e & {}-ɐ & {}-ɐ\\
		\textsc{pl} & {}-ǝ & {}-ǝ & {}-ǝ\\\midrule
 & \multicolumn{3}{c}{schwach}\\
 & \NOM & \AKK & \DAT\\\midrule
		\textsc{m.sg.} & {}-ø & {}-ɐ & {}-ɐ\\
		\textsc{n.sg.} & {}-ø & {}-ø & {}-ɐ\\
		\textsc{f.sg.} & {}-ø & {}-ø & {}-ɐ\\
		\textsc{pl} & {}-ǝ & {}-ǝ & {}-ǝ\\
		\lspbottomrule
	\end{tabular}

\end{table}

% \textbf{Paradigma 37: Starke und schwache \isi{Adjektive} im Kaiserstuhl \citep[407-410]{Noth1993}}

\begin{table}[H]
	\caption{Starke und schwache Adjektive im Kaiserstuhl \citep[407-410]{Noth1993}}\label{table37}
	\begin{tabular}{lccc}
\lsptoprule
 & \multicolumn{3}{c}{stark}\\
 & \NOM & \AKK & \DAT\\\midrule
		\textsc{m.sg.} & {}-ɐ & {}-ɐ & {}-əm\\
		\textsc{n.sg.} & {}-ø/-s & {}-ø/-s & {}-əm\\
		\textsc{f.sg.} & {}-i & {}-i & {}-ər\\
		\textsc{pl} & {}-i & {}-i & {}-ɐ\\\midrule
 & \multicolumn{3}{c}{schwach}\\
 & \NOM & \AKK & \DAT\\\midrule
		\textsc{m.sg.} & {}-ø & {}-ø & {}-ɐ\\
		\textsc{n.sg.} & {}-ø & {}-ø & {}-ɐ\\
		\textsc{f.sg.} & {}-ø & {}-ø & {}-ɐ\\
		\textsc{pl} & {}-ɐ & {}-ɐ & {}-ɐ\\
		\lspbottomrule
	\end{tabular}
\end{table}

% \textbf{Paradigma 38: Starke und schwache \isi{Adjektive} im Münstertal \citep[44-45]{Mankel1886}}

\begin{table}[H]
	\caption{Starke und schwache Adjektive im Münstertal \citep[44-45]{Mankel1886}}\label{table38}
	\begin{tabular}{lccc}
\lsptoprule
 & \multicolumn{3}{c}{stark}\\
 & \NOM & \AKK & \DAT\\\midrule
		\textsc{m.sg.} & {}-ər & {}-ər & {}-əm\\
		\textsc{n.sg.} & {}-ø & {}-ø & {}-əm\\
		\textsc{f.sg.} & {}-i & {}-i & {}-ər\\
		\textsc{pl} & {}-i & {}-i & {}-ə\\\midrule
 & \multicolumn{3}{c}{schwach}\\
 & \NOM & \AKK & \DAT\\\midrule
		\textsc{m.sg.} & {}-ø & {}-ø & {}-ə\\
		\textsc{n.sg.} & {}-ø & {}-ø & {}-ə\\
		\textsc{f.sg.} & {}-ø & {}-ø & {}-ø\\
		\textsc{pl} & {}-ə & {}-ə & {}-ə\\
		\lspbottomrule
	\end{tabular}
\end{table}

% \textbf{Paradigma 39: Starke und schwache \isi{Adjektive} in Colmar \citep[77-80]{Henry1900}}

\begin{table}[H]
	\caption{Starke und schwache Adjektive in Colmar \citep[77-80]{Henry1900}}\label{table39}
	\begin{tabular}{lccc}
\lsptoprule
 & \multicolumn{3}{c}{stark}\\
 & \NOM & \AKK & \DAT\\\midrule
		\textsc{m.sg.} & {}-r & {}-r & {}-m\\
		\textsc{n.sg.} & {}-s & {}-s & {}-m\\
		\textsc{f.sg.} & {}-i & {}-i & {}-r\\
		\textsc{pl} & {}-i & {}-i & {}-e\\\midrule
 & \multicolumn{3}{c}{schwach}\\
 & \NOM & \AKK & \DAT\\\midrule
		\textsc{m.sg.} & {}-ø/-e & {}-e & {}-e\\
		\textsc{n.sg.} & {}-ø/-e & {}-ø/-e & {}-e\\
		\textsc{f.sg.} & {}-ø/-i & {}-ø/-i & {}-e\\
		\textsc{pl} & {}-i & {}-i & {}-e\\
		\lspbottomrule
	\end{tabular}

\end{table}

% \textbf{Paradigma 40: Starke und schwache \isi{Adjektive} im Elsass (Ebene) \citep[114-146]{Beyer1963}}

\begin{table}[H]
	\caption{Starke und schwache Adjektive im Elsass (Ebene) \citep[114-146]{Beyer1963}}\label{table40}
	\begin{tabular}{lccc}
\lsptoprule
 & \multicolumn{3}{c}{stark}\\
 & \NOM & \AKK & \DAT\\\midrule
		\textsc{m.sg.} & {}-ə & {}-ə & {}-əm\\
		\textsc{n.sg.} & {}-ø & {}-ø & {}-əm\\
		\textsc{f.sg.} & {}-i & {}-i & {}-rə\\
		\textsc{pl} & {}-i & {}-i & {}-ə\\\midrule
 & \multicolumn{3}{c}{schwach}\\
 & \NOM & \AKK & \DAT\\\midrule
		\textsc{m.sg.} & {}-ø & {}-ø & {}-ə\\
		\textsc{n.sg.} & {}-ə & {}-ə & {}-ə\\
		\textsc{f.sg.} & {}-ø & {}-ø & {}-ø\\
		\textsc{pl} & {}-i & {}-i & {}-ə\\
		\lspbottomrule
	\end{tabular}
\end{table}

\section{Personalpronomen}

% \textbf{Paradigma 41: \isi{Personalpronomen} des Althochdeutschen \citep[241-245]{Braune2004}}

\begin{table}[H]
	\caption{Personalpronomen des Althochdeutschen \citep[241-245]{Braune2004}}\label{table41}
	\begin{tabular}{l>{\scshape}lllll}
		\lsptoprule
		\multicolumn{6}{c}{betont}\\
		& & \NOM & \AKK & \DAT & \GEN\\\midrule
		\textsc{sg} & 1. & ich & mich & mir & m\=in\\
		& 2. & du & dich & dir & d\=in\\
		& 3.m. & er & inan & imu & s\=in\\
		& 3.n. & iz & iz & imu & es\\
		& 3.f. & siu & sia & iru & ira\\
		\textsc{pl} & 1. & wir & unsich & uns & uns\=er\\
		& 2. & \=ir & iuwich & iu & iuw\=er\\
		& 3.m. & sie & sie & in & iro\\
		& 3.n. & siu & siu & in & iro\\
		& 3.f. & sio & sio & in & iro\\\midrule
 \multicolumn{6}{c}{unbetont}\\
 & & \NOM & \AKK & \DAT & \GEN\\\midrule
		\textsc{sg} & 3.m. & r & nan & mu & sin\\
		& 3.n. & z & z & mu & s\\
		& 3.f. & si & sa & ru & ra\\
		\textsc{pl} & 3.m. & se & se & n & ro\\
		& 3.n. & si & si & n & ro\\
		& 3.f. & so & so & n & ro\\
		\lspbottomrule
	\end{tabular}
\end{table}

% \textbf{Paradigma 42: \isi{Personalpronomen} des Mittelhochdeutschen \citep[210-214]{Paul2007}}

\begin{table}[H]
	\caption{Personalpronomen des Mittelhochdeutschen \citep[210-214]{Paul2007}}\label{table42}
	\begin{tabular}{l>{\scshape}lllll}
		\lsptoprule
		\multicolumn{6}{c}{betont}\\
 & & \NOM & \AKK & \DAT & \GEN\\\midrule
		\textsc{sg} & 1. & ich & mich & mir & m\=in\\
		& 2. & du & dich & dir & d\=in\\
		& 3.m. & ɛr & in & im & s\=in\\
		& 3.n. & ɛs & ɛs & im & s\=in\\
		& 3.f. & siu & siə & ir & ir\\
		\textsc{pl} & 1. & wir & uns & uns & unsər\\
		& 2. & ir & ǖch & ǖ & ǖwər\\
		& 3.m. & siə & siə & in & ir\\
		& 3.n. & siu & siu & in & ir\\
		& 3.f. & siə & siə & in & ir\\\midrule
 \multicolumn{6}{c}{unbetont}\\
 & & \NOM & \AKK & \DAT & \GEN\\\midrule
		\textsc{sg} & 3.m. & ər & ən & əm & əm\\
		& 3.n. & əm & əm & əm & əm\\
		& 3.f. & iu & ə & ər & ər\\
		\textsc{pl} & 3.m. & ə & ə & ən & ər\\
		& 3.n. & ǖ & ǖ & ən & ər\\
		& 3.f. & ə & ə & ən & ər\\
		\lspbottomrule
	\end{tabular}

\end{table}

% \textbf{Paradigma 43: \isi{Personalpronomen} der deutschen Standardsprache \citep[169-177]{Eisenberg2006}}

\begin{table}[H]
	\caption{Personalpronomen der deutschen Standardsprache \citep[169-177]{Eisenberg2006}}\label{table43}
	\begin{tabular}{l>{\scshape}lllll}
		\lsptoprule
		\multicolumn{6}{c}{betont und unbetont}\\
		&  & \NOM & \AKK & \DAT & \GEN\\\midrule
		\textsc{sg} & 1. & ich & mich & mir & meinər\\
		& 2. & du & dich & dir & deinər\\
		& 3.m. & er & \=in & \=im & seinər\\
		& 3.n. & es & es & \=im & seinər\\
		& 3.f. & s\=i & s\=i & \=ir & \=irər\\
		\textsc{pl} & 1. & wir & uns & uns & unsər\\
		& 2. & \=ir & euch & euch & euər\\
		& 3. & s\=i & s\=i & \=inən & \=irər\\
		\lspbottomrule
	\end{tabular}
\end{table}

% \textbf{Paradigma 44: \isi{Personalpronomen} von Issime \citep[206-312]{Zürrer1999}}

\begin{table}[H]
	\caption{Personalpronomen von Issime \citep[206-312]{Zürrer1999}}\label{table44}
	\begin{tabular}{l>{\scshape}lllll}
		\lsptoprule
		\multicolumn{6}{c}{betont}\\
 & & \NOM & \AKK & \DAT & \GEN\\\midrule
		\textsc{sg} & 1. & ich & mich & m\=ir & meir\\
		& 2. & dou & dich & dir & deir\\
		& 3.m. & eer & im & im & dscheir\\
		& 3.n. & \=is & \=is & im & dscheir\\
		& 3.f. & dschi & dschi & irra & irra\\
		\textsc{pl} & 1. & wir & ündsch & ündsch & ündsch-uru\\
		& 2. & ir & auw & auw & auw-uru\\
		& 3. & dschi & dschi & ürj-u & ürj-u, ürj-uru\\
		& 1. & wir-endri & ündsch-endri & ündsch-enandre & ündsch-erandru\\
		& 2. & ir-endri & auw-endri & auw-enandre & auw-erandru\\
		& 3. & dschi-endri & dschi-endri & ürj-enandre & ürj-erandru\\\midrule
 \multicolumn{6}{c}{unbetont}\\
 & & \NOM & \AKK & \DAT & \GEN\\\midrule
		\textsc{sg} & 1. & i & mi & mer & meir\\
		& 2. & de & di & der & deir\\
		& 3.m. & er & ne & mu & dschi\\
		& 3.n. & is & is & mu & dschi\\
		& 3.f. & dschi & dscha & ara & ara\\
		\textsc{pl} & 1. & wer & ünsch & ünsch & ündsch-uru\\
		& 2. & ir & ni & ni & auw-uru\\
		& 3.m. & dschi & dschu & ne & eru\\
		& 3.n. & dschi & dschi & ne & eru\\
		& 3.f. & dschi & dschu & ne & eru\\
		\lspbottomrule
	\end{tabular}
\end{table}

% \textbf{Paradigma 45: \isi{Personalpronomen} von Visperterminen \citep[139-141]{Wipf1911}}

\begin{table}[H]
	\caption{Personalpronomen von Visperterminen \citep[139-141]{Wipf1911}}\label{table45}
	\begin{tabular}{l>{\scshape}lllll}
		\lsptoprule
		\multicolumn{6}{c}{betont}\\
 & & \NOM & \AKK & \DAT & \GEN\\\midrule
		\textsc{sg} & 1. & \=ich & m\=ich & miər & m\=ine\\
		& 2. & dǖ & d\=ich & diər & d\=ine\\
		& 3.m. & ǣr & inu & imu & sch\=ine\\
		& 3.n. & ǣs & ǣs & imu & sch\=ine\\
		& 3.f. & sch\=i & sch\=i & ira & ira\\
		\textsc{pl} & 1. & wiər & \=isch & \=isch & \=ische\\
		& 2. & iər & eww & eww & ewwe\\
		& 3. & sch\=i & sch\=i & ine & iro\\\midrule
 \multicolumn{6}{c}{unbetont}\\
 & & \NOM & \AKK & \DAT & \GEN\\\midrule
		\textsc{sg} & 1. & ich & mich & mr & m\=ine\\
		& 2. & d & dich & dr & d\=ine\\
		& 3.m. & ær & nu & mu & schi\\
		& 3.n. & æs & æs/s/sus & mu & schi\\
		& 3.f. & schi & scha & ra & ra\\
		\textsc{pl} & 1. & wr & schi & schi & \=ische\\
		& 2. & r & ew & ew & ewe\\
		& 3. & schi & schi & ne & ro\\
		\lspbottomrule
	\end{tabular}
\end{table}

% \textbf{Paradigma 46: \isi{Personalpronomen} von Jaun \citep[280-282]{Stucki1917}}

\begin{table}[H]
	\caption{Personalpronomen von Jaun \citep[280-282]{Stucki1917}}\label{table46}
	\begin{tabular}{l>{\scshape}lllll}
		\lsptoprule
		\multicolumn{6}{c}{betont}\\
 & & \NOM & \AKK & \DAT & \GEN\\\midrule
		\textsc{sg} & 1. & ich & mir & mir & m\=ina/m\=inərə\\
		& 2. & du & dir & dir & d\=ina/d\=inərə\\
		& 3.m. & ær & imm/ẽ & im & s\=ina/s\=inərə\\
		& 3.n.unbel. & ǣs & ǣs & im & \\
		& 3.n.bel. & ǣs & ẽs & im & \\
		& 3.f. & sia & sia & ira & ira\\
		\textsc{pl} & 1. & wir & ȫs & ȫs & ȫsa/ȫsərə\\
		& 2. & ir & ȫch & ȫch & ȫwa/ȫwərə\\
		& 3.m. & s\=\i & s\=\i & inə & iru/irəru\\
		& 3.n. & s\=\i & s\=\i & inə & iru/irəru\\
		& 3.f. & siu & siu & inə & iru/irəru\\\midrule
 \multicolumn{6}{c}{unbetont}\\
 & & \NOM & \AKK & \DAT & \GEN\\\midrule
		\textsc{sg} & 1. & i & mər/mi & mər & \\
		& 2. & t & dər/di & dər & \\
		& 3.m. & ər & na & mu & si\\
		& 3.n.unbel. & əs & əs & mu & si\\
		& 3.n.bel. & əs & is & mu & si\\
		& 3.f. & si & sa & ra & ra\\
		\textsc{pl} & 1. & wər & nus & nus & \\
		& 2. & ər & nuch & nuch & \\
		& 3. & si & si & nə & ru\\
		\lspbottomrule
	\end{tabular}
\end{table}

% \textbf{Paradigma 47: \isi{Personalpronomen} des Sensebezirks \citep[196-198]{Henzen1927}}

\begin{table}[H]
	\caption{Personalpronomen des Sensebezirks \citep[196-198]{Henzen1927}}\label{table47}
	\begin{tabular}{l>{\scshape}lllll}
		\lsptoprule
		\multicolumn{6}{c}{betont}\\
 & & \NOM & \AKK & \DAT & \GEN\\\midrule
		\textsc{sg} & 1. & \=i & miər & miər & minərə\\
		& 2. & d\=u & diər & diər & dinərə\\
		& 3.m. & ær & \=im & \=im & sinərə\\
		& 3.n.unbel. & æs & \=im & \=im & sinərə\\
		& 3.n.bel. & æs & æs & \=im & sinərə\\
		& 3.f. & sia/si & sia & ira & \=ira, \=irə\\
		\textsc{pl} & 1. & wiər/miər & ǖs & ǖs & üsərə\\
		& 2. & iər & ȫch & ȫch & öwərə\\
		& 3. & s\=i & s\=i & \=inə & \=irə/\=irərə\\\midrule
 \multicolumn{6}{c}{unbetont}\\
 & & \NOM & \AKK & \DAT & \GEN\\\midrule
		\textsc{sg} & 1. & i & mi & mər & \\
		& 2. & du & di & dər & \\
		& 3.m. & ər & mu & mu & \\
		& 3.n.unbel. & əs & mu & mu & \\
		& 3.n.bel. & əs & əs & mu & \\
		& 3.f. & si & sa & əra & \\
		\textsc{pl} & 1. & wər/mər & nis & nis & \\
		& 2. & er & nuch & nuch & \\
		& 3. & si & si & nə & rə\\
		\lspbottomrule
	\end{tabular}
\end{table}

% \textbf{Paradigma 48: \isi{Personalpronomen} von Uri \citep[190-192]{Clauß1929}}

\begin{table}[H]
	\caption{Personalpronomen von Uri \citep[190-192]{Clauß1929}}\label{table48}
	\begin{tabular}{l>{\scshape}lllll}
		\lsptoprule
		\multicolumn{6}{c}{betont}\\
 & & \NOM & \AKK & \DAT & \GEN\\\midrule
		\textsc{sg} & 1. & ich & mich & miar & m\=inɐ/m\=inərtnɐ\\
		& 2. & dǖ/dǖɐ & dich & diar & d\=inɐ/d\=inərtnɐ\\
		& 3.m. & \=ar & inɐ & im & s\=inɐ/s\=inərtnɐ\\
		& 3.n.unbel. & \=as & \=as & im & s\=inɐ/s\=inərtnɐ\\
		& 3.n.bel. & \=as & inəs & im & s\=inɐ/s\=inərtnɐ\\
		& 3.f. & s\=i & s\=i & irɐ & irɐ\\
		\textsc{pl} & 1. & miar & \=is & \=is & \=isərɐ/\=isərtnɐ\\
		& 2. & iar & \=ich & \=ich & \=iwərɐ/\=iwərtnɐ\\
		& 3. & s\=i & s\=i & inɐ & irər/irərtnɐ\\\midrule
		\multicolumn{6}{c}{unbetont}\\
		&  & \NOM & \AKK & \DAT & \\\midrule
		\textsc{sg} & 1. & i & mi & mər & \\
		& 2. & dü & di & dər & \\
		& 3.m. & ər & a & əm & \\
		& 3.n. & s & s & əm & \\
		& 3.f. & si & si & ərɐ & \\
		\textsc{pl} & 1. & mər & is & is & \\
		& 2. & ər & əch & əch & \\
		& 3. & si & si & nɐ & \\
		\lspbottomrule
	\end{tabular}
\end{table}

% \textbf{Paradigma 49: \isi{Personalpronomen} von Vorarlberg \citep[271-274]{Jutz1925}}

\begin{table}[H]
	\caption{Personalpronomen von Vorarlberg \citep[271-274]{Jutz1925}}\label{table49}
	\begin{tabular}{l>{\scshape}llll}
		\lsptoprule
		\multicolumn{5}{c}{betont}\\
 & & \NOM & \AKK & \DAT\\\midrule
		\textsc{sg} & 1. & \=i & m\=i & m\=iər\\
		& 2. & d\=u & d\=i & d\=iər\\
		& 3.m. & \=er/ər & iən & im\\
		& 3.n. & \=es & \=es & im\\
		& 3.f. & s\=i/si & s\=i/si & \=iərə\\
		\textsc{pl} & 1. & m\=iər & ǖs & ǖs\\
		& 2. & \=iər & öu & öu\\
		& 3. & s\=i/si & s\=i/si & \=inə\\\midrule
 \multicolumn{5}{c}{unbetont}\\
 & & \NOM & \AKK & \DAT\\\midrule
		\textsc{sg} & 1. & i & mi & mr\\
		& 2. & du/də & di & dr\\
		& 3.m. & r & ən & əm\\
		& 3.n. & əs & əs & əm\\
		& 3.f. & sı & sı & ərə\\
		\textsc{pl} & 1. & mr & is & is\\
		& 2. & ər/r & ni & ni\\
		& 3. & sı & sı & nə\\
		\lspbottomrule
	\end{tabular}
\end{table}

% \textbf{Paradigma 50: \isi{Personalpronomen} von Zürich \citep[153-162]{Weber1987}}

\begin{table}[H]
	\caption{Personalpronomen von Zürich \citep[153-162]{Weber1987}}\label{table50}
	\begin{tabular}{l>{\scshape}llll}
		\lsptoprule
		\multicolumn{5}{c}{betont}\\
 & & \NOM & \AKK & \DAT\\\midrule
		\textsc{sg} & 1. & ich & mich & mir\\
		& 2. & du & dich & dir\\
		& 3.m. & ɛr & in & im\\
		& 3.n.unbel. & ɛs & ɛs & im\\
		& 3.n.bel. & ɛs & ins & im\\
		& 3.f. & si & si & irə\\
		\textsc{pl} & 1. & mir & öis & öis\\
		& 2. & ir & öi & öi \\
		& 3. & si & si & inə\\\midrule
 \multicolumn{5}{c}{unbetont}\\
 & & \NOM & \AKK & \DAT\\\midrule
		\textsc{sg} & 1. & i & mi & mər\\
		& 2. & de/d & di & dər\\
		& 3.m. & ər & ə & əm\\
		& 3.n. & əs/s & s & əm\\
		& 3.f. & si/s & si & ərə\\
		\textsc{pl} & 1. & mər & is & is\\
		& 2. & ər & i & i \\
		& 3. & si/s & si/s & ənə\\
		\lspbottomrule
	\end{tabular}
\end{table}

% \textbf{Paradigma 51: \isi{Personalpronomen} von Bern \citep[92-97]{Marti1985}}

\begin{table}[H]
	\caption{Personalpronomen von Bern \citep[92-97]{Marti1985}}\label{table51}
	\begin{tabular}{l>{\scshape}lllll}
		\lsptoprule
		\multicolumn{6}{c}{betont}\\
		&  & \NOM & \AKK & \DAT & \\\midrule
		\textsc{sg} & 1. & \=i/\=ig & m\=i & m\=ir & \\
		& 2. & d\=u & d\=i & d\=ir & \\
		& 3.m. & ǣr & \=in & \=im & \\
		& 3.n.unbel. & ǣs & ǣs & \=im & \\
		& 3.n.bel. & ǣs & \=ins & \=im & \\
		& 3.f. & seiə/s\=i & seiə/s\=i & \=irə & \\
		\textsc{pl} & 1. & m\=ir/miər & ǖs & ǖs & \\
		& 2. & d\=ir & öich & öich & \\
		& 3. & seiə & seiə/s\=i & \=inə & \\\midrule
 \multicolumn{6}{c}{unbetont}\\
 & & \NOM & \AKK & \DAT & \GEN\\\midrule
		\textsc{sg} & 1. & i & mi & mər & \\
		& 2. & de & di & dər & \\
		& 3.m. & ər & nə & im & \\
		& 3.n.unbel. & əs/s & əs/s & im & \\
		& 3.n.bel. & əs/s & s & im & \\
		& 3.f. & si & sə & ərə & \\
		\textsc{pl} & 1. & mər & is & is & \\
		& 2. & ər & əch & əch & \\
		& 3. & si & sə & nə & rə\\
		\lspbottomrule
	\end{tabular}
\end{table}

% \textbf{Paradigma 52: \isi{Personalpronomen} von Huzenbach \citep[102-103]{Baur1967}}

\begin{table}[H]
	\caption{Personalpronomen von Huzenbach \citep[102-103]{Baur1967}}\label{table52}
	\begin{tabular}{l>{\scshape}llll}
		\lsptoprule
		\multicolumn{5}{c}{betont}\\
 & & \NOM & \AKK & \DAT\\\midrule
		\textsc{sg} & 1. & \=i & m\=i & m\=ir\\
		& 2. & dǝu/d\=u & d\=i & d\=ir\\
		& 3.m. & \=er & \=e͂n & \=e͂m\\
		& 3.n. & \=es & \=es & \=e͂m\\
		& 3.f. & s\=i & s\=i & \=irǝ\\
		\textsc{pl} & 1. & m\=ir & ãẽs & ãẽs\\
		& 2. & \=ir & ǝich & ǝich\\
		& 3. & s\=i & s\=i & \=e͂nǝ\\\midrule
 \multicolumn{5}{c}{unbetont}\\
 & & \NOM & \AKK & \DAT\\\midrule
		\textsc{sg} & 1. & i, e & me & mǝr\\
		& 2. & dǝ/d & dǝ & dǝr\\
		& 3.m. & ǝr & ǝn/ǝ & ǝm\\
		& 3.n. & s & s & ǝm/m\\
		& 3.f. & se & se & ǝrǝ\\
		\textsc{pl} & 1. & mǝr & ech/ich & ech/ich\\
		& 2. & ǝr & ich/ech & ich/ech\\
		& 3. & se/s & se/s & ǝnǝ/nǝ\\
		\lspbottomrule
	\end{tabular}
\end{table}

% \textbf{Paradigma 53: \isi{Personalpronomen} von Saulgau \citep[116-117]{Raichle1932}}

\begin{table}[H]
	\caption{Personalpronomen von Saulgau \citep[116-117]{Raichle1932}}\label{table53}
	\begin{tabular}{l>{\scshape}llll}
		\lsptoprule
		\multicolumn{5}{c}{betont}\\
 & & \NOM & \AKK & \DAT\\\midrule
		\textsc{sg} & 1. & \=i & m\=i & m\=iǝr\\
		& 2. & d\=u & d\=i & d\=iǝr\\
		& 3.m. & ɛr & \=e͂n & \=e͂m\\
		& 3.n. & ɛs & ɛs & \=e͂m\\
		& 3.f. & s\=iǝ & s\=iǝ & \=iǝrǝ\\
		\textsc{pl} & 1. & m\=iǝr & ǝis & ǝis\\
		& 2. & \=iǝr & ui & ui\\
		& 3. & s\=iǝ & s\=iǝ & \=e͂ne\\\midrule
 \multicolumn{5}{c}{unbetont}\\
 & & \NOM & \AKK & \DAT\\\midrule
		\textsc{sg} & 1. & e & me & mr\\
		& 2. & dǝ & de & dr\\
		& 3.m. & r & n & m\\
		& 3.n. & s & s & m\\
		& 3.f. & se & se & ǝrǝ\\
		\textsc{pl} & 1. & mr & es & es\\
		& 2. & r & ui & ui\\
		& 3. & se & se & ǝnǝ\\
		\lspbottomrule
	\end{tabular}
\end{table}

% \textbf{Paradigma 54: \isi{Personalpronomen} von Stuttgart \citep[160-161]{Frey1975}}

\begin{table}[H]
	\caption{Personalpronomen von Stuttgart \citep[160-161]{Frey1975}}\label{table54}
	\begin{tabular}{l>{\scshape}llll}
		\lsptoprule
		\multicolumn{5}{c}{betont}\\
 & & \NOM & \AKK & \DAT\\\midrule
		\textsc{sg} & 1. & \=i & m\=i & m\=ir\\
		& 2. & d\=u & d\=i & d\=ir\\
		& 3.m. & ɛr & \=en & \=em\\
		& 3.n. & des & des & \=em\\
		& 3.f. & s\=i/diǝ & s\=i/diǝ & iǝrǝ\\
		\textsc{pl} & 1. & m\=ir & ons & ons\\
		& 2. & \=ir & ǝix & ǝix\\
		& 3. & s\=i/diǝ & s\=\i/diǝ & \=ene\\\midrule
 \multicolumn{5}{c}{unbetont}\\
 & & \NOM & \AKK & \DAT\\\midrule
		\textsc{sg} & 1. & i & mi & mir\\
		& 2. & du & di & dir\\
		& 3.m. & ǝr & den & dem\\
		& 3.n. & s & s & dem\\
		& 3.f. & se & se & dɛrə\\
		\textsc{pl} & 1. & mǝr & ons & ons\\
		& 2. & ǝr & ǝix & ǝix\\
		& 3. & se & se & dene\\
		\lspbottomrule
	\end{tabular}

\end{table}

% \textbf{Paradigma 55: \isi{Personalpronomen} von Petrifeld \citep[64-65]{Moser1937}}

\begin{table}[H]
	\caption{Personalpronomen von Petrifeld \citep[64-65]{Moser1937}}\label{table55}
	\begin{tabular}{l>{\scshape}llll}
		\lsptoprule
		\multicolumn{5}{c}{betont}\\
 & & \NOM & \AKK & \DAT\\\midrule
		\textsc{sg} & 1. & \=i & m\=i & m\=iǝr\\
		& 2. & d\=u & d\=i & d\=iǝr\\
		& 3.m. & i\=eǝr & in & im\\
		& 3.n. &  &  & im\\
		& 3.f. & s\=iǝ & s\=iǝ & \=iǝrǝ\\
		\textsc{pl} & 1. & m\=iǝr & \=ais & \=ais\\
		& 2. & \=iǝr & ui & ui\\
		& 3. & s\=iǝ & ine & ine\\\midrule
 \multicolumn{5}{c}{unbetont}\\
 & & \NOM & \AKK & \DAT\\\midrule
		\textsc{sg} & 1. & i & me & mr\\
		& 2. & dǝ & de & dr\\
		& 3.m. & r & ǝ & ǝm\\
		& 3.n. & s/ǝs & s & ǝm\\
		& 3.f. & se & se & rǝ\\
		\textsc{pl} & 1. & mr & es & es\\
		& 2. & r & ene & ene\\
		& 3. & se & s & ene\\
		\lspbottomrule
	\end{tabular}
\end{table}

% \textbf{Paradigma 56: \isi{Personalpronomen} von Elisabethtal \citep[52]{Žirmunskij1928/29}}

\begin{table}[H]
	\caption{Personalpronomen von Elisabethtal \citep[52]{Žirmunskij1928/29}}\label{table56}
	\begin{tabular}{l>{\scshape}llll}
		\lsptoprule
		\multicolumn{5}{c}{betont}\\
 & & \NOM & \AKK & \DAT\\\midrule
		\textsc{sg} & 1. & \=i & m\=i & miǝr\\
		& 2. & d\=u & d\=i & diǝr\\
		& 3.m. & deɐr & deɐ͂/\=en & deɐm/\=em\\
		& 3.n. & d\=es & d\=es & deɐm/\=em\\
		& 3.f. & dui/sui & dui/sui & iǝr/deɐ͂rɐ\\
		\textsc{pl} & 1. & miǝr/mr & ons & ons\\
		& 2. & iǝr & uich/\=enɐ & uich/\=enɐ\\
		& 3. & diǝ/siǝ & diǝ/siǝ & deɐnɐ\\\midrule
 \multicolumn{5}{c}{unbetont}\\
 & & \NOM & \AKK & \DAT\\\midrule
		\textsc{sg} & 3.m. &  & n & m\\
		& 3.n. & ts & ts & m\\
		\lspbottomrule
	\end{tabular}
\end{table}

% \textbf{Paradigma 57: \isi{Personalpronomen} des Kaiserstuhls \citep[393-398]{Noth1993}}

\begin{table}[H]
	\caption{Personalpronomen des Kaiserstuhls \citep[393-398]{Noth1993}}\label{table57}
	\begin{tabular}{l>{\scshape}llll}
		\lsptoprule
		\multicolumn{5}{c}{betont}\\
 & & \NOM & \AKK & \DAT\\\midrule
		\textsc{sg} & 1. & ich & mich & m\=ir\\
		& 2. & d\=u & dich & d\=ir\\
		& 3.m. & \=ar & \=inɐ & \=im\\
		& 3.n.unbel. & \=as & \=as & \=im\\
		& 3.n.bel. & \=as & \=inəs & \=im\\
		& 3.f. & s\=i & s\=i & \=irɐ\\
		\textsc{pl} & 1. & m\=ir & uns & uns\\
		& 2. & \=ir & æich & æich\\
		& 3. & s\=i & s\=i & \=inɐ\\\midrule
 \multicolumn{5}{c}{unbetont}\\
 & & \NOM & \AKK & \DAT\\\midrule
		\textsc{sg} & 1. & i & mi & mr\\
		& 2. & dr & di & dr\\
		& 3.m. & dr & ɐ & əm\\
		& 3.n. & s & s & əm\\
		& 3.f. & si & si & erɐ\\
		\textsc{pl} & 1. & mr & is & is\\
		& 2. & dr & ich & ich\\
		& 3. & si & si & enɐ\\
		\lspbottomrule
	\end{tabular}
\end{table}

% \textbf{Paradigma 58: \isi{Personalpronomen} des Münstertals \citep[46-47]{Mankel1886}}

\begin{table}[H]
	\caption{Personalpronomen des Münstertals \citep[46-47]{Mankel1886}}\label{table58}
	\begin{tabular}{l>{\scshape}llll}
		\lsptoprule
		\multicolumn{5}{c}{betont}\\
 & & \NOM & \AKK & \DAT\\\midrule
		\textsc{sg} & 1. & ich & mich & m\=er\\
		& 2. & tǖ & tich & t\=er\\
		& 3.m. & \=ar & ǣne & ǣm\\
		& 3.n. & as & es & ǣm\\
		& 3.f. & s\=e & s\=e & \=er\\
		\textsc{pl} & 1. & m\=er & \=us & \=us\\
		& 2. & \=er & ich & ich\\
		& 3. & s\=e & s\=e & ǣne\\\midrule
 \multicolumn{5}{c}{unbetont}\\
 & & \NOM & \AKK & \DAT\\\midrule
		\textsc{sg} & 1. & i & mi & mər\\
		& 2. & tə & ti & tər\\
		& 3.m. & ər & nə & əm\\
		& 3.n. & s & s & əm\\
		& 3.f. & sə & sə & ər\\
		\textsc{pl} & 1. & mər & əs & əs\\
		& 2. & ər & i & i\\
		& 3. & sə & sə & nə\\
		\lspbottomrule
	\end{tabular}
\end{table}

% \textbf{Paradigma 59: \isi{Personalpronomen} von Colmar \citep[81-83]{Henry1900}}

\begin{table}[H]
	\caption{Personalpronomen von Colmar \citep[81-83]{Henry1900}}\label{table59}
	\begin{tabular}{l>{\scshape}llll}
		\lsptoprule
		\multicolumn{5}{c}{betont}\\
 & & \NOM & \AKK & \DAT\\\midrule
		\textsc{sg} & 1. & ich & mich & mer\\
		& 2. & tü & tich & ter\\
		& 3.m. & ar & ene & em\\
		& 3.n. & as & as & em\\
		& 3.f. & se & se & ere\\
		\textsc{pl} & 1. & mer & ons & ons\\
		& 2. & \=er & eich & eich\\
		& 3. & se & se & \=ene\\\midrule
 \multicolumn{5}{c}{unbetont}\\
 & & \NOM & \AKK & \DAT\\\midrule
		\textsc{sg} & 1. & i & mi & mr\\
		& 2. & te & ti & tr\\
		& 3.m. & er & ne & m\\
		& 3.n. & s & s & m\\
		& 3.f. & si & si & re\\
		\textsc{pl} & 1. & mr & is & is\\
		& 2. & er & i & i\\
		& 3. & si & si & ene\\
		\lspbottomrule
	\end{tabular}
\end{table}

% \textbf{Paradigma 60: \isi{Personalpronomen} des Elsass (Ebene) \citep[ 151-159]{Beyer1963}}

\begin{table}[H]
	\caption{Personalpronomen des Elsass (Ebene) \citep[ 151-159]{Beyer1963}}\label{table60}
	\begin{tabular}{l>{\scshape}llll}
		\lsptoprule
		\multicolumn{5}{c}{betont}\\
 & & \NOM & \AKK & \DAT\\\midrule
		\textsc{sg} & 1. & ich & mich & mir\\
		& 2. & dü & dich & dir\\
		& 3.m. & ær & inə & im\\
		& 3.n.unbel. & æs & æs & im\\
		& 3.n.bel. & æs & inəs & im\\
		& 3.f. & si & si & irə\\
		\textsc{pl} & 1. & mir & uns & uns\\
		& 2. & ir & eich & eich\\
		& 3. & si & si & inə\\\midrule
 \multicolumn{5}{c}{unbetont}\\
 & & \NOM & \AKK & \DAT\\\midrule
		\textsc{sg} & 1. & i & mi & mər\\
		& 2. & də & də & dər\\
		& 3.m. & ər & nə & əm\\
		& 3.n. & əs/s & s & əm\\
		& 3.f. & si & si & ərə\\
		\textsc{pl} & 1. & mər & i & i\\
		& 2. & ər & uch & uch\\
		& 3. & si & si & nə\\
		\lspbottomrule
	\end{tabular}
\end{table}

\pagebreak
\section{Interrogativpronomen}

% \textbf{Paradigma 61: \isi{Interrogativpronomen} des Althochdeutschen \citep[252-253]{Braune2004}}

\begin{table}[H]
	\caption{Interrogativpronomen des Althochdeutschen \citep[252-253]{Braune2004}}\label{table61}
	\begin{tabular}{llllll}
		\lsptoprule
		& \NOM & \AKK & \DAT & \GEN & \INSTR\\\midrule
 belebt & wer & wenan & wemu & wes & \\
		unbelebt & waz & waz & wemu/wiu & wes & wiu\\
		\lspbottomrule
	\end{tabular}
\end{table}

% \textbf{Paradigma 62: \isi{Interrogativpronomen} des Mittelhochdeutschen \citep[222-223]{Paul2007}}

\begin{table}[H]
	\caption{Interrogativpronomen des Mittelhochdeutschen \citep[222-223]{Paul2007}}\label{table62}
	\begin{tabular}{lllll}
		\lsptoprule
		& \NOM & \AKK & \DAT & \GEN\\\midrule
 belebt & wɛr & wɛn & wɛm & wɛs\\
		unbelebt & was & was & wɛm & wɛs\\
		\lspbottomrule
	\end{tabular}
\end{table}

% \textbf{Paradigma 63: \isi{Interrogativpronomen} der deutschen Standardsprache \citep[169-177]{Eisenberg2006}}

\begin{table}[H]
	\caption{Interrogativpronomen der deutschen Standardsprache \citep[169-177]{Eisenberg2006}}\label{table63}
	\begin{tabular}{lllll}
		\lsptoprule
		& \NOM & \AKK & \DAT & \GEN\\\midrule
 belebt & wer & wen & wem & wessən\\
		unbelebt & was & was & wem & wessən\\
		\lspbottomrule
	\end{tabular}
\end{table}

% \textbf{Paradigma 64: \isi{Interrogativpronomen} von Issime \citep[258-259, 306]{Zürrer1999}}

\begin{table}[H]
	\caption{Interrogativpronomen von Issime \citep[258-259, 306]{Zürrer1999}}\label{table64}
	\begin{tabular}{llll}
		\lsptoprule
		& \NOM & \AKK & \DAT\\\midrule
 belebt & wer & wem & wem\\
		unbelebt & was & was & wem\\
		\lspbottomrule
	\end{tabular}
\end{table}

% \textbf{Paradigma 65: \isi{Interrogativpronomen} von Visperterminen \citep[143]{Wipf1911}}

\begin{table}[H]
	\caption{Interrogativpronomen von Visperterminen \citep[143]{Wipf1911}}\label{table65}
	\begin{tabular}{lllll}
		\lsptoprule
		& \NOM & \AKK & \DAT & \GEN\\\midrule
 belebt & wær & wær & wem & weschi\\
		unbelebt & was & was & wem & weschi\\
		\lspbottomrule
	\end{tabular}
\end{table}

% \textbf{Paradigma 66: \isi{Interrogativpronomen} von Jaun \citep[285-286]{Stucki1917}}

\begin{table}[H]
	\caption{Interrogativpronomen von Jaun \citep[285-286]{Stucki1917}}\label{table66}
	\begin{tabular}{llll}
		\lsptoprule
		& \NOM & \AKK & \DAT\\\midrule
 belebt & wær & wær/wɛm & wɛm\\
		unbelebt & was & was & wɛm\\
		\lspbottomrule
	\end{tabular}
\end{table}

% \textbf{Paradigma 67: \isi{Interrogativpronomen} des Sensebezirks \citep[201-202]{Henzen1927}}

\begin{table}[H]
	\caption{Interrogativpronomen des Sensebezirks \citep[201-202]{Henzen1927}}\label{table67}
	\begin{tabular}{llll}
		\lsptoprule
		& \NOM & \AKK & \DAT\\\midrule
 belebt & wær & wær & wæm\\
		unbelebt & was & was & wæm\\
		\lspbottomrule
	\end{tabular}
\end{table}

% \textbf{Paradigma 68: \isi{Interrogativpronomen} von Uri \citep[196]{Clauß1929}}

\begin{table}[H]
	\caption{Interrogativpronomen von Uri \citep[196]{Clauß1929}}\label{table68}
	\begin{tabular}{lllll}
		\lsptoprule
		& \NOM & \AKK & \DAT & \GEN\\\midrule
 belebt & w\=er & w\=er & wɛm & wessɐ\\
		unbelebt & wɑs & wɑs & wɑs & \\
		\lspbottomrule
	\end{tabular}
\end{table}

% \textbf{Paradigma 69: \isi{Interrogativpronomen} von Vorarlberg \citep[282]{Jutz1925}}

\begin{table}[H]
	\caption{Interrogativpronomen von Vorarlberg \citep[282]{Jutz1925}}\label{table69}
	\begin{tabular}{llll}
		\lsptoprule
		& \NOM & \AKK & \DAT\\\midrule
 belebt & w\=er & w\=e & wem\\
		unbelebt & was & was & wem\\
		\lspbottomrule
	\end{tabular}
\end{table}

% \textbf{Paradigma 70: \isi{Interrogativpronomen} von Zürich \citep[144-145]{Weber1987}}

\begin{table}[H]
	\caption{Interrogativpronomen von Zürich \citep[144-145]{Weber1987}}\label{table70}
	\begin{tabular}{llll}
		\lsptoprule
		& \NOM & \AKK & \DAT\\\midrule
 belebt & wɛ & wɛ & wɛm\\
		unbelebt & was & was & was\\
		\lspbottomrule
	\end{tabular}
\end{table}

% \textbf{Paradigma 71: \isi{Interrogativpronomen} von Bern \citep[106]{Marti1985}}

\begin{table}[H]
	\caption{Interrogativpronomen von Bern \citep[106]{Marti1985}}\label{table71}
	\begin{tabular}{llll}
		\lsptoprule
		& \NOM & \AKK & \DAT\\\midrule
 belebt & wær & wær & wæm\\
		unbelebt & was & was & wasəm/was\\
		\lspbottomrule
	\end{tabular}
\end{table}

% \textbf{Paradigma 72: \isi{Interrogativpronomen} von Huzenbach \citep[105]{Baur1967}}

\begin{table}[H]
	\caption{Interrogativpronomen von Huzenbach \citep[105]{Baur1967}}\label{table72}
	\begin{tabular}{llll}
		\lsptoprule
		& \NOM & \AKK & \DAT\\\midrule
 belebt & w\=er/w\=eǝr & w\=e͂n & w\=e͂m\\
		unbelebt & w\=as/wa & w\=as/wa & w\=as/wa\\
		\lspbottomrule
	\end{tabular}
\end{table}

% \textbf{Paradigma 73: \isi{Interrogativpronomen} von Saulgau \citep[120]{Raichle1932}}

\begin{table}[H]
	\caption{Interrogativpronomen von Saulgau \citep[120]{Raichle1932}}\label{table73}
	\begin{tabular}{llll}
		\lsptoprule
		& \NOM & \AKK & \DAT\\\midrule
 belebt & wɛǝr/weǝr & wɛǝr/weǝr & wɛm\\
		unbelebt & w\=a/wa & w\=a/wa & wɛm\\
		\lspbottomrule
	\end{tabular}
\end{table}

% \textbf{Paradigma 74: \isi{Interrogativpronomen} von Stuttgart \citep[162]{Frey1975}}

\begin{table}[H]
	\caption{Interrogativpronomen von Stuttgart \citep[162]{Frey1975}}\label{table74}
	\begin{tabular}{llll}
		\lsptoprule
		& \NOM & \AKK & \DAT\\\midrule
 belebt & wɛr & w\=en & w\=em\\
		unbelebt & was/w\=as & was/w\=as & was/w\=as\\
		\lspbottomrule
	\end{tabular}
\end{table}

% \textbf{Paradigma 75: \isi{Interrogativpronomen} von Petrifeld \citep[66-67]{Moser1937}}

\begin{table}[H]
	\caption{Interrogativpronomen von Petrifeld \citep[66-67]{Moser1937}}\label{table75}
	\begin{tabular}{llll}
		\lsptoprule
		& \NOM & \AKK & \DAT\\\midrule
 belebt & wi\=eǝr/wieǝr & wi\=eǝr/wieǝr & wi\=eǝm/wieǝm\\
		unbelebt & w\=a/wa & w\=a/wa & wi\=eǝm/wieǝm\\
		\lspbottomrule
	\end{tabular}
\end{table}

% \textbf{Paradigma 76: \isi{Interrogativpronomen} von Elisabethtal \citep[53]{Žirmunskij1928/29}}

\begin{table}[H]
	\caption{Interrogativpronomen von Elisabethtal \citep[53]{Žirmunskij1928/29}}\label{table76}
	\begin{tabular}{llll}
		\lsptoprule
		& \NOM & \AKK & \DAT\\\midrule
 belebt & weɐr & weɐn & weɐm\\
		unbelebt & w\=as/w\=a & w\=as/w\=a & w\=as/w\=a\\
		\lspbottomrule
	\end{tabular}
\end{table}

% \textbf{Paradigma 77: \isi{Interrogativpronomen} des Kaiserstuhls \citep[385]{Noth1993}}

\begin{table}[H]
	\caption{Interrogativpronomen des Kaiserstuhls \citep[385]{Noth1993}}\label{table77}
	\begin{tabular}{llll}
		\lsptoprule
		& \NOM & \AKK & \DAT\\\midrule
 belebt & war & war & wam\\
		unbelebt & was & was & wam\\
		\lspbottomrule
	\end{tabular}
\end{table}

% \textbf{Paradigma 78: \isi{Interrogativpronomen} des Münstertals \citep[48]{Mankel1886}}

\begin{table}[H]
	\caption{Interrogativpronomen des Münstertals \citep[48]{Mankel1886}}\label{table78}
	\begin{tabular}{llll}
		\lsptoprule
		& \NOM & \AKK & \DAT\\\midrule
 belebt & wɛr & wɛr & wæm\\
		unbelebt & wɑs & wɑs & wɑs\\
		\lspbottomrule
	\end{tabular}
\end{table}

% \textbf{Paradigma 79: \isi{Interrogativpronomen} von Colmar \citep[85-86]{Henry1900}}

\begin{table}[H]
	\caption{Interrogativpronomen von Colmar \citep[85-86]{Henry1900}}\label{table79}
	\begin{tabular}{llll}
		\lsptoprule
		& \NOM & \AKK & \DAT\\\midrule
 belebt & war & war & wam\\
		unbelebt & was & was & was\\
		\lspbottomrule
	\end{tabular}
\end{table}

% \textbf{Paradigma 80: \isi{Interrogativpronomen} des Elsass (Ebene) \citep[164-167]{Beyer1963}}

\begin{table}[H]
	\caption{Interrogativpronomen des Elsass (Ebene) \citep[164-167]{Beyer1963}}\label{table80}
	\begin{tabular}{llll}
		\lsptoprule
		& \NOM & \AKK & \DAT\\\midrule
 belebt & wer & wer/wennə & wennə\\
		unbelebt & was & was & wennə\\
		\lspbottomrule
	\end{tabular}
\end{table}

\section{Bestimmter Artikel und Demonstrativpronomen}

% \textbf{Paradigma 81: \isi{Demonstrativpronomen} des Althochdeutschen \citep[247-249]{Braune2004}}

\begin{table}[H]
	\caption{Demonstrativpronomen des Althochdeutschen \citep[247-249]{Braune2004}}\label{table81}
		\begin{tabular}{*{10}{l}}
		\lsptoprule
		& \multicolumn{5}{c}{\textsc{sg}}  & \multicolumn{4}{c}{\textsc{pl}} \\\cmidrule(lr){2-6}\cmidrule(lr){7-10}
		& \NOM & \AKK & \DAT & \GEN & \INSTR & \NOM & \AKK & \DAT & \GEN\\\midrule
		\textsc{m} & der & den & demu & des & diu & die & die & d\=em & dero\\
		\textsc{n} & daz & daz & demu & des & diu & diu & diu & d\=em & dero\\
		\textsc{f} & diu & dia & deru & dera &  & dio & dio & d\=em & dero\\
		\lspbottomrule
	\end{tabular}
\end{table}

% \textbf{Paradigma 82: Bestimmter Artikel und \isi{Demonstrativpronomen} des Mittelhochdeutschen \citep[217-219]{Paul2007}}

\begin{table}[H]
	\caption{Bestimmter Artikel und Demonstrativpronomen des Mittelhochdeutschen \citep[217-219]{Paul2007}}\label{table82}
	\resizebox{\textwidth}{!}{\begin{tabular}{*{9}{l}}
		\lsptoprule
		& \multicolumn{4}{c}{\textsc{sg}} & \multicolumn{4}{c}{\textsc{pl}} \\\cmidrule(lr){2-5}\cmidrule{6-9}		
		& \NOM & \AKK & \DAT & \GEN & \NOM & \AKK & \DAT & \GEN\\\midrule
		\textsc{m.sg.} & dɛr & dɛn & dɛm/dɛmo & dës & diə & diə & dɛn/diən & dɛr/dɛro\\
		\textsc{n.sg.} & das & das & dɛm/dɛmo & dës & diu & diu & dɛn/diən & dɛr/dɛro\\
		\textsc{f.sg.} & diu & diə & dɛr/dɛro & dɛr/dɛro & diə & diə & dɛn/diən & dɛr/dɛro\\
		\lspbottomrule
	\end{tabular}}
\end{table}

% \textbf{Paradigma 83: Bestimmter Artikel und \isi{Demonstrativpronomen} der deutschen Standardsprache \citep[169-177]{Eisenberg2006}}

\begin{table}[H]
	\caption{Bestimmter Artikel und Demonstrativpronomen der deutschen Standardsprache \citep[169-177]{Eisenberg2006}}\label{table83}
	\begin{tabular}{lllll}
		\lsptoprule
		& \NOM & \AKK & \DAT & \GEN\\\midrule
		\textsc{m.sg.} & der & den & dem/m & des\\
		\textsc{n.sg.} & das & das & dem/m & des\\
		\textsc{f.sg.} & die & die & der & der\\
		\textsc{pl} & die & die & den & der\\
		\lspbottomrule
	\end{tabular}
\end{table}

% \textbf{Paradigma 84: Bestimmter Artikel und \isi{Demonstrativpronomen} von Issime \citep[4-12, 81]{Perinetto1981}}

\begin{table}[H]
	\caption{Bestimmter Artikel und Demonstrativpronomen von Issime \citep[4-12, 81]{Perinetto1981}}\label{table84}
	\begin{tabular}{lllllllll}
		\lsptoprule
		& \multicolumn{4}{c}{bestimmter Artikel} &  \multicolumn{4}{c}{Demonstrativpronomen} \\\cmidrule(lr){2-5}\cmidrule(lr){6-9}
		& \NOM & \AKK & \DAT & \GEN & \NOM & \AKK & \DAT & \GEN\\\midrule
		\textsc{m.sg.} & dær/da & dær/da & dam & ds & dæ & dæ & dem & desch\\
		\textsc{n.sg.} & ds & ds & dam & ds & das & das & dem & desch\\
		\textsc{f.sg.} & di & di & der & der & dei & dei & der & der\\
		\textsc{pl} & di & di & da & der & dei & dei & dene & der\\
		\lspbottomrule
	\end{tabular}
\end{table}

% \textbf{Paradigma 85: Bestimmter Artikel und \isi{Demonstrativpronomen} von Visperterminen \citep[141]{Wipf1911}}

\begin{table}[H]
	\caption{Bestimmter Artikel und Demonstrativpronomen von Visperterminen \citep[141]{Wipf1911}}\label{table85}
	\begin{tabular}{lllllllll}
		\lsptoprule
		& \multicolumn{4}{c}{bestimmter Artikel} &  \multicolumn{4}{c}{Demonstrativpronomen} \\\cmidrule(lr){2-5}\cmidrule(lr){6-9}
		& \NOM & \AKK & \DAT & \GEN & \NOM & \AKK & \DAT & \GEN\\\midrule
		\textsc{m.sg.} & dr & dun/dr & dm/m & ds & dær & dɛnu/dær & dɛm & des\\
		\textsc{n.sg.} & ds & ds & dm/m & ds & das & das & dɛm & des\\
		\textsc{f.sg.} & d & d & dr & dr & di & di & dær & dær\\
		\textsc{pl} & d & d & du/de & dr & di & di & dɛne & dær\\
		\lspbottomrule
	\end{tabular}
\end{table}

% \textbf{Paradigma 86: Bestimmter Artikel und \isi{Demonstrativpronomen} von Jaun \citep[282-283]{Stucki1917}}

\begin{table}[H]
	\caption{Bestimmter Artikel und Demonstrativpronomen von Jaun \citep[282-283]{Stucki1917}}\label{table86}
	\begin{tabular}{lllllllll}
		\lsptoprule
		& \multicolumn{4}{c}{bestimmter Artikel} &  & \multicolumn{3}{c}{Demonstrativpronomen}\\\cmidrule(lr){2-5}\cmidrule(lr){7-9}
		& \NOM & \AKK & \DAT & \GEN &  & \NOM & \AKK & \DAT\\\midrule
		\textsc{m.sg.} & dər & dər/ə & dəm/əm & ts & \textsc{m.sg.} & dær & dær & dɛm\\
		\textsc{n.sg.} & ts & ts & dəm/əm & ts & \textsc{n.sg.} & das & das & dɛm\\
		\textsc{f.sg.} & di/t & di/t & dər & dər & \textsc{f.sg.} & di & di & dɛr\\
		\textsc{pl} & di/t & di/t & də & dər & \textsc{m./n.pl.} & di & di & dɛnə\\
		&  &  &  &  & \textsc{fem.pl.} & diu & diu & dɛnə\\
		\lspbottomrule
	\end{tabular}
\end{table}

% \textbf{Paradigma 87: Bestimmter Artikel und \isi{Demonstrativpronomen} des Sensebezirks \citep[200-201]{Henzen1927}}

\begin{table}[H]
	\caption{Bestimmter Artikel und Demonstrativpronomen des Sensebezirks \citep[200-201]{Henzen1927}}\label{table87}
	\begin{tabular}{lllllll}
		\lsptoprule
		& \multicolumn{3}{c}{bestimmter Artikel} & \multicolumn{3}{c}{Demonstrativpronomen} \\\cmidrule(lr){2-4}\cmidrule(lr){5-7}
		& \NOM & \AKK & \DAT & \NOM & \AKK & \DAT\\\midrule
		\textsc{m.sg.} & dər & dər/ə & dum/um & dæ & dæ & dɛm\\
		\textsc{n.sg.} & ts & ts & dum/um & das & das & dɛm\\
		\textsc{f.sg.} & di/d & di/d & dər & di & di & dər\\
		\textsc{pl} & di/d & di/d & də & di & di & dɛnə\\
		\lspbottomrule
	\end{tabular}
\end{table}

% \textbf{Paradigma 88: Bestimmter Artikel und \isi{Demonstrativpronomen} von Uri \citep[194-195]{Clauß1929}}

\begin{table}[H]
	\caption{Bestimmter Artikel und Demonstrativpronomen von Uri \citep[194-195]{Clauß1929}}\label{table88}
	\begin{tabular}{llllllll}
		\lsptoprule
		& \multicolumn{4}{c}{bestimmter Artikel} & \multicolumn{3}{c}{Demonstrativpronomen}\\\cmidrule(lr){2-5}\cmidrule(lr){6-8}
		& \NOM & \AKK & \DAT & \textsc{poss} & \NOM & \AKK & \DAT\\\midrule
		\textsc{m.sg.} & dər & dər/ə & dəm/əm & ts & dɛr & dɛr & dɛm\\
		\textsc{n.sg.} & ts & ts & dəm/əm & ts & das & das & dɛm\\
		\textsc{f.sg.} & t/di & t/di & dər & ts & diɐ & diɐ & dɛrɐ\\
		\textsc{pl} & t/di & t/di & də &  & diɐ & diɐ & dɛnɐ\\
		\lspbottomrule
	\end{tabular}
\end{table}

% \textbf{Paradigma 89: Bestimmter Artikel und \isi{Demonstrativpronomen} von Vorarlberg \citep[276-279]{Jutz1925}}

\begin{table}[H]
	\caption{Bestimmter Artikel und Demonstrativpronomen von Vorarlberg \citep[276-279]{Jutz1925}}\label{table89}
	\begin{tabular}{llllllll}
		\lsptoprule
		& \multicolumn{4}{c}{bestimmter Artikel}  & \multicolumn{3}{c}{Demonstrativpronomen} \\\cmidrule(lr){2-5}\cmidrule(lr){6-8}
		& \NOM & \AKK & \DAT & \textsc{poss} & \NOM & \AKK & \DAT\\\midrule
		\textsc{m.sg.} & dr & də/e & dem/m & s & d\=er & d\=e & dem\\
		\textsc{n.sg.} & ts & ts & dem/m & s & das & das & dem\\
		\textsc{f.sg.} & di/t & di/t & der & s & diə & diə & der\\
		\textsc{pl} & di/t & di/t & də &  & diə & diə & denə\\
		\lspbottomrule
	\end{tabular}
\end{table}

% \textbf{Paradigma 90: Bestimmter Artikel und \isi{Demonstrativpronomen} von Zürich \citep[101-104, 139-140]{Weber1987}}

\begin{table}[H]
	\caption{Bestimmter Artikel und Demonstrativpronomen von Zürich \citep[101-104, 139-140]{Weber1987}}\label{table90}
	\begin{tabular}{lllllll}
		\lsptoprule
		& \multicolumn{3}{c}{bestimmter Artikel} & \multicolumn{3}{c}{Demonstrativpronomen} \\\cmidrule(lr){2-4}\cmidrule(lr){5-7}		
		& \NOM & \AKK & \DAT & \NOM & \AKK & \DAT\\\midrule
		\textsc{m.sg.} & də & də & əm & dɛ & dɛ & dɛm\\
		\textsc{n.sg.} & s & s & əm & d\=as & d\=as & dɛm\\
		\textsc{f.sg.} & d & d & dər & diə & diə & dɛrə\\
		\textsc{pl} & d & d & də & diə & diə & dɛnə\\
		\lspbottomrule
	\end{tabular}
\end{table}

% \textbf{Paradigma 91: Bestimmter Artikel und \isi{Demonstrativpronomen} von Bern \citep[77-79, 102-103]{Marti1985}}

\begin{table}[H]
	\caption{Bestimmter Artikel und Demonstrativpronomen von Bern \citep[77-79, 102-103]{Marti1985}}\label{table91}
	\begin{tabular}{lllllll}
		\lsptoprule
		& \multicolumn{3}{c}{bestimmter Artikel} & \multicolumn{3}{c}{Demonstrativpronomen} \\\cmidrule(lr){2-4}\cmidrule(lr){5-7}		
		& \NOM & \AKK & \DAT & \NOM & \AKK & \DAT\\\midrule
		\textsc{m.sg.} & dər & dər/ə & əm & dæ & dæ & dæm\\
		\textsc{n.sg.} & ds & ds & əm & das & das & dæm\\
		\textsc{f.sg.} & d/di & d/di & dər & diə & diə & dɛr/dɛrə\\
		\textsc{pl} & d/di & d/di & də & diə & diə & dɛnə\\
		\lspbottomrule
	\end{tabular}
\end{table}

% \textbf{Paradigma 92: Bestimmter Artikel und \isi{Demonstrativpronomen} von Huzenbach \citep[100, 104-105]{Baur1967}}

\begin{table}[H]
	\caption{Bestimmter Artikel und Demonstrativpronomen von Huzenbach \citep[100, 104-105]{Baur1967}}\label{table92}
	\begin{tabular}{llllllll}
		\lsptoprule
		& \multicolumn{4}{c}{bestimmter Artikel} & \multicolumn{3}{c}{Demonstrativpronomen} \\\cmidrule(lr){2-5}\cmidrule(lr){6-8}
		& \NOM & \AKK & \DAT & \textsc{poss} & \NOM & \AKK & \DAT\\\midrule
		\textsc{m.sg.} & dǝ & dǝ & ǝm & s & d\=er & d\=e͂n & d\=e͂m\\
		\textsc{n.sg.} & s & s & ǝm & s & d\=es & d\=es & d\=e͂m\\
		\textsc{f.sg.} & d & d & dǝ & s & d\=i/d\=iǝ & d\=i/d\=iǝ & d\=erǝ\\
		\textsc{pl} & d/de/diǝ & d/de/diǝ & de &  & d\=iǝ & d\=iǝ & d\=e͂əne\\
		\lspbottomrule
	\end{tabular}
\end{table}

% \textbf{Paradigma 93: Bestimmter Artikel und \isi{Demonstrativpronomen} von Saulgau \citep[115-116, 119-120]{Raichle1932}}

\begin{table}[H]
	\caption{Bestimmter Artikel und Demonstrativpronomen von Saulgau \citep[115-116, 119-120]{Raichle1932}}\label{table93}
	\begin{tabular}{llllllll}
		\lsptoprule
		& \multicolumn{4}{c}{bestimmter Artikel}  & \multicolumn{3}{c}{Demonstrativpronomen} \\\cmidrule(lr){2-5}\cmidrule(lr){6-8}
		& \NOM & \AKK & \DAT & \textsc{poss} & \NOM & \AKK & \DAT\\\midrule
		\textsc{m.sg.} & dr & dǝ/n & m & s & dɛǝr & dɛǝn & dǝm\\
		\textsc{n.sg.} & s & s & m & s & d\=es & d\=es & dǝm\\
		\textsc{f.sg.} & d & d & dr & s & d\=iǝ & d\=iǝ & dɛǝrǝ\\
		\textsc{pl} & d & d & de &  & d\=iǝ & d\=iǝ & dǝnne\\
		\lspbottomrule
	\end{tabular}
\end{table}

% \textbf{Paradigma 94: Bestimmter Artikel und \isi{Demonstrativpronomen} von Stuttgart \citep[154-155]{Frey1975}}

\begin{table}[H]
	\caption{Bestimmter Artikel und Demonstrativpronomen von Stuttgart \citep[154-155]{Frey1975}}\label{table94}
	\begin{tabular}{lllllll}
		\lsptoprule
		& \multicolumn{3}{c}{bestimmter Artikel} & \multicolumn{3}{c}{Demonstrativpronomen} \\\cmidrule(lr){2-4}\cmidrule(lr){5-7}
		& \NOM & \AKK & \DAT & \NOM & \AKK & \DAT\\\midrule
		\textsc{m.sg.} & dǝr & dǝ & ǝm & dɛr & den & dem\\
		\textsc{n.sg.} & s & s & ǝm & des & des & dem\\
		\textsc{f.sg.} & d & d & dǝr & d\=i & d\=i & dɛrǝ\\
		\textsc{pl} & d & d & də & d\=i & d\=i & dene\\
		\lspbottomrule
	\end{tabular}
\end{table}

% \textbf{Paradigma 95: Bestimmter Artikel und \isi{Demonstrativpronomen} von Petrifeld \citep[63-64, 65]{Moser1937}}

\begin{table}[H]
	\caption{Bestimmter Artikel und Demonstrativpronomen von Petrifeld \citep[63-64, 65]{Moser1937}}\label{table95}
	\begin{tabular}{llllllll}
		\lsptoprule
		& \multicolumn{4}{c}{bestimmter Artikel}  & \multicolumn{3}{c}{Demonstrativpronomen} \\\cmidrule(lr){2-5}\cmidrule(lr){6-8}
		& \NOM & \AKK & \DAT & \textsc{poss} & \NOM & \AKK & \DAT\\\midrule
		\textsc{m.sg.} & dr & dr & im/ǝm/m & s & di\=eǝr & di\=eǝr & dieǝm\\
		\textsc{n.sg.} & s & s & im/ǝm/m & s & d\=es & d\=es & dieǝm\\
		\textsc{f.sg.} & d & d & dr & s & diǝ & diǝ & di\=eǝrǝ\\
		\textsc{pl} & d/de & d/de & de & s & dieǝne & dieǝne & dieǝne\\
		\lspbottomrule
	\end{tabular}
\end{table}

% \textbf{Paradigma 96: Bestimmter Artikel und \isi{Demonstrativpronomen} von Elisabethtal \citep[52]{Žirmunskij1928/29}}

\begin{table}[H]
	\caption{Bestimmter Artikel und Demonstrativpronomen von Elisabethtal \citep[52]{Žirmunskij1928/29}}\label{table96}
	\begin{tabular}{lllllll}
		\lsptoprule
		& \multicolumn{3}{c}{bestimmter Artikel} & \multicolumn{3}{c}{Demonstrativpronomen} \\\cmidrule(lr){2-4}\cmidrule(lr){5-7}
		& \NOM & \AKK & \DAT & \NOM & \AKK & \DAT\\\midrule
		\textsc{m.sg.} & dr & dɐ͂ & ǝm/m & deɐr & deɐ͂ & deɐm\\
		\textsc{n.sg.} & ts & ts & ǝm/m & d\=es & d\=es & deɐm\\
		\textsc{f.sg.} & d & d & dr & dui & dui & deɐ͂rɐ\\
		\textsc{pl} & d & d & d\=e͂ & diǝ & diǝ & deɐnǝ\\
		\lspbottomrule
	\end{tabular}
\end{table}

% \textbf{Paradigma 97: Bestimmter Artikel und \isi{Demonstrativpronomen} des Kaiserstuhls \citep[359-378]{Noth1993}}

\begin{table}[H]
	\caption{Bestimmter Artikel und Demonstrativpronomen des Kaiserstuhls \citep[359-378]{Noth1993}}\label{table97}
	\begin{tabular}{llllllll}
		\lsptoprule
		& \multicolumn{4}{c}{bestimmter Artikel}  & \multicolumn{3}{c}{Demonstrativpronomen} \\\cmidrule(lr){2-5}\cmidrule(lr){6-8}
		& \NOM & \AKK & \DAT & \textsc{poss} & \NOM & \AKK & \DAT\\\midrule
		\textsc{m.sg.} & dr & dr & im & s & d\=a & d\=a & dam\\
		\textsc{n.sg.} & s & s & im & s & des & des & dam\\
		\textsc{f.sg.} & d & d & dr & s & diɐ & diɐ & d\=arɐ\\
		\textsc{pl} & d & d & dr &  & diɐ & diɐ & d\=anɐ\\
		\lspbottomrule
	\end{tabular}
\end{table}

% \textbf{Paradigma 98: Bestimmter Artikel und \isi{Demonstrativpronomen} des Münstertals \citep[47-48]{Mankel1886}}

\begin{table}[H]
	\caption{Bestimmter Artikel und Demonstrativpronomen des Münstertals \citep[47-48]{Mankel1886}}\label{table98}
	\begin{tabular}{lllllll}
		\lsptoprule
		& \multicolumn{3}{c}{bestimmter Artikel} & \multicolumn{3}{c}{Demonstrativpronomen} \\\cmidrule(lr){2-4}\cmidrule(lr){5-7}
		& \NOM & \AKK & \DAT & \NOM & \AKK & \DAT\\\midrule
		\textsc{m.sg.} & tər & tər & æm/əm & t\=ar & t\=ar & tam\\
		\textsc{n.sg.} & s & s & æm/əm & t\=as & t\=as & tam\\
		\textsc{f.sg.} & ti & ti & tə & tie & tie & t\=ar\\
		\textsc{pl} & tə & tə & tə & ti & ti & t\=anə\\
		\lspbottomrule
	\end{tabular}
\end{table}

% \textbf{Paradigma 99: Bestimmter Artikel und \isi{Demonstrativpronomen} von Colmar \citep[68-70, 83]{Henry1900}}

\begin{table}[H]
	\caption{Bestimmter Artikel und Demonstrativpronomen von Colmar \citep[68-70, 83]{Henry1900}}\label{table99}
	\begin{tabular}{lllllll}
		\lsptoprule
		& \multicolumn{3}{c}{bestimmter Artikel} & \multicolumn{3}{c}{Demonstrativpronomen} \\\cmidrule(lr){2-4}\cmidrule(lr){5-7}
		& \NOM & \AKK & \DAT & \NOM & \AKK & \DAT\\\midrule
		\textsc{m.sg.} & tr & tr & em/m & tar/ta & tar/ta & tam\\
		\textsc{n.sg.} & s & s & em/m & tes & tes & tam\\
		\textsc{f.sg.} & t & t & tr & tiə/te & tiə/te & tare\\
		\textsc{pl} & t & t & te & tiə/te & tiə/te & tane\\
		\lspbottomrule
	\end{tabular}

\end{table}

% \textbf{Paradigma 100: Bestimmter Artikel und \isi{Demonstrativpronomen} des Elsass (Ebene) \citep[72-78, 84-88]{Beyer1963}}

\begin{table}[H]
	\caption{Bestimmter Artikel und Demonstrativpronomen des Elsass (Ebene) \citep[72-78, 84-88]{Beyer1963}}\label{table100}
	\begin{tabular}{lllllll}
		\lsptoprule
		& \multicolumn{3}{c}{bestimmter Artikel} & \multicolumn{3}{c}{Demonstrativpronomen} \\\cmidrule(lr){2-4}\cmidrule(lr){5-7}
		& \NOM & \AKK & \DAT & \NOM & \AKK & \DAT\\\midrule
		\textsc{m.sg.} & dər & dər/də & im/m & der & der & dem\\
		\textsc{n.sg.} & s & s & im/m & dɑs & dɑs & dem\\
		\textsc{f.sg.} & d & d & dər & diǝ & diǝ & der\\
		\textsc{pl} & d & d & də & diǝ & diǝ & denə\\
		\lspbottomrule
	\end{tabular}

\end{table}

\section{Unbestimmter Artikel und Possessivpronomen}

% \textbf{Paradigma 101: \isi{Possessivpronomen} des Althochdeutschen \citep[245-246]{Braune2004}}

\begin{table}[H]
	\caption{Possessivpronomen des Althochdeutschen \citep[245-246]{Braune2004}}\label{table101}
	\begin{tabular}{llllllllll}
		\lsptoprule
		& \multicolumn{5}{c}{\textsc{sg}}  & \multicolumn{4}{c}{\textsc{pl}} \\\cmidrule(lr){2-6}\cmidrule(lr){7-10}
		& \NOM & \AKK & \DAT & \GEN & \INSTR & \NOM & \AKK & \DAT & \GEN\\\midrule
		\textsc{m} & \mbox{{}-\=er/-ø} & {}-an & {}-emu & {}-es & {}-u & {}-e & {}-e & {}-\=em & {}-ero\\
		\textsc{n} & \mbox{{}-az/-ø} & \mbox{{}-az/-ø} & {}-emu & {}-es & {}-u & {}-u & {}-u & {}-\=em & {}-ero\\
		\textsc{f} & {}-u/-ø & {}-a & {}-eru & {}-era &  & {}-o & {}-o & {}-\=em & {}-ero\\
		\midrule
		\textsc{3.sg.f.} & ira &  &  &  &  &  &  &  & \\
		\textsc{3.pl.} & iro &  &  &  &  &  &  &  & \\
		\lspbottomrule
	\end{tabular}
\end{table}

% \mbox{3.Sg.f.} &  & ira &  &  &  &  &  &  & \\
% 		\mbox{3.Pl.} &  & iro &  &  &  &  &  &  & \\

% \textbf{Paradigma 102: Unbestimmter Artikel und \isi{Possessivpronomen} des Mittelhochdeutschen \citep[216-217, 231]{Paul2007}}

\begin{table}[H]
	\caption{Unbestimmter Artikel und Possessivpronomen des Mittelhochdeutschen \citep[216-217, 231]{Paul2007}}\label{table102}
	\begin{tabular}{llllllllll}
		\lsptoprule
		& \multicolumn{4}{p{4cm}}{\mbox{unbestimmter Artikel}\newline \mbox{+ Possessivpronomen}}  & & \multicolumn{4}{l}{Possessivpronomen}\\\cmidrule(lr){2-5}\cmidrule(lr){7-10}
		& \NOM & \AKK & \DAT & \GEN &  & \NOM & \AKK & \DAT & \GEN\\\midrule
		\textsc{m.sg.} & {}-ø & {}-ən & {}-əm & {}-əs & \textsc{m.sg.} & {}-ə & {}-ən & {}-ən & {}-ən\\
		\textsc{n.sg.} & {}-ø & {}-ø & {}-əm & {}-əs & \textsc{n.sg.} & {}-ə & {}-ə & {}-ən & {}-ən\\
		\textsc{f.sg.} & {}-ø & {}-ø/-ə & {}-ər & {}-ər & \textsc{f.sg.} & {}-ə & {}-ən & {}-ən & {}-ən\\
		\textsc{m.pl.} & {}-ə & {}-ə & {}-ən & {}-ər & \textsc{pl} & {}-ən & {}-ən & {}-ən & {}-ən\\
		\textsc{n.pl.} & {}-iu & {}-iu & {}-ən & {}-ər &  &  &  &  & \\
		\textsc{f.pl.} & {}-ə & {}-ə & {}-ən & {}-ər &  &  &  &  & \\
		\midrule
		\textsc{3.sg.f.+ 3.pl.} & ir &  &  &  &  &  &  &  & \\
		\lspbottomrule
	\end{tabular}
\end{table}

% \textbf{Paradigma 103: Unbestimmter Artikel und \isi{Possessivpronomen} der deutschen Standardsprache \citep[169-177]{Eisenberg2006}}

\begin{table}[H]
	\caption{Unbestimmter Artikel und Possessivpronomen der deutschen Standardsprache \citep[169-177]{Eisenberg2006}}\label{table103}
	\begin{tabular}{llllllllll}
		\lsptoprule
		& \multicolumn{4}{c}{unbestimmter Artikel}  &  & \multicolumn{4}{c}{Possessivpronomen} \\\cmidrule(lr){2-5}\cmidrule(lr){7-10}
		& \NOM & \AKK & \DAT & \GEN &  & \NOM & \AKK & \DAT & \GEN\\\midrule
		\textsc{m.sg.} & {}-ø & {}-ən & {}-əm & {}-əs & \textsc{m.sg.} & {}-ø & {}-ən & {}-əm & {}-əs\\
		\textsc{n.sg.} & {}-ø & {}-ø & {}-əm & {}-əs & \textsc{n.sg.} & {}-ø & {}-ø & {}-əm & {}-əs\\
		\textsc{f.sg.} & {}-ə & {}-ə & {}-ər & {}-ər & \textsc{f.sg.} & {}-ə & {}-ə & {}-ər & {}-ər\\
		&  &  &  &  & \textsc{pl} & {}-ə & {}-ə & {}-ən & {}-ər\\
		\lspbottomrule
	\end{tabular}
\end{table}

% \textbf{Paradigma 104: Unbestimmter Artikel und \isi{Possessivpronomen} von Issime \citep[13-16, 82-84]{Perinetto1981}}

\begin{table}[H]
	\caption{Unbestimmter Artikel und Possessivpronomen von Issime \citep[13-16, 82-84]{Perinetto1981}}\label{table104}
	\begin{tabular}{llllll}
		\lsptoprule
		\multicolumn{6}{l}{unbestimmter Artikel}\\
		&  & \NOM & \AKK & \DAT & \GEN\\\midrule
		& \textsc{m} & {}-ø & {}-ø & {}-m & {}-s\\
		& \textsc{n} & {}-s & {}-s & {}-m & {}-s\\
		& \textsc{f} & {}-ø & {}-ø & {}-r & {}-r\\\midrule
		\multicolumn{6}{l}{Possessivpronomen}\\
		& & \multicolumn{4}{c}{\scshape 1.-3. \textsc{sg}}\\\cmidrule(lr){1-6}
		&  & \NOM & \AKK & \DAT & \GEN\\\midrule
		\textsc{sg} & \textsc{m} & {}-ø & {}-ø & {}-m & {}-sch\\
		& \textsc{n} & {}-s & {}-s & {}-m & {}-sch\\
		& \textsc{f} & {}-ø & {}-ø & {}-r & {}-r\\
		\textsc{pl} & \textsc{m} & {}-ø & {}-ø & {}-e & {}-r\\
		& \textsc{n} & {}-i & {}-i & {}-e & {}-r\\
		& \textsc{f} & {}-ø & {}-ø & {}-e & {}-r\\\midrule
		& & \multicolumn{4}{c}{\scshape 1./2. \textsc{pl}} \\\cmidrule(lr){1-6}
		&  & \NOM & \AKK & \DAT & \GEN\\\midrule
		\textsc{sg} & \textsc{m} & {}-e & {}-e & {}-em & {}-s\\
		& \textsc{n} & {}-s & {}-s & {}-em & {}-s\\
		& \textsc{f} & {}-ø & {}-ø & {}-er & {}-er\\
		\textsc{pl} & \textsc{m} & {}-ø & {}-ø & {}-e & {}-er\\
		& \textsc{n} & {}-i & {}-i & {}-ene & {}-er\\
		& \textsc{f} & {}-ø & {}-ø & {}-e & {}-er\\\midrule
		& & \multicolumn{4}{c}{\scshape 3. \textsc{pl}} \\\cmidrule(lr){1-6}
		&  & \NOM & \AKK & \DAT & \GEN\\\midrule
		\textsc{sg} & \textsc{m} & {}-ø & {}-ø & {}-ø & {}-ø\\
		& \textsc{n} & {}-ø & {}-ø & {}-ø & {}-ø\\
		& \textsc{f} & {}-ø & {}-ø & {}-er & {}-er\\
		\textsc{pl} & \textsc{m} & {}-ø & {}-ø & {}-e & {}-er\\
		& \textsc{n} & {}-ø & {}-ø & {}-ene & {}-er\\
		& \textsc{f} & {}-ø & {}-ø & {}-e & {}-er\\
		\midrule
		\textsc{3.sg.f.} & irra &  &  &  & \\
		\lspbottomrule
	\end{tabular}
\end{table}

% \textbf{Paradigma 105: Unbestimmter Artikel und \isi{Possessivpronomen} von Visperterminen \citep[137, 144]{Wipf1911}}

\begin{table}[H]
	\caption{Unbestimmter Artikel und Possessivpronomen von Visperterminen \citep[137, 144]{Wipf1911}}\label{table105}
	\begin{tabular}{lllll}
		\lsptoprule
		\multicolumn{5}{c}{unbestimmter Artikel} \\
		& \NOM & \AKK & \DAT & \GEN\\\midrule
		\textsc{m} & {}-ø & {}-ø & \mbox{{}-um/-am/-mu} & {}-s\\
		\textsc{n} & {}-s & {}-s & \mbox{{}-um/-am/-mu} & {}-s\\
		\textsc{f} & {}-ø & {}-ø & {}-ar & {}-ar\\\midrule
		\multicolumn{5}{c}{Possessivpronomen} \\
		& \NOM & \AKK & \DAT & \GEN\\\midrule
		\textsc{m.sg.} & {}-e & {}-e & {}-um & {}-s\\
		\textsc{n.sg.} & {}-s & {}-s & {}-um & {}-s\\
		\textsc{f.sg.} & {}-i & {}-i & {}-er & {}-er\\
		\textsc{pl} & {}-i & {}-i & {}-e & {}-er\\
		\midrule
		\textsc{3.sg.f.} & ira &  &  & \\
		\textsc{3.pl.} & iro &  &  & \\
		\lspbottomrule
	\end{tabular}
\end{table}

% \textbf{Paradigma 106: Unbestimmter Artikel und \isi{Possessivpronomen} von Jaun \citep[277-278, 284-285]{Stucki1917}}

\begin{table}[H]
	\caption{Unbestimmter Artikel und Possessivpronomen von Jaun \citep[277-278, 284-285]{Stucki1917}}\label{table106}
	\begin{tabular}{llllll}
		\lsptoprule
		\multicolumn{6}{l}{unbestimmter Artikel} \\
		&  & \NOM & \AKK & \DAT & \\\midrule
		& \textsc{m} & a & a/ən-a & əmənə/əmə & \\
		& \textsc{n} & as & as/ən-as & əmənə/əmə & \\
		& \textsc{f} & a & a/ən-a & andərə/ərə & \\ \midrule
		\multicolumn{6}{l}{Possessivpronomen} \\
		& & \multicolumn{4}{c}{\scshape 1./2. \textsc{sg}, 3. \textsc{sg} m./n.}\\\cmidrule(lr){1-6}
		&  & \NOM & \AKK & \DAT & \GEN\\\midrule
		\textsc{sg} & \textsc{m} & {}-a/-ø & {}-a/-ø & {}-m & {}-s\\
		& \textsc{n} & {}-s & {}-s & {}-m & {}-s\\
		& \textsc{f} & {}-i/-ø & {}-i/-ø & {}-r & {}-r\\
		\textsc{pl} & \textsc{m} & {}-ø & {}-ø & {}-ə & {}-r\\
		& \textsc{n} & {}-i & {}-i & {}-ə & {}-r\\
		& \textsc{f} & {}-ø & {}-ø & {}-ə & {}-r\\\midrule
		& & \multicolumn{4}{c}{\scshape 1./2. \textsc{pl}}\\\cmidrule(lr){1-6}
		&  & \NOM & \AKK & \DAT & \GEN\\\midrule
		\textsc{sg} & \textsc{m} & {}-a & {}-a & {}-m & {}-ø\\
		& \textsc{n} & {}-ø & {}-ø & {}-m & {}-ø\\
		& \textsc{f} & {}-i & {}-i & {}-r & {}-ø\\
		\textsc{pl} & \textsc{m} & {}-ø & {}-ø & {}-ə & \\
		& \textsc{n} & {}-i & {}-i & {}-ə & \\
		& \textsc{f} & {}-ø & {}-ø & {}-ə & \\
		\midrule
		\textsc{3.sg.f.} & ira &  &  &  & \\
		\textsc{3.pl.} & iru &  &  &  & \\
		\lspbottomrule
	\end{tabular}
\end{table}

% \textbf{Paradigma 107: Unbestimmter Artikel und \isi{Possessivpronomen} des Sensebezirks \citep[194, 198-199]{Henzen1927}}

\begin{table}[H]
	\caption{Unbestimmter Artikel und Possessivpronomen des Sensebezirks \citep[194, 198-199]{Henzen1927}}\label{table107}
	\begin{tabular}{llll}
		\lsptoprule
		\multicolumn{4}{l}{unbestimmter Artikel}\\
		& \NOM & \AKK & \DAT\\\midrule
		\textsc{m} & a & a/ən-a & ama/ima/aməna/iməna/əməna\\
		\textsc{n} & as & as/ən-as & ama/ima/aməna/iməna/əməna\\
		\textsc{f} & a & a/ən-a & anəra/inəra/əra/ənəra\\ \midrule
		\multicolumn{4}{l}{Possessivpronomen}\\
		\multicolumn{4}{c}{\scshape 1./2. \textsc{sg}, 3. \textsc{sg} m./n.} \\\cmidrule(lr){1-4}
		& \NOM & \AKK & \DAT\\\midrule
		\textsc{m.sg.} & {}-ø & {}-ø & {}-um/-m\\
		\textsc{n.sg.} & {}-s & {}-s & {}-um/-m\\
		\textsc{f.sg.} & {}-ø & {}-ø & {}-ər\\
		\textsc{pl} & {}-ər/-ø & {}-ər/-ø & {}-ə\\ \midrule
		\multicolumn{4}{c}{\scshape 1./2. \textsc{pl}}  \\\cmidrule(lr){1-4}
		& \NOM & \AKK & \DAT\\\midrule
		\textsc{m.sg.} & {}-a & {}-a & {}-um\\
		\textsc{n.sg.} & {}-ərsch & {}-ərsch & {}-um\\
		\textsc{f.sg.} & {}-i & {}-i & {}-ər\\
		\textsc{pl} & {}-ər & {}-ər & {}-ə\\
		\midrule
		\textsc{3.sg.f.} & \=ira/\=iras &  & \\
		\textsc{3.pl.} & \=irə/\=irəs &  & \\
		\lspbottomrule
	\end{tabular}
\end{table}

% \textbf{Paradigma 108: Unbestimmter Artikel und \isi{Possessivpronomen} von Uri \citep[189, 193-194]{Clauß1929}}

\begin{table}[H]
	\caption{Unbestimmter Artikel und Possessivpronomen von Uri \citep[189, 193-194]{Clauß1929}}\label{table108}
	\begin{tabular}{llllll}
		\lsptoprule
		\multicolumn{6}{l}{unbestimmter Artikel} \\
		&  & \NOM & \AKK & \DAT & \\\midrule
		& \textsc{m} & ɐ & ɐ/ən-ɐ & \mbox{ɑmənɐ/ɑmɐ/əmɐ} & \\
		& \textsc{n} & ɐs & ɐs/ən- ɐs & \mbox{ɑmənɐ/ɑmɐ/əmɐ} & \\
		& \textsc{f} & ɐ & ɐ/ən-ɐ & \mbox{ɑnərɐ/ərɐ/nərɐ} & \\ \midrule
		\multicolumn{6}{l}{Possessivpronomen}  \\
		\multicolumn{6}{c}{\scshape 1./2. \textsc{sg}, 3. \textsc{sg} m./n.} \\\cmidrule(lr){1-6}
		&  & \NOM & \AKK & \DAT & \GEN\\\midrule
		\textsc{sg} & \textsc{m} & {}-ɐ/-ø & {}-ɐ/-ø & {}-m & {}-s\\
		& \textsc{n} & {}-s/-ø & {}-s/-ø & {}-m & {}-s\\
		& \textsc{f} & {}-i/-ø & {}-i/-ø & {}-ər & \\
		\textsc{pl} & - & {}-i & {}-i & {}-ɐ & \\ \midrule
		\multicolumn{6}{c}{\scshape 1./2. \textsc{pl}} \\\cmidrule(lr){1-6}
		&  & \NOM & \AKK & \DAT & \\\midrule
		\textsc{sg} & \textsc{m} & {}-ɐ & {}-ɐ & {}-m & \\
		& \textsc{n} & {}-s & {}-s & {}-m & \\
		& \textsc{f} & {}-i & {}-i & {}-ər & \\
		\textsc{pl} & \textsc{m} & {}-ø & {}-ø & {}-nɐ & \\
		& \textsc{n} & {}-i & {}-i & {}-nɐ & \\
		& \textsc{f} & {}-ø & {}-ø & {}-nɐ & \\ \midrule
		\multicolumn{6}{c}{\scshape 3.sg.f., 3.pl.} \\\cmidrule(lr){1-6}
		&  & \NOM & \AKK & \DAT & \\\midrule
		\textsc{sg} & \textsc{m} & {}-ɐ & {}-ɐ & {}-m & \\
		& \textsc{n} & {}-əss & {}-əss & {}-m & \\
		& \textsc{f} & {}-i & {}-i & {}-ər & \\
		\textsc{pl} & - & {}-i & {}-i & {}-nɐ & \\
		\lspbottomrule
	\end{tabular}
\end{table}

% \textbf{Paradigma 109: Unbestimmter Artikel und \isi{Possessivpronomen} von Vorarlberg \citep[269-270, 274-276]{Jutz1925}}

\begin{table}[H]
	\caption{Unbestimmter Artikel und Possessivpronomen von Vorarlberg \citep[269-270, 274-276]{Jutz1925}}\label{table109}
	\begin{tabular}{llll}
		\lsptoprule
		\multicolumn{4}{l}{unbestimmter Artikel}\\
		& \NOM & \AKK & \DAT\\\midrule
		\textsc{m} & ən & ən & əmənə/əmə\\
		\textsc{n} & ə & ə & əmənə/əmə\\
		\textsc{f} & ə & ə & ənərə/ərə\\ \midrule
		\multicolumn{4}{l}{Possessivpronomen}\\
		\multicolumn{4}{c}{\scshape 1./2. \textsc{sg}, 3. \textsc{sg} m./n.} \\\cmidrule(lr){1-4}
		& \NOM & \AKK & \DAT\\\midrule
		\textsc{m.sg.} & {}-ə & {}-ə & {}-m\\
		\textsc{n.sg.} & {}-ø & {}-ø & {}-m\\
		\textsc{f.sg.} & {}-i & {}-i & {}-r/-ərə\\
		\textsc{pl} & {}-i & {}-i & {}-ə\\ \midrule
		\multicolumn{4}{c}{\scshape 1./2. \textsc{pl}} \\\cmidrule(lr){1-4}
		& \NOM & \AKK & \DAT\\\midrule
		\textsc{m.sg.} & {}-ə & {}-ə & {}-m\\
		\textsc{n.sg.} & {}-ø/üsə & {}-ø/üsə & {}-m\\
		\textsc{f.sg.} & {}-i & {}-i & {}-ər\\
		\textsc{pl} & {}-i & {}-i & {}-nə\\
		\lspbottomrule
	\end{tabular}
\end{table}

% \textbf{Paradigma 110: Unbestimmter Artikel und \isi{Possessivpronomen} von Zürich \citep[104-107, 135-139]{Weber1987}}

\begin{table}[H]
	\caption{Unbestimmter Artikel und Possessivpronomen von Zürich \citep[104-107, 135-139]{Weber1987}}\label{table110}
	\begin{tabular}{llll}
		\lsptoprule
		\multicolumn{4}{l}{unbestimmter Artikel}\\\cmidrule(lr){1-4}
		& \NOM & \AKK & \DAT\\\midrule
		\textsc{m} & ən & ən/ən-ən & əmənə/əmə\\
		\textsc{n} & əs & əs/ən-əs & əmənə/əmə\\
		\textsc{f} & ə & ə/ən-ə & ənərə/ərə\\ \midrule
		\multicolumn{4}{l}{Possessivpronomen}\\
		\multicolumn{4}{c}{\scshape 1./2. \textsc{sg}, 3. \textsc{sg} m./n.} \\\cmidrule(lr){1-4}
		& \NOM & \AKK & \DAT\\\midrule
		\textsc{m.sg.} & {}-ø & {}-ø & {}-m\\
		\textsc{n.sg.} & {}-s & {}-s & {}-m\\
		\textsc{f.sg.} & {}-i & {}-i & {}-ərə\\
		\textsc{pl} & {}-i & {}-i & {}-ə\\ \midrule
		\multicolumn{4}{c}{\scshape 1.-3. \textsc{pl}, 3. \textsc{sg} f.} \\\cmidrule(lr){1-4}
		& \NOM & \AKK & \DAT\\\midrule
		\textsc{m.sg.} & {}-ə & {}-ə & {}-m\\
		\textsc{n.sg.} & {}-s & {}-s & {}-m\\
		\textsc{f.sg.} & {}-i & {}-i & {}-ə\\
		\textsc{pl} & {}-i & {}-i & {}-ə\\
		\lspbottomrule
	\end{tabular}
\end{table}

% \textbf{Paradigma 111: Unbestimmter Artikel und \isi{Possessivpronomen} von Bern \citep[79, 98-101]{Marti1985}}

\begin{table}[H]
	\caption{Unbestimmter Artikel und Possessivpronomen von Bern \citep[79, 98-101]{Marti1985}}\label{table111}
	\begin{tabular}{llll}
		\lsptoprule
		\multicolumn{4}{l}{unbestimmter Artikel}\\	
		& \NOM & \AKK & \DAT\\\midrule
		\textsc{m} & ə & ə/n-ə & əmənə/əmnə/əmə\\
		\textsc{n} & əs & əs/n-əs & əmənə/əmnə/əmə\\
		\textsc{f} & ə & ə/n-ə & ənərə/ərə\\ \midrule
		\multicolumn{4}{l}{Possessivpronomen}\\
		\multicolumn{4}{c}{\scshape1. \textsc{sg}} \\\cmidrule(lr){1-4}
		& \NOM & \AKK & \DAT\\\midrule
		\textsc{m.sg.} & {}-ø & {}-ø & {}-m\\
		\textsc{n.sg.} & {}-s & {}-s & {}-m\\
		\textsc{f.sg.} & {}-ø & {}-ø & {}-rə\\
		\textsc{pl} & {}-i & {}-i & {}-ər\\ \midrule
		\multicolumn{4}{c}{\scshape 1. \textsc{pl}} \\\cmidrule(lr){1-4}
		& \NOM & \AKK & \DAT\\\midrule
		\textsc{m.sg.} & {}-ə & {}-ə & {}-m\\
		\textsc{n.sg.} & {}-s & {}-s & {}-m\\
		\textsc{f.sg.} & {}-i & {}-i & {}-ər\\
		\textsc{pl} & {}-i & {}-i & {}-ər\\ \midrule
		\multicolumn{4}{c}{\scshape 2. \textsc{sg}, 3. \textsc{sg} m./n.} \\\cmidrule(lr){1-4}
		& \NOM & \AKK & \DAT\\\midrule
		\textsc{m.sg.} & {}-ø & {}-ø & {}-m\\
		\textsc{n.sg.} & {}-s & {}-s & {}-m\\
		\textsc{f.sg.} & {}-ø & {}-ø & {}-rə\\
		\textsc{pl} & {}-i/-ər & {}-i/-ər & {}-ə\\ \midrule
		\multicolumn{4}{c}{\scshape 2./3. \textsc{pl}, 3. \textsc{sg} f.} \\
		& \NOM & \AKK & \DAT\\\midrule
		\textsc{m.sg.} & {}-ə & {}-ə & {}-m\\
		\textsc{n.sg.} & {}-s & {}-s & {}-m\\
		\textsc{f.sg.} & {}-i & {}-i & {}-er\\
		\textsc{pl} & {}-i/-ər & {}-i/-ər & {}-nə\\
		\lspbottomrule
	\end{tabular}
\end{table}

% \textbf{Paradigma 112: Unbestimmter Artikel und \isi{Possessivpronomen} von Huzenbach \citep[101, 104]{Baur1967}}

\begin{table}[H]
	\caption{Unbestimmter Artikel und Possessivpronomen von Huzenbach \citep[101, 104]{Baur1967}}\label{table112}
	\begin{tabular}{llll}
		\lsptoprule
		\multicolumn{4}{c}{unbestimmter Artikel}\\
		& \NOM & \AKK & \DAT\\\midrule
		\textsc{m} & ə/ǝn & ǝn & ǝmǝ\\
		\textsc{n} & ǝ & ǝ & ǝmǝ\\
		\textsc{f} & ǝ & ǝ & ǝrǝ\\\midrule
		\multicolumn{4}{c}{Possessivpronomen} \\       
		& \NOM & \AKK & \DAT\\\midrule
		\textsc{m.sg.} & {}-ø & {}-n & {}-m\\
		\textsc{n.sg.} & {}-ø & {}-ø & {}-m\\
		\textsc{f.sg.} & {}-ø & {}-ø & {}-nǝr/-nǝrǝ/-ǝrǝ\\
		\textsc{pl} & {}-nə & {}-nə & {}-nə\\
		\lspbottomrule
	\end{tabular}
\end{table}

% \textbf{Paradigma 113: Unbestimmter Artikel und \isi{Possessivpronomen} von Saulgau \citep[116-119]{Raichle1932}}

\begin{table}[H]
	\caption{Unbestimmter Artikel und Possessivpronomen von Saulgau \citep[116-119]{Raichle1932}}\label{table113}
	\begin{tabular}{llll}
		\lsptoprule
		\multicolumn{4}{l}{unbestimmter Artikel}\\
		& \NOM & \AKK & \DAT\\\midrule
		 \textsc{m} & n & n & e͂mǝ/ǝmǝ\\
		 \textsc{n} & ǝ & ǝ & e͂mǝ/ǝmǝ\\
		 \textsc{f} & ǝ & ǝ & e͂nǝrǝ/ǝrǝ\\ \midrule
		\multicolumn{4}{l}{Possessivpronomen}\\
		\multicolumn{4}{c}{\scshape 1./2. \textsc{sg}, 3. \textsc{sg} m./n.}  \\\cmidrule(lr){1-4}
		& \NOM & \AKK & \DAT\\\midrule
		\textsc{m.sg.} & {}-n & {}-n & {}-m\\
		\textsc{n.sg.} & {}-ø & {}-ø & {}-m\\
		\textsc{f.sg.} & {}-ø & {}-ø & {}-ǝrǝ\\
		\textsc{pl} & {}-e & {}-e & {}-e\\ \midrule
		\multicolumn{4}{c}{\scshape 1./2. \textsc{pl}} \\   \cmidrule(lr){1-4}
		& \NOM & \AKK & \DAT\\\midrule
		\textsc{m.sg.} & {}-ø/-n & {}-ø/-n & {}-m\\
		\textsc{n.sg.} & {}-ø & {}-ø & {}-m\\
		\textsc{f.sg.} & {}-ø & {}-ø & {}-ǝ\\
		\textsc{pl} & {}-e & {}-e & {}-e\\ \midrule
		\multicolumn{4}{c}{\scshape 3. \textsc{sg} f., 3. \textsc{pl}} \\ \cmidrule(lr){1-4}
		& \NOM & \AKK & \DAT\\\midrule
		\textsc{m.sg.} & {}-ø/-n & {}-n & {}-m\\
		\textsc{n.sg.} & {}-ø & {}-ø & {}-m\\
		\textsc{f.sg.} & {}-ø & {}-ø & {}-ø\\
		\textsc{pl} & {}-e & {}-e & {}-e\\
		\lspbottomrule
	\end{tabular}
\end{table}

% \textbf{Paradigma 114: Unbestimmter Artikel und \isi{Possessivpronomen} von Stuttgart \citep[156]{Frey1975}}

\begin{table}[H]
	\caption{Unbestimmter Artikel und Possessivpronomen von Stuttgart \citep[156]{Frey1975}}\label{table114}
	\begin{tabular}{llll}
		\lsptoprule
		\multicolumn{4}{l}{unbestimmter Artikel}\\
		& \NOM & \AKK & \DAT\\\midrule
		 \textsc{m} & ǝn & ǝn & ǝmǝ\\
		 \textsc{n} & ǝ & ǝ & ǝmǝ\\
		 \textsc{f} & ǝ & ǝ & ǝrǝ\\ \midrule
		\multicolumn{4}{c}{Possessivpronomen}\\		
		\multicolumn{4}{c}{\scshape 1./2. \textsc{sg}, 3. \textsc{sg} m./n.} \\\cmidrule(lr){1-4}
		& \NOM & \AKK & \DAT\\\midrule
		\textsc{m.sg.} & {}-ø & {}-n & {}-m\\
		\textsc{n.sg.} & {}-ø & {}-ø & {}-m\\
		\textsc{f.sg.} & {}-ø & {}-ø & {}-ǝr/-ǝrǝ\\
		\textsc{pl} & {}-e & {}-e & {}-e\\ \midrule
		\multicolumn{4}{c}{\scshape 1.-3. \textsc{pl}, 3. \textsc{sg} f.}  \\\cmidrule(lr){1-4}
		& \NOM & \AKK & \DAT\\\midrule
		\textsc{m.sg.} & {}-ø & {}-n & {}-m\\
		\textsc{n.sg.} & {}-ø & {}-ø & {}-m\\
		\textsc{f.sg.} & {}-ø & {}-ø & {}-ø/-r\\
		\textsc{pl} & {}-e & {}-e & {}-e\\
		\lspbottomrule
	\end{tabular}

\end{table}

% \textbf{Paradigma 115: Unbestimmter Artikel und \isi{Possessivpronomen} von Petrifeld \citep[64-66]{Moser1937}}

\begin{table}[H]
	\caption{Unbestimmter Artikel und Possessivpronomen von Petrifeld \citep[64-66]{Moser1937}}\label{table115}
	\begin{tabular}{llll}
		\lsptoprule
		\multicolumn{4}{l}{unbestimmter Artikel}\\	
		& \NOM & \AKK & \DAT\\\midrule
		 \textsc{m} & ǝn/n & ǝn/n & imǝ/ǝmǝ\\
		 \textsc{n} & ǝ & ǝ & imǝ/ǝmǝ\\
		 \textsc{f} & ǝ & ǝ & inrǝ/rǝ\\ \midrule
		\multicolumn{4}{l}{Possessivpronomen}\\
		\multicolumn{4}{c}{\scshape 1./2. \textsc{sg}, 3. \textsc{sg} m./n.} \\\cmidrule(lr){1-4}
		& \NOM & \AKK & \DAT\\\midrule
		\textsc{m.sg.} & {}-ø & {}-ø & {}-m\\
		\textsc{n.sg.} & {}-ø & {}-ø & {}-m\\
		\textsc{f.sg.} & {}-ø & {}-ø & {}-rǝ\\
		\textsc{pl} & {}-e & {}-e & {}-e\\ \midrule
		\multicolumn{4}{c}{\scshape1./2. \textsc{pl}, 3. \textsc{sg} f.} \\\cmidrule(lr){1-4}
		\textsc{m.sg.} & {}-ø & {}-ø & {}-m\\
		\textsc{n.sg.} & {}-ø & {}-ø & {}-m\\
		\textsc{f.sg.} & {}-ø & {}-ø & {}-ǝ\\
		\textsc{pl} & {}-e & {}-e & {}-e\\ \midrule
		\multicolumn{4}{c}{\scshape 3. \textsc{pl}} \\\cmidrule(lr){1-4}
		\textsc{m.sg.} & {}-ø & {}-ø & {}-m\\
		\textsc{n.sg.} & {}-ø & {}-ø & {}-m\\
		\textsc{f.sg.} & {}-ø & {}-ø & {}-m\\
		\textsc{pl} & {}-ø & {}-ø & {}-ø\\
		\lspbottomrule
	\end{tabular}
\end{table}

% \textbf{Paradigma 116: Unbestimmter Artikel und \isi{Possessivpronomen} von Elisabethtal \citep[52]{Žirmunskij1928/29}}

\begin{table}[H]
	\caption{Unbestimmter Artikel und Possessivpronomen von Elisabethtal \citep[52]{Žirmunskij1928/29}}\label{table116}
	\begin{tabular}{llll}
		\lsptoprule
		\multicolumn{4}{l}{unbestimmter Artikel}\\
		& \NOM & \AKK & \DAT\\\midrule
		 \textsc{m} & ɐ͂ & en & emɐ\\
		 \textsc{n} & ɐ͂ & ɐ͂ & emɐ\\
		 \textsc{f} & ɐ͂ & ɐ͂ & ǝrɐ\\ \midrule
		\multicolumn{4}{l}{Possessivpronomen}\\
		\multicolumn{4}{c}{\scshape 1./2. \textsc{sg}, 3. \textsc{sg} m./n.}  \\\cmidrule(lr){1-4}
		& \NOM & \AKK & \DAT\\\midrule
		\textsc{m.sg.} & {}-ø & {}-n & {}-m\\
		\textsc{n.sg.} & {}-ø & {}-ø & {}-m\\
		\textsc{f.sg.} & {}-ø & {}-ø & {}-rɐ\\
		\textsc{pl} & {}-ǝ & {}-ǝ & {}-ǝ\\ \midrule
		\multicolumn{4}{c}{\scshape 1./2. \textsc{pl}}  \\\cmidrule(lr){1-4}
		& \NOM & \AKK & \DAT\\\midrule
		\textsc{m.sg.} & {}-ø & {}-n & {}-m\\
		\textsc{n.sg.} & {}-ø & {}-ø & {}-m\\
		\textsc{f.sg.} & {}-ø & {}-ø & {}-ǝr\\
		\textsc{pl} & {}-ǝ & {}-ǝ & {}-ǝ\\
		\lspbottomrule
	\end{tabular}
\end{table}

% \textbf{Paradigma 117: Unbestimmter Artikel und \isi{Possessivpronomen} des Kaiserstuhls \citep[376, 380-384]{Noth1993}}

\begin{table}[H]
	\caption{Unbestimmter Artikel und Possessivpronomen des Kaiserstuhls \citep[376, 380-384]{Noth1993}}\label{table117}
	\begin{tabular}{llll}
		\lsptoprule
		\multicolumn{4}{l}{unbestimmter Artikel}\\		
		& \NOM & \AKK & \DAT\\\midrule
		 \textsc{m} & a & a & imɐ\\
		 \textsc{n} & a & a & imɐ\\
		 \textsc{f} & a & a & inərɐ\\ \midrule
		\multicolumn{4}{l}{Possessivpronomen}\\	
		\multicolumn{4}{c}{\scshape 1./2. \textsc{sg}, 3. \textsc{sg} m./n.} \\\cmidrule(lr){1-4}	
		& \NOM & \AKK & \DAT\\\midrule
		\textsc{m.sg.} & {}-ø & {}-ø & {}-əm/-m\\
		\textsc{n.sg.} & {}-ø & {}-ø & {}-əm/-m\\
		\textsc{f.sg.} & {}-ø & {}-ø & {}-ərɐ/-rɐ\\
		\textsc{pl} & {}-i & {}-i & {}-ənɐ/-ɐ\\ \midrule
		\multicolumn{4}{c}{\scshape 1./2. \textsc{pl}} \\\cmidrule(lr){1-4}	
		& \NOM & \AKK & \DAT\\\midrule
		\textsc{m.sg.} & {}-ø & {}-ø & {}-əm\\
		\textsc{n.sg.} & {}-ø & {}-ø & {}-əm\\
		\textsc{f.sg.} & {}-ø & {}-ø & {}-ɐ\\
		\textsc{pl} & {}-i & {}-i & {}-ɐ\\ \midrule
		\multicolumn{4}{c}{\scshape 3. \textsc{sg} f., 3. pl} \\\cmidrule(lr){1-4}	
		& \NOM & \AKK & \DAT\\\midrule
		\textsc{m.sg.} & {}-ɐ & {}-ɐ & {}-ənəm/-əm\\
		\textsc{n.sg.} & {}-ɐ & {}-ɐ & {}-ənəm/-əm\\
		\textsc{f.sg.} & {}-i & {}-i & {}-ərɐ/-ɐ\\
		\textsc{pl} & {}-i & {}-i & {}-ənɐ/-ɐ\\
		\lspbottomrule
	\end{tabular}
\end{table}

% \textbf{Paradigma 118: Unbestimmter Artikel und \isi{Possessivpronomen} des Münstertals \citep[45-47]{Mankel1886}}

\begin{table}[H]
	\caption{Unbestimmter Artikel und Possessivpronomen des Münstertals \citep[45-47]{Mankel1886}}\label{table118}
	\begin{tabular}{llll}
		\lsptoprule
		\multicolumn{4}{c}{unbestimmter Artikel}\\
		& \NOM & \AKK & \DAT\\\midrule
		 \textsc{m} & ə & ə & æmə/əmə\\
		 \textsc{n} & ə & ə & æmə/əmə\\
		 \textsc{f} & ə & ə & ænərə/ənərə\\\midrule
               	 \multicolumn{4}{c}{Possessivpronomen}\\
                & \NOM & \AKK & \DAT\\\midrule
		\textsc{m.sg.} & {}-ø & {}-ø & {}-m\\
		\textsc{n.sg.} & {}-ø & {}-ø & {}-m\\
		\textsc{f.sg.} & {}-i & {}-i & {}-ər\\
		\textsc{pl} & {}-i & {}-i & {}-ə\\
		\lspbottomrule
	\end{tabular}
\end{table}

% \textbf{Paradigma 119: Unbestimmter Artikel und \isi{Possessivpronomen} von Colmar \citep[70-71, 84-85]{Henry1900}}

\begin{table}[H]
	\caption{Unbestimmter Artikel und Possessivpronomen von Colmar \citep[70-71, 84-85]{Henry1900}}\label{table119}
	\begin{tabular}{llll}
		\lsptoprule
		\multicolumn{4}{c}{unbestimmter Artikel}\\	
		& \NOM & \AKK & \DAT\\\midrule
		 \textsc{m} & e & e & eme/me\\
		 \textsc{n} & e & e & eme/me\\
		 \textsc{f} & e & e & enre/re\\\midrule
               \multicolumn{4}{c}{Possessivpronomen}\\
               & \NOM & \AKK & \DAT\\\midrule
		\textsc{m.sg.} & {}-ø & {}-ø & {}-m\\
		\textsc{n.sg.} & {}-ø & {}-ø & {}-m\\
		\textsc{f.sg.} & {}-i & {}-i & {}-re\\
		\textsc{pl} & {}-i & {}-i & {}-e\\
		\lspbottomrule
	\end{tabular}
\end{table}

% \textbf{Paradigma 120: Unbestimmter Artikel und \isi{Possessivpronomen} des Elsass (Ebene) \citep[78-83, 98-109]{Beyer1963}}

\begin{table}[H]
	\caption{Unbestimmter Artikel und Possessivpronomen des Elsass (Ebene) \citep[78-83, 98-109]{Beyer1963}}\label{table120}
	\begin{tabular}{llll}
		\lsptoprule
	  \multicolumn{4}{c}{unbestimmter Artikel}\\
		& \NOM & \AKK & \DAT\\\midrule
		 \textsc{m} & ə & ə & imə/əmə\\
		 \textsc{n} & ə & ə & imə/əmə\\
		 \textsc{f} & ə & ə & inərə/ərə\\\midrule
               \multicolumn{4}{c}{Possessivpronomen}\\
               & \NOM & \AKK & \DAT\\\midrule
		\textsc{m.sg.} & {}-ə & {}-ə & {}-m\\
		\textsc{n.sg.} & {}-ø & {}-ø & {}-m\\
		\textsc{f.sg.} & {}-i & {}-i & {}-ərə\\
		\textsc{pl} & {}-i & {}-i & {}-ə\\
		\lspbottomrule
	\end{tabular}

\end{table}

% \textbf{1. Substantive}

% % \textbf{Paradigma 1: Althochdeutsches \isi{Substantiv} \citep[183-217]{Braune2004}}

% \begin{table}[H]
% \caption{Althochdeutsches Substantiv \citep[183-217]{Braune2004}}\label{table1}
% \begin{tabular}{lll}
% \lsptoprule
% TABLE\\
% \lspbottomrule
% \end{tabular}
% \end{table}

% % \textbf{Paradigma 2: Mittelhochdeutsches \isi{Substantiv} \citep[183-199]{Paul2007}}

% \begin{table}[H]
% \caption{Mittelhochdeutsches Substantiv \citep[183-199]{Paul2007}}\label{table2}
% \begin{tabular}{lll}
% \lsptoprule
% TABLE\\
% \lspbottomrule
% \end{tabular}
% \end{table}

% % \textbf{Paradigma 3: Substantivflexion der deutschen Standardsprache \citep[158-169]{Eisenberg2006}}

% \begin{table}[H]
% \caption{Substantivflexion der deutschen Standardsprache \citep[158-169]{Eisenberg2006}}\label{table3}
% \begin{tabular}{lll}
% \lsptoprule
% TABLE\\
% \lspbottomrule
% \end{tabular}
% \end{table}

% % \textbf{Paradigma 4: Substantivflexion von Issime \citep[144-205]{Zürrer1999}}

% \begin{table}[H]
% \caption{Substantivflexion von Issime \citep[144-205]{Zürrer1999}}\label{table4}
% \begin{tabular}{lll}
% \lsptoprule
% TABLE\\
% \lspbottomrule
% \end{tabular}
% \end{table}

% % \textbf{Paradigma 5: Substantivflexion von Visperterminen \citep[119-134]{Wipf1911}}

% \begin{table}[H]
% \caption{Substantivflexion von Visperterminen \citep[119-134]{Wipf1911}}\label{table5}
% \begin{tabular}{lll}
% \lsptoprule
% TABLE\\
% \lspbottomrule
% \end{tabular}
% \end{table}

% % \textbf{Paradigma 6: Substantivflexion von Jaun \citep[255-272]{Stucki1917}}

% \begin{table}[H]
% \caption{Substantivflexion von Jaun \citep[255-272]{Stucki1917}}\label{table6}
% \begin{tabular}{lll}
% \lsptoprule
% TABLE\\
% \lspbottomrule
% \end{tabular}
% \end{table}

% % \textbf{Paradigma 7: Substantivflexion des Sensebezirks \citep[179-190]{Henzen1927}}

% \begin{table}[H]
% \caption{Substantivflexion des Sensebezirks \citep[179-190]{Henzen1927}}\label{table7}
% \begin{tabular}{lll}
% \lsptoprule
% \textsc{fk} & \textsc{sg} & Plural\\
% 1 & ascht & ɛscht\\
% 2 & bach & bæch\\
% 3 & bærg & bærg-ə\\
% 4 & pfana & pfan-ə\\
% 5 & malər & malər\\
% 6 & nɛts & nɛts-əni\\
% 7 & blati & blatə{}-(n)i\\
% 8 & rad & rɛd-ər\\
% 9 & tach & tæch-ər\\
% 10 & lamm & lamm-ər\\
% & \multicolumn{2}{X}{Possessiv-S}\\
% \lspbottomrule
% \end{tabular}
% \end{table}


% % \textbf{Paradigma 8: Substantivflexion von Uri \citep[173-185]{Clauß1929}}

% \begin{table}[H]
% \caption{Substantivflexion von Uri \citep[173-185]{Clauß1929}}\label{table8}
% \begin{tabular}{llll}
% \lsptoprule
% & \textsc{sg} & \multicolumn{2}{X}{Plural}\\
% \textsc{fk} &  & \NOM\slash\AKK & \DAT\\\midrule
% 1 & ɑscht & escht & escht-ɐ\\
% 2 & chrɑmpf & chrampf & chrampf-ɐ\\
% 3 & mɑlər & mɑlər & mɑlər-ɐ\\
% 4 & chnacht & chnacht-ɐ & chnacht-ɐ\\
% 5 & fɑttər & fattər-ɐ & fattər-ɐ\\
% 6 & fadɐ & fad-əm & fap-m-ɐ\\
% 7 & schnidəri & schnidər-ɐ & schnidər-ɐ\\
% 8 & kcharli & kcharl-əs-ɐ & kcharl-əs-ɐ\\
% 9 & blɑt & blet-ǝr & blet-ǝr-ɐ\\
% 10 & tɑch & tach-ər & tach-ər-ɐ\\
% 11 & bet & bet-i & bet-ənɐ\\
% 12 & nets & nets-i & nets-ɐ\\
% & \multicolumn{3}{X}{Possessiv-S}\\
% \lspbottomrule
% \end{tabular}
% \end{table}

% % \textbf{Paradigma 9: Substantivflexion von Vorarlberg \citep[231-261]{Jutz1925}}

% \begin{table}[H]
% \caption{Substantivflexion von Vorarlberg \citep[231-261]{Jutz1925}}\label{table9}
% \begin{tabular}{llll}
% \lsptoprule
% & \textsc{sg} & \multicolumn{2}{X}{Plural}\\
% \textsc{fk} &  & \NOM\slash\AKK & \DAT\\\midrule
% 1 & schlag & schleg & \\
% 2 & bǣrg & bǣrg & bǣrg-ə\\
% 3 & kchind & kchind & \\
% 4 & wart & wart & wart-ə\\
% 5 & bett & bett-ər & \\
% 6 & \=arm & \=arm-ə & \\
% 7 & fuədər & füədər-ə & \\
% & \multicolumn{3}{X}{Possessiv-S}\\
% \lspbottomrule
% \end{tabular}
% \end{table}

% % \textbf{Paradigma 10: Substantivflexion von Zürich \citep[108-119]{Weber1987}}

% \begin{table}[H]
% \caption{Substantivflexion von Zürich \citep[108-119]{Weber1987}}\label{table10}
% \begin{tabular}{llll}\lsptoprule
% & \textsc{sg} & \textsc{pl} & \\
% \textsc{fk} &  & \NOM\slash\AKK & \DAT\\\midrule
% 1 & gascht & gescht & {}-ə\\
% 2 & bank & bænk & {}-ə\\
% 3 & h\=aggə & hȫggə & {}-ə\\
% 4 & rad & red-ər & {}-ə\\
% 5 & fass & fæss-ər & \\
% 6 & vattər & vætter-ə & {}-ə\\
% 7 & p\=ur & p\=ur-ə & {}-ə\\
% 8 & fisch & fisch & {}-ə\\
% & \multicolumn{3}{X}{Possessiv-S}\\
% \lspbottomrule
% \end{tabular}
% \end{table}

% % \textbf{Paradigma 11: Substantivflexion von Bern \citep[82-90]{Marti1985}}

% \begin{table}[H]
% \caption{Substantivflexion von Bern \citep[82-90]{Marti1985}}\label{table11}
% \begin{tabular}{lll}
% \lsptoprule
% \textsc{fk} & \textsc{sg} & Plural\\
% 1 & tannə & tannə\\
% 2 & darm & dærm\\
% 3 & gascht & gescht\\
% 4 & tal & tæl-ər\\
% 5 & gl\=as & glɛs-ər\\
% 6 & h\=as & has-ə\\
% 7 & tochtər & töchtər-ə\\
% & \multicolumn{2}{X}{Possessiv-S}\\
% \lspbottomrule
% \end{tabular}
% \end{table}

% % \textbf{Paradigma 12: Substantivflexion von Huzenbach \citep[92-98]{Baur1967}}

% \begin{table}[H]
% \caption{Substantivflexion von Huzenbach \citep[92-98]{Baur1967}}\label{table12}
% \begin{tabular}{lll}
% \lsptoprule
% \textsc{fk} & \textsc{sg} & Plural\\
% 1 & schuə & schuə\\
% 2 & schdal & schdel\\
% 3 & wald & weld-ǝr\\
% 4 & dan & dan-ǝ\\
% 5 & muədər & miədər-ə\\
% 6 & wiǝrde & wiǝrd-ǝnǝ\\
% & \multicolumn{2}{X}{Possessiv-S}\\
% \lspbottomrule
% \end{tabular}
% \end{table}

% % \textbf{Paradigma 13: Substantivflexion von Saulgau \citep[100-109]{Raichle1932}}

% \begin{table}[H]
% \caption{Substantivflexion von Saulgau \citep[100-109]{Raichle1932}}\label{table13}
% \end{table}
% \begin{tabular}{lll}
% \lsptoprule
% \textsc{fk} & \textsc{sg} & Plural\\
% 1 & molr & molr\\
% 2 & gascht & gescht\\
% 3 & arm & ɛrm\\
% 4 & bot & bot-ǝ\\
% 5 & {}-le & {}-lǝ\\
% 6 & blat & blet-r\\
% & \multicolumn{2}{X}{Possessiv-S}\\
% \lspbottomrule
% \end{tabular}

% % \textbf{Paradigma 14: Substantivflexion von Stuttgart \citep[149-152]{Frey1975}}

% \begin{table}[H]
% \caption{Substantivflexion von Stuttgart \citep[149-152]{Frey1975}}\label{table14}
% table\\
% \end{table}

% % \textbf{Paradigma 15: Substantivflexion von Petrifeld \citep[59-62]{Moser1937}}

% \begin{table}[H]
% \caption{Substantivflexion von Petrifeld \citep[59-62]{Moser1937}}\label{table15}
% table\\
% \end{table}

% % \textbf{Paradigma 16: Substantivflexion von Elisabethtal \citep[50-52]{Žirmunskij1928/29}}

% \begin{table}[H]
% \caption{Substantivflexion von Elisabethtal \citep[50-52]{Žirmunskij1928/29}}\label{table16}
% table\\
% \end{table}

% % \textbf{Paradigma 17: Substantivflexion des Kaiserstuhls \citep[359-373]{Noth1993}}

% \begin{table}[H]
% \caption{Substantivflexion des Kaiserstuhls \citep[359-373]{Noth1993}}\label{table17}
% table\\
% \end{table}

% % \textbf{Paradigma 18: Substantivflexion des Münstertals \citep[40-44]{Mankel1886}}

% \begin{table}[H]
% \caption{Substantivflexion des Münstertals \citep[40-44]{Mankel1886}}\label{table18}
% table\\
% \end{table}

% % \textbf{Paradigma 19: Substantivflexion von Colmar \citep[71-76]{Henry1900}}

% \begin{table}[H]
% \caption{Substantivflexion von Colmar \citep[71-76]{Henry1900}}\label{table19}
% table\\
% \end{table}

% % \textbf{Paradigma 20: Substantivflexion des Elsass (Ebene) \citep[25-71]{Beyer1963}}

% \begin{table}[H]
% \caption{Substantivflexion des Elsass (Ebene) \citep[25-71]{Beyer1963}}\label{table20}
% table\\
% \end{table}

% \textbf{2. Stark und schwach flektierte Adjektive}\\

% % \textbf{Paradigma 21: Starke und schwache \isi{Adjektive} im Althochdeutschen \citep[217-227]{Braune2004}}

% \begin{table}[H]
% \caption{Starke und schwache Adjektive im Althochdeutschen \citep[217-227]{Braune2004}}\label{table21}
% table\\
% \end{table}

% % \textbf{Paradigma 22: Starke und schwache \isi{Adjektive} im Mittelhochdeutschen \citep[200-203]{Paul2007}}

% \begin{table}[H]
% \caption{Starke und schwache Adjektive im Mittelhochdeutschen \citep[200-203]{Paul2007}}\label{table22}
% table\\
% \end{table}

% % \textbf{Paradigma 23: Starke und schwache \isi{Adjektive} in der Standardsprache \citep[177-184]{Eisenberg2006}}

% \begin{table}[H]
% \caption{Starke und schwache Adjektive in der Standardsprache \citep[177-184]{Eisenberg2006}}\label{table23}
% table\\
% \end{table}

% % \textbf{Paradigma 24: Starke und schwache \isi{Adjektive} in Issime (\citealt[267-268]{Zürrer1999}, \citealt[90-97]{Perinetto1981})}

% \begin{table}[H]
% \caption{TeStarke und schwache Adjektive in Issime (\citealt[267-268]{Zürrer1999}, \citealt[90-97]{Perinetto1981})st}\label{table24}
% table\\
% \end{table}

% % \textbf{Paradigma 25: Starke und schwache \isi{Adjektive} in Visperterminen \citep[134-135]{Wipf1911}}

% \begin{table}[H]
% \caption{Starke und schwache Adjektive in Visperterminen \citep[134-135]{Wipf1911}}\label{table25}
% table\\
% \end{table}

% % \textbf{Paradigma 26: Starke und schwache \isi{Adjektive} in Jaun \citep[272-275]{Stucki1917}}

% \begin{table}[H]
% \caption{Starke und schwache Adjektive in Jaun \citep[272-275]{Stucki1917}}\label{table26}
% table\\
% \end{table}

% % \textbf{Paradigma 27: Starke und schwache \isi{Adjektive} im Sensebezirk \citep[190-192]{Henzen1927}}

% \begin{table}[H]
% \caption{Starke und schwache Adjektive im Sensebezirk \citep[190-192]{Henzen1927}}\label{table27}
% table\\
% \end{table}

% % \textbf{Paradigma 28: Starke und schwache \isi{Adjektive} in Uri \citep[185-187]{Clauß1929}}

% \begin{table}[H]
% \caption{Starke und schwache Adjektive in Uri \citep[185-187]{Clauß1929}}\label{table28}
% table\\
% \end{table}

% % \textbf{Paradigma 29: Starke und schwache \isi{Adjektive} in Vorarlberg \citep[261-267]{Jutz1925}}

% \begin{table}[H]
% \caption{Starke und schwache Adjektive in Vorarlberg \citep[261-267]{Jutz1925}}\label{table29}
% table\\
% \end{table}

% % \textbf{Paradigma 30: Starke und schwache \isi{Adjektive} in Zürich \citep[121-126]{Weber1987}}

% \begin{table}[H]
% \caption{Starke und schwache Adjektive in Zürich \citep[121-126]{Weber1987}}\label{table30}
% table\\
% \end{table}

% % \textbf{Paradigma 31: Starke und schwache \isi{Adjektive} in Bern \citep[117-120]{Marti1985}}

% \begin{table}[H]
% \caption{Starke und schwache Adjektive in Bern \citep[117-120]{Marti1985}}\label{table31}
% table\\
% \end{table}

% % \textbf{Paradigma 32: Starke und schwache \isi{Adjektive} in Huzenbach \citep[98-99]{Baur1967}}

% \begin{table}[H]
% \caption{Starke und schwache Adjektive in Huzenbach \citep[98-99]{Baur1967}}\label{table32}
% table\\
% \end{table}

% % \textbf{Paradigma 33: Starke und schwache \isi{Adjektive} in Saulgau \citep[109-113]{Raichle1932}}

% \begin{table}[H]
% \caption{Starke und schwache Adjektive in Saulgau \citep[109-113]{Raichle1932}}\label{table33}
% table\\
% \end{table}

% % \textbf{Paradigma 34: Starke und schwache \isi{Adjektive} in Stuttgart \citep[157-159]{Frey1975}}

% \begin{table}[H]
% \caption{Starke und schwache Adjektive in Stuttgart \citep[157-159]{Frey1975}}\label{table34}
% table\\
% \end{table}

% % \textbf{Paradigma 35: Starke und schwache \isi{Adjektive} in Stuttgart \citep[157-159]{Frey1975}}

% \begin{table}[H]
% \caption{Starke und schwache Adjektive in Stuttgart \citep[157-159]{Frey1975}}\label{table35}
% table\\
% \end{table}

% % \textbf{Paradigma 36: Starke und schwache \isi{Adjektive} in Elisabethtal \citep[52]{Žirmunskij1928/29}}

% \begin{table}[H]
% \caption{Starke und schwache Adjektive in Elisabethtal \citep[52]{Žirmunskij1928/29}}\label{table36}
% table\\
% \end{table}

% % \textbf{Paradigma 37: Starke und schwache \isi{Adjektive} im Kaiserstuhl \citep[407-410]{Noth1993}}

% \begin{table}[H]
% \caption{Starke und schwache Adjektive im Kaiserstuhl \citep[407-410]{Noth1993}}\label{table37}
% table\\
% \end{table}

% % \textbf{Paradigma 38: Starke und schwache \isi{Adjektive} im Münstertal \citep[44-45]{Mankel1886}}

% \begin{table}[H]
% \caption{Starke und schwache Adjektive im Münstertal \citep[44-45]{Mankel1886}}\label{table38}
% table\\
% \end{table}

% % \textbf{Paradigma 39: Starke und schwache \isi{Adjektive} in Colmar \citep[77-80]{Henry1900}}

% \begin{table}[H]
% \caption{Starke und schwache Adjektive in Colmar \citep[77-80]{Henry1900}}\label{table39}
% table\\
% \end{table}

% % \textbf{Paradigma 40: Starke und schwache \isi{Adjektive} im Elsass (Ebene) \citep[114-146]{Beyer1963}}

% \begin{table}[H]
% \caption{Test}\label{table40}
% table\\
% \end{table}

% \textbf{3. Personalpronomen}\\

% % \textbf{Paradigma 41: \isi{Personalpronomen} des Althochdeutschen \citep[241-245]{Braune2004}}

% \begin{table}[H]
% \caption{Personalpronomen des Althochdeutschen \citep[241-245]{Braune2004}}\label{table41}
% table\\
% \end{table}

% % \textbf{Paradigma 42: \isi{Personalpronomen} des Mittelhochdeutschen \citep[210-214]{Paul2007}}

% \begin{table}[H]
% \caption{Personalpronomen des Mittelhochdeutschen \citep[210-214]{Paul2007}}\label{table42}
% table\\
% \end{table}

% % \textbf{Paradigma 43: \isi{Personalpronomen} der deutschen Standardsprache \citep[169-177]{Eisenberg2006}}

% \begin{table}[H]
% \caption{Personalpronomen der deutschen Standardsprache \citep[169-177]{Eisenberg2006}}\label{table43}
% table\\
% \end{table}

% % \textbf{Paradigma 44: \isi{Personalpronomen} von Issime \citep[206-312]{Zürrer1999}}

% \begin{table}[H]
% \caption{Personalpronomen von Issime \citep[206-312]{Zürrer1999}}\label{table44}
% table\\
% \end{table}

% % \textbf{Paradigma 45: \isi{Personalpronomen} von Visperterminen \citep[139-141]{Wipf1911}}

% \begin{table}[H]
% \caption{Personalpronomen von Visperterminen \citep[139-141]{Wipf1911}}\label{table45}
% table\\
% \end{table}

% % \textbf{Paradigma 46: \isi{Personalpronomen} von Jaun \citep[280-282]{Stucki1917}}

% \begin{table}[H]
% \caption{Personalpronomen von Jaun \citep[280-282]{Stucki1917}}\label{table46}
% table\\
% \end{table}

% % \textbf{Paradigma 47: \isi{Personalpronomen} des Sensebezirks \citep[196-198]{Henzen1927}}

% \begin{table}[H]
% \caption{Personalpronomen des Sensebezirks \citep[196-198]{Henzen1927}}\label{table47}
% table\\
% \end{table}

% % \textbf{Paradigma 48: \isi{Personalpronomen} von Uri \citep[190-192]{Clauß1929}}

% \begin{table}[H]
% \caption{Personalpronomen von Uri \citep[190-192]{Clauß1929}}\label{table48}
% table\\
% \end{table}

% % \textbf{Paradigma 49: \isi{Personalpronomen} von Vorarlberg \citep[271-274]{Jutz1925}}

% \begin{table}[H]
% \caption{Personalpronomen von Vorarlberg \citep[271-274]{Jutz1925}}\label{table49}
% table\\
% \end{table}

% % \textbf{Paradigma 50: \isi{Personalpronomen} von Zürich \citep[153-162]{Weber1987}}

% \begin{table}[H]
% \caption{Personalpronomen von Zürich \citep[153-162]{Weber1987}}\label{table50}
% table\\
% \end{table}

% % \textbf{Paradigma 51: \isi{Personalpronomen} von Bern \citep[92-97]{Marti1985}}

% \begin{table}[H]
% \caption{Personalpronomen von Bern \citep[92-97]{Marti1985}}\label{table51}
% table\\
% \end{table}

% % \textbf{Paradigma 52: \isi{Personalpronomen} von Huzenbach \citep[102-103]{Baur1967}}

% \begin{table}[H]
% \caption{Personalpronomen von Huzenbach \citep[102-103]{Baur1967}}\label{table52}
% table\\
% \end{table}

% % \textbf{Paradigma 53: \isi{Personalpronomen} von Saulgau \citep[116-117]{Raichle1932}}

% \begin{table}[H]
% \caption{Personalpronomen von Saulgau \citep[116-117]{Raichle1932}}\label{table53}
% table\\
% \end{table}

% % \textbf{Paradigma 54: \isi{Personalpronomen} von Stuttgart \citep[160-161]{Frey1975}}

% \begin{table}[H]
% \caption{Personalpronomen von Stuttgart \citep[160-161]{Frey1975}}\label{table54}
% table\\
% \end{table}

% % \textbf{Paradigma 55: \isi{Personalpronomen} von Petrifeld \citep[64-65]{Moser1937}}

% \begin{table}[H]
% \caption{Personalpronomen von Petrifeld \citep[64-65]{Moser1937}}\label{table55}
% table\\
% \end{table}

% % \textbf{Paradigma 56: \isi{Personalpronomen} von Elisabethtal \citep[52]{Žirmunskij1928/29}}

% \begin{table}[H]
% \caption{Personalpronomen von Elisabethtal \citep[52]{Žirmunskij1928/29}}\label{table56}
% table\\
% \end{table}

% % \textbf{Paradigma 57: \isi{Personalpronomen} des Kaiserstuhls \citep[393-398]{Noth1993}}

% \begin{table}[H]
% \caption{Personalpronomen des Kaiserstuhls \citep[393-398]{Noth1993}}\label{table57}
% table\\
% \end{table}

% % \textbf{Paradigma 58: \isi{Personalpronomen} des Münstertals \citep[46-47]{Mankel1886}}

% \begin{table}[H]
% \caption{Personalpronomen des Münstertals \citep[46-47]{Mankel1886}}\label{table58}
% table\\
% \end{table}

% % \textbf{Paradigma 59: \isi{Personalpronomen} von Colmar \citep[81-83]{Henry1900}}

% \begin{table}[H]
% \caption{Personalpronomen von Colmar \citep[81-83]{Henry1900}}\label{table59}
% table\\
% \end{table}

% % \textbf{Paradigma 60: \isi{Personalpronomen} des Elsass (Ebene) \citep[ 151-159]{Beyer1963}}

% \begin{table}[H]
% \caption{Personalpronomen des Elsass (Ebene) \citep[ 151-159]{Beyer1963}}\label{table60}
% table\\
% \end{table}

% \textbf{4. Interrogativpronomen}\\

% % \textbf{Paradigma 61: \isi{Interrogativpronomen} des Althochdeutschen \citep[252-253]{Braune2004}}

% \begin{table}[H]
% \caption{Interrogativpronomen des Althochdeutschen \citep[252-253]{Braune2004}}\label{table61}
% table\\
% \end{table}

% % \textbf{Paradigma 62: \isi{Interrogativpronomen} des Mittelhochdeutschen \citep[222-223]{Paul2007}}

% \begin{table}[H]
% \caption{Interrogativpronomen des Mittelhochdeutschen \citep[222-223]{Paul2007}}\label{table62}
% table\\
% \end{table}

% % \textbf{Paradigma 63: \isi{Interrogativpronomen} der deutschen Standardsprache \citep[169-177]{Eisenberg2006}}

% \begin{table}[H]
% \caption{Interrogativpronomen der deutschen Standardsprache \citep[169-177]{Eisenberg2006}}\label{table63}
% table\\
% \end{table}

% % \textbf{Paradigma 64: \isi{Interrogativpronomen} von Issime \citep[258-259, 306]{Zürrer1999}}

% \begin{table}[H]
% \caption{Interrogativpronomen von Issime \citep[258-259, 306]{Zürrer1999}}\label{table64}
% table\\
% \end{table}

% % \textbf{Paradigma 65: \isi{Interrogativpronomen} von Visperterminen \citep[143]{Wipf1911}}

% \begin{table}[H]
% \caption{Interrogativpronomen von Visperterminen \citep[143]{Wipf1911}}\label{table65}
% table\\
% \end{table}

% % \textbf{Paradigma 66: \isi{Interrogativpronomen} von Jaun \citep[285-286]{Stucki1917}}

% \begin{table}[H]
% \caption{Interrogativpronomen von Jaun \citep[285-286]{Stucki1917}}\label{table66}
% table\\
% \end{table}

% % \textbf{Paradigma 67: \isi{Interrogativpronomen} des Sensebezirks \citep[201-202]{Henzen1927}}

% \begin{table}[H]
% \caption{Interrogativpronomen des Sensebezirks \citep[201-202]{Henzen1927}}\label{table67}
% table\\
% \end{table}

% % \textbf{Paradigma 68: \isi{Interrogativpronomen} von Uri \citep[196]{Clauß1929}}

% \begin{table}[H]
% \caption{Interrogativpronomen von Uri \citep[196]{Clauß1929}}\label{table68}
% table\\
% \end{table}

% % \textbf{Paradigma 69: \isi{Interrogativpronomen} von Vorarlberg \citep[282]{Jutz1925}}

% \begin{table}[H]
% \caption{Interrogativpronomen von Vorarlberg \citep[282]{Jutz1925}}\label{table69}
% table\\
% \end{table}

% % \textbf{Paradigma 70: \isi{Interrogativpronomen} von Zürich \citep[144-145]{Weber1987}}

% \begin{table}[H]
% \caption{Interrogativpronomen von Zürich \citep[144-145]{Weber1987}}\label{table70}
% table\\
% \end{table}

% % \textbf{Paradigma 71: \isi{Interrogativpronomen} von Bern \citep[106]{Marti1985}}

% \begin{table}[H]
% \caption{Interrogativpronomen von Bern \citep[106]{Marti1985}}\label{table71}
% table\\
% \end{table}

% % \textbf{Paradigma 72: \isi{Interrogativpronomen} von Huzenbach \citep[105]{Baur1967}}

% \begin{table}[H]
% \caption{Interrogativpronomen von Huzenbach \citep[105]{Baur1967}}\label{table72}
% table\\
% \end{table}

% % \textbf{Paradigma 73: \isi{Interrogativpronomen} von Saulgau \citep[120]{Raichle1932}}

% \begin{table}[H]
% \caption{Interrogativpronomen von Saulgau \citep[120]{Raichle1932}}\label{table73}
% table\\
% \end{table}

% % \textbf{Paradigma 74: \isi{Interrogativpronomen} von Stuttgart \citep[162]{Frey1975}}

% \begin{table}[H]
% \caption{Interrogativpronomen von Stuttgart \citep[162]{Frey1975}}\label{table74}
% table\\
% \end{table}

% % \textbf{Paradigma 75: \isi{Interrogativpronomen} von Petrifeld \citep[66-67]{Moser1937}}

% \begin{table}[H]
% \caption{Interrogativpronomen von Petrifeld \citep[66-67]{Moser1937}}\label{table75}
% table\\
% \end{table}

% % \textbf{Paradigma 76: \isi{Interrogativpronomen} von Elisabethtal \citep[53]{Žirmunskij1928/29}}

% \begin{table}[H]
% \caption{Interrogativpronomen von Elisabethtal \citep[53]{Žirmunskij1928/29}}\label{table76}
% table\\
% \end{table}

% % \textbf{Paradigma 77: \isi{Interrogativpronomen} des Kaiserstuhls \citep[385]{Noth1993}}

% \begin{table}[H]
% \caption{Interrogativpronomen des Kaiserstuhls \citep[385]{Noth1993}}\label{table77}
% table\\
% \end{table}

% % \textbf{Paradigma 78: \isi{Interrogativpronomen} des Münstertals \citep[48]{Mankel1886}}

% \begin{table}[H]
% \caption{Interrogativpronomen des Münstertals \citep[48]{Mankel1886}}\label{table78}
% table\\
% \end{table}

% % \textbf{Paradigma 79: \isi{Interrogativpronomen} von Colmar \citep[85-86]{Henry1900}}

% \begin{table}[H]
% \caption{Interrogativpronomen von Colmar \citep[85-86]{Henry1900}}\label{table79}
% table\\
% \end{table}

% % \textbf{Paradigma 80: \isi{Interrogativpronomen} des Elsass (Ebene) \citep[164-167]{Beyer1963}}

% \begin{table}[H]
% \caption{Interrogativpronomen des Elsass (Ebene) \citep[164-167]{Beyer1963}}\label{table80}
% table\\
% \end{table}

% \textbf{5. Bestimmter Artikel und Demonstrativpronomen}\\

% % \textbf{Paradigma 81: \isi{Demonstrativpronomen} des Althochdeutschen \citep[247-249]{Braune2004}}

% \begin{table}[H]
% \caption{Demonstrativpronomen des Althochdeutschen \citep[247-249]{Braune2004}}\label{table81}
% table\\
% \end{table}

% % \textbf{Paradigma 82: Bestimmter Artikel und \isi{Demonstrativpronomen} des Mittelhochdeutschen \citep[217-219]{Paul2007}}

% \begin{table}[H]
% \caption{Bestimmter Artikel und Demonstrativpronomen des Mittelhochdeutschen \citep[217-219]{Paul2007}}\label{table82}
% table\\
% \end{table}

% % \textbf{Paradigma 83: Bestimmter Artikel und \isi{Demonstrativpronomen} der deutschen Standardsprache \citep[169-177]{Eisenberg2006}}

% \begin{table}[H]
% \caption{Bestimmter Artikel und Demonstrativpronomen der deutschen Standardsprache \citep[169-177]{Eisenberg2006}}\label{table83}
% table\\
% \end{table}

% % \textbf{Paradigma 84: Bestimmter Artikel und \isi{Demonstrativpronomen} von Issime \citep[4-12, 81]{Perinetto1981}}

% \begin{table}[H]
% \caption{Bestimmter Artikel und Demonstrativpronomen von Issime \citep[4-12, 81]{Perinetto1981}}\label{table84}
% table\\
% \end{table}

% % \textbf{Paradigma 85: Bestimmter Artikel und \isi{Demonstrativpronomen} von Visperterminen \citep[141]{Wipf1911}}

% \begin{table}[H]
% \caption{Bestimmter Artikel und Demonstrativpronomen von Visperterminen \citep[141]{Wipf1911}}\label{table85}
% table\\
% \end{table}

% % \textbf{Paradigma 86: Bestimmter Artikel und \isi{Demonstrativpronomen} von Jaun \citep[282-283]{Stucki1917}}

% \begin{table}[H]
% \caption{Bestimmter Artikel und Demonstrativpronomen von Jaun \citep[282-283]{Stucki1917}}\label{table86}
% table\\
% \end{table}

% % \textbf{Paradigma 87: Bestimmter Artikel und \isi{Demonstrativpronomen} des Sensebezirks \citep[200-201]{Henzen1927}}

% \begin{table}[H]
% \caption{Bestimmter Artikel und Demonstrativpronomen des Sensebezirks \citep[200-201]{Henzen1927}}\label{table87}
% table\\
% \end{table}

% % \textbf{Paradigma 88: Bestimmter Artikel und \isi{Demonstrativpronomen} von Uri \citep[194-195]{Clauß1929}}

% \begin{table}[H]
% \caption{Bestimmter Artikel und Demonstrativpronomen von Uri \citep[194-195]{Clauß1929}}\label{table88}
% table\\
% \end{table}

% % \textbf{Paradigma 89: Bestimmter Artikel und \isi{Demonstrativpronomen} von Vorarlberg \citep[276-279]{Jutz1925}}

% \begin{table}[H]
% \caption{Bestimmter Artikel und Demonstrativpronomen von Vorarlberg \citep[276-279]{Jutz1925}}\label{table89}
% table\\
% \end{table}

% % \textbf{Paradigma 90: Bestimmter Artikel und \isi{Demonstrativpronomen} von Zürich \citep[101-104, 139-140]{Weber1987}}

% \begin{table}[H]
% \caption{Bestimmter Artikel und Demonstrativpronomen von Zürich \citep[101-104, 139-140]{Weber1987}}\label{table90}
% table\\
% \end{table}

% % \textbf{Paradigma 91: Bestimmter Artikel und \isi{Demonstrativpronomen} von Bern \citep[77-79, 102-103]{Marti1985}}

% \begin{table}[H]
% \caption{Bestimmter Artikel und Demonstrativpronomen von Bern \citep[77-79, 102-103]{Marti1985}}\label{table91}
% table\\
% \end{table}

% % \textbf{Paradigma 92: Bestimmter Artikel und \isi{Demonstrativpronomen} von Huzenbach \citep[100, 104-105]{Baur1967}}

% \begin{table}[H]
% \caption{Bestimmter Artikel und Demonstrativpronomen von Huzenbach \citep[100, 104-105]{Baur1967}}\label{table92}
% table\\
% \end{table}

% % \textbf{Paradigma 93: Bestimmter Artikel und \isi{Demonstrativpronomen} von Saulgau \citep[115-116, 119-120]{Raichle1932}}

% \begin{table}[H]
% \caption{Bestimmter Artikel und Demonstrativpronomen von Saulgau \citep[115-116, 119-120]{Raichle1932}}\label{table93}
% table\\
% \end{table}

% % \textbf{Paradigma 94: Bestimmter Artikel und \isi{Demonstrativpronomen} von Stuttgart \citep[154-155]{Frey1975}}

% \begin{table}[H]
% \caption{Bestimmter Artikel und Demonstrativpronomen von Stuttgart \citep[154-155]{Frey1975}}\label{table94}
% table\\
% \end{table}

% % \textbf{Paradigma 95: Bestimmter Artikel und \isi{Demonstrativpronomen} von Petrifeld \citep[63-64, 65]{Moser1937}}

% \begin{table}[H]
% \caption{Bestimmter Artikel und Demonstrativpronomen von Petrifeld \citep[63-64, 65]{Moser1937}}\label{table95}
% table\\
% \end{table}

% % \textbf{Paradigma 96: Bestimmter Artikel und \isi{Demonstrativpronomen} von Elisabethtal \citep[52]{Žirmunskij1928/29}}

% \begin{table}[H]
% \caption{Bestimmter Artikel und Demonstrativpronomen von Elisabethtal \citep[52]{Žirmunskij1928/29}}\label{table96}
% table\\
% \end{table}

% % \textbf{Paradigma 97: Bestimmter Artikel und \isi{Demonstrativpronomen} des Kaiserstuhls \citep[359-378]{Noth1993}}

% \begin{table}[H]
% \caption{Bestimmter Artikel und Demonstrativpronomen des Kaiserstuhls \citep[359-378]{Noth1993}}\label{table97}
% table\\
% \end{table}

% % \textbf{Paradigma 98: Bestimmter Artikel und \isi{Demonstrativpronomen} des Münstertals \citep[47-48]{Mankel1886}}

% \begin{table}[H]
% \caption{Bestimmter Artikel und Demonstrativpronomen des Münstertals \citep[47-48]{Mankel1886}}\label{table98}
% table\\
% \end{table}

% % \textbf{Paradigma 99: Bestimmter Artikel und \isi{Demonstrativpronomen} von Colmar \citep[68-70, 83]{Henry1900}}

% \begin{table}[H]
% \caption{Bestimmter Artikel und Demonstrativpronomen von Colmar \citep[68-70, 83]{Henry1900}}\label{table99}
% table\\
% \end{table}

% % \textbf{Paradigma 100: Bestimmter Artikel und \isi{Demonstrativpronomen} des Elsass (Ebene) \citep[72-78, 84-88]{Beyer1963}}

% \begin{table}[H]
% \caption{Bestimmter Artikel und Demonstrativpronomen des Elsass (Ebene) \citep[72-78, 84-88]{Beyer1963}}\label{table100}
% table\\
% \end{table}

% \textbf{6. Unbestimmter Artikel und Possessivpronomen}\\

% % \textbf{Paradigma 101: \isi{Possessivpronomen} des Althochdeutschen \citep[245-246]{Braune2004}}

% \begin{table}[H]
% \caption{Possessivpronomen des Althochdeutschen \citep[245-246]{Braune2004}}\label{table101}
% table\\
% \end{table}

% % \textbf{Paradigma 102: Unbestimmter Artikel und \isi{Possessivpronomen} des Mittelhochdeutschen \citep[216-217, 231]{Paul2007}}

% \begin{table}[H]
% \caption{Unbestimmter Artikel und Possessivpronomen des Mittelhochdeutschen \citep[216-217, 231]{Paul2007}}\label{table102}
% table\\
% \end{table}

% % \textbf{Paradigma 103: Unbestimmter Artikel und \isi{Possessivpronomen} der deutschen Standardsprache \citep[169-177]{Eisenberg2006}}

% \begin{table}[H]
% \caption{Unbestimmter Artikel und Possessivpronomen der deutschen Standardsprache \citep[169-177]{Eisenberg2006}}\label{table103}
% table\\
% \end{table}

% % \textbf{Paradigma 104: Unbestimmter Artikel und \isi{Possessivpronomen} von Issime \citep[13-16, 82-84]{Perinetto1981}}

% \begin{table}[H]
% \caption{Unbestimmter Artikel und Possessivpronomen von Issime \citep[13-16, 82-84]{Perinetto1981}}\label{table104}
% table\\
% \end{table}

% % \textbf{Paradigma 105: Unbestimmter Artikel und \isi{Possessivpronomen} von Visperterminen \citep[137, 144]{Wipf1911}}

% \begin{table}[H]
% \caption{Unbestimmter Artikel und Possessivpronomen von Visperterminen \citep[137, 144]{Wipf1911}}\label{table105}
% table\\
% \end{table}

% % \textbf{Paradigma 106: Unbestimmter Artikel und \isi{Possessivpronomen} von Jaun \citep[277-278, 284-285]{Stucki1917}}

% \begin{table}[H]
% \caption{Unbestimmter Artikel und Possessivpronomen von Jaun \citep[277-278, 284-285]{Stucki1917}}\label{table106}
% table\\
% \end{table}

% % \textbf{Paradigma 107: Unbestimmter Artikel und \isi{Possessivpronomen} des Sensebezirks \citep[194, 198-199]{Henzen1927}}

% \begin{table}[H]
% \caption{Unbestimmter Artikel und Possessivpronomen des Sensebezirks \citep[194, 198-199]{Henzen1927}}\label{table107}
% table\\
% \end{table}

% % \textbf{Paradigma 108: Unbestimmter Artikel und \isi{Possessivpronomen} von Uri \citep[189, 193-194]{Clauß1929}}

% \begin{table}[H]
% \caption{Unbestimmter Artikel und Possessivpronomen von Uri \citep[189, 193-194]{Clauß1929}}\label{table108}
% table\\
% \end{table}

% % \textbf{Paradigma 109: Unbestimmter Artikel und \isi{Possessivpronomen} von Vorarlberg \citep[269-270, 274-276]{Jutz1925}}

% \begin{table}[H]
% \caption{Unbestimmter Artikel und Possessivpronomen von Vorarlberg \citep[269-270, 274-276]{Jutz1925}}\label{table109}
% table\\
% \end{table}

% % \textbf{Paradigma 110: Unbestimmter Artikel und \isi{Possessivpronomen} von Zürich \citep[104-107, 135-139]{Weber1987}}

% \begin{table}[H]
% \caption{Unbestimmter Artikel und Possessivpronomen von Zürich \citep[104-107, 135-139]{Weber1987}}\label{table110}
% table\\
% \end{table}

% % \textbf{Paradigma 111: Unbestimmter Artikel und \isi{Possessivpronomen} von Bern \citep[79, 98-101]{Marti1985}}

% \begin{table}[H]
% \caption{Unbestimmter Artikel und Possessivpronomen von Bern \citep[79, 98-101]{Marti1985}}\label{table111}
% table\\
% \end{table}

% % \textbf{Paradigma 112: Unbestimmter Artikel und \isi{Possessivpronomen} von Huzenbach \citep[101, 104]{Baur1967}}

% \begin{table}[H]
% \caption{Unbestimmter Artikel und Possessivpronomen von Huzenbach \citep[101, 104]{Baur1967}}\label{table112}
% table\\
% \end{table}

% % \textbf{Paradigma 113: Unbestimmter Artikel und \isi{Possessivpronomen} von Saulgau \citep[116-119]{Raichle1932}}

% \begin{table}[H]
% \caption{ Unbestimmter Artikel und Possessivpronomen von Saulgau \citep[116-119]{Raichle1932}}\label{table113}
% table\\
% \end{table}

% % \textbf{Paradigma 114: Unbestimmter Artikel und \isi{Possessivpronomen} von Stuttgart \citep[156]{Frey1975}}

% \begin{table}[H]
% \caption{nbestimmter Artikel und Possessivpronomen von Stuttgart \citep[156]{Frey1975}}\label{table114}
% table\\
% \end{table}

% % \textbf{Paradigma 115: Unbestimmter Artikel und \isi{Possessivpronomen} von Petrifeld \citep[64-66]{Moser1937}}

% \begin{table}[H]
% \caption{Unbestimmter Artikel und Possessivpronomen von Petrifeld \citep[64-66]{Moser1937}}\label{table115}
% table\\
% \end{table}

% % \textbf{Paradigma 116: Unbestimmter Artikel und \isi{Possessivpronomen} von Elisabethtal \citep[52]{Žirmunskij1928/29}}

% \begin{table}[H]
% \caption{Unbestimmter Artikel und Possessivpronomen von Elisabethtal \citep[52]{Žirmunskij1928/29}}\label{table116}
% table\\
% \end{table}

% % \textbf{Paradigma 117: Unbestimmter Artikel und \isi{Possessivpronomen} des Kaiserstuhls \citep[376, 380-384]{Noth1993}}

% \begin{table}[H]
% \caption{Unbestimmter Artikel und Possessivpronomen des Kaiserstuhls \citep[376, 380-384]{Noth1993}}\label{table117}
% table\\
% \end{table}

% % \textbf{Paradigma 118: Unbestimmter Artikel und \isi{Possessivpronomen} des Münstertals \citep[45-47]{Mankel1886}}

% \begin{table}[H]
% \caption{Unbestimmter Artikel und Possessivpronomen des Münstertals \citep[45-47]{Mankel1886}}\label{table118}
% table\\
% \end{table}

% % \textbf{Paradigma 119: Unbestimmter Artikel und \isi{Possessivpronomen} von Colmar \citep[70-71, 84-85]{Henry1900}}

% \begin{table}[H]
% \caption{Unbestimmter Artikel und Possessivpronomen von Colmar \citep[70-71, 84-85]{Henry1900}}\label{table119}
% table\\
% \end{table}

% % \textbf{Paradigma 120: Unbestimmter Artikel und \isi{Possessivpronomen} des Elsass (Ebene) \citep[78-83, 98-109]{Beyer1963}}

% \begin{table}[H]
% \caption{Unbestimmter Artikel und Possessivpronomen des Elsass (Ebene) \citep[78-83, 98-109]{Beyer1963}}\label{table120}
% table\\
% \end{table}


% \begin{verbatim}%%move bib entries to  localbibliography.bib
% \end{verbatim}
